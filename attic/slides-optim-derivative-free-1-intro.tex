\documentclass[11pt,compress,t,notes=noshow, xcolor=table]{beamer}

\input{../../style/preamble}
\input{../../latex-math/basic-math}
\input{../../latex-math/basic-ml}


\newcommand{\titlefigure}{figure_man/convex_programs.png}
\newcommand{\learninggoals}{
\item Problems with gradient based methods
}


%\usepackage{animate} % only use if you want the animation for Taylor2D

\title{Optimization in Machine Learning}
%\author{Bernd Bischl}
\date{}

\begin{document}

\lecturechapter{Introduction to derivative-free optimization}
\lecture{Optimization in Machine Learning}
\sloppy
%%%%%%%%%%%%%%%%%%%%%%%%%%%%%%%%%%%%%%%%%%%%%%%%%%%%%%%%%%%%%%%%%%%%%%%%%%%%%%%%%%%

\begin{vbframe}{Problems with gradient based methods}

Where do problems possibly arise using a gradient based method?

%\vspace*{-0.2cm}
%\begin{center}
%\includegraphics[width = 0.5\textwidth]{figure_man/griewank1.png} \\
%\footnotesize{Griewank function on $[-400; 400]^2$}
%\end{center}

%\framebreak

%\begin{center}
%\includegraphics{figure_man/griewank2.png} \\
%\footnotesize{Still Griewank function, but zoomed closer on $[-200; 200]^2$.}
%\end{center}

%\framebreak

%\begin{center}
%\includegraphics{figure_man/griewank3.png} \\
%\footnotesize{Griewank function on $[-4, 4]^2$. The function is multimodal.}
%\end{center}

%\framebreak

\vspace{0.3cm}
\begin{center}
\includegraphics[width = 0.45\linewidth]{figure_man/griewank1.png} ~~~ \includegraphics[width = 0.45\linewidth]{figure_man/griewank3.png}
\end{center}

Left: Griewank function on $[-400; 400]^2$\\Right: Griewank function on $[-4, 4]^2$. The function is multimodal.

\framebreak

\begin{center}
\includegraphics{figure_man/problems.png}
\end{center}

\begin{enumerate}
\item Discrete feasible range
\item Non-continuous / differentiable objective function / constraints
\item Multi-modal functions
\item Discontinuous feasible range
\item Analytical form of objective function unknown
\item .....
\end{enumerate}

In such cases, optimization methods that are based on the local gradient should not be used. 
\end{vbframe}

\endlecture
\end{document}


