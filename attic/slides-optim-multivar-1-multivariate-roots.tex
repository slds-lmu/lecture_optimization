\documentclass[11pt,compress,t,notes=noshow, xcolor=table]{beamer}


\input{../../style/preamble}
\input{../../latex-math/basic-math}
\input{../../latex-math/basic-ml}

%\newcommand{\titlefigure}{figure_man/}
\newcommand{\learninggoals}{
\item LEARNING GOAL 1
\item LEARNING GOAL 2}


%\usepackage{animate} % only use if you want the animation for Taylor2D

\title{Optimization}
%\author{}
%\date{}

\begin{document}

\lecturechapter{Multivariate Roots}
\lecture{Optimization}
\sloppy

% --------------------------------------------------------------------------------------------

\begin{vbframe}{Overdetermined systems of equations}

When solving an overdetermined system of equations, i.e.

\vspace*{-0.3cm}
$$
f(\xv) = 0
$$

with $f: \R^d \to \R^m$ with $m > n$ (more equations than unknowns), there is usually no solution for the corresponding equation.

\lz

However, in order to find an approximate solution, we interpret the problem of root search as an \textbf{optimization problem}:
$$
\min_{\xv}\|f(\xv)\|_2^2
$$


To solve this, general optimization techniques or special methods for (nonlinear) \textbf{least squares problems} (see Gauss-Newton algorithm later in this chapter) can be used.

\end{vbframe}

\begin{vbframe}{Convergence of Newton's method}

Under certain conditions, Newton's method converges \textbf{quadratically}. Otherwise, the procedure may converge slowly or not at all.

\vspace*{0.2cm}

\textbf{Possible reasons}:

\begin{itemize}
\item Stationary point is hit: Method terminates because $\nabla f(\xv)= 0$, although root has not yet been found.
\item Bad starting point: Starting point must be close enough to root.
\item \enquote{Overshot}: Function not sufficiently smooth. 
\end{itemize}

\begin{center}
\includegraphics[width=0.7\textwidth]{figure_man/newtonfail1.png} \\
\begin{footnotesize} Newton's method for $f(x) = |x|^{1/4}$ and starting point $x = 0.05$. Although the starting point is close to the root, we move further away from it in each step.\end{footnotesize}
\end{center}

\end{vbframe}

\endlecture
\end{document}


