\documentclass[11pt,compress,t,notes=noshow, xcolor=table]{beamer}

\input{../../style/preamble}
\input{../../latex-math/basic-math}
\input{../../latex-math/basic-ml}

\title{Optimization in Machine Learning}

\begin{document}

\titlemeta{% Chunk title (example: CART, Forests, Boosting, ...), can be empty
}{% Lecture title  
Nelder-Mead method
}{% Relative path to title page image: Can be empty but must not start with slides/
figure_man/Nelder04.png
}{
\item General idea 
\item Reflection, expansion, contraction
\item Advantages \& disadvantages
\item Examples
}


\begin{vbframe}{Nelder-Mead Method}

\begin{itemize}
\item Derivative-free method $\Rightarrow$ heuristic
\item Generalization of bisection in $d$-dimensional space
\item Based on $d$-simplex, defined by $d + 1$ points:
\begin{itemize}
\small
\item $d = 1$ interval
\item $d = 2$ triangle
\item $d = 3$ tetrahedron
\item $\cdots$
\end{itemize}
\end{itemize}

\framebreak

A version of the \textbf{Nelder-Mead} method:

\medskip

\textbf{Initialization:} Choose $d + 1$ random, affinely independent points $\mathbf{v}_i$ ($\mathbf{v}_i$ are vertices: corner points of the simplex/polytope).

\medskip

\begin{enumerate}
\item \textbf{Order}: Order points according to ascending function values
$$
f(\mathbf{v}_1) \leq f(\mathbf{v}_2) \leq \ldots \leq f(\mathbf{v}_d) \leq f(\mathbf{v}_{d + 1}).
$$
with $\mathbf{v}_1$ best point, $\mathbf{v}_{d + 1}$ worst point.

\begin{figure}
\includegraphics[width = 0.5\linewidth]{figure_man/Nelder01.png}
\end{figure}

\item Compute \textbf{centroid} without worst point
$$
\bar{\mathbf{v}} = \frac{1}{d} \sum_{i = 1}^d \mathbf{v}_i.
$$

\begin{figure}
\includegraphics[width = 0.43\linewidth]{figure_man/Nelder02.png} ~~~ \includegraphics[width = 0.43\linewidth]{figure_man/Nelder03.png}
\end{figure}

\framebreak

\item \textbf{Reflection:} Compute reflection point
$$
\mathbf{v}_r = \bar{\mathbf{v}} + \rho (\bar{\mathbf{v}} - \mathbf{v}_{d + 1}),
$$
with $\rho > 0$.
Compute $f(\mathbf{v}_r)$.

\vspace{\baselineskip}

\begin{figure}
\includegraphics[width = 0.43\linewidth]{figure_man/Nelder04.png} 
\end{figure}

\vspace{\baselineskip}

\textbf{Note:} Default value for reflection coefficient: $\rho = 1$

\framebreak

\small
Distinguish three cases:

\begin{itemize}
\small
\item \textbf{Case 1:} $f(\mathbf{v}_1) \leq f(\mathbf{v}_r) < f(\mathbf{v}_d)$

\medskip

$\Rightarrow$ Accept $\mathbf{v}_r$ and discard $\mathbf{v}_{d + 1}$
\end{itemize}


\medskip

\begin{minipage}{0.57\textwidth}
\begin{itemize}
\small
\item \textbf{Case 2:} $f(\mathbf{v}_r) < f(\mathbf{v}_1)$

\medskip

$\Rightarrow$ \textbf{Expansion:} 
\begin{equation*}
\mathbf{v}_e = \bar{\mathbf{v}} + \chi (\mathbf{v}_{r} - \bar{\mathbf{v}}), \quad \chi > 1.
\end{equation*}

We discard $\mathbf{v}_{d + 1}$ and accept the better of $\mathbf{v}_r$ and $\mathbf{v}_e$.
\end{itemize}
\end{minipage}
\begin{minipage}{0.35\textwidth}
\begin{center}
\includegraphics[width = 1\linewidth]{figure_man/Nelder06.png}
\end{center}
\end{minipage}

\vspace{\baselineskip}

\textbf{Note:} Default value for expansion coefficient: $\chi = 2$

\framebreak

\begin{itemize}
\small
\item \textbf{Case 3:} $f(\mathbf{v}_r) \ge f(\mathbf{v}_d)$

\medskip

$\Rightarrow$ \textbf{Contraction:}
\begin{equation*}
\mathbf{v}_c = \bar{\mathbf{v}} + \gamma (\mathbf{v}_{d + 1} - \bar{\mathbf{v}})
\end{equation*}
with $0 < \gamma \le 1/2$.

\begin{itemize}
\item If $f(\mathbf{v}_c) < f(\mathbf{v}_{d + 1})$, accept $\mathbf{v}_c$.
\item Otherwise, shrink \textbf{entire} simplex (\textbf{Shrinking}):
\begin{equation*}
\mathbf{v}_{i} = \mathbf{v}_1 + \sigma (\mathbf{v}_{i} - \mathbf{v}_{1}) \quad \forall i
\end{equation*}
\end{itemize}

\medskip

\textbf{Note:} Default values for contraction and shrinking coefficient: $\gamma = \sigma = 1/2$
\end{itemize}

\item \textbf{Repeat} all steps until stopping criterion met.

\end{enumerate}

\end{vbframe}

\begin{vbframe}{Nelder-Mead}
\small

\textbf{Advantages:}
\begin{itemize}
\item No gradients needed
\item Robust, often works well for non-differentiable functions.
\end{itemize}
\textbf{Drawbacks:}
\begin{itemize}
\item Relatively slow (not applicable in high dimensions)
\item Not each step improves solution, only mean of corner values is reduced.
\item No guarantee for convergence to local optimum / stationary point.
\end{itemize}

\textbf{Visualization:}
\begin{center}
\url{http://www.benfrederickson.com/numerical-optimization/}
\end{center}

\vspace{0.3cm}
\textbf{Note:} Nelder-Mead is default method of \texttt{R} function \texttt{optim()}.
If gradient is available and cheap, L-BFGS is preferred.

\end{vbframe}


\begin{frame}{Nelder-Mead Visualization in 2D}

$$\min_\xv f(x_1,x_2) = x_1^{2} + x_2^{2} + x_1\cdot \sin x_2 + x_2 \cdot \sin x_1 $$ 


\only<1>{\includegraphics{figure_man/nm_animation2d_1.PNG}}
\only<2>{\includegraphics{figure_man/nm_animation2d_2.PNG}}
\only<3>{\includegraphics{figure_man/nm_animation2d_3.PNG}}
\only<4>{\includegraphics{figure_man/nm_animation2d_4.PNG}}

\end{frame}



\begin{vbframe}{Nelder-Mead vs. GD}
%% http://www.benfrederickson.com/numerical-optimization/
\vspace*{-0.5cm}
\begin{figure}
\centering
\begin{minipage}{0.45\textwidth}
\centering
\includegraphics[width = 0.8\linewidth]{figure_man/nm_gd_cities_1.PNG}
\end{minipage}\hfill
\begin{minipage}{0.45\textwidth}
\centering
\includegraphics[width = 0.8\linewidth]{figure_man/nm_gd_cities_2.PNG}
\end{minipage}
\end{figure}
\begin{figure}
\centering
\begin{minipage}{0.45\textwidth}
\centering
\includegraphics[width = 0.8\linewidth]{figure_man/nm_gd_cities_3.PNG}
\end{minipage}\hfill
\begin{minipage}{0.45\textwidth}
\centering
\includegraphics[width = 0.8\linewidth]{figure_man/nm_gd_cities_4.PNG}
\end{minipage}
\end{figure}

\begin{footnotesize}
Nelder-Mead in multiple dimensions:
Organize points (US cities) to keep predefined mutual distances.
For 10 cities, gradient descent (top) converges well for a suitable learning rate.
Nelder-Mead (bottom) fails to converge, even after many iterations.
\end{footnotesize}


\framebreak
\vspace*{-0.8cm}
\begin{figure}
\centering
\begin{minipage}{0.45\textwidth}
\centering
\includegraphics[width = 0.8\linewidth]{figure_man/nm_gd_cities_5.PNG}
\end{minipage}\hfill
\begin{minipage}{0.45\textwidth}
\centering
\includegraphics[width = 0.8\linewidth]{figure_man/nm_gd_cities_6.PNG}
\end{minipage}
\end{figure}
\begin{figure}
\centering
\begin{minipage}{0.45\textwidth}
\centering
\includegraphics[width = 0.8\linewidth]{figure_man/nm_gd_cities_7.PNG}
\end{minipage}\hfill
\begin{minipage}{0.45\textwidth}
\centering
\includegraphics[width = 0.8\linewidth]{figure_man/nm_gd_cities_8.PNG}
\end{minipage}
\end{figure}
\vspace*{0.5cm}
\begin{footnotesize}
Even for only 5 cities, Nelder-Mead (bottom) performs poorly.
However, gradient descent (top) still works.
\end{footnotesize}

\end{vbframe}


\endlecture
\end{document}

