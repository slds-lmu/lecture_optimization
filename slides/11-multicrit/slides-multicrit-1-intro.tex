\documentclass[11pt,compress,t,notes=noshow,xcolor=table]{beamer}

\usepackage{algorithm}
\usepackage[ruled,vlined,algo2e,linesnumbered]{algorithm2e}
%\usepackage[boxed,vlined]{algorithm2e}

\input{../../style/preamble}
\input{../../latex-math/basic-math}
\input{../../latex-math/basic-ml}
\input{../../latex-math/ml-mbo}
\input{slides/11-multicrit/macros}

% -- Macros from original code --------------------------------------
\newcommand{\xx}{\conf}     % x of the optimizer
\newcommand{\xxi}[1][i]{\lambda_{#1}} % i-th component of xx
\newcommand{\XX}{\pcs}     % search space / domain of f
\newcommand{\f}{\cost}     % objective function
\newcommand{\yy}{\cost}    % outcome of objective function
% -------------------------------------------------------------------

\title{Optimization in Machine Learning}

\begin{document}

% Example adaptation of the old \title / \subtitle to the new style:
\titlemeta{%
  Multi-criteria Optimization % chunk title
}{%
  Introduction              % lecture title
}{%
  figure_man/expedia-5-1.pdf % example path to a title image (optional)
}{%
  \item What is multi-criteria optimization?
  \item Motivating examples
  \item Pareto optimality
}

% -------------------------------------------------------------------
\begin{vbframe}{Introductory example}

Often we want to solve optimization problems concerning several goals.

\bigskip
\textbf{General applications:}
\begin{itemize}
\item Medicine: maximum effect, but minimum side effect of a drug.
\item Finances: maximum return, but minimum risk of an equity portfolio.
\item Production planning: maximum revenue, but minimum costs.
\item Booking a hotel: maximum rating, but minimum costs.
\end{itemize}

\bigskip
\textbf{In machine learning:}
\begin{itemize}
\item Sparse models: maximum predictive performance, but minimal number of features.
\item Fast models: maximum predictive performance, but short prediction time.
\item ...
\end{itemize}

\framebreak

\textbf{Example}: Choose the best hotel by maximizing ratings subject to a maximum price per night.

\bigskip
\textbf{Problems}:
\begin{itemize}
 \item The result depends on how we select the maximum price; different price bounds give different solutions.
 \item We could also choose a minimum rating and optimize the price.
 \item The more objectives we include, the harder it gets to fix cutoffs like these.
\end{itemize}

\bigskip
\textbf{Goal}: Find a more general approach to solve multi-criteria problems.

\begin{center}
\includegraphics[width=0.35\linewidth]{figure_man/booking1.png}
\quad
\includegraphics[width=0.35\linewidth]{figure_man/booking2.png}
\end{center}

When booking a hotel: find the hotel with
\begin{itemize}
\item minimum price per night (\textbf{costs}), and
\item maximum user rating (\textbf{performance}).
\end{itemize}

\begin{footnotesize}
Since our standard is to minimize objectives, we minimize negative ratings.
\end{footnotesize}

\framebreak

The objectives often conflict with each other:
\begin{itemize}
\item Lower price $\to$ usually lower hotel rating.
\item Better rating $\to$ usually higher price.
\end{itemize}

Example: (negative) average rating by hotel guests (1--5) vs.\ average price per night.

\bigskip
\begin{center}
\includegraphics[width=0.8\textwidth]{figure_man/expedia-1-1.pdf}
\end{center}

\framebreak

{\footnotesize
We cannot directly establish a total order on the objective space:
\begin{itemize}
  \item Left: A hotel with rating 4 for 89~Euro 
        \((\textcolor{green}{\cost^{(1)}} = (89, -4.0))\) 
        is better than a hotel for 108~Euro with the same rating
        \((\textcolor{red}{\cost^{(2)}} = (108, -4.0))\).
  \item Right: How to decide if 
        \(\textcolor{orange}{\cost^{(1)}}=(89, -4.0)\) is better or worse than 
        \(\textcolor{orange}{\cost^{(2)}}=(95, -4.5)\)?
  \item How much is one \emph{rating point} worth?
\end{itemize}

\begin{center}
\includegraphics[scale=1]{figure_man/expedia-2-1.pdf}
\end{center}

\vspace{-0.1cm}

Let \(\cost = (\text{price}, -\text{rating})\). The candidate 
\(\textcolor{green}{\cost^{(1)}}=(89,-4.0)\)
dominates 
\(\textcolor{red}{\cost^{(2)}}=(108,-4.0)\). 
That is, \(\textcolor{green}{\cost^{(1)}}\) is at least as good 
in every dimension, and strictly better in one dimension.
Hence $\textcolor{green}{\cost^{(1)}} \prec \textcolor{red}{\cost^{(2)}}$. But for
\(\textcolor{orange}{\cost^{(1)}}=(89,-4.0)\) and \(\textcolor{orange}{\cost^{(2)}}=(95,-4.5)\)
we cannot say any is strictly better.


}


\end{vbframe}

% -------------------------------------------------------------------
\begin{vbframe}{How to define optimality?}


\begin{itemize}\setlength{\itemsep}{0.1em}
  \item We call these \textbf{incomparable} (or \emph{equivalent} in the Pareto sense):
\[
\textcolor{orange}{\cost^{(1)}} 
   \not\prec 
\textcolor{orange}{\cost^{(2)}} 
\quad\text{and}\quad
\textcolor{orange}{\cost^{(2)}} 
   \not\prec 
\textcolor{orange}{\cost^{(1)}}.
\]
\item All such non-dominated points form the \textbf{Pareto front}.
\end{itemize}

\bigskip
\begin{columns}[c]
        \column{0.5\textwidth}
        \includegraphics[width=\textwidth]{figure_man/expedia-4-1.pdf}
        
        \column{0.5\textwidth}
        \includegraphics[width=\textwidth]{figure_man/expedia-5-1.pdf}
    \end{columns}

\end{vbframe}


% -------------------------------------------------------------------
\begin{vbframe}{Definition: multi-criteria optimization}

A \textbf{multi-criteria optimization problem} is defined by
\[
\min_{\conf \in \pcs}  \cost(\conf) 
\;\;\Leftrightarrow\;\;
\min_{\conf \in \pcs} 
\bigl(\cost_1(\conf), \cost_2(\conf), \dots, \cost_m(\conf)\bigr),
\]
with \(\pcs \subset \mathbb{R}^n\) and a multi-criteria objective function
\(\cost: \pcs \to \mathbb{R}^m,\; m \ge 2.\)

\begin{itemize}
\item \textbf{Goal:} minimize multiple target functions simultaneously.
\item \((\cost_1(\conf), \dots, \cost_m(\conf))^\top\) maps each candidate 
      \(\conf\) into the objective space \(\mathbb{R}^m\).
\item Typically no single best solution exists, as the objectives are often 
      in conflict and cannot be totally ordered in \(\mathbb{R}^m\).
\item W.l.o.g.\ we always minimize.
\item Also called multi-objective optimization, Pareto optimization, etc.
\end{itemize}

\end{vbframe}

% -------------------------------------------------------------------
\begin{vbframe}{Pareto sets and Pareto optimality}

\textbf{Definition:}

Given 
\[
\min_{\conf \in \pcs} 
   \bigl(\cost_1(\conf), \dots, \cost_m(\conf)\bigr), 
\quad \cost_i: \pcs \to \mathbb{R}.
\]

\begin{itemize}
  \item A candidate \(\conf^{(1)}\) \textbf{(Pareto-)dominates} \(\conf^{(2)}\) 
        if \(\cost(\conf^{(1)}) \prec \cost(\conf^{(2)})\), i.e.:
    \begin{enumerate}
      \item \(\cost_i(\conf^{(1)}) \le \cost_i(\conf^{(2)})\) for all \(i\),
      \item \(\cost_j(\conf^{(1)}) < \cost_j(\conf^{(2)})\) for at least one \(j\).
    \end{enumerate}
    \item A candidate \(\optconf\) not dominated by any other candidate is 
          \textbf{Pareto optimal}.
    \item The set of all Pareto optimal candidates is the \textbf{Pareto set} 
          \(\mathcal{P} = \{\conf \in \pcs \mid \nexists \,\tilde{\conf} :
               \cost(\tilde{\conf}) \prec \cost(\conf)\}\).
    \item \(\mathcal{F} = \cost(\mathcal{P})\), the \textbf{Pareto front}, 
          is the image of that set in objective space.
\end{itemize}

\end{vbframe}


% -------------------------------------------------------------------
\begin{vbframe}{Example: One objective function}

We consider 
\[
\min_{x} \cost(x) = (x - 1)^2, 
\quad 0 \le x \le 3.
\]
The optimum is at \(x^* = 1\).

%\bigskip
\begin{center}
\includegraphics[width=0.5\textwidth]{figure_man/graph1.png}
\end{center}

\end{vbframe}

% -------------------------------------------------------------------
\begin{vbframe}{Example: Two target functions}

We extend the above problem to two objectives 
\(\cost_1(x) = (x - 1)^2\) 
and 
\(\cost_2(x) = 3(x - 2)^2\). Thus
\[
  \min_{x} \cost(x) 
    = \bigl(\cost_1(x),\, \cost_2(x)\bigr),
  \quad 0 \le x \le 3.
\]

\begin{center}
\includegraphics[width=0.5\textwidth]{figure_man/graph2.png}
\end{center}

\framebreak

Visualizing both objectives:
$\bigl(\cost_1(x), \;\cost_2(x)\bigr)
$
for all \(x \in [0,3]\) in the objective space:

\begin{center}
\includegraphics[width=0.45\textwidth]{figure_man/graph3.png}
\end{center}

The Pareto front is shown in green. 

\end{vbframe}

% -------------------------------------------------------------------
\begin{vbframe}{A-priori vs. A-posteriori}

\begin{itemize}
\item The Pareto set is a set of equally valid solutions.
\item Often, we only want a \textbf{single} solution in practice.
\item Without extra info, there is no unique best solution: one must choose 
      based on further criteria (cost constraints, business preference, etc.).
\end{itemize}

\bigskip
Two overall strategies:
\begin{itemize}
\item \textbf{A-priori approach}: incorporate user preferences \emph{before} the optimization.
\item \textbf{A-posteriori approach}: first find (an approximation of) the entire Pareto set, 
      then let the user choose.
\end{itemize}

\end{vbframe}

% -------------------------------------------------------------------
\begin{vbframe}{A-priori procedure}

\textbf{Example: Weighted total}

$$ \min_{\conf} \sum_{i=1}^m w_i \cost_i(\conf) $$

\textbf{Prior knowledge}: 
``One rating point is worth 50~Euro to this customer.'' 

\[
\min_{\text{Hotel}} \text{(Price)} \;-\; 50 \times \text{(Rating)}.
\]
\vspace{-0.2cm}
\begin{center}
\includegraphics[width=0.6\textwidth]{figure_man/expedia-9-1.pdf}
\end{center}


\framebreak

\textbf{Example: Lexicographic method}

\textbf{Prior knowledge}: The rating is more important than the price. 
Hence we optimize the rating first, then the price among those best ratings:

\begin{center}
\includegraphics[width=1\textwidth]{figure_man/expedia-10-1.pdf}
\end{center}

\framebreak
\vspace{-0.3cm}
Lexicographic approach in general:

{\small
\[
\begin{aligned}
\yy_1^* &= \min_{\conf \in \pcs} \;\cost_1(\conf),\\
\yy_2^* &= \min_{\substack{\conf \in \pcs:\\ \cost_1(\conf)=\yy_1^*}} \;\cost_2(\conf),\\
\yy_3^* &= \min_{\substack{\conf \in \pcs:\\ \cost_1(\conf)=\yy_1^*,\,\cost_2(\conf)=\yy_2^*}} 
          \;\cost_3(\conf),\\
&\quad\vdots
\end{aligned}
\]
}

But different orderings yield different solutions.

%\framebreak
{\footnotesize
\textbf{Summary (a-priori)}:
\vspace{-0.1cm}
\begin{itemize}\setlength{\itemsep}{0.7em}
\item Implicit assumption: single-objective optimization is “easy.”
\item We get only \emph{one} solution, which depends strongly on the chosen weighting/order/etc.
\item Varying these parameters systematically can produce several solutions, but may miss large parts of the non-dominated set.
\end{itemize}
}


\end{vbframe}

% -------------------------------------------------------------------
\begin{vbframe}{A-posteriori procedure}

{\footnotesize
A-posteriori methods aim to
\begin{itemize}
  \item find \emph{all} optimal candidates (or a good approximation of them),
  \item let the user pick a single solution based on personal preferences or 
        additional info (e.g.\ hotel location).
\end{itemize}
Hence, a-posteriori methods are more general.
}
%\framebreak

{\footnotesize
\textbf{Example}: 
All Pareto-optimal hotels are shown on the left. 
User then picks from that subset on the right, based on hidden preferences 
(location, brand loyalty, etc.):
}
%\bigskip
\begin{center}
\includegraphics[scale=0.8]{figure_man/expedia-11-1.pdf}
\end{center}

\end{vbframe}

% -------------------------------------------------------------------
\begin{vbframe}{Evaluation of solutions}

\begin{columns}
\begin{column}{0.5\textwidth}
{\footnotesize A common measure for the performance of a set of candidates 
\(\mathcal{P}\subset\pcs\) is the \textbf{dominated hypervolume}:
\[
S(\mathcal{P},R) 
  = \lambda\Bigl(
      \bigcup_{\tilde{\conf}\in \mathcal{P}} 
         \{\;\conf \;\mid\; 
              \tilde{\conf}\prec \conf \prec R
         \}
    \Bigr),
\]
where \(\lambda(\cdot)\) is the Lebesgue measure and R a pre-specified reference point every point in the set must dominate}
\end{column}
\begin{column}{0.5\textwidth}
\begin{center}
\includegraphics[width=0.6\textwidth]{figure_man/dominated_hypervolume.png}
\end{center}
\end{column}
\end{columns}

% add here sentence about reference point
% "and R is a pre-specified reference point and every point in the set must dominate it" (such as maximum +1 of all objectives

%\framebreak

{\footnotesize
\begin{itemize}
\item HV is measured w.r.t.\ the reference point \(R\), often chosen so each 
      component is a known “worst” (upper bound) for that objective.
\item HV is also called the \textbf{S-metric}.
\item Its computation is \(\mathcal{O}(n^{m-1})\) in the number of points 
      and dimension \(m\).
\item Efficient algorithms exist for small \(m\). 
      In ML, we rarely exceed \(m=3\) in practice.
\end{itemize}
}

\end{vbframe}

% -------------------------------------------------------------------
\endlecture
\end{document}
