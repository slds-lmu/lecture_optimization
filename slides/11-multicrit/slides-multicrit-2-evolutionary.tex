\documentclass[11pt,compress,t,notes=noshow,xcolor=table]{beamer}

\usepackage{algorithm}
% or \usepackage{algpseudocode} for a more modern environment
\usepackage[ruled,vlined,algo2e,linesnumbered]{algorithm2e}

\input{../../style/preamble}
\input{../../latex-math/basic-math}
\input{../../latex-math/basic-ml}
\input{../../latex-math/ml-mbo}
\input{macros}

% -- Macros from original code --------------------------------------
\newcommand{\q}[0]{\mathbf{q}}
\newcommand{\xx}{\conf}     % x of the optimizer
\newcommand{\xxi}[1][i]{\lambda_{#1}} % i-th component of xx
\newcommand{\XX}{\pcs}     % search space / domain of f
\newcommand{\f}{\cost}     % objective function
% \newcommand{\y}{\cost}    % (commented out in original code)
% -------------------------------------------------------------------

\title{Optimization in Machine Learning}

\begin{document}

% Adaptation of the old \title / \subtitle to the new style:
\titlemeta{%
  Multi-criteria Optimization % chunk title
}{%
  Evolutionary Approaches    % lecture title
}{%
  figure_man/NSGA2_steps.png % optional title image path
}{%
  \item EMOAs and population-based approach
  \item NSGA-II
  \item SMS-EMOA
}

% -------------------------------------------------------------------
\begin{vbframe}{A-posteriori methods and evolutionary algorithms}

Evolutionary multi-objective algorithms (EMOAs) evolve a diverse population over time to approximate the Pareto front.

\vspace{-0.6cm}

\begin{columns}
\begin{column}{0.3\textwidth}
\begin{center}
\begin{figure}
\centering
\begin{tikzpicture}[node distance=-1.8cm, auto,]
  \begin{footnotesize}
    \node (init) {Initialize population};
    \node[below = 0.1cm of init](rating1) {Eval population};
    \node[below = 0.1cm of rating1](selection1) {Parent selection};
    \node[below = 0.1cm of selection1](variation) {Variation};
    \node[below = 0.1cm of variation](rating2) {Eval offspring};
    \node[below = 0.1cm of rating2](selection2) {Survival selection};
    \node[below = 0.1cm of selection2](stop) {Stop};
    \node[below = 0.2cm of stop](dummy2) {};
    \node[below = 0.2cm of stop](dummy3) {};
    \node[right = 0.01cm of dummy3](dummy4) {yes};
    \node[left = 0.2cm of rating2](dummy1) {no};
    \draw[->] (init) to (rating1);
    \draw[->] (rating1) to (selection1);
    \draw[->] (selection1) to (variation);
    \draw[->] (variation) to (rating2);
    \draw[->] (rating2) to (selection2);
    \draw[->] (selection2) to (stop);
    \draw[->] (stop) to (dummy2);
    \draw[->] (stop) to [bend left=90, looseness=2](selection1);
  \end{footnotesize}
\end{tikzpicture}
\end{figure}
\end{center}
\end{column}
\hspace{0.2cm}
\begin{column}{0.7\textwidth}
\begin{center}
\vspace{0.7cm}
\includegraphics[width=0.98\textwidth]{figure_man/NSGA2_steps.png}
\end{center}
\end{column}
\end{columns}

\begin{itemize}{\small
\item Population-based search approximates a whole Pareto set in one run.
\item The main steps (initialize, select, vary, survive, repeat) mirror standard EAs but must handle multiple objectives.}
\end{itemize}

\end{vbframe}

% -------------------------------------------------------------------
\begin{vbframe}{Basic EA template loop}

\begin{algorithm2e}[H]
\SetAlgoLined         % or \DontPrintSemicolon, whichever style you prefer
\caption{Basic EA template loop}
%\KwCommentSty{}       % optional: adjust comment style if you like

% If you want input / output placeholders, you can do:
% \KwIn{$\XX$ search space, population size $\mu, \lambda$, \dots}
% \KwOut{A set of evaluated solutions \dots}

Initialize and evaluate population $\mathcal{P}_0 \subset \XX$ with $|\mathcal{P}_0| = \mu$\;
$t \gets 0$\;

\Repeat{\textit{Stop criterion is fulfilled}}{%
  Select parents from $\mathcal{P}_t$ and generate offspring $\mathcal{Q}_t$, with $|\mathcal{Q}_t| = \lambda$\;
  Select $\mu$ survivors to form $\mathcal{P}_{t+1}$\;
  $t \gets t + 1$\;
}
\end{algorithm2e}


\bigskip
\begin{itemize}
  \item As in standard EAs, we have parent selection, variation, and survival selection.
  \item For multi-objective EAs (EMOAs), ranking solutions by “fitness” is no longer straightforward. We need special selection strategies (non-dominated sorting, etc.).
\end{itemize}

\end{vbframe}

% -------------------------------------------------------------------
\begin{vbframe}{NSGA-II}

The \textbf{non-dominated sorting genetic algorithm (NSGA-II)} was published by
\cite{DEB2002FAST}.

\begin{itemize}
\item Follows a $(\mu + \lambda)$ strategy for survival.
\item Any standard variation operators (like polynomial mutation, SBX) can be used.
\item Parent and survival selection both use:
  \begin{enumerate}
    \item \textbf{Non-dominated sorting (NDS)} as the main ranking criterion,
    \item \textbf{Crowding distance} as a tie-break.
  \end{enumerate}
\end{itemize}

\end{vbframe}

% -------------------------------------------------------------------
\begin{vbframe}{NSGA-II: non-dominated sorting (I)}

NDS partitions an objective space set into fronts $\mathcal{F}_1 \prec \mathcal{F}_2 \prec \mathcal{F}_3 \prec ... $.

\begin{columns}
\begin{column}{0.4\textwidth}
\begin{itemize}{\small
    \item $\mathcal{F}_1$ is non-dominated, 
      each $\xx \in \mathcal{F}_2$ is dominated, but only by points in $\mathcal{F}_1$, 
      each $\xx \in \mathcal{F}_3$ is dominated, but only by points in $\mathcal{F}_1$ and $\mathcal{F}_2$, 
      and so on. 
    \item We can easily compute the partitioning by computing all non-dominated points  $\mathcal{F}_1$,
        removing them, then computing the next layer of non-dominated points $\mathcal{F}_2$, and so on.}
\end{itemize}
\end{column}

\begin{column}{0.6\textwidth}
\vspace{0.3cm}
\begin{center}
\includegraphics[width = 0.9\textwidth]{figure_man/NSGA2_NDS.png}
\end{center}
\end{column}
\end{columns}

\end{vbframe}

% -------------------------------------------------------------------
\begin{vbframe}{NSGA-II: non-dominated sorting (II)}

For \textbf{survival} in $(\mu + \lambda)$ selection:
\begin{itemize}
\item We rank all $\mu + \lambda$ individuals by front index.
\item We start filling the next population from $\mathcal{F}_1$ upward.
\item If front $\mathcal{F}_i$ cannot fully fit (because we would exceed $\mu$), we partially fill with a subset of $\mathcal{F}_i$.
\end{itemize}

\begin{center}
\includegraphics[width=0.45\linewidth]{figure_man/NSGA2_2.png}
\end{center}

\textbf{But}: Which individuals survive from this partially included front?  
$\to$ \emph{crowding distance} sorting.

\smallskip
\footnotesize
(NDS is also used for parent selection with binary tournaments.)
\normalsize

\end{vbframe}

% -------------------------------------------------------------------
\begin{vbframe}{NSGA-II: crowding distance (I)}

\textbf{Idea}: Prefer individuals that are more “spread out” (less crowded).

\begin{center}
\includegraphics[height=0.55\textheight]{figure_man/NSGA2_CS1.png}
\end{center}

Left: not good, solutions bunched up.  
Right: better spread.

\end{vbframe}

% -------------------------------------------------------------------
\begin{vbframe}{NSGA-II: crowding distance (II)}

\begin{columns}
\begin{column}{0.55\textwidth}
\textbf{Crowding distance}:

For each objective $f_j$:
\begin{itemize}
\item Sort points by $f_j$ (in ascending order).
\item Normalize $f_j$ values to [0, 1].
\item “Border” points (lowest/highest $f_j$) get infinite distance (ensuring they always survive if possible).
\item For internal points, distance is difference in $f_j$-values between next neighbors.
\end{itemize}
Then sum (or average) across all $m$ objectives to get the final crowding score.
\end{column}

\begin{column}{0.45\textwidth}
\begin{center}
\includegraphics[width=0.95\textwidth]{figure_man/NSGA2_CS2.png}
\medskip

\begin{footnotesize}
Red: High CD \quad Blue: Low CD
\end{footnotesize}
\end{center}
\end{column}
\end{columns}

\end{vbframe}

% -------------------------------------------------------------------
% (Commented-out SPEA-2 section from original)

% \begin{vbframe}{SPEA-2}
% ...
% \end{vbframe}

% -------------------------------------------------------------------
\begin{vbframe}{Hypervolume contribution}

\textbf{S-Metric-Selection} uses the \emph{hypervolume contribution} $\Delta s(\xx,\mathcal{P})$ of each individual:
\[
  \Delta s(\xx, \mathcal{P})
  = S(\mathcal{P}, R) - S\bigl(\mathcal{P}\setminus\{\xx\},\,R\bigr),
\]
where $S(\cdot,R)$ is the dominated HV w.r.t.\ reference point $R$.

{\small
\begin{columns}
\begin{column}{0.5\textwidth}
\begin{itemize}
\item Dark rectangles: unique HV part dominated only by that dot
\item Gray cross: reference point $R$.
\item The individual with the smallest HV contribution is “least important” for covering objective space.
\item This is used by the SMS-EMOA to remove individuals in survival selection.
\end{itemize}
\end{column}
\begin{column}{0.5\textwidth}
\begin{center}
\includegraphics[width=0.8\textwidth]{figure_man/hv_contrib.png}
\end{center}
\end{column}
\end{columns}
}
\end{vbframe}

% -------------------------------------------------------------------
\begin{vbframe}{SMS-EMOA algorithm}

\begin{algorithm2e}[H]
\SetAlgoLined
\caption{SMS-EMOA}
Generate initial population $\mathcal{P}_0$ of size $\mu$\;
$t \gets 0$\;

\Repeat{\textit{Termination}}{
  Create exactly one offspring $\q$ by recombination + mutation\;
  Perform non-dominated sorting on $(\mathcal{P}_t \cup \{\q\})$\;
  Let $\mathcal{F}_k$ be the last front\;
  $\tilde{\xx} \gets \arg\min_{\xx \in \mathcal{F}_k} \Delta s\bigl(\xx,\mathcal{F}_k\bigr)$\;
  $\mathcal{P}_{t+1} \gets \bigl(\mathcal{P}_t \cup \{\q\}\bigr)\setminus\{\tilde{\xx}\}$\;
  $t \gets t + 1$\;
}
\end{algorithm2e}

%\begin{algorithm}[H]
%  \begin{center}
%  \caption{SMS-EMOA}
%  \begin{algorithmic}[1]
%    \STATE Generate initial population $\mathcal{P}_0$ of size $\mu$
%    \STATE $t \leftarrow 0$
%    \REPEAT
%      \STATE Create exactly \emph{one} offspring $\q$ by recombination + mutation
%      \STATE Perform non-dominated sorting on $\mathcal{P}_t \cup \{\q\}$
%      \STATE Let $\mathcal{F}_k$ be the last front
%      \STATE $\tilde{\xx} \leftarrow \arg\min_{\xx \in \mathcal{F}_k} \Delta s\bigl(\xx,\mathcal{F}_k\bigr)$
%      \STATE $\mathcal{P}_{t+1} \leftarrow \bigl(\mathcal{P}_t \cup \{\q\}\bigr)\setminus\{\tilde{\xx}\}$
%      \STATE $t \leftarrow t + 1$
%    \UNTIL{Termination}
%  \end{algorithmic}
%  \end{center}
%\end{algorithm}

\begin{itemize}\setlength{\itemsep}{0.3em}{\footnotesize
\item As soon as we exceed size $\mu+1$, remove the worst HV contributor from the \emph{last front}.
\item That way, each iteration keeps exactly $\mu$ individuals.
\item Ranking is primarily by front membership; hypervolume contrib. is the tiebreak.}
\end{itemize}
\end{vbframe}

% -------------------------------------------------------------------
\endlecture
\end{document}
