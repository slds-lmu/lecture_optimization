\documentclass[11pt,compress,t,notes=noshow, xcolor=table]{beamer}

\input{../../style/preamble}
\input{../../latex-math/basic-math}
\input{../../latex-math/basic-ml}


\newcommand{\titlefigure}%{figure_man/hinge_vs_l2.pdf}
{figure_man/tangent.png} 
\newcommand{\learninggoals}{
\item Definition of smoothness
\item Uni- \& multivariate differentiation
\item Gradient, partial derivatives 
\item Jacobi-Matrix
\item Hessian Matrix
}


%\usepackage{animate} % only use if you want the animation for Taylor2D

\title{Optimization in machine learning}
%\author{Bernd Bischl}
\date{}

\begin{document}

\lecturechapter{Matrix calculus}
\lecture{Optimization}
\sloppy

% ------------------------------------------------------------------------------


\begin{vbframe}{Univariate differentiability}

\textbf{Definition:} A function $f: \mathcal{S} \subseteq \R \to \R$ is said to be differentiable for each inner point $x \in \mathcal{S}$ if the following limit exists:

$$
f'(x) := \lim_{h \to 0} \frac{f(x + h) - f(x)}{h}
$$

Intuitively: $f$ can be approxed locally by a lin. fun. with slope $m = f'(x)$. 

\begin{center}
\includegraphics[width = 0.8\textwidth]{figure_man/tangent.png} \\
\begin{footnotesize}
Left: Function is differentiable everywhere. Right: Not differentiable at the red point. 
\end{footnotesize}
\end{center}

% \framebreak

% \textbf{Äquivalente Definition}:

% $f$ ist genau dann differenzierbar bei $\tilde x \in I$, wenn sich $f$ lokal durch eine \textbf{lineare Funktion} (Tangente) approximieren lässt. Das heißt, es existieren

% \begin{itemize}
% \item $m_{\tilde x} \in \R$ (Steigung)
% \item eine Funktion $r(\cdot)$ (Fehler der Approximation),
% \end{itemize}

% sodass

% \begin{eqnarray*}
% % f(x) &=& f(\tilde x) + f'(\tilde x)(x - \tilde x) + r(x - \tilde x) \quad \text{bzw.}\\
% f(\tilde x + h) &=& f(\tilde x) + m_{\tilde x} \cdot h + r(h)\\
% \text{mit } && \lim_{h \to 0}\frac{|r(h)|}{|h|} = 0
% \end{eqnarray*}

% Ist $f$ differenzierbar, dann entspricht $m_{\tilde x} = f'(\tilde x)$ (aus 1. Definition).

\end{vbframe}





\endlecture
\end{document}