\documentclass[11pt,compress,t,notes=noshow, xcolor=table]{beamer}

\usepackage[]{graphicx}
\usepackage[]{color}
% maxwidth is the original width if it is less than linewidth
% otherwise use linewidth (to make sure the graphics do not exceed the margin)
\makeatletter
\def\maxwidth{ %
  \ifdim\Gin@nat@width>\linewidth
    \linewidth
  \else
    \Gin@nat@width
  \fi
}
\makeatother

% ---------------------------------%
% latex-math dependencies, do not remove:
% - mathtools
% - bm
% - siunitx
% - dsfont
% - xspace
% ---------------------------------%

%--------------------------------------------------------%
%       Language, encoding, typography
%--------------------------------------------------------%

\usepackage[english]{babel}
\usepackage[utf8]{inputenc} % Enables inputting UTF-8 symbols
% Standard AMS suite
\usepackage{amsmath,amsfonts,amssymb}

% Font for double-stroke / blackboard letters for sets of numbers (N, R, ...)
% Distribution name is "doublestroke"
% According to https://mirror.physik.tu-berlin.de/pub/CTAN/fonts/doublestroke/dsdoc.pdf
% the "bbm" package does a similar thing and may be superfluous.
% Required for latex-math
\usepackage{dsfont}

% bbm – "Blackboard-style" cm fonts (https://www.ctan.org/pkg/bbm)
% Used to be in common.tex, loaded directly after this file
% Maybe superfluous given dsfont is loaded
% TODO: Check if really unused?
% \usepackage{bbm}

% bm – Access bold symbols in maths mode - https://ctan.org/pkg/bm
% Required for latex-math
% https://tex.stackexchange.com/questions/3238/bm-package-versus-boldsymbol
\usepackage{bm}

% pifont – Access to PostScript standard Symbol and Dingbats fonts
% Used for \newcommand{\xmark}{\ding{55}, which is never used
% aside from lecture_advml/attic/xx-automl/slides.Rnw
% \usepackage{pifont}

% Quotes (inline and display), provdes \enquote
% https://ctan.org/pkg/csquotes
\usepackage{csquotes}

% Adds arg to enumerate env, technically superseded by enumitem according
% to https://ctan.org/pkg/enumerate
% Replace with https://ctan.org/pkg/enumitem ?
\usepackage{enumerate}

% Line spacing - provides \singlespacing \doublespacing \onehalfspacing
% https://ctan.org/pkg/setspace
% \usepackage{setspace}

% mathtools – Mathematical tools to use with amsmath
% https://ctan.org/pkg/mathtools?lang=en
% latex-math dependency according to latex-math repo
\usepackage{mathtools}

% Maybe not great to use this https://tex.stackexchange.com/a/197/19093
% Use align instead -- TODO: Global search & replace to check, eqnarray is used a lot
% $ rg -f -u "\begin{eqnarray" -l | grep -v attic | awk -F '/' '{print $1}' | sort | uniq -c
%   13 lecture_advml
%   14 lecture_i2ml
%    2 lecture_iml
%   27 lecture_optimization
%   45 lecture_sl
\usepackage{eqnarray}

% For shaded regions / boxes
% Used sometimes in optim
% https://www.ctan.org/pkg/framed
\usepackage{framed}

%--------------------------------------------------------%
%       Cite button (version 2024-05)
%--------------------------------------------------------%
% Note this requires biber to be in $PATH when running,
% telltale error in log would be e.g. Package biblatex Info: ... file 'authoryear.dbx' not found
% aside from obvious "biber: command not found" or similar.

\usepackage{hyperref}
\usepackage{usebib}
\usepackage[backend=biber, style=authoryear]{biblatex}

% Only try adding a references file if it exists, otherwise
% this would compile error when references.bib is not found
\IfFileExists{references.bib} {
  \addbibresource{references.bib}
  \bibinput{references}
}

\newcommand{\citelink}[1]{%
\NoCaseChange{\resizebox{!}{9pt}{\protect\beamergotobutton{\href{\usebibentry{\NoCaseChange{#1}}{url}}{\begin{NoHyper}\cite{#1}\end{NoHyper}}}}}%
}

%--------------------------------------------------------%
%       Displaying code and algorithms
%--------------------------------------------------------%

% Reimplements verbatim environments: https://ctan.org/pkg/verbatim
% verbatim used sed at least once in
% supervised-classification/slides-classification-tasks.tex
\usepackage{verbatim}

% Both used together for algorithm typesetting, see also overleaf: https://www.overleaf.com/learn/latex/Algorithms
% algorithmic env is also used, but part of the bundle:
%   "algpseudocode is part of the algorithmicx bundle, it gives you an improved version of algorithmic besides providing some other features"
% According to https://tex.stackexchange.com/questions/229355/algorithm-algorithmic-algorithmicx-algorithm2e-algpseudocode-confused
\usepackage{algorithm}
\usepackage{algpseudocode}

%--------------------------------------------------------%
%       Tables
%--------------------------------------------------------%

% multi-row table cells: https://www.namsu.de/Extra/pakete/Multirow.html
% Provides \multirow
% Used e.g. in evaluation/slides-evaluation-measures-classification.tex
\usepackage{multirow}

% colortbl: https://ctan.org/pkg/colortbl
% "The package allows rows and columns to be coloured, and even individual cells." well.
% Provides \columncolor and \rowcolor
% \rowcolor is used multiple times, e.g. in knn/slides-knn.tex
\usepackage{colortbl}

% long/multi-page tables: https://texdoc.org/serve/longtable.pdf/0
% Not used in slides
% \usepackage{longtable}

% pretty table env: https://ctan.org/pkg/booktabs
% Is used
% Defines \toprule
\usepackage{booktabs}

%--------------------------------------------------------%
%       Figures: Creating, placing, verbing
%--------------------------------------------------------%

% wrapfig - Wrapping text around figures https://de.overleaf.com/learn/latex/Wrapping_text_around_figures
% Provides wrapfigure environment -used in lecture_optimization
\usepackage{wrapfig}

% Sub figures in figures and tables
% https://ctan.org/pkg/subfig -- supersedes subfigure package
% Provides \subfigure
% \subfigure not used in slides but slides-tuning-practical.pdf errors without this pkg, error due to \captionsetup undefined
\usepackage{subfig}

% Actually it's pronounced PGF https://en.wikibooks.org/wiki/LaTeX/PGF/TikZ
\usepackage{tikz}

% No idea what/why these settings are what they are but I assume they're there on purpose
\usetikzlibrary{shapes,arrows,automata,positioning,calc,chains,trees, shadows}
\tikzset{
  %Define standard arrow tip
  >=stealth',
  %Define style for boxes
  punkt/.style={
    rectangle,
    rounded corners,
    draw=black, very thick,
    text width=6.5em,
    minimum height=2em,
    text centered},
  % Define arrow style
  pil/.style={
    ->,
    thick,
    shorten <=2pt,
    shorten >=2pt,}
}

% Defines macros and environments
\usepackage{../../style/lmu-lecture}

\let\code=\texttt     % Used regularly

% Not sure what/why this does
\setkeys{Gin}{width=0.9\textwidth}

% -- knitr leftovers --
% Used often in conjunction with \definecolor{shadecolor}{rgb}{0.969, 0.969, 0.969}
% Removing definitions requires chaning _many many_ slides, which then need checking to see if output still ok
\definecolor{fgcolor}{rgb}{0.345, 0.345, 0.345}
\definecolor{shadecolor}{rgb}{0.969, 0.969, 0.969}
\newenvironment{knitrout}{}{} % an empty environment to be redefined in TeX

%-------------------------------------------------------------------------------------------------------%
%  Unused stuff that needs to go but is kept here currently juuuust in case it was important after all  %
%-------------------------------------------------------------------------------------------------------%

% \newcommand{\hlnum}[1]{\textcolor[rgb]{0.686,0.059,0.569}{#1}}%
% \newcommand{\hlstr}[1]{\textcolor[rgb]{0.192,0.494,0.8}{#1}}%
% \newcommand{\hlcom}[1]{\textcolor[rgb]{0.678,0.584,0.686}{\textit{#1}}}%
% \newcommand{\hlopt}[1]{\textcolor[rgb]{0,0,0}{#1}}%
% \newcommand{\hlstd}[1]{\textcolor[rgb]{0.345,0.345,0.345}{#1}}%
% \newcommand{\hlkwa}[1]{\textcolor[rgb]{0.161,0.373,0.58}{\textbf{#1}}}%
% \newcommand{\hlkwb}[1]{\textcolor[rgb]{0.69,0.353,0.396}{#1}}%
% \newcommand{\hlkwc}[1]{\textcolor[rgb]{0.333,0.667,0.333}{#1}}%
% \newcommand{\hlkwd}[1]{\textcolor[rgb]{0.737,0.353,0.396}{\textbf{#1}}}%
% \let\hlipl\hlkwb

% \makeatletter
% \newenvironment{kframe}{%
%  \def\at@end@of@kframe{}%
%  \ifinner\ifhmode%
%   \def\at@end@of@kframe{\end{minipage}}%
%   \begin{minipage}{\columnwidth}%
%  \fi\fi%
%  \def\FrameCommand##1{\hskip\@totalleftmargin \hskip-\fboxsep
%  \colorbox{shadecolor}{##1}\hskip-\fboxsep
%      % There is no \\@totalrightmargin, so:
%      \hskip-\linewidth \hskip-\@totalleftmargin \hskip\columnwidth}%
%  \MakeFramed {\advance\hsize-\width
%    \@totalleftmargin\z@ \linewidth\hsize
%    \@setminipage}}%
%  {\par\unskip\endMakeFramed%
%  \at@end@of@kframe}
% \makeatother

% \definecolor{shadecolor}{rgb}{.97, .97, .97}
% \definecolor{messagecolor}{rgb}{0, 0, 0}
% \definecolor{warningcolor}{rgb}{1, 0, 1}
% \definecolor{errorcolor}{rgb}{1, 0, 0}
% \newenvironment{knitrout}{}{} % an empty environment to be redefined in TeX

% \usepackage{alltt}
% \newcommand{\SweaveOpts}[1]{}  % do not interfere with LaTeX
% \newcommand{\SweaveInput}[1]{} % because they are not real TeX commands
% \newcommand{\Sexpr}[1]{}       % will only be parsed by R
% \newcommand{\xmark}{\ding{55}}%

% textpos – Place boxes at arbitrary positions on the LATEX page
% https://ctan.org/pkg/textpos
% Provides \begin{textblock}
% TODO: Check if really unused?
% \usepackage[absolute,overlay]{textpos}

% -----------------------%
% Likely knitr leftovers %
% -----------------------%

% psfrag – Replace strings in encapsulated PostScript figures
% https://www.overleaf.com/latex/examples/psfrag-example/tggxhgzwrzhn
% https://ftp.mpi-inf.mpg.de/pub/tex/mirror/ftp.dante.de/pub/tex/macros/latex/contrib/psfrag/pfgguide.pdf
% Can't tell if this is needed
% TODO: Check if really unused?
% \usepackage{psfrag}

% arydshln – Draw dash-lines in array/tabular
% https://www.ctan.org/pkg/arydshln
% !! "arydshln has to be loaded after array, longtable, colortab and/or colortbl"
% Provides \hdashline and \cdashline
% Not used in slides
% \usepackage{arydshln}

% tabularx – Tabulars with adjustable-width columns
% https://ctan.org/pkg/tabularx
% Provides \begin{tabularx}
% Not used in slides
% \usepackage{tabularx}

% placeins – Control float placement
% https://ctan.org/pkg/placeins
% Defines a \FloatBarrier command
% TODO: Check if really unused?
% \usepackage{placeins}

% Can't find a reason why common.tex is not just part of this file?
% This file is included in slides and exercises

% Rarely used fontstyle for R packages, used only in 
% - forests/slides-forests-benchmark.tex
% - exercises/single-exercises/methods_l_1.Rnw
% - slides/cart/attic/slides_extra_trees.Rnw
\newcommand{\pkg}[1]{{\fontseries{b}\selectfont #1}}

% Spacing helpers, used often (mostly in exercises for \dlz)
\newcommand{\lz}{\vspace{0.5cm}} % vertical space (used often in slides)
\newcommand{\dlz}{\vspace{1cm}}  % double vertical space (used often in exercises, never in slides)

% Don't know if this is used or needed, remove?
% textcolor that works in mathmode
% https://tex.stackexchange.com/a/261480
% Used e.g. in forests/slides-forests-bagging.tex
% [...] \textcolor{blue}{\tfrac{1}{M}\sum^M_{m} [...]
% \makeatletter
% \renewcommand*{\@textcolor}[3]{%
%   \protect\leavevmode
%   \begingroup
%     \color#1{#2}#3%
%   \endgroup
% }
% \makeatother


% dependencies: amsmath, amssymb, dsfont
% math spaces
\ifdefined\N
\renewcommand{\N}{\mathds{N}} % N, naturals
\else \newcommand{\N}{\mathds{N}} \fi
\newcommand{\Z}{\mathds{Z}} % Z, integers
\newcommand{\Q}{\mathds{Q}} % Q, rationals
\newcommand{\R}{\mathds{R}} % R, reals
\ifdefined\C
\renewcommand{\C}{\mathds{C}} % C, complex
\else \newcommand{\C}{\mathds{C}} \fi
\newcommand{\continuous}{\mathcal{C}} % C, space of continuous functions
\newcommand{\M}{\mathcal{M}} % machine numbers
\newcommand{\epsm}{\epsilon_m} % maximum error

% counting / finite sets
\newcommand{\setzo}{\{0, 1\}} % set 0, 1
\newcommand{\setmp}{\{-1, +1\}} % set -1, 1
\newcommand{\unitint}{[0, 1]} % unit interval

% basic math stuff
\newcommand{\xt}{\tilde x} % x tilde
\DeclareMathOperator*{\argmax}{arg\,max} % argmax
\DeclareMathOperator*{\argmin}{arg\,min} % argmin
\newcommand{\argminlim}{\mathop{\mathrm{arg\,min}}\limits} % argmax with limits
\newcommand{\argmaxlim}{\mathop{\mathrm{arg\,max}}\limits} % argmin with limits
\newcommand{\sign}{\operatorname{sign}} % sign, signum
\newcommand{\I}{\mathbb{I}} % I, indicator
\newcommand{\order}{\mathcal{O}} % O, order
\newcommand{\bigO}{\mathcal{O}} % Big-O Landau
\newcommand{\littleo}{{o}} % Little-o Landau
\newcommand{\pd}[2]{\frac{\partial{#1}}{\partial #2}} % partial derivative
\newcommand{\floorlr}[1]{\left\lfloor #1 \right\rfloor} % floor
\newcommand{\ceillr}[1]{\left\lceil #1 \right\rceil} % ceiling
\newcommand{\indep}{\perp \!\!\! \perp} % independence symbol

% sums and products
\newcommand{\sumin}{\sum\limits_{i=1}^n} % summation from i=1 to n
\newcommand{\sumim}{\sum\limits_{i=1}^m} % summation from i=1 to m
\newcommand{\sumjn}{\sum\limits_{j=1}^n} % summation from j=1 to p
\newcommand{\sumjp}{\sum\limits_{j=1}^p} % summation from j=1 to p
\newcommand{\sumik}{\sum\limits_{i=1}^k} % summation from i=1 to k
\newcommand{\sumkg}{\sum\limits_{k=1}^g} % summation from k=1 to g
\newcommand{\sumjg}{\sum\limits_{j=1}^g} % summation from j=1 to g
\newcommand{\summM}{\sum\limits_{m=1}^M} % summation from m=1 to M
\newcommand{\meanin}{\frac{1}{n} \sum\limits_{i=1}^n} % mean from i=1 to n
\newcommand{\meanim}{\frac{1}{m} \sum\limits_{i=1}^m} % mean from i=1 to n
\newcommand{\meankg}{\frac{1}{g} \sum\limits_{k=1}^g} % mean from k=1 to g
\newcommand{\meanmM}{\frac{1}{M} \sum\limits_{m=1}^M} % mean from m=1 to M
\newcommand{\prodin}{\prod\limits_{i=1}^n} % product from i=1 to n
\newcommand{\prodkg}{\prod\limits_{k=1}^g} % product from k=1 to g
\newcommand{\prodjp}{\prod\limits_{j=1}^p} % product from j=1 to p

% linear algebra
\newcommand{\one}{\bm{1}} % 1, unitvector
\newcommand{\zero}{\mathbf{0}} % 0-vector
\newcommand{\id}{\bm{I}} % I, identity
\newcommand{\diag}{\operatorname{diag}} % diag, diagonal
\newcommand{\trace}{\operatorname{tr}} % tr, trace
\newcommand{\spn}{\operatorname{span}} % span
\newcommand{\scp}[2]{\left\langle #1, #2 \right\rangle} % <.,.>, scalarproduct
\newcommand{\mat}[1]{\begin{pmatrix} #1 \end{pmatrix}} % short pmatrix command
\newcommand{\Amat}{\mathbf{A}} % matrix A
\newcommand{\Deltab}{\mathbf{\Delta}} % error term for vectors

% basic probability + stats
\renewcommand{\P}{\mathds{P}} % P, probability
\newcommand{\E}{\mathds{E}} % E, expectation
\newcommand{\var}{\mathsf{Var}} % Var, variance
\newcommand{\cov}{\mathsf{Cov}} % Cov, covariance
\newcommand{\corr}{\mathsf{Corr}} % Corr, correlation
\newcommand{\normal}{\mathcal{N}} % N of the normal distribution
\newcommand{\iid}{\overset{i.i.d}{\sim}} % dist with i.i.d superscript
\newcommand{\distas}[1]{\overset{#1}{\sim}} % ... is distributed as ...

% machine learning
\newcommand{\Xspace}{\mathcal{X}} % X, input space
\newcommand{\Yspace}{\mathcal{Y}} % Y, output space
\newcommand{\Zspace}{\mathcal{Z}} % Z, space of sampled datapoints
\newcommand{\nset}{\{1, \ldots, n\}} % set from 1 to n
\newcommand{\pset}{\{1, \ldots, p\}} % set from 1 to p
\newcommand{\gset}{\{1, \ldots, g\}} % set from 1 to g
\newcommand{\Pxy}{\mathbb{P}_{xy}} % P_xy
\newcommand{\Exy}{\mathbb{E}_{xy}} % E_xy: Expectation over random variables xy
\newcommand{\xv}{\mathbf{x}} % vector x (bold)
\newcommand{\xtil}{\tilde{\mathbf{x}}} % vector x-tilde (bold)
\newcommand{\yv}{\mathbf{y}} % vector y (bold)
\newcommand{\xy}{(\xv, y)} % observation (x, y)
\newcommand{\xvec}{\left(x_1, \ldots, x_p\right)^\top} % (x1, ..., xp)
\newcommand{\Xmat}{\mathbf{X}} % Design matrix
\newcommand{\allDatasets}{\mathds{D}} % The set of all datasets
\newcommand{\allDatasetsn}{\mathds{D}_n}  % The set of all datasets of size n
\newcommand{\D}{\mathcal{D}} % D, data
\newcommand{\Dn}{\D_n} % D_n, data of size n
\newcommand{\Dtrain}{\mathcal{D}_{\text{train}}} % D_train, training set
\newcommand{\Dtest}{\mathcal{D}_{\text{test}}} % D_test, test set
\newcommand{\xyi}[1][i]{\left(\xv^{(#1)}, y^{(#1)}\right)} % (x^i, y^i), i-th observation
\newcommand{\Dset}{\left( \xyi[1], \ldots, \xyi[n]\right)} % {(x1,y1)), ..., (xn,yn)}, data
\newcommand{\defAllDatasetsn}{(\Xspace \times \Yspace)^n} % Def. of the set of all datasets of size n
\newcommand{\defAllDatasets}{\bigcup_{n \in \N}(\Xspace \times \Yspace)^n} % Def. of the set of all datasets
\newcommand{\xdat}{\left\{ \xv^{(1)}, \ldots, \xv^{(n)}\right\}} % {x1, ..., xn}, input data
\newcommand{\ydat}{\left\{ \yv^{(1)}, \ldots, \yv^{(n)}\right\}} % {y1, ..., yn}, input data
\newcommand{\yvec}{\left(y^{(1)}, \hdots, y^{(n)}\right)^\top} % (y1, ..., yn), vector of outcomes
\newcommand{\greekxi}{\xi} % Greek letter xi
\renewcommand{\xi}[1][i]{\xv^{(#1)}} % x^i, i-th observed value of x
\newcommand{\yi}[1][i]{y^{(#1)}} % y^i, i-th observed value of y
\newcommand{\xivec}{\left(x^{(i)}_1, \ldots, x^{(i)}_p\right)^\top} % (x1^i, ..., xp^i), i-th observation vector
\newcommand{\xj}{\xv_j} % x_j, j-th feature
\newcommand{\xjvec}{\left(x^{(1)}_j, \ldots, x^{(n)}_j\right)^\top} % (x^1_j, ..., x^n_j), j-th feature vector
\newcommand{\phiv}{\mathbf{\phi}} % Basis transformation function phi
\newcommand{\phixi}{\mathbf{\phi}^{(i)}} % Basis transformation of xi: phi^i := phi(xi)

%%%%%% ml - models general
\newcommand{\lamv}{\bm{\lambda}} % lambda vector, hyperconfiguration vector
\newcommand{\Lam}{\bm{\Lambda}}	 % Lambda, space of all hpos
% Inducer / Inducing algorithm
\newcommand{\preimageInducer}{\left(\defAllDatasets\right)\times\Lam} % Set of all datasets times the hyperparameter space
\newcommand{\preimageInducerShort}{\allDatasets\times\Lam} % Set of all datasets times the hyperparameter space
% Inducer / Inducing algorithm
\newcommand{\ind}{\mathcal{I}} % Inducer, inducing algorithm, learning algorithm

% continuous prediction function f
\newcommand{\ftrue}{f_{\text{true}}}  % True underlying function (if a statistical model is assumed)
\newcommand{\ftruex}{\ftrue(\xv)} % True underlying function (if a statistical model is assumed)
\newcommand{\fx}{f(\xv)} % f(x), continuous prediction function
\newcommand{\fdomains}{f: \Xspace \rightarrow \R^g} % f with domain and co-domain
\newcommand{\Hspace}{\mathcal{H}} % hypothesis space where f is from
\newcommand{\fbayes}{f^{\ast}} % Bayes-optimal model
\newcommand{\fxbayes}{f^{\ast}(\xv)} % Bayes-optimal model
\newcommand{\fkx}[1][k]{f_{#1}(\xv)} % f_j(x), discriminant component function
\newcommand{\fh}{\hat{f}} % f hat, estimated prediction function
\newcommand{\fxh}{\fh(\xv)} % fhat(x)
\newcommand{\fxt}{f(\xv ~|~ \thetav)} % f(x | theta)
\newcommand{\fxi}{f\left(\xv^{(i)}\right)} % f(x^(i))
\newcommand{\fxih}{\hat{f}\left(\xv^{(i)}\right)} % f(x^(i))
\newcommand{\fxit}{f\left(\xv^{(i)} ~|~ \thetav\right)} % f(x^(i) | theta)
\newcommand{\fhD}{\fh_{\D}} % fhat_D, estimate of f based on D
\newcommand{\fhDtrain}{\fh_{\Dtrain}} % fhat_Dtrain, estimate of f based on D
\newcommand{\fhDnlam}{\fh_{\Dn, \lamv}} %model learned on Dn with hp lambda
\newcommand{\fhDlam}{\fh_{\D, \lamv}} %model learned on D with hp lambda
\newcommand{\fhDnlams}{\fh_{\Dn, \lamv^\ast}} %model learned on Dn with optimal hp lambda
\newcommand{\fhDlams}{\fh_{\D, \lamv^\ast}} %model learned on D with optimal hp lambda

% discrete prediction function h
\newcommand{\hx}{h(\xv)} % h(x), discrete prediction function
\newcommand{\hh}{\hat{h}} % h hat
\newcommand{\hxh}{\hat{h}(\xv)} % hhat(x)
\newcommand{\hxt}{h(\xv | \thetav)} % h(x | theta)
\newcommand{\hxi}{h\left(\xi\right)} % h(x^(i))
\newcommand{\hxit}{h\left(\xi ~|~ \thetav\right)} % h(x^(i) | theta)
\newcommand{\hbayes}{h^{\ast}} % Bayes-optimal classification model
\newcommand{\hxbayes}{h^{\ast}(\xv)} % Bayes-optimal classification model

% yhat
\newcommand{\yh}{\hat{y}} % yhat for prediction of target
\newcommand{\yih}{\hat{y}^{(i)}} % yhat^(i) for prediction of ith targiet
\newcommand{\resi}{\yi- \yih}

% theta
\newcommand{\thetah}{\hat{\theta}} % theta hat
\newcommand{\thetav}{\bm{\theta}} % theta vector
\newcommand{\thetavh}{\bm{\hat\theta}} % theta vector hat
\newcommand{\thetat}[1][t]{\thetav^{[#1]}} % theta^[t] in optimization
\newcommand{\thetatn}[1][t]{\thetav^{[#1 +1]}} % theta^[t+1] in optimization
\newcommand{\thetahDnlam}{\thetavh_{\Dn, \lamv}} %theta learned on Dn with hp lambda
\newcommand{\thetahDlam}{\thetavh_{\D, \lamv}} %theta learned on D with hp lambda
\newcommand{\mint}{\min_{\thetav \in \Theta}} % min problem theta
\newcommand{\argmint}{\argmin_{\thetav \in \Theta}} % argmin theta
% LS 29.10.2024 addin thetab back for now because apparently this broke and nobody updated slides to reflect thetab -> thetav changes?
\newcommand{\thetab}{\bm{\theta}} % theta vector


% densities + probabilities
% pdf of x
\newcommand{\pdf}{p} % p
\newcommand{\pdfx}{p(\xv)} % p(x)
\newcommand{\pixt}{\pi(\xv~|~ \thetav)} % pi(x|theta), pdf of x given theta
\newcommand{\pixit}[1][i]{\pi\left(\xi[#1] ~|~ \thetav\right)} % pi(x^i|theta), pdf of x given theta
\newcommand{\pixii}[1][i]{\pi\left(\xi[#1]\right)} % pi(x^i), pdf of i-th x

% pdf of (x, y)
\newcommand{\pdfxy}{p(\xv,y)} % p(x, y)
\newcommand{\pdfxyt}{p(\xv, y ~|~ \thetav)} % p(x, y | theta)
\newcommand{\pdfxyit}{p\left(\xi, \yi ~|~ \thetav\right)} % p(x^(i), y^(i) | theta)

% pdf of x given y
\newcommand{\pdfxyk}[1][k]{p(\xv | y= #1)} % p(x | y = k)
\newcommand{\lpdfxyk}[1][k]{\log p(\xv | y= #1)} % log p(x | y = k)
\newcommand{\pdfxiyk}[1][k]{p\left(\xi | y= #1 \right)} % p(x^i | y = k)

% prior probabilities
\newcommand{\pik}[1][k]{\pi_{#1}} % pi_k, prior
\newcommand{\lpik}[1][k]{\log \pi_{#1}} % log pi_k, log of the prior
\newcommand{\pit}{\pi(\thetav)} % Prior probability of parameter theta

% posterior probabilities
\newcommand{\post}{\P(y = 1 ~|~ \xv)} % P(y = 1 | x), post. prob for y=1
\newcommand{\postk}[1][k]{\P(y = #1 ~|~ \xv)} % P(y = k | y), post. prob for y=k
\newcommand{\pidomains}{\pi: \Xspace \rightarrow \unitint} % pi with domain and co-domain
\newcommand{\pibayes}{\pi^{\ast}} % Bayes-optimal classification model
\newcommand{\pixbayes}{\pi^{\ast}(\xv)} % Bayes-optimal classification model
\newcommand{\pix}{\pi(\xv)} % pi(x), P(y = 1 | x)
\newcommand{\piv}{\bm{\pi}} % pi, bold, as vector
\newcommand{\pikx}[1][k]{\pi_{#1}(\xv)} % pi_k(x), P(y = k | x)
\newcommand{\pikxt}[1][k]{\pi_{#1}(\xv ~|~ \thetav)} % pi_k(x | theta), P(y = k | x, theta)
\newcommand{\pixh}{\hat \pi(\xv)} % pi(x) hat, P(y = 1 | x) hat
\newcommand{\pikxh}[1][k]{\hat \pi_{#1}(\xv)} % pi_k(x) hat, P(y = k | x) hat
\newcommand{\pixih}{\hat \pi(\xi)} % pi(x^(i)) with hat
\newcommand{\pikxih}[1][k]{\hat \pi_{#1}(\xi)} % pi_k(x^(i)) with hat
\newcommand{\pdfygxt}{p(y ~|~\xv, \thetav)} % p(y | x, theta)
\newcommand{\pdfyigxit}{p\left(\yi ~|~\xi, \thetav\right)} % p(y^i |x^i, theta)
\newcommand{\lpdfygxt}{\log \pdfygxt } % log p(y | x, theta)
\newcommand{\lpdfyigxit}{\log \pdfyigxit} % log p(y^i |x^i, theta)

% probababilistic
\newcommand{\bayesrulek}[1][k]{\frac{\P(\xv | y= #1) \P(y= #1)}{\P(\xv)}} % Bayes rule
\newcommand{\muk}{\bm{\mu_k}} % mean vector of class-k Gaussian (discr analysis)

% residual and margin
\newcommand{\eps}{\epsilon} % residual, stochastic
\newcommand{\epsv}{\bm{\epsilon}} % residual, stochastic, as vector
\newcommand{\epsi}{\epsilon^{(i)}} % epsilon^i, residual, stochastic
\newcommand{\epsh}{\hat{\epsilon}} % residual, estimated
\newcommand{\epsvh}{\hat{\epsv}} % residual, estimated, vector
\newcommand{\yf}{y \fx} % y f(x), margin
\newcommand{\yfi}{\yi \fxi} % y^i f(x^i), margin
\newcommand{\Sigmah}{\hat \Sigma} % estimated covariance matrix
\newcommand{\Sigmahj}{\hat \Sigma_j} % estimated covariance matrix for the j-th class

% ml - loss, risk, likelihood
\newcommand{\Lyf}{L\left(y, f\right)} % L(y, f), loss function
\newcommand{\Lypi}{L\left(y, \pi\right)} % L(y, pi), loss function
\newcommand{\Lxy}{L\left(y, \fx\right)} % L(y, f(x)), loss function
\newcommand{\Lxyi}{L\left(\yi, \fxi\right)} % loss of observation
\newcommand{\Lxyt}{L\left(y, \fxt\right)} % loss with f parameterized
\newcommand{\Lxyit}{L\left(\yi, \fxit\right)} % loss of observation with f parameterized
\newcommand{\Lxym}{L\left(\yi, f\left(\bm{\tilde{x}}^{(i)} ~|~ \thetav\right)\right)} % loss of observation with f parameterized
\newcommand{\Lpixy}{L\left(y, \pix\right)} % loss in classification
\newcommand{\Lpiy}{L\left(y, \pi\right)} % loss in classification
\newcommand{\Lpiv}{L\left(y, \piv\right)} % loss in classification
\newcommand{\Lpixyi}{L\left(\yi, \pixii\right)} % loss of observation in classification
\newcommand{\Lpixyt}{L\left(y, \pixt\right)} % loss with pi parameterized
\newcommand{\Lpixyit}{L\left(\yi, \pixit\right)} % loss of observation with pi parameterized
\newcommand{\Lhy}{L\left(y, h\right)} % L(y, h), loss function on discrete classes
\newcommand{\Lhxy}{L\left(y, \hx\right)} % L(y, h(x)), loss function on discrete classes
\newcommand{\Lr}{L\left(r\right)} % L(r), loss defined on residual (reg) / margin (classif)
\newcommand{\lone}{|y - \fx|} % L1 loss
\newcommand{\ltwo}{\left(y - \fx\right)^2} % L2 loss
\newcommand{\lbernoullimp}{\ln(1 + \exp(-y \cdot \fx))} % Bernoulli loss for -1, +1 encoding
\newcommand{\lbernoullizo}{- y \cdot \fx + \log(1 + \exp(\fx))} % Bernoulli loss for 0, 1 encoding
\newcommand{\lcrossent}{- y \log \left(\pix\right) - (1 - y) \log \left(1 - \pix\right)} % cross-entropy loss
\newcommand{\lbrier}{\left(\pix - y \right)^2} % Brier score
\newcommand{\risk}{\mathcal{R}} % R, risk
\newcommand{\riskbayes}{\mathcal{R}^\ast}
\newcommand{\riskf}{\risk(f)} % R(f), risk
\newcommand{\riskdef}{\E_{y|\xv}\left(\Lxy \right)} % risk def (expected loss)
\newcommand{\riskt}{\mathcal{R}(\thetav)} % R(theta), risk
\newcommand{\riske}{\mathcal{R}_{\text{emp}}} % R_emp, empirical risk w/o factor 1 / n
\newcommand{\riskeb}{\bar{\mathcal{R}}_{\text{emp}}} % R_emp, empirical risk w/ factor 1 / n
\newcommand{\riskef}{\riske(f)} % R_emp(f)
\newcommand{\risket}{\mathcal{R}_{\text{emp}}(\thetav)} % R_emp(theta)
\newcommand{\riskr}{\mathcal{R}_{\text{reg}}} % R_reg, regularized risk
\newcommand{\riskrt}{\mathcal{R}_{\text{reg}}(\thetav)} % R_reg(theta)
\newcommand{\riskrf}{\riskr(f)} % R_reg(f)
\newcommand{\riskrth}{\hat{\mathcal{R}}_{\text{reg}}(\thetav)} % hat R_reg(theta)
\newcommand{\risketh}{\hat{\mathcal{R}}_{\text{emp}}(\thetav)} % hat R_emp(theta)
\newcommand{\LL}{\mathcal{L}} % L, likelihood
\newcommand{\LLt}{\mathcal{L}(\thetav)} % L(theta), likelihood
\newcommand{\LLtx}{\mathcal{L}(\thetav | \xv)} % L(theta|x), likelihood
\newcommand{\logl}{\ell} % l, log-likelihood
\newcommand{\loglt}{\logl(\thetav)} % l(theta), log-likelihood
\newcommand{\logltx}{\logl(\thetav | \xv)} % l(theta|x), log-likelihood
\newcommand{\errtrain}{\text{err}_{\text{train}}} % training error
\newcommand{\errtest}{\text{err}_{\text{test}}} % test error
\newcommand{\errexp}{\overline{\text{err}_{\text{test}}}} % avg training error

% lm
\newcommand{\thx}{\thetav^\top \xv} % linear model
\newcommand{\olsest}{(\Xmat^\top \Xmat)^{-1} \Xmat^\top \yv} % OLS estimator in LM


\definecolor{customblue}{HTML}{517FF7}
\newcommand{\deriv}{d}

\title{Optimization in Machine Learning}

\begin{document}

\titlemeta{% Chunk title (example: CART, Forests, Boosting, ...), can be empty
  Mathematical Concepts 
  }{% Lecture title  
  Matrix Calculus
  }{% Relative path to title page image: Can be empty but must not start with slides/
  figure_man/1920px-Greek_lc_delta.svg.png
  }{
    \item Rules of matrix calculus 
    \item Connection of gradient, Jacobian and Hessian
}

% ------------------------------------------------------------------------------

\begin{vbframe}{Scope}
\begin{itemize}
    \setlength{\itemsep}{0.5\baselineskip}
    \item $\mathcal{X}$/$\mathcal{Y}$ denote space of \textbf{independent}/\textbf{dependent} variables
    \item Identify dependent variable $y$ with a \textbf{function} $f: \mathcal{X} \to \mathcal{Y}, x\mapsto f(x)$
    \item Assume $y$ sufficiently smooth
    \item In matrix calculus, $x$ and $y$ can be \textbf{scalars}, \textbf{vectors}, or \textbf{matrices}
        \item We denote vectors/matrices in \textbf{bold} lowercase/uppercase letters
        \vspace{0.5\baselineskip}
        \begin{table}
            \centering
            \begin{tabular}{c||c|c|c}
                 Type & scalar $x$ & vector $\xv$ & matrix $\mathbf{X}$ \\ \hline\hline
                 scalar $y$ & $\deriv y / \deriv x$ & $\deriv y / \deriv\xv$ & $\deriv y / \deriv\mathbf{X}$ \\ \hline
                 vector $\yv$ & $\deriv\yv / \deriv x$ & $\deriv\yv / \deriv\xv$ & -- \\ \hline
                 matrix $\mathbf{Y}$ & $\deriv\mathbf{Y} / \deriv x$ & -- & --
            \end{tabular}
        \end{table}
        \item This notation is also reffered to as \emph{Leibniz Notation}
\end{itemize}
\end{vbframe}

\begin{vbframe}{Leibniz Notation convention}
    \begin{itemize} \setlength{\itemsep}{0.5\baselineskip}
        \item Instead of writing $f(x)$ everywhere, we replace the function $f$ with the variable $y$.
\item This helps clarify relationships when multiple functions or variables are involved, especially in contexts like partial derivatives or matrix calculus.
  \item Also applicable to partial derivatives: For \( y = f:\R^n\longrightarrow\R,\; \xv\mapsto\fx=f(x_1, x_2, \ldots, x_n) \), the partial derivative of $y$ w.r.t. \( x_i \) is $
      \deriv y/\deriv x_i$.
\item \textbf{Examples:}
\begin{itemize}
    \item    $
      y = x^3 + 5x \Longrightarrow
      \dfrac{dy}{dx} = 3x^2 + 5
   $\\[10pt]
   \item $y = x_1^2 + 3x_2 \Longrightarrow
      \dfrac{\deriv y}{\deriv x_1} = 2x_1, \quad
      \dfrac{\deriv y}{\deriv x_2} = 3$
\end{itemize}
    \end{itemize}
\end{vbframe}

\begin{vbframe}{Derivatives of scalar-valued functions}
        \vspace{0.5\baselineskip}
        \begin{table}
            \centering
            \begin{tabular}{c||c|c|c}
                 Type & scalar $x$ & vector $\xv$ & matrix $\mathbf{X}$ \\ \hline\hline
                 scalar $y$ & $\deriv y / \deriv x$ & \cellcolor{customblue}$\deriv y / \deriv\xv$ & \cellcolor{customblue}$\deriv y / \deriv\mathbf{X}$ \\ \hline
                 vector $\yv$ & $\deriv\yv / \deriv x$ & $\deriv\yv / \deriv\xv$ & -- \\ \hline
                 matrix $\mathbf{Y}$ & $\deriv\mathbf{Y} / \deriv x$ & -- & --
            \end{tabular}
        \end{table}\,\\
        \begin{itemize}
        \item $\deriv y / \deriv\xv$ is the gradient from the previous slide deck\\[\baselineskip]
            \item When the input is a matrix the concept remains the same, i.e. for $y= f:\R^{m\times n}\longrightarrow\R,\; \mathbf{X}\mapsto f(\mathbf{X})$
$$
\frac{\deriv y}{\deriv\bm{X}}= \left(\begin{array}{ccc}
\dfrac{\partial f}{\partial x_{11}} & \cdots & \dfrac{\partial f}{\partial x_{1 n}} \\
\vdots & \ddots & \vdots \\
\dfrac{\partial f}{\partial x_{m 1}} & \cdots & \dfrac{\partial f }{\partial x_{m n}}
\end{array}\right) \in \mathbb{R}^{m \times n}
$$
        \end{itemize}
\end{vbframe}


\begin{vbframe}{Derivatives: Univariate and Jacobian}
        \vspace{0.5\baselineskip}
        \begin{table}
            \centering
            \begin{tabular}{c||c|c|c}
                 Type & scalar $x$ & vector $\xv$ & matrix $\mathbf{X}$ \\ \hline\hline
                 scalar $y$ &\cellcolor{customblue} $\deriv y / \deriv x$ & $\deriv y / \deriv\xv$ & $\deriv y / \deriv\mathbf{X}$ \\ \hline
                 vector $\yv$ & $\deriv\yv / \deriv x$ & \cellcolor{customblue}$\deriv\yv / \deriv\xv$ & -- \\ \hline
                 matrix $\mathbf{Y}$ & $\deriv\mathbf{Y} / \deriv x$ & -- & --
            \end{tabular}
        \end{table}\,\\
        \begin{itemize}
        \item $\deriv y / \deriv x$ is the univariate derivative $y'$\\[\baselineskip]
            \item $\deriv\yv / \deriv\xv$ is the Jacobian from the previous slide deck
        \end{itemize}
\end{vbframe}



\begin{vbframe}{Derivatives of functions with scalars as inputs}
       % \vspace{0.5\baselineskip}
        \begin{table}
            \centering
            \begin{tabular}{c||c|c|c}
                 Type & scalar $x$ & vector $\xv$ & matrix $\mathbf{X}$ \\ \hline\hline
                 scalar $y$ & $\deriv y / \deriv x$ & $\deriv y / \deriv\xv$ & $\deriv y / \deriv\mathbf{X}$ \\ \hline
                 vector $\yv$ & \cellcolor{customblue}$\deriv\yv / \deriv x$ & $\deriv\yv / \deriv\xv$ & -- \\ \hline
                 matrix $\mathbf{Y}$ & \cellcolor{customblue}$\deriv\mathbf{Y} / \deriv x$ & -- & --
            \end{tabular}
        \end{table}\,\\
        \begin{itemize}
        \item  Here, for univariate $f_{ij}:\R\rightarrow\R$, $\yv$ ($n=1$) or $\mathbf{Y}$ ($n>1$) are equal to a function $f: \R\longrightarrow\R^{m\times n}, x\mapsto \big(f_{ij}(x)\big)_{i=1,\dots,m;\,j=1,\dots,n}$ and the derivatives are, respectively given by $$
        \frac{\deriv\yv}{\deriv x}=\begin{pmatrix}
            \dfrac{\partial f_{1}}{\partial x}\\\vdots\\ \dfrac{\partial f_{m}}{\partial x}
        \end{pmatrix}\in\R^m;\quad
        \frac{\deriv\mathbf{Y}}{\deriv x}=\left(\begin{array}{ccc}
\dfrac{\partial f_{11}}{\partial x} & \cdots & \dfrac{\partial f_{1 n}}{\partial x} \\
\vdots & \ddots & \vdots \\
\dfrac{\partial f_{m 1}}{\partial x} & \cdots & \dfrac{\partial f_{m n}}{\partial x}
\end{array}\right) \in \mathbb{R}^{m \times n}$$
        \end{itemize}
\end{vbframe}

\begin{vbframe}{Multivariate Differentiation Rules}
  \begin{itemize}
    \item Basic rules from single-variable calculus still apply.
    \item But, for \(\boldsymbol{x}\in\mathbb{R}^{n}\): gradients are vectors/matrices (order matters).
  \end{itemize}
  
  \vspace{1em}
  \textbf{Key Rules:}\\
  
  \begin{itemize}
      \item \textbf{Sum:} 
      \[
      \frac{\deriv}{\deriv \boldsymbol{x}}\bigl(f+g\bigr)=\frac{\deriv f}{\deriv \boldsymbol{x}}+\frac{\deriv g}{\deriv \boldsymbol{x}}
      \]
    \item \textbf{Product:} 
      \[
      \frac{\deriv}{\deriv \boldsymbol{x}}\bigl(fg\bigr)=\frac{\deriv f}{\deriv \boldsymbol{x}}\,g+f\,\frac{\deriv g}{\deriv \boldsymbol{x}}
      \]
    \item \textbf{Chain:} 
      \[
      \frac{\deriv}{\deriv \boldsymbol{x}}\Bigl((f\circ g)(\boldsymbol{x})\Bigr)=\dfrac{\deriv}{\deriv \boldsymbol{x}}(f(g(\boldsymbol{x})))=\frac{\deriv f}{\deriv g}\,\frac{\deriv g}{\deriv \boldsymbol{x}}
      \]
  \end{itemize}
\end{vbframe}

\begin{frame}[allowframebreaks]{Details on the Chain Rule}
\vspace*{-5pt}
\begin{itemize}
    \item Suppose\begin{itemize}
    \item  we have functions $\boldsymbol{g}: S\subseteq\mathbb{R}^n \rightarrow \mathbb{R}^m$ and $\boldsymbol{f}: T\subseteq\mathbb{R}^m \rightarrow \mathbb{R}^{\ell}$
    \item $\boldsymbol{a} \in S$ is a point such that $\boldsymbol{g}(\boldsymbol{a}) \in T$ $\Rightarrow$ $\boldsymbol{f} \circ \boldsymbol{g}(\boldsymbol{x})=\boldsymbol{f}(\boldsymbol{g}(\boldsymbol{x}))$ is well-defined for all $\boldsymbol{x}$ close to $\boldsymbol{a}$
\end{itemize}
\item Then, if $\boldsymbol{g}$ is differentiable at $\boldsymbol{a}$ and $\boldsymbol{f}$ is differentiable at $\boldsymbol{g}(\boldsymbol{a})$ $\Rightarrow\boldsymbol{f} \circ \boldsymbol{g}$ is differentiable at $\boldsymbol{a}$, and the derivative $\dfrac{\deriv \boldsymbol f}{\deriv \boldsymbol g} \dfrac{\deriv \boldsymbol g}{\deriv \boldsymbol{x}}$, is equal to
\begin{align*}
\nabla_{\boldsymbol{a}} \boldsymbol{f} \circ \boldsymbol{g}\;\hat{=}\;\boldsymbol{J}_{\boldsymbol{f} \circ \boldsymbol{g}}(\boldsymbol{a})=\boldsymbol{J}_{\boldsymbol{f}}(\boldsymbol{g}(\boldsymbol{a}))\boldsymbol{J}_ {\boldsymbol{g}}(\boldsymbol{a})\;\hat{=}\;\nabla_{\boldsymbol{g}(\boldsymbol{a})}{\boldsymbol{f}}\;\nabla_{\boldsymbol{a}} {\boldsymbol{g}}\in\R^{l\times n}
\end{align*}
\noindent {\footnotesize (See
\href{https://www.math.utoronto.ca/courses/mat237y1/20199/notes/Chapter2/S2.3.html\#sect-2.3.4}{Chapter 2.3 of the UofT course \emph{MAT237 - Multivariable Calculus} for proof})}\\[5pt]

\item We can also write $\boldsymbol{f}$ as a function of $\boldsymbol{y}=\left(y_1, \ldots, y_m\right) \in \mathbb{R}^m$, and $\boldsymbol{g}$ as a function of $\boldsymbol{x}=\left(x_1, \ldots, x_n\right) \in \mathbb{R}^n$. Then for each $k=1, \ldots, \ell$ and $j=1, \ldots, n$
\begin{equation*}
\left[\boldsymbol{J}_{\boldsymbol{f} \circ \boldsymbol{g}}(\boldsymbol{a})\right]_{kj} = \frac{\partial}{\partial x_j}\left(f_k \circ \boldsymbol{g}\right)(\boldsymbol{a})=\sum_{i=1}^m \frac{\partial f_k}{\partial y_i}(\boldsymbol{g}(\boldsymbol{a})) \frac{\partial g_i}{\partial x_j}(\boldsymbol{a})
\end{equation*}
\end{itemize}
\end{frame}

\begin{vbframe}{Helpful Calculation Rules}
Let $\boldsymbol{a}$, $\boldsymbol{b}$ denote vectors, $\boldsymbol{X}$, $\boldsymbol{A}$ matrices, and $f(\boldsymbol{X})^{-1}$ the inverse of $f(\boldsymbol{X})$ if it exists. 
 \begin{itemize}
    \item $ \frac{\deriv \boldsymbol{x}^{\top} \boldsymbol{a}}{\deriv \boldsymbol{x}}=\boldsymbol{a}^{\top}$, $\frac{\deriv \boldsymbol{a}^{\top} \boldsymbol{x}}{\deriv \boldsymbol{x}}=\boldsymbol{a}^{\top}$
    \item $\frac{\deriv\boldsymbol{X}\boldsymbol{a}}{\deriv\boldsymbol{a}} = \boldsymbol{X}$, $\frac{\deriv\boldsymbol{a}^T\boldsymbol{X}}{\deriv\boldsymbol{a}} = \boldsymbol{X}^T$
    \item $\frac{\deriv \boldsymbol{a}^{\top} \boldsymbol{X} \boldsymbol{b}}{\deriv \boldsymbol{X}}=\boldsymbol{a} \boldsymbol{b}^{\top}$
    \item $\frac{\deriv \boldsymbol{x}^{\top} \boldsymbol{A} \boldsymbol{x}}{\deriv \boldsymbol{x}}=\boldsymbol{x}^{\top}\left(\boldsymbol{A}+\boldsymbol{A}^{\top}\right)$
    \item For a symmetric matrix $\boldsymbol{W}$, $\frac{\deriv}{\deriv \boldsymbol{s}}(\boldsymbol{x}-\boldsymbol{A} \boldsymbol{s})^{\top} \boldsymbol{W}(\boldsymbol{x}-\boldsymbol{A} \boldsymbol{s})=-2(\boldsymbol{x}-\boldsymbol{A} \boldsymbol{s})^{\top} \boldsymbol{W} \boldsymbol{A}$
    \item $\frac{\deriv}{\deriv \boldsymbol{X}} \boldsymbol{f}(\boldsymbol{X})^{\top}=\left(\frac{\deriv \boldsymbol{f}(\boldsymbol{X})}{\deriv \boldsymbol{X}}\right)^{\top}$
    \item $\frac{\deriv}{\deriv \boldsymbol{X}} \boldsymbol{f}(\boldsymbol{X})^{-1}=-\boldsymbol{f}(\boldsymbol{X})^{-1} \frac{\deriv \boldsymbol{f}(\boldsymbol{X})}{\deriv \boldsymbol{X}} \boldsymbol{f}(\boldsymbol{X})^{-1}$
    \item $\frac{\deriv \boldsymbol{a}^{\top} \boldsymbol{X}^{-1} \boldsymbol{b}}{\deriv \boldsymbol{X}}=-\left(\boldsymbol{X}^{-1}\right)^{\top} \boldsymbol{a} \boldsymbol{b}^{\top}\left(\boldsymbol{X}^{-1}\right)^{\top}$
    \end{itemize}  \,\\
    {\small\textbf{Note:} to compute  gradients of matrices
with respect to vectors (or other matrices) we need \emph{tensors}, see chapter 5.4 of \href{https://mml-book.github.io/book/mml-book.pdf}{\textcolor{blue}{Deisenroth}} for more.}
\end{vbframe}

\begin{vbframe}{Example: Logistic Regression I}
\begin{itemize}    \setlength{\itemsep}{0.5\baselineskip}
    \item Let's say, for data in $\R^{n\times m}$ we're trying to minimize the risk in logistic regression by finding the gradient for negative log loss:
\[
-\ell({\theta})=\sum_{i=1}^n-y^{(i)} \log \left(\pi\left(\mathbf{x}^{(i)} \mid \theta\right)\right)-\left(1-y^{(i)}\right) \log \left(1-\pi\left(\mathbf{x}^{(i)} \mid \theta\right)\right)
\]
where $
\pi(\mathbf{x} \mid {\theta})=s(f(\theta,\xv))
$\\[5pt] with 
$f(\theta,\xv)=\theta^{\top} \xv$ and $s(x)=\frac{1}{1+\exp (-x)}$.
\item[$\Rightarrow$] We want to find \begin{align*}
\nabla_\theta -\ell (\theta)=-\nabla_\theta \ell (\theta)&=-\sumin\underbrace{ y^{(i)} \log (s(f(\theta,\xv^{(i)}))+(1-y^{(i)}) \log (1-s(f(\theta,\xv^{(i)}))}_{=:y \log (s_i)+(1-y_i) \log (1-s_i)}
\end{align*}
\item Define $h_i:=-[y \log (s_i)+(1-y_i) \log (1-s_i)]$  and $f_i:=\theta^{\top} \xv^{(i)}$
\end{itemize}
\end{vbframe}

\begin{vbframe}{Example: Logistic Regression II}
\begin{itemize}    \setlength{\itemsep}{0.5\baselineskip}
\item We can now directly apply the Chain Rule:
$$
-\nabla_\theta \ell (\theta)=\sumin\nabla_\theta h_i\circ s_i \circ f_i=\sumin\frac{\deriv h_i}{\deriv s_i} \frac{\deriv s_i}{\deriv f_i} \frac{\deriv f_i}{\deriv \theta}
$$
    \item Given that, $\forall i\in\{1,\dots,n\}$ $$\frac{\deriv h_i}{\deriv s_i}=\frac{1-y_i}{1-s_i}-\frac{y_i}{s_i};\quad\frac{\deriv s_i}{\deriv f_i}=s(f_i)\big(1-s_i(f_i)\big);\quad\frac{\deriv f_i}{\deriv \theta}=\left(\xv^{(i)}\right)^\top$$
    \item we get that $-\nabla_\theta \ell (\theta)=\sumin\frac{\deriv h_i}{\deriv s_i} \frac{\deriv s_i}{\deriv f_i} \frac{\deriv f_i}{\deriv \theta}$  equals\begin{align*}
        &\sumin \left[\frac{1-y^{(i)}}{1-s(f(\theta, \xv^{(i)}))}-\frac{y^{(i)}}{s(f(\theta, \xv^{(i)}))}\right] \cdot \left[s(f(\theta, \xv^{(i)}))(1-s(f(\theta, \xv^{(i)})))\right] \left(\xv^{(i)}\right)^\top\\
        =&\sumin\left(s\left(f\left(\theta,\xv^{(i)}\right)\right)-y^{(i)}\right)\left(\xv^{(i)}\right)^\top\in\R^{1\times m}
    \end{align*}
\end{itemize}
\end{vbframe}

\begin{vbframe}{Example: Logistic Regression -- Different Notation}
This example highlights how using Leibniz notation can make things easier.\\
Alternatively, we could write this example out as follows:
\begin{itemize}
    \item Given the sum rule, it suffices to find the derivative of 
\[
-[y \log (s(f(\theta,\xv))+(1-y) \log (1-s(f(\theta,\xv))]
\]
\item Define $h(y,z):=-[y \log (z)+(1-y) \log (1-z)]$.
\item Now, what we are looking for is 
\[
\nabla_\theta h(y,\cdot)\circ s \circ f(\cdot,\xv).
\]
where $f$ is a function from $\R^p$ to $\R$ and both $h$ and $s$ are functions from $\R$ to $\R$
$\Longrightarrow$ by the chain rule
$$
\nabla_\theta h(y,\cdot)\circ s \circ f(\cdot,\xv)=\frac{\deriv h}{\deriv s} \frac{\deriv s}{\deriv f} \frac{\deriv f}{\deriv \theta}\in\R^{1\times p}.
$$
\end{itemize}\framebreak
\begin{itemize}
    \item Using the fact that $$\frac{\partial h}{\partial z}=\frac{1-y}{1-z}-\frac{y}{z};\quad\frac{\partial s}{\partial x}=s(x)\big(1-s(x)\big);\quad\frac{\partial f}{\partial \theta}=\xv$$ we get \begin{align*}
   [\nabla_\theta h(y,\cdot)\circ s \circ f(\cdot,\xv)]_j&=\left[\frac{1-y}{1-s(f(\theta,\xv))}-\frac{y}{s(f(\theta,\xv))}\right] \cdot \left[s(f(\theta,\xv))(1-s(f(\theta,\xv)))\right]\cdot \frac{\partial f}{\partial\theta_j}\\
   &=(s(f(\theta,\xv))-y)\xv_j
\end{align*}
$\Longrightarrow \nabla_\theta h(y,\cdot)\circ s \circ f(\cdot,\xv)=(s(f(\theta,\xv))-y)\xv^\top$
\item Plugging this back into the original sum term we get $$
-\nabla_\theta \ell (\theta)=\sumin\left(s\left(f\left(\theta,\xv^{(i)}\right)\right)-y^{(i)}\right)\left(\xv^{(i)}\right)^\top\in\R^{1\times m},
$$ the same as before.
\end{itemize}
\end{vbframe}


\begin{comment}
\begin{vbframe}{Numerator layout}

\begin{itemize}
    \item \textbf{Matrix calculus:} collect derivative of each component of dependent variable w.r.t. each component of independent variable
    \item We use so-called \textbf{numerator layout} convention:
        \begin{align*}
            \frac{\partial y}{\partial\xv} &= \left(\frac{\partial y}{\partial x_1}, \cdots, \frac{\partial y}{\partial x_d}\right) = \nabla y^T \in \R^{1\times d} \\
            \frac{\partial\yv}{\partial x} &= \left(\frac{\partial y_1}{\partial x}, \cdots, \frac{\partial y_m}{\partial x}\right)^T \in \R^{m} \\
            \frac{\partial\yv}{\partial\xv} =
            \begin{pmatrix}
                \frac{\partial y_1}{\partial\xv} \\
                \vdots \\
                \frac{\partial y_m}{\partial\xv} \\
            \end{pmatrix} &=
            \left(\frac{\partial\yv}{\partial x_1} \cdots \frac{\partial\yv}{\partial x_d}\right) =
            \begin{pmatrix}
                \frac{\partial y_1}{\partial x_1} & \cdots & \frac{\partial y_1}{\partial x_d} \\
                \vdots & \ddots & \vdots \\ 
                \frac{\partial y_m}{\partial x_1} & \cdots & \frac{\partial y_m}{\partial x_d}
            \end{pmatrix} = J_\yv \in \R^{m \times d}
        \end{align*}
\end{itemize}

\end{vbframe}

\begin{vbframe}{Scalar-by-vector}

Let $\xv \in \R^d$, $y,z : \R^d \to \R$ and $\Amat$ be a matrix.

\medskip

\begin{itemize}
    \item If $y$ is a \textbf{constant} function: $\frac{\partial y}{\partial\xv} = \mathbf{0}^T \in \R^{1\times d}$
    \item \textbf{Linearity}: $\frac{\partial (a \cdot y + z)}{\partial\xv} = a\frac{\partial y}{\partial\xv} + \frac{\partial z}{\partial\xv}$ \quad ($a$ constant)
    \item \textbf{Product} rule: $\frac{\partial (y \cdot z)}{\partial\xv} = y \frac{\partial z}{\partial\xv} + \frac{\partial y}{\partial \xv}z$
    \item \textbf{Chain} rule: $\frac{\partial g(y)}{\partial\xv} = \frac{\partial g(y)}{\partial y}\frac{\partial y}{\partial \xv}$ \quad ($g$ scalar-valued function)
    \item \textbf{Second} derivative: $\frac{\partial^2 y}{\partial\xv \partial\xv^T} = \nabla^2 y^T$ ($=\nabla^2 y$ if~$y\in\mathcal{C}^2$) (Hessian)
    \item $\frac{\partial(\xv^T\Amat\xv)}{\partial\xv} =  \xv^T(\Amat+\Amat^T)$
    \item $\frac{\partial(\yv^T \Amat \mathbf{z})}{\partial\xv} = \yv^T \Amat \frac{\partial\mathbf{z}}{\partial\xv} + \mathbf{z}^T \Amat^T \frac{\partial\yv}{\partial\xv}$ \quad ($\yv$, $\mathbf{z}$ vector-valued functions of~$\xv$)
\end{itemize}

\end{vbframe}

\begin{vbframe}{Vector-by-scalar}

Let $x \in \R$ and $\yv, \mathbf{z} : \R \to \R^m$.

\medskip

\begin{itemize}
    \item If $\mathbf{y}$ is a \textbf{constant} function: $\frac{\partial \mathbf{y}}{\partial x} = \mathbf{0} \in \R^{m}$
    \item \textbf{Linearity}: $\frac{\partial (a \cdot \yv + \mathbf{z})}{\partial x} = a \frac{\partial\yv}{\partial x} + \frac{\partial \mathbf{z}}{\partial x}$ \quad ($a$ constant)
    \item \textbf{Chain} rule: $\frac{\partial\mathbf{g}(\yv)}{\partial x} = \frac{\partial\mathbf{g}(\yv)}{\partial\yv}\frac{\partial \yv}{\partial x}$ \quad ($\mathbf{g}$ vector-valued function)
    \item $\frac{\partial(\Amat\yv)}{\partial x} = \Amat\frac{\partial\yv}{\partial x}$  \quad ($\mathbf{A}$ matrix)
\end{itemize}
\end{vbframe}

\begin{vbframe}{Vector-by-vector}

Let $\xv \in \R^d$ and $\yv, \mathbf{z} : \R^d \to \R^m$.

\medskip

\begin{itemize}
    \item If $\yv$ is a \textbf{constant} function: $\frac{\partial\yv}{\partial\xv} = \mathbf{0} \in \R^{m \times d}$
    \item $\frac{\partial \mathbf{x}}{\partial \mathbf{x}} = \mathbf{I} \in \R^{d \times d}$
    \item \textbf{Linearity}: $\frac{\partial (a \cdot \yv + \mathbf{z})}{\partial\xv} = a \frac{\partial\yv}{\partial\xv} + \frac{\partial\mathbf{z}}{\partial\xv}$ \quad ($a$ constant)
    \item \textbf{Chain} rule: $\frac{\partial\mathbf{g}(\yv)}{\partial\xv} = \frac{\partial\mathbf{g}(\yv)}{\partial\yv} \frac{\partial\yv}{\partial\xv}$ \quad ($\mathbf{g}$ vector-valued function)
    \item $\frac{\partial(\Amat\xv)}{\partial\xv} = \Amat$, $\frac{\partial(\xv^T\mathbf{B})}{\partial\xv} = \mathbf{B}^T$ \quad ($\Amat,\mathbf{B}$ matrices)
\end{itemize}

\end{vbframe}

\begin{vbframe}{Example}

Consider $f : \R^2 \to \R$ with
\begin{equation*}
    f(\xv) = \exp\left(-(\xv - \mathbf{c})^T \Amat (\xv - \mathbf{c})\right),
\end{equation*}

\vspace{-0.5\baselineskip}

where $\mathbf{c} = (1, 1)^T$ and $\Amat = \begin{pmatrix}1 & 1/2 \\ 1/2 & 1\end{pmatrix}$.

\medskip

Compute $\nabla f(\xv)$ at $\xv^\ast = \mathbf{0}$:

\medskip

\begin{enumerate}
    \item Write $f(\xv) = \exp(g(\mathbf{u}(\xv)))$ with $g(\mathbf{u}) = -\mathbf{u}^T\Amat\mathbf{u}$ and $\mathbf{u}(\xv) = \xv-\mathbf{c}$
    \item \textbf{Chain} rule: $\frac{\partial f(\xv)}{\partial\xv} = \exp(g(\mathbf{u}(\xv))) \frac{\partial g(\mathbf{u})}{\partial\mathbf{u}} \frac{\partial\mathbf{u}(\xv)}{\partial\xv}$
    \item $\mathbf{u}^\ast := \mathbf{u}(\xv^\ast) = (-1,-1)^T$, $g(\mathbf{u^\ast}) = -3$
    \item $\frac{\partial g(\mathbf{u})}{\partial\mathbf{u}} = -2\mathbf{u}^T\Amat$, $\frac{\partial g(\mathbf{u}^\ast)}{\partial\mathbf{u}} = (3, 3)$
    \item \textbf{Linearity}: $\frac{\partial\mathbf{u}(\xv)}{\partial\xv} = \frac{\partial(\xv - \mathbf{c})}{\partial\xv} = \frac{\partial\xv}{\partial\xv} - \frac{\partial\mathbf{c}}{\partial\xv} = \mathbf{I}_2$
    \item $\nabla f(\xv^\ast) = \frac{\partial f(\xv^\ast)}{\partial{\xv}}^T = (\exp(-3) \cdot (3, 3) \cdot \mathbf{I}_2)^T = \exp(-3) \begin{pmatrix}3 \\ 3\end{pmatrix}$
\end{enumerate}

\end{vbframe}
\end{comment}





\endlecture

\end{document}
