\documentclass[11pt,compress,t,notes=noshow, xcolor=table]{beamer}

\usepackage[]{graphicx}
\usepackage[]{color}
% maxwidth is the original width if it is less than linewidth
% otherwise use linewidth (to make sure the graphics do not exceed the margin)
\makeatletter
\def\maxwidth{ %
  \ifdim\Gin@nat@width>\linewidth
    \linewidth
  \else
    \Gin@nat@width
  \fi
}
\makeatother

% ---------------------------------%
% latex-math dependencies, do not remove:
% - mathtools
% - bm
% - siunitx
% - dsfont
% - xspace
% ---------------------------------%

%--------------------------------------------------------%
%       Language, encoding, typography
%--------------------------------------------------------%

\usepackage[english]{babel}
\usepackage[utf8]{inputenc} % Enables inputting UTF-8 symbols
% Standard AMS suite
\usepackage{amsmath,amsfonts,amssymb}

% Font for double-stroke / blackboard letters for sets of numbers (N, R, ...)
% Distribution name is "doublestroke"
% According to https://mirror.physik.tu-berlin.de/pub/CTAN/fonts/doublestroke/dsdoc.pdf
% the "bbm" package does a similar thing and may be superfluous.
% Required for latex-math
\usepackage{dsfont}

% bbm – "Blackboard-style" cm fonts (https://www.ctan.org/pkg/bbm)
% Used to be in common.tex, loaded directly after this file
% Maybe superfluous given dsfont is loaded
% TODO: Check if really unused?
% \usepackage{bbm}

% bm – Access bold symbols in maths mode - https://ctan.org/pkg/bm
% Required for latex-math
% https://tex.stackexchange.com/questions/3238/bm-package-versus-boldsymbol
\usepackage{bm}

% pifont – Access to PostScript standard Symbol and Dingbats fonts
% Used for \newcommand{\xmark}{\ding{55}, which is never used
% aside from lecture_advml/attic/xx-automl/slides.Rnw
% \usepackage{pifont}

% Quotes (inline and display), provdes \enquote
% https://ctan.org/pkg/csquotes
\usepackage{csquotes}

% Adds arg to enumerate env, technically superseded by enumitem according
% to https://ctan.org/pkg/enumerate
% Replace with https://ctan.org/pkg/enumitem ?
\usepackage{enumerate}

% Line spacing - provides \singlespacing \doublespacing \onehalfspacing
% https://ctan.org/pkg/setspace
% \usepackage{setspace}

% mathtools – Mathematical tools to use with amsmath
% https://ctan.org/pkg/mathtools?lang=en
% latex-math dependency according to latex-math repo
\usepackage{mathtools}

% Maybe not great to use this https://tex.stackexchange.com/a/197/19093
% Use align instead -- TODO: Global search & replace to check, eqnarray is used a lot
% $ rg -f -u "\begin{eqnarray" -l | grep -v attic | awk -F '/' '{print $1}' | sort | uniq -c
%   13 lecture_advml
%   14 lecture_i2ml
%    2 lecture_iml
%   27 lecture_optimization
%   45 lecture_sl
\usepackage{eqnarray}

% For shaded regions / boxes
% Used sometimes in optim
% https://www.ctan.org/pkg/framed
\usepackage{framed}

%--------------------------------------------------------%
%       Cite button (version 2024-05)
%--------------------------------------------------------%
% Note this requires biber to be in $PATH when running,
% telltale error in log would be e.g. Package biblatex Info: ... file 'authoryear.dbx' not found
% aside from obvious "biber: command not found" or similar.

\usepackage{hyperref}
\usepackage{usebib}
\usepackage[backend=biber, style=authoryear]{biblatex}

% Only try adding a references file if it exists, otherwise
% this would compile error when references.bib is not found
\IfFileExists{references.bib} {
  \addbibresource{references.bib}
  \bibinput{references}
}

\newcommand{\citelink}[1]{%
\NoCaseChange{\resizebox{!}{9pt}{\protect\beamergotobutton{\href{\usebibentry{\NoCaseChange{#1}}{url}}{\begin{NoHyper}\cite{#1}\end{NoHyper}}}}}%
}

%--------------------------------------------------------%
%       Displaying code and algorithms
%--------------------------------------------------------%

% Reimplements verbatim environments: https://ctan.org/pkg/verbatim
% verbatim used sed at least once in
% supervised-classification/slides-classification-tasks.tex
\usepackage{verbatim}

% Both used together for algorithm typesetting, see also overleaf: https://www.overleaf.com/learn/latex/Algorithms
% algorithmic env is also used, but part of the bundle:
%   "algpseudocode is part of the algorithmicx bundle, it gives you an improved version of algorithmic besides providing some other features"
% According to https://tex.stackexchange.com/questions/229355/algorithm-algorithmic-algorithmicx-algorithm2e-algpseudocode-confused
\usepackage{algorithm}
\usepackage{algpseudocode}

%--------------------------------------------------------%
%       Tables
%--------------------------------------------------------%

% multi-row table cells: https://www.namsu.de/Extra/pakete/Multirow.html
% Provides \multirow
% Used e.g. in evaluation/slides-evaluation-measures-classification.tex
\usepackage{multirow}

% colortbl: https://ctan.org/pkg/colortbl
% "The package allows rows and columns to be coloured, and even individual cells." well.
% Provides \columncolor and \rowcolor
% \rowcolor is used multiple times, e.g. in knn/slides-knn.tex
\usepackage{colortbl}

% long/multi-page tables: https://texdoc.org/serve/longtable.pdf/0
% Not used in slides
% \usepackage{longtable}

% pretty table env: https://ctan.org/pkg/booktabs
% Is used
% Defines \toprule
\usepackage{booktabs}

%--------------------------------------------------------%
%       Figures: Creating, placing, verbing
%--------------------------------------------------------%

% wrapfig - Wrapping text around figures https://de.overleaf.com/learn/latex/Wrapping_text_around_figures
% Provides wrapfigure environment -used in lecture_optimization
\usepackage{wrapfig}

% Sub figures in figures and tables
% https://ctan.org/pkg/subfig -- supersedes subfigure package
% Provides \subfigure
% \subfigure not used in slides but slides-tuning-practical.pdf errors without this pkg, error due to \captionsetup undefined
\usepackage{subfig}

% Actually it's pronounced PGF https://en.wikibooks.org/wiki/LaTeX/PGF/TikZ
\usepackage{tikz}

% No idea what/why these settings are what they are but I assume they're there on purpose
\usetikzlibrary{shapes,arrows,automata,positioning,calc,chains,trees, shadows}
\tikzset{
  %Define standard arrow tip
  >=stealth',
  %Define style for boxes
  punkt/.style={
    rectangle,
    rounded corners,
    draw=black, very thick,
    text width=6.5em,
    minimum height=2em,
    text centered},
  % Define arrow style
  pil/.style={
    ->,
    thick,
    shorten <=2pt,
    shorten >=2pt,}
}

% Defines macros and environments
\usepackage{../../style/lmu-lecture}

\let\code=\texttt     % Used regularly

% Not sure what/why this does
\setkeys{Gin}{width=0.9\textwidth}

% -- knitr leftovers --
% Used often in conjunction with \definecolor{shadecolor}{rgb}{0.969, 0.969, 0.969}
% Removing definitions requires chaning _many many_ slides, which then need checking to see if output still ok
\definecolor{fgcolor}{rgb}{0.345, 0.345, 0.345}
\definecolor{shadecolor}{rgb}{0.969, 0.969, 0.969}
\newenvironment{knitrout}{}{} % an empty environment to be redefined in TeX

%-------------------------------------------------------------------------------------------------------%
%  Unused stuff that needs to go but is kept here currently juuuust in case it was important after all  %
%-------------------------------------------------------------------------------------------------------%

% \newcommand{\hlnum}[1]{\textcolor[rgb]{0.686,0.059,0.569}{#1}}%
% \newcommand{\hlstr}[1]{\textcolor[rgb]{0.192,0.494,0.8}{#1}}%
% \newcommand{\hlcom}[1]{\textcolor[rgb]{0.678,0.584,0.686}{\textit{#1}}}%
% \newcommand{\hlopt}[1]{\textcolor[rgb]{0,0,0}{#1}}%
% \newcommand{\hlstd}[1]{\textcolor[rgb]{0.345,0.345,0.345}{#1}}%
% \newcommand{\hlkwa}[1]{\textcolor[rgb]{0.161,0.373,0.58}{\textbf{#1}}}%
% \newcommand{\hlkwb}[1]{\textcolor[rgb]{0.69,0.353,0.396}{#1}}%
% \newcommand{\hlkwc}[1]{\textcolor[rgb]{0.333,0.667,0.333}{#1}}%
% \newcommand{\hlkwd}[1]{\textcolor[rgb]{0.737,0.353,0.396}{\textbf{#1}}}%
% \let\hlipl\hlkwb

% \makeatletter
% \newenvironment{kframe}{%
%  \def\at@end@of@kframe{}%
%  \ifinner\ifhmode%
%   \def\at@end@of@kframe{\end{minipage}}%
%   \begin{minipage}{\columnwidth}%
%  \fi\fi%
%  \def\FrameCommand##1{\hskip\@totalleftmargin \hskip-\fboxsep
%  \colorbox{shadecolor}{##1}\hskip-\fboxsep
%      % There is no \\@totalrightmargin, so:
%      \hskip-\linewidth \hskip-\@totalleftmargin \hskip\columnwidth}%
%  \MakeFramed {\advance\hsize-\width
%    \@totalleftmargin\z@ \linewidth\hsize
%    \@setminipage}}%
%  {\par\unskip\endMakeFramed%
%  \at@end@of@kframe}
% \makeatother

% \definecolor{shadecolor}{rgb}{.97, .97, .97}
% \definecolor{messagecolor}{rgb}{0, 0, 0}
% \definecolor{warningcolor}{rgb}{1, 0, 1}
% \definecolor{errorcolor}{rgb}{1, 0, 0}
% \newenvironment{knitrout}{}{} % an empty environment to be redefined in TeX

% \usepackage{alltt}
% \newcommand{\SweaveOpts}[1]{}  % do not interfere with LaTeX
% \newcommand{\SweaveInput}[1]{} % because they are not real TeX commands
% \newcommand{\Sexpr}[1]{}       % will only be parsed by R
% \newcommand{\xmark}{\ding{55}}%

% textpos – Place boxes at arbitrary positions on the LATEX page
% https://ctan.org/pkg/textpos
% Provides \begin{textblock}
% TODO: Check if really unused?
% \usepackage[absolute,overlay]{textpos}

% -----------------------%
% Likely knitr leftovers %
% -----------------------%

% psfrag – Replace strings in encapsulated PostScript figures
% https://www.overleaf.com/latex/examples/psfrag-example/tggxhgzwrzhn
% https://ftp.mpi-inf.mpg.de/pub/tex/mirror/ftp.dante.de/pub/tex/macros/latex/contrib/psfrag/pfgguide.pdf
% Can't tell if this is needed
% TODO: Check if really unused?
% \usepackage{psfrag}

% arydshln – Draw dash-lines in array/tabular
% https://www.ctan.org/pkg/arydshln
% !! "arydshln has to be loaded after array, longtable, colortab and/or colortbl"
% Provides \hdashline and \cdashline
% Not used in slides
% \usepackage{arydshln}

% tabularx – Tabulars with adjustable-width columns
% https://ctan.org/pkg/tabularx
% Provides \begin{tabularx}
% Not used in slides
% \usepackage{tabularx}

% placeins – Control float placement
% https://ctan.org/pkg/placeins
% Defines a \FloatBarrier command
% TODO: Check if really unused?
% \usepackage{placeins}

% Can't find a reason why common.tex is not just part of this file?
% This file is included in slides and exercises

% Rarely used fontstyle for R packages, used only in 
% - forests/slides-forests-benchmark.tex
% - exercises/single-exercises/methods_l_1.Rnw
% - slides/cart/attic/slides_extra_trees.Rnw
\newcommand{\pkg}[1]{{\fontseries{b}\selectfont #1}}

% Spacing helpers, used often (mostly in exercises for \dlz)
\newcommand{\lz}{\vspace{0.5cm}} % vertical space (used often in slides)
\newcommand{\dlz}{\vspace{1cm}}  % double vertical space (used often in exercises, never in slides)

% Don't know if this is used or needed, remove?
% textcolor that works in mathmode
% https://tex.stackexchange.com/a/261480
% Used e.g. in forests/slides-forests-bagging.tex
% [...] \textcolor{blue}{\tfrac{1}{M}\sum^M_{m} [...]
% \makeatletter
% \renewcommand*{\@textcolor}[3]{%
%   \protect\leavevmode
%   \begingroup
%     \color#1{#2}#3%
%   \endgroup
% }
% \makeatother


% dependencies: amsmath, amssymb, dsfont
% math spaces
\ifdefined\N
\renewcommand{\N}{\mathds{N}} % N, naturals
\else \newcommand{\N}{\mathds{N}} \fi
\newcommand{\Z}{\mathds{Z}} % Z, integers
\newcommand{\Q}{\mathds{Q}} % Q, rationals
\newcommand{\R}{\mathds{R}} % R, reals
\ifdefined\C
\renewcommand{\C}{\mathds{C}} % C, complex
\else \newcommand{\C}{\mathds{C}} \fi
\newcommand{\continuous}{\mathcal{C}} % C, space of continuous functions
\newcommand{\M}{\mathcal{M}} % machine numbers
\newcommand{\epsm}{\epsilon_m} % maximum error

% counting / finite sets
\newcommand{\setzo}{\{0, 1\}} % set 0, 1
\newcommand{\setmp}{\{-1, +1\}} % set -1, 1
\newcommand{\unitint}{[0, 1]} % unit interval

% basic math stuff
\newcommand{\xt}{\tilde x} % x tilde
\DeclareMathOperator*{\argmax}{arg\,max} % argmax
\DeclareMathOperator*{\argmin}{arg\,min} % argmin
\newcommand{\argminlim}{\mathop{\mathrm{arg\,min}}\limits} % argmax with limits
\newcommand{\argmaxlim}{\mathop{\mathrm{arg\,max}}\limits} % argmin with limits
\newcommand{\sign}{\operatorname{sign}} % sign, signum
\newcommand{\I}{\mathbb{I}} % I, indicator
\newcommand{\order}{\mathcal{O}} % O, order
\newcommand{\bigO}{\mathcal{O}} % Big-O Landau
\newcommand{\littleo}{{o}} % Little-o Landau
\newcommand{\pd}[2]{\frac{\partial{#1}}{\partial #2}} % partial derivative
\newcommand{\floorlr}[1]{\left\lfloor #1 \right\rfloor} % floor
\newcommand{\ceillr}[1]{\left\lceil #1 \right\rceil} % ceiling
\newcommand{\indep}{\perp \!\!\! \perp} % independence symbol

% sums and products
\newcommand{\sumin}{\sum\limits_{i=1}^n} % summation from i=1 to n
\newcommand{\sumim}{\sum\limits_{i=1}^m} % summation from i=1 to m
\newcommand{\sumjn}{\sum\limits_{j=1}^n} % summation from j=1 to p
\newcommand{\sumjp}{\sum\limits_{j=1}^p} % summation from j=1 to p
\newcommand{\sumik}{\sum\limits_{i=1}^k} % summation from i=1 to k
\newcommand{\sumkg}{\sum\limits_{k=1}^g} % summation from k=1 to g
\newcommand{\sumjg}{\sum\limits_{j=1}^g} % summation from j=1 to g
\newcommand{\summM}{\sum\limits_{m=1}^M} % summation from m=1 to M
\newcommand{\meanin}{\frac{1}{n} \sum\limits_{i=1}^n} % mean from i=1 to n
\newcommand{\meanim}{\frac{1}{m} \sum\limits_{i=1}^m} % mean from i=1 to n
\newcommand{\meankg}{\frac{1}{g} \sum\limits_{k=1}^g} % mean from k=1 to g
\newcommand{\meanmM}{\frac{1}{M} \sum\limits_{m=1}^M} % mean from m=1 to M
\newcommand{\prodin}{\prod\limits_{i=1}^n} % product from i=1 to n
\newcommand{\prodkg}{\prod\limits_{k=1}^g} % product from k=1 to g
\newcommand{\prodjp}{\prod\limits_{j=1}^p} % product from j=1 to p

% linear algebra
\newcommand{\one}{\bm{1}} % 1, unitvector
\newcommand{\zero}{\mathbf{0}} % 0-vector
\newcommand{\id}{\bm{I}} % I, identity
\newcommand{\diag}{\operatorname{diag}} % diag, diagonal
\newcommand{\trace}{\operatorname{tr}} % tr, trace
\newcommand{\spn}{\operatorname{span}} % span
\newcommand{\scp}[2]{\left\langle #1, #2 \right\rangle} % <.,.>, scalarproduct
\newcommand{\mat}[1]{\begin{pmatrix} #1 \end{pmatrix}} % short pmatrix command
\newcommand{\Amat}{\mathbf{A}} % matrix A
\newcommand{\Deltab}{\mathbf{\Delta}} % error term for vectors

% basic probability + stats
\renewcommand{\P}{\mathds{P}} % P, probability
\newcommand{\E}{\mathds{E}} % E, expectation
\newcommand{\var}{\mathsf{Var}} % Var, variance
\newcommand{\cov}{\mathsf{Cov}} % Cov, covariance
\newcommand{\corr}{\mathsf{Corr}} % Corr, correlation
\newcommand{\normal}{\mathcal{N}} % N of the normal distribution
\newcommand{\iid}{\overset{i.i.d}{\sim}} % dist with i.i.d superscript
\newcommand{\distas}[1]{\overset{#1}{\sim}} % ... is distributed as ...


% sets
\newcommand{\CC}[1]{\mathcal{C}^{#1}}         % class C^k
\newcommand{\CCinf}{\mathcal{C}^{\infty}}     % class C^infty


% objectives, domains, points
\renewcommand{\S}{\mathcal{S}}              % domain of f
\newcommand{\xv}{\bm{x}}                  % vector x (bold)
\newcommand{\yv}{\bm{y}}                  % vector y (bold)
\newcommand{\vv}{\bm{v}}                  % vector v (bold)
\newcommand{\wv}{\bm{w}}                  % vector w (bold)
\newcommand{\xvs}{\bm{x}^\ast}            % x-star,  theoretical optimum
\newcommand{\xs}{x^\ast}                  % x-star,  theoretical optimum
\newcommand{\fxvs}{f(\xvs)}               % f at x-star, theoretical optimum
\renewcommand{\H}{\mathbf{H}}                     % Hessian matrix



% some other stuff
\providecommand{\eps}{\epsilon}               % epsilon
\newcommand{\A}{\mathbf{A}}                     % Hessian matrix
\newcommand{\bv}{\mathbf{b}}                     % Hessian matrix
\newcommand{\zv}{\mathbf{z}}                     % Hessian matrix
\newcommand{\V}{\mathbf{V}}                     % Hessian matrix


\newcommand{\vvmax}{\bm{v}_\text{max}}                  % vector v (bold)
\newcommand{\vvmin}{\bm{v}_\text{min}}                  % vector v (bold)
\newcommand{\lammax}{\lambda_\text{max}}                  % vector v (bold)
\newcommand{\lammin}{\lambda_\text{min}}                  % vector v (bold)

\newcommand{\sumid}{\sum\limits_{i=1}^d} % summation from i=1 to n


\title{Optimization in Machine Learning}

\begin{document}

\titlemeta{
Mathematical Concepts 
}{
Quadratic functions II
}{
figure_man/quadratic_functions_2D_example_2_4.png
}{
\item Geometry of quadratic forms
\item Spectrum of Hessian
}

% ------------------------------------------------------------------------------

\begin{framei}{Properties of quadratic functions}
\item $q(\xv) = \xv^T \A \xv + \bv^T \xv + c$
\item Under symmetry: $\H = 2 \A$
\item Convexity/concavity of $q$ depend on eigenvalues of $\H$
\end{framei}
  
\begin{vbframe}{Geometry of quadratic functions}
  
\textbf{Example:} $\Amat = \begin{pmatrix} 2 & -1 \\ -1 & 2\end{pmatrix}$ $\implies$ $\H = 2\Amat = \begin{pmatrix} 4 & -2 \\ -2 & 4\end{pmatrix}$

\begin{itemize}
    \item Since~$\H$ symmetric, eigendecomposition $\H = \V\Lambda\V^T$ with
        \begin{equation*}
            \V = \begin{pmatrix}
                    | & | \\
                    \textcolor{magenta}{v_\text{max}} & \textcolor{orange}{v_\text{min}} \\
                    | & |
                \end{pmatrix}
                = \frac{1}{\sqrt{2}} \begin{pmatrix}
                    1 & 1 \\
                    -1 & 1
                \end{pmatrix}
            \text{ orthogonal}
        \end{equation*}
        \begin{equation*}
            \text{and }
            \Lambda = \begin{pmatrix}
                \textcolor{magenta}{\lambda_\text{max}} & 0 \\
                0 & \textcolor{orange}{\lambda_\text{min}}
            \end{pmatrix}
            = \begin{pmatrix}6 & 0 \\ 0 & 2\end{pmatrix}.
        \end{equation*}
\end{itemize}

\vspace{-0.5\baselineskip}

\begin{figure}
    \includegraphics[height=0.28\textwidth,keepaspectratio]{figure_man/quadr-eigenv.png}
\end{figure}
  
\framebreak
    
\begin{itemize}
    \item $\textcolor{magenta}{\vv_\text{max}}$ ($\textcolor{orange}{\vv_\text{min}}$) direction of highest (lowest) curvature

        \vspace{0.25\baselineskip}
    
        \begin{footnotesize}
            \textbf{Proof:} With $\vv=\V^T\xv$:

            \vspace{-\baselineskip}
            
            \begin{equation*}
                \xv^T \H \xv = \xv^T\V\Lambda\V^T\xv = \vv^T\Lambda\vv = \sum_{i=1}^d \lambda_iv_i^2 \leq \textcolor{magenta}{\lambda_\text{max}} \sum_{i=1}^d v_i^2 = \textcolor{magenta}{\lambda_\text{max}}\|\vv\|^2
            \end{equation*}
            Since $\|\vv\| = \|\xv\|$ ($\V$ orthogonal): $\max_{\|\xv\|=1} \xv^T \H \xv \leq \textcolor{magenta}{\lambda_\text{max}}$
            
            Additional: $\textcolor{magenta}{\vv_\text{max}}^T \H \textcolor{magenta}{\vv_\text{max}} = \mathbf{e}_1^T\Lambda\mathbf{e}_1 = \textcolor{magenta}{\lambda_\text{max}}$

            Analogous: $\min_{\|\xv\|=1} \xv^T \H \xv \geq \textcolor{orange}{\lambda_\text{min}}$ and $\textcolor{orange}{\vv_\text{min}}^T \H \textcolor{orange}{\vv_\text{min}} = \textcolor{orange}{\lambda_\text{min}}$
        \end{footnotesize}

    \medskip

    \item Contour lines of any quadratic form are ellipses \\
    (with eigenvectors of A as principal axes, principal axis theorem)
        \vspace{0.25\baselineskip}
    
        \begin{footnotesize}
        Look at $q(\xv) = \xv^T \A \xv + \bv^T \xv + c$ \\
        Now use $\yv = \xv - \wv = \xv + \frac{1}{2} \A^{-1} \bv$\\
        This already gives us the general form of an ellipse:\\
        $\yv^T \A \yv = (\xv-\wv)^T \A (\xv-\wv) = q(\xv) + const$\\
        If we use $\zv = \vv^T y$ we obtain it in standard form\\
        $\sumin \lambda_i z_i^2 = \zv^T \bm{\Lambda} \zv = y^T \vv \bm{\Lambda} \vv^T y = \yv^T \A \yv = q(\xv) + const $
        \end{footnotesize}

        
        %     \textbf{Proof:} With $\vv=\V^T\xv$:

        %     \vspace{-0.5\baselineskip}

        %     \begin{equation*}
        %         q(\xv) = \xv^T \H \xv + \bv^T\xv + c = \vv^T\Lambda\vv + \bv^T V \vv + c =: \tilde{q}(\vv)
        %     \end{equation*}

        %     Now:
        %     \begin{equation*}
        %         q(\vv_j) = \mathbf{e}_j^T\Lambda\mathbf{e}_j + \bv^T V \mathbf{e}_j + c = \tilde{q}(\mathbf{e}_j)
        %     \end{equation*}

        %     Especially: $q(\textcolor{magenta}{\vv_\text{max}}) = \textcolor{magenta}{\lambda_\text{max}} = \tilde{q}(\mathbf{e}_1) \quad\text{and}\quad q(\textcolor{orange}{\vv_\text{min}}) = \textcolor{orange}{\lambda_\text{min}} = \tilde{q}(\mathbf{e}_d)$
        % \end{footnotesize}
\end{itemize}

Recall: \textbf{Second order condition for optimality} is \textbf{sufficient}.

\medskip

We skipped the \textbf{proof} at first, but can now catch up on it.


    \footnotesize
     If $H(\xv^\ast) \succ 0$ at stationary point~$\xv^\ast$, then $\xv^\ast$ is local minimum ($\prec$ for maximum).

     \medskip

    \textbf{Proof:}
    Let $\textcolor{orange}{\lambda_\text{min}}>0$ denote the smallest eigenvalue of $H(\xv^\ast)$.
    Then:
    
    %\vspace{-1.25\baselineskip}
    
    \begin{equation*}
        f(\xv) = f(\xv^\ast) + \underbrace{\nabla f(\xv^\ast)}_{=0}{}(\xv-\xv^\ast) + \frac{1}{2}\underbrace{(\xv-\xv^\ast)^T H(\xv^\ast)(\xv-\xv^\ast)}_{\geq \textcolor{orange}{\lambda_\text{min}} \|\xv-\xv^\ast\|^2 \text{ (see above)}} + \underbrace{R_2(\xv,\xv^\ast)}_{=o(\|\xv-\xv^\ast\|^2)}.
    \end{equation*}

    Choose~$\eps>0$ s.t. $|R_2(\xv,\xv^\ast)| < \frac{1}{2} \textcolor{orange}{\lambda_\text{min}} \|\xv-\xv^\ast\|^2$ for each~$\xv \neq \xv^\ast$ with $\|\xv-\xv^\ast\|<\eps$.
    Then:

    \vspace{-1.25\baselineskip}

    \begin{equation*}
        f(\xv) \geq f(\xv^\ast) + \underbrace{\frac{1}{2} \textcolor{orange}{\lambda_\text{min}} \|\xv-\xv^\ast\|^2 + R_2(\xv,\xv^\ast)}_{>0} > f(\xv^\ast) \quad\text{for each $\xv \neq \xv^\ast$ with $\|\xv-\xv^\ast\|<\eps$}.
    \end{equation*}


\framebreak

If spectrum of~$\Amat$ is known, also that of $\H = 2\Amat$ is known.

\begin{itemize}
    \item If \textbf{all} eigenvalues of $\H$ $\overset{(>)}{\geq} 0$ ($\Leftrightarrow$ $\H \overset{(\succ)}{\succcurlyeq} 0$):
        \begin{itemize} 
            \item $q$ (strictly) convex,
            \item there is a (unique) global minimum. 
        \end{itemize}
    \item If \textbf{all} eigenvalues of $\H$ $\overset{(<)}{\leq} 0$ ($\Leftrightarrow$ $\H \overset{(\prec)}{\preccurlyeq} 0$):
        \begin{itemize} 
            \item $q$ (strictly) concave,
            \item there is a (unique) global maximum. 
        \end{itemize}
    \item If~$\H$ has both positive and negative eigenvalues ($\Leftrightarrow$ $\H$ indefinite):
        \begin{itemize}
            \item $q$ neither convex nor concave,
            \item there is a saddle point.
        \end{itemize}
\end{itemize}

% \begin{figure}
%     \includegraphics[width=0.3\textwidth, keepaspectratio]{figure_man/minmaxsaddle_2.png}
% \end{figure}

\begin{figure}
    \centering
    \includegraphics[width=0.67\textwidth]{figure_man/minmaxsaddle.png}
\end{figure}

\end{vbframe}

\begin{framei}{Geometry of quadratic functions}
\item Example: $\Amat = \begin{pmatrix} 2 & -1 \\ -1 & 2\end{pmatrix}$ $\Rightarrow$ $\H = 2\Amat = \begin{pmatrix} 4 & -2 \\ -2 & 4\end{pmatrix}$
\item Since $\H$ symmetric: eigendecomposition $\H = \V\Lambda\V^T$
\begin{equation*}
\V = \begin{pmatrix}
| & | \\
\textcolor{magenta}{v_\text{max}} & \textcolor{orange}{v_\text{min}} \\
| & |
\end{pmatrix}
= \frac{1}{\sqrt{2}} \begin{pmatrix}
1 & 1 \\
-1 & 1
\end{pmatrix}\;\text{orthogonal}
\end{equation*}
\begin{equation*}
\Lambda = \begin{pmatrix}
\textcolor{magenta}{\lambda_\text{max}} & 0 \\
0 & \textcolor{orange}{\lambda_\text{min}}
\end{pmatrix}
= \begin{pmatrix}6 & 0 \\ 0 & 2\end{pmatrix}
\end{equation*}
\imageC[0.35]{figure_man/quadr-eigenv.png}
\end{framei}

\begin{framei}[fs=footnotesize]{Spectrum and curvature}
\item $\vvmax$ direction of highest curvature, with curvature value $\lammax$
$$
\vv^T \H \vv = \vv^T \V \Lambda \V^T \vv = \wv^T \Lambda \wv = 
\sumid \lambda_i w_i^2 \le \lammax \sumid w_i^2 = \lammax || \wv ||^2
$$
\item Since $|| \vv || = || \xv ||$ ($\V$ orthogonal): 
$\max_{|| \vv || = 1} \vv^T \H \vv \le \lammax$
\item For $\vvmax$ we obtain this upper bound:
$\vvmax^T \H \vvmax = \mathbf{e}_1^T \Lambda \mathbf{e}_1 = \lammax$ 
\item Analogously, $\vvmin$ direction of lowest curvature, with curvature value $\lammin$
\item Contour lines of any quadratic form are ellipses (principal axis theorem)
\item[] Look at $q(\xv) = \xv^T \A \xv + \bv^T \xv + c$, set $\yv = \xv - \wv = \xv + \tfrac{1}{2} \A^{-1} \bv$
\item[] Then $(\xv-\wv)^T \A (\xv-\wv) = \yv^T \A \yv = q(\xv) + \text{const}$
\item[] With $\zv = \vv^T \yv$ we obtain standard form: $\sumin \lambda_i z_i^2 = \zv^T \bm{\Lambda} \zv = \yv^T \A \yv = q(\xv) + \text{const}$
\end{framei}

\begin{framei}[fs=footnotesize]{Second order condition}
\item Recall: Second order condition for optimality is sufficient
\item If $H(\xv^\ast) \succ 0$ at stationary point $\xv^\ast$, then $\xv^\ast$ local minimum ($\prec$ for maximum)
\item[] \begin{equation*}
f(\xv) = f(\xv^\ast) + \underbrace{\nabla f(\xv^\ast)}_{=0}(\xv-\xv^\ast) + \tfrac{1}{2}\underbrace{(\xv-\xv^\ast)^T H(\xv^\ast)(\xv-\xv^\ast)}_{\ge \textcolor{orange}{\lambda_\text{min}} \|\xv-\xv^\ast\|^2} + \underbrace{R_2(\xv,\xv^\ast)}_{=o(\|\xv-\xv^\ast\|^2)}
\end{equation*}
\item Choose $\eps>0$ s.t. $|R_2(\xv,\xv^\ast)| < \tfrac{1}{2} \textcolor{orange}{\lambda_\text{min}} \|\xv-\xv^\ast\|^2$ for $\xv \neq \xv^\ast$, $\|\xv-\xv^\ast\|<\eps$
\item[] \begin{equation*}
f(\xv) \ge f(\xv^\ast) + \underbrace{\tfrac{1}{2} \textcolor{orange}{\lambda_\text{min}} \|\xv-\xv^\ast\|^2 + R_2(\xv,\xv^\ast)}_{>0} > f(\xv^\ast)
\end{equation*}
\end{framei}

\begin{framei}{Eigenvalues and shape}
\item If spectrum of $\Amat$ is known, also that of $\H = 2\Amat$ is known
\item If all eigenvalues of $\H$ $\overset{(>)}{\ge 0}$ ($\Leftrightarrow$ $\H \overset{(\succ)}{\succcurlyeq} 0$):
\begin{itemize}
\item $q$ (strictly) convex
\item (Unique) global minimum
\end{itemize}
\item If all eigenvalues of $\H$ $\overset{(<)}{\le 0}$ ($\Leftrightarrow$ $\H \overset{(\prec)}{\preccurlyeq} 0$):
\begin{itemize}
\item $q$ (strictly) concave
\item (Unique) global maximum
\end{itemize}
\item If $\H$ has both positive and negative eigenvalues ($\Leftrightarrow$ $\H$ indefinite):
\begin{itemize}
\item $q$ neither convex nor concave
\item there is a saddle point
\end{itemize}
\imageC[0.67]{figure_man/minmaxsaddle.png}
\end{framei}

\begin{framei}{Condition and curvature}
\item $\kappa(\H) = \kappa(\Amat) = |\textcolor{magenta}{\lambda_\text{max}}| / |\textcolor{orange}{\lambda_\text{min}}|$
\item High condition means
\begin{itemize}
\item $|\textcolor{magenta}{\lambda_\text{max}}| \gg |\textcolor{orange}{\lambda_\text{min}}|$
\item Curvature along $\textcolor{magenta}{\vv_\text{max}} \gg$ along $\textcolor{orange}{\vv_\text{min}}$
\item Problem for algorithms like gradient descent
\end{itemize}
\imageC[1]{figure_man/quadr-conds.png}
{\footnotesize Left: Excellent condition. Middle: Good condition. Right: Bad condition.}
\end{framei}

\begin{framei}{Approximation of smooth functions}
\item Any $f \in \mathcal{C}^2$ can be locally approximated by quadratic function (second order Taylor)
\item[] \begin{equation*}
f(\xv) \approx f(\bm{\tilde{x}}) + \nabla f(\bm{\tilde{x}})(\xv-\bm{\tilde{x}}) + \tfrac 12(\xv-\bm{\tilde{x}})^T\nabla^2 f(\bm{\tilde{x}})(\xv-\bm{\tilde{x}})
\end{equation*}
\imageC[0.3]{figure_man/taylor_2D_quadratic.png}
{\footnotesize $f$ and second order approximation: dark vs bright grid. (Source: \url{daniloroccatano.blog})}
\item $\implies$ Hessians provide information about \textbf{local} geometry of a function
\end{framei}
  
\endlecture

\end{document}
  
  