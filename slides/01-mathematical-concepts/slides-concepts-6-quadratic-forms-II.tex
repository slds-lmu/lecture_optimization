\input{../../style/preamble}
\input{../../latex-math/basic-math}
\input{../../latex-math/basic-ml}


\newcommand{\titlefigure}{figure_man/quadratic_functions_2D_example_2_4.png}
\newcommand{\learninggoals}{
\item Geometry of quadratic forms
\item Eigenspectrum}


\title{Optimization}
%\author{Bernd Bischl, Christoph Molnar, Daniel Schalk, Fabian Scheipl}
\date{}

\begin{document}

\lecturechapter{Quadratic forms II}
\lecture{Optimization}
\sloppy

% ------------------------------------------------------------------------------

\begin{vbframe}{Properties of quadratic functions}

  \textbf{Univariate function} \vspace*{0.2cm}
  
  2nd derivative is a $q''(x) = 2 \cdot a$. Basic properties of $q$ can be read-off: 
  
  \begin{itemize}
    \item $q''(x) > 0$: $q$ convex; $q''(x) < 0$: $q$ concave
    \item High (lows) absolute values of $q''(x)$: high (low) curvature
  \end{itemize}
  
  \vspace*{0.2cm}
  
  \textbf{Multivariate function}\vspace*{0.2cm}
  
  2nd derivative is a symmetric matrix of values $\bf{H}$ (called Hessian). 
  
  \lz 
  
  Now: See how Eigenspectrum of $\bf{H}$ encodes the basic properties of $q$. 
  
  \end{vbframe}
  
  \begin{frame}{Properties of quadratic functions (diag)}
  
  \textbf{Example 1}: Function composed of two univariate quadratic terms
  
  \vspace*{-0.3cm}
  
  \only<1>{
  \begin{eqnarray*}
    q(\xv) &=& \xv^\top \Amat \xv = \xv^\top\begin{pmatrix} 2 & 0 \\ 0 & 1\end{pmatrix} \xv = 2 \cdot x_1^2 + x_2^2 \\
    \text{with } \nabla q(\xv) &=& 2 \cdot \Amat \cdot \xv = 4 \cdot x_1 + 2 \cdot x_2, \quad \bm{H} = 2 \cdot \Amat = \begin{pmatrix} \textcolor{orange}{4} & 0 \\ 0 & 2 \end{pmatrix} 
  \end{eqnarray*}
  }
  
  \only<2>{
  \begin{eqnarray*}
    q(\xv) &=& \xv^\top \Amat \xv = \xv^\top\begin{pmatrix} 2 & 0 \\ 0 & 1\end{pmatrix} \xv = 2 \cdot x_1^2 + x_2^2 \\
    \text{with } \nabla q(\xv) &=& 2 \cdot \Amat \cdot \xv = 4 \cdot x_1 + 2 \cdot x_2, \quad \bm{H} = 2 \cdot \Amat = \begin{pmatrix} \textcolor{orange}{4} & 0 \\ 0 & \textcolor{magenta}{2} \end{pmatrix} 
  \end{eqnarray*}
  }
  
  \begin{figure}
    \only<1>{\includegraphics[height=0.4\textwidth, keepaspectratio]{figure_man/quadratic_functions_2D_example_diag_2.png} \\
    \begin{footnotesize} 
    $q$ has a high positive curvature of $\textcolor{orange}{4}$ in the direction of $\textcolor{orange}{v = (1, 0)^\top}$, \phantom{mand a lower (positive) curvature off $\textcolor{magenta}{2}$ in direction of $\textcolor{magenta}{v = (0, 1)^\top}$.}
    \end{footnotesize}
    }
    \only<2>{\includegraphics[height=0.4\textwidth, keepaspectratio]{figure_man/quadratic_functions_2D_example_diag_3.png} \\
    \begin{footnotesize} 
    $q$ has a high positive curvature of $\textcolor{orange}{4}$ in the direction of $\textcolor{orange}{v = (1, 0)^\top}$, and a lower (positive) curvature of $\textcolor{magenta}{2}$ in direction of $\textcolor{magenta}{v = (0, 1)^\top}$.
    \end{footnotesize}
    }
  \end{figure}
  
  \end{frame}
  
  \begin{frame}{Properties of quadratic functions (diag)}
  
  \textbf{Takeaway I}: 
  
  \begin{itemize}
    \item Hessian encodes curvature 
    \item If the Hessian $\mathbf{H}$ is diagonal, the diagonal elements encode the curvature of the function: 
  
    \begin{itemize}
      \item $i$-th diagonal element gives us the curvature in the direction of $\bm{v} = \bm{e}_i$ because 
      $$
      \bm{v}^\top \bm{H} \bm{v} = \bm{e}_i^\top \bm{H} \bm{e}_i = h_{ii}.
      $$
      \item The curvature in an arbitrary direction $\bm{v}
      \in \R^d$, $\|\bm{v}\| = 1$, is 
      $$
      \bm{v}^\top\mathbf{H}\bm{v} = h_{11} v_1^2 + h_{22}v_2^2 + ... + h_{dd}v_d^2. 
      $$
    \end{itemize}
  
    \item<2-> For general (non-diagonal) matrices we analyze the \textbf{eigenspectrum} of $\mathbf{H}$ 
    \vspace*{0.2cm}
    \item<2->[] \begin{footnotesize}
    \textbf{Note: } For diagonal matrices the eigenspectrum is is to read-off: Diagonal elements of $\mathbf{H}$ \textbf{eigenvalues}, unit vectors \textbf{eigenvectors}
    \begin{eqnarray*}
      \mathbf{H} \bm{e}_1 &=& \begin{pmatrix} \textcolor{orange}{4} & 0 \\ 0 & \textcolor{magenta}{2} \end{pmatrix} = \textcolor{orange}{4} \cdot \bm{e}_1; \qquad \mathbf{H} \bm{e}_2 = \begin{pmatrix} \textcolor{orange}{4} & 0 \\ 0 & \textcolor{magenta}{2} \end{pmatrix} = \textcolor{magenta}{2} \cdot \bm{e}_2 \\		
    \end{eqnarray*}	
    \end{footnotesize}
  
  \end{itemize}
  
  
  \end{frame}
  
  
  \begin{frame}{Properties of quadratic functions}
  
  \textbf{Example 2}:
  
  \vspace*{-0.3cm}
  
  \begin{eqnarray*}
    q(\xv) &=& \xv^\top \Amat \xv = \xv^\top\begin{pmatrix} 2 & -1 \\ -1 & 2\end{pmatrix} \xv, \\
    \text{with } \nabla q(\xv) &=& 2 \cdot \Amat \cdot \xv, \qquad
    \nabla^2 q(\xv) = \bm{H} = 2 \cdot \Amat = \begin{pmatrix} 4 & -2 \\ -2 & 4\end{pmatrix} 
  \end{eqnarray*}
  
  % \only<1>{
  % 	\begin{figure}
  % 		\includegraphics[height=0.4\textwidth, keepaspectratio]{figure_man/quadratic_functions_2D_example_1_3.png} \\
  % 		\begin{footnotesize} 
  % 			We can again look at the (directional) curvature along the $x_1$ axis, \phantom{ or $x_2$ axis. However, there is a direction of maximum and minimum curvature. } 
  % 		\end{footnotesize}
  % 	\end{figure}
  % }
  
  % \only<2>{
  % 	\begin{figure}
  % 		\includegraphics[height=0.4\textwidth, keepaspectratio]{figure_man/quadratic_functions_2D_example_1_4.png} \\
  % 		\begin{footnotesize} 
  % 			We can again look at the (directional) curvature along the $x_1$ axis, or $x_2$ axis. \phantom{The directions of maximum and minimum curvature are along the eigenvectors of $\bm{H}$. } 
  % 		\end{footnotesize}
  % 	\end{figure}
  % }
  
  \only<1> {
    \begin{figure}
      \includegraphics[height=0.35\textwidth, keepaspectratio]{figure_man/quadratic_functions_2D_example_1_7.png} \\
      \begin{footnotesize} 
        In the general case, the curvature is determined by the Eigenspectrum of $\bm{H}$. 
      \end{footnotesize}
    \end{figure}
  
  }
  
  \end{frame}
  
  
  \begin{frame}{Properties of quadratic functions}
  
  \textbf{Takeaway II}:
  
  \begin{itemize}
    \item Geometrically, directions of highest / lowest curvature along main axes of ellipses representing the contour lines of $q$. 
    \item Mathematically, the direction with the highest (lowest) curvature is the direction of the eigenvector $\bm{v}_\text{max}$ ($\bm{v}_\text{min}$) belonging to largest (smallest) eigenvalue $\lambda_\text{max}$ ($\lambda_\text{min}$) of $\bm{H}$. 
  \end{itemize}
  
  \begin{columns}
  
  \begin{column}{0.5\textwidth}  
      \begin{center}
        \includegraphics{figure_man/quadratic_functions_2D_example_1_7.png}      
       \end{center}
  \end{column}
  
  \begin{column}{0.5\textwidth}
  
  \begin{footnotesize}
   The eigenvectors and eigenvalues of $\bm{H} = \begin{pmatrix} 4 & -2 \\ -2 & 4\end{pmatrix}$ are: 
  
  \begin{eqnarray*}
    \textcolor{orange}{\bm{v}_\text{min}} &=& \frac{1}{\sqrt{2}}\begin{pmatrix} 1 \\ 1 \end{pmatrix}, \qquad \textcolor{orange}{\lambda_\text{min} = 2} \\
    \textcolor{magenta}{\bm{v}_\text{max}} &=& \frac{1}{\sqrt{2}}\begin{pmatrix} - 1 \\ 1 \end{pmatrix}, \qquad \textcolor{magenta}{\lambda_\text{max} = 3}
  \end{eqnarray*}  
  
  \end{footnotesize}
  \end{column}
  
  \end{columns}
  
  
  
  \end{frame}
  
  
  \begin{vbframe}{Properties of quadratic functions}
  
  Direction $\textcolor{magenta}{\bm{v}_\text{max}}$ is also direction in which the function increases fastest. 
  
  \begin{figure}
    \includegraphics[width=0.6\textwidth]{figure_man/quadratic_functions_2D_example_1_7.png} \\
    \begin{footnotesize}
    \enquote{Walking} the same distance along $\bm{v}_\text{max}$ (magenta) makes us pass more level curves than walking along any other direction.
    \end{footnotesize} 
  \end{figure}
  
  
  \framebreak 
  
  If eigenspectrum of $\Amat$ is known, i.e. the set of its eigenvalues $\left\{\lambda_1, \lambda_2, ..., \lambda_d\right\}$, also eigenspectrum of $\bm{H} = 2\cdot \Amat$ is known and we can read off:
  
  \begin{itemize}
    \item If \textbf{all} eigenvalues of the $\bm{H}$ are $>0$ (we call $\bm{H}$ positive definite): 
    \begin{itemize} 
      \item the function $q$ is convex,
      \item there is a unique global minimum. 
    \end{itemize}
    \item If \textbf{all} eigenvalues of the $\bm{H}$ are $<0$ (we call $\bm{H}$ negative definite): 
    \begin{itemize} 
      \item the function $q$ is concave,
      \item there is a unique global maximum. 
    \end{itemize}
  
  \begin{figure}
    \includegraphics[width = 0.6\textwidth, keepaspectratio]{figure_man/minmaxsaddle_1.png} \\
  \end{figure}
  
  \framebreak 
  
    \item If there are both positive and negative eigenvalues (we call $\bm{H}$ indefinite):
    \begin{itemize}
      \item the function $q$ is neither concave nor convex,
      \item there is a saddle point. 
    \end{itemize}
    \begin{figure}
      \includegraphics[width = 0.3\textwidth, keepaspectratio]{figure_man/minmaxsaddle_2.png} \\
    \end{figure}
  \end{itemize}
  
  \end{vbframe}
  
  
  \begin{frame}{Properties of quadratic functions}
  
  \textbf{Example}: Sketch the following function
  
  \begin{eqnarray*}
    q(\xv) &=& \xv^\top \begin{pmatrix} -1 & -1 \\ -1 & 1\end{pmatrix} \xv % + \begin{pmatrix} 0 & 1 \end{pmatrix} \xv
  \end{eqnarray*}
  
  \begin{columns}
  
  \begin{column}{0.5\textwidth}
  
  \begin{footnotesize}
  
    \only<1>{
      \textbf{Step 1: } Compute the Hessian 
  
      \begin{eqnarray*}
        \bm{H} &=& 2 \cdot \Amat = \begin{pmatrix} -2 & -2 \\ -2 & 2 \end{pmatrix}
      \end{eqnarray*}
    }
  
    \only<2->{
      \textbf{Step 2: } Compute eigenvectors / -values:  
  
      \begin{eqnarray*}
        \textcolor{orange}{\bm{v}_1} &=& \begin{pmatrix} 1 - \sqrt{2} \\ 1 \end{pmatrix}, \qquad \textcolor{orange}{\lambda_1 = 2\sqrt{2}} \\
        \textcolor{magenta}{\bm{v}_2} &=& \begin{pmatrix} 1+ \sqrt{2} \\ 1 \end{pmatrix}, \qquad \textcolor{magenta}{\lambda_2 = - 2\sqrt{2}}.
      \end{eqnarray*}  
  
    }
  
  
  \end{footnotesize}
  \end{column}
  
  \begin{column}{0.5\textwidth}  %%<--- here
      \begin{center}
        \only<2>{\includegraphics{figure_man/quadratic_functions_2D_example_2_3.png}}
        \only<3>{\includegraphics{figure_man/quadratic_functions_2D_example_2_4.png}}
       \end{center}
  \end{column}
  
  \end{columns}
  
  
  \end{frame}
  
  \begin{vbframe}{Properties of quadratic functions}
  
  \textbf{Example}: Sketch the following function
  
  \begin{eqnarray*}
    q(\xv) &=& \xv^\top \begin{pmatrix} -1 & -1 \\ -1 & 1\end{pmatrix} \xv % + \begin{pmatrix} 0 & 1 \end{pmatrix} \xv
  \end{eqnarray*}
  
    \vspace*{-1cm}
  
  \begin{figure}
    \includegraphics[width = 0.48\textwidth]{figure_man/quadratic_functions_2D_example_2_1.png} ~~ \includegraphics[width = 0.48\textwidth, keepaspectratio]{figure_man/quadratic_functions_2D_example_2_4.png}
  \end{figure}
  
  \end{vbframe}
  
  
  
  
  
  \begin{vbframe}{Eigenspectrum and Condition}
  
  Also the condition can be read off from Eigenspectrum: $\kappa(\Amat) = \frac{|\lambda_\text{max}|}{|\lambda_\text{min}|}$. A high condition means: 
  
  \begin{itemize}
    \item The absolute value of the biggest eigenvalue $\lambda_\text{max}$ is much larger than the absolute value of the lowest eigenvalue $\lambda_\text{min}$. 
    \item The curvature in the direction of minimum curvature ($\textbf{v}_\text{max}$) is much lower than the one in the direction of maximum curvature ($\textbf{v}_\text{min}$).
    \item We will see later: optimization algorithms like gradient descent will have difficulties optimizing such functions. 
  \end{itemize}
  
  \vspace*{-0.3cm}
  
  \begin{figure}
    \includegraphics[height=0.25\textwidth, width = 0.25\textwidth]{figure_man/quadratic_functions_2D_bad_cond_1.png} ~~ \includegraphics[height=0.25\textwidth, width = 0.25\textwidth]{figure_man/quadratic_functions_2D_bad_cond_2.png}\\
    \begin{footnotesize} 
    % todo - also add this as a 2d plot
    \end{footnotesize}
  \end{figure}
  
  
  \end{vbframe}
  
  \begin{vbframe}{Interpretation of general functions}
  Every function can be locally approximated by a quadratic function via 2nd order Taylor approximation: 
  
  \vspace*{-0.3cm}
  
  $$
  f(x) \approx f(\bm{\tilde{x}}) + \nabla f(\bm{\tilde{x}})^\top(\xv-\bm{\tilde{x}}) +
  \frac 12(\xv-\bm{\tilde{x}})^\top\nabla^2 f(\bm{\tilde{x}})(\xv-\bm{\tilde{x}})
  $$
  
  \begin{figure}
    \includegraphics[width=0.3\textwidth]{figure_man/taylor_2D_quadratic.png} \\
    \begin{footnotesize} 
    $f$ is shown as the hollow grid and its second-order approximation at $(0, 0)$ as a continuous surface. Source: \url{daniloroccatano.blog}. 
    \end{footnotesize}
  \end{figure}
  
  By analyzing $\nabla^2 f(\bm{\tilde{x}})$ we can gain a local understanding of a function's geometry. 
  
  \end{vbframe}
  
  
  
  \endlecture
\end{document}
  
  