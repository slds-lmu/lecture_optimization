\documentclass[11pt,compress,t,notes=noshow, xcolor=table]{beamer}

\input{../../style/preamble}
\input{../../latex-math/basic-math}

% sets
\newcommand{\CC}[1]{\mathcal{C}^{#1}}         % class C^k
\newcommand{\CCinf}{\mathcal{C}^{\infty}}     % class C^infty


% objectives, domains, points
\renewcommand{\S}{\mathcal{S}}              % domain of f
\newcommand{\xv}{\bm{x}}                  % vector x (bold)
\newcommand{\yv}{\bm{y}}                  % vector y (bold)
\newcommand{\vv}{\bm{v}}                  % vector v (bold)
\newcommand{\wv}{\bm{w}}                  % vector w (bold)
\newcommand{\xvs}{\bm{x}^\ast}            % x-star,  theoretical optimum
\newcommand{\xs}{x^\ast}                  % x-star,  theoretical optimum
\newcommand{\fxvs}{f(\xvs)}               % f at x-star, theoretical optimum
\renewcommand{\H}{\mathbf{H}}                     % Hessian matrix



% some other stuff
\providecommand{\eps}{\epsilon}               % epsilon
\newcommand{\A}{\mathbf{A}}                     % Hessian matrix
\newcommand{\bv}{\mathbf{b}}                     % Hessian matrix
\newcommand{\zv}{\mathbf{z}}                     % Hessian matrix
\newcommand{\V}{\mathbf{V}}                     % Hessian matrix


\newcommand{\vvmax}{\bm{v}_\text{max}}                  % vector v (bold)
\newcommand{\vvmin}{\bm{v}_\text{min}}                  % vector v (bold)
\newcommand{\lammax}{\lambda_\text{max}}                  % vector v (bold)
\newcommand{\lammin}{\lambda_\text{min}}                  % vector v (bold)

\newcommand{\sumid}{\sum\limits_{i=1}^d} % summation from i=1 to n


\title{Optimization in Machine Learning}

\begin{document}

\titlemeta{
Mathematical Concepts 
}{
Quadratic functions II
}{
figure_man/quadratic_functions_2D_example_2_4.png
}{
\item Geometry of quadratic functions
\item Spectrum of Hessian
}

% ------------------------------------------------------------------------------

\begin{framei}{Properties of quadratic functions}
\item $q(\xv) = \xv^T \A \xv + \bv^T \xv + c$
\item Under symmetry: $\H = 2 \A$
\item Convexity/concavity of $q$ depend on eigenvalues of $\H$
\end{framei}
  
% \begin{vbframe}{Geometry of quadratic functions}
  
% \textbf{Example:} $\Amat = \begin{pmatrix} 2 & -1 \\ -1 & 2\end{pmatrix}$ $\implies$ $\H = 2\Amat = \begin{pmatrix} 4 & -2 \\ -2 & 4\end{pmatrix}$

% \begin{itemize}
%     \item Since~$\H$ symmetric, eigendecomposition $\H = \V\Lambda\V^T$ with
%         \begin{equation*}
%             \V = \begin{pmatrix}
%                     | & | \\
%                     \textcolor{magenta}{v_\text{max}} & \textcolor{orange}{v_\text{min}} \\
%                     | & |
%                 \end{pmatrix}
%                 = \frac{1}{\sqrt{2}} \begin{pmatrix}
%                     1 & 1 \\
%                     -1 & 1
%                 \end{pmatrix}
%             \text{ orthogonal}
%         \end{equation*}
%         \begin{equation*}
%             \text{and }
%             \Lambda = \begin{pmatrix}
%                 \textcolor{magenta}{\lambda_\text{max}} & 0 \\
%                 0 & \textcolor{orange}{\lambda_\text{min}}
%             \end{pmatrix}
%             = \begin{pmatrix}6 & 0 \\ 0 & 2\end{pmatrix}.
%         \end{equation*}
% \end{itemize}

% \vspace{-0.5\baselineskip}

% \begin{figure}
%     \includegraphics[height=0.28\textwidth,keepaspectratio]{figure_man/quadr-eigenv.png}
% \end{figure}
  
% \framebreak
    
% \begin{itemize}
%     \item $\textcolor{magenta}{\vv_\text{max}}$ ($\textcolor{orange}{\vv_\text{min}}$) direction of highest (lowest) curvature

%         \vspace{0.25\baselineskip}
    
%         \begin{footnotesize}
%             \textbf{Proof:} With $\vv=\V^T\xv$:

%             \vspace{-\baselineskip}
            
%             \begin{equation*}
%                 \xv^T \H \xv = \xv^T\V\Lambda\V^T\xv = \vv^T\Lambda\vv = \sum_{i=1}^d \lambda_iv_i^2 \leq \textcolor{magenta}{\lambda_\text{max}} \sum_{i=1}^d v_i^2 = \textcolor{magenta}{\lambda_\text{max}}\|\vv\|^2
%             \end{equation*}
%             Since $\|\vv\| = \|\xv\|$ ($\V$ orthogonal): $\max_{\|\xv\|=1} \xv^T \H \xv \leq \textcolor{magenta}{\lambda_\text{max}}$
            
%             Additional: $\textcolor{magenta}{\vv_\text{max}}^T \H \textcolor{magenta}{\vv_\text{max}} = \mathbf{e}_1^T\Lambda\mathbf{e}_1 = \textcolor{magenta}{\lambda_\text{max}}$

%             Analogous: $\min_{\|\xv\|=1} \xv^T \H \xv \geq \textcolor{orange}{\lambda_\text{min}}$ and $\textcolor{orange}{\vv_\text{min}}^T \H \textcolor{orange}{\vv_\text{min}} = \textcolor{orange}{\lambda_\text{min}}$
%         \end{footnotesize}

%     \medskip

%     \item Contour lines of any quadratic functions are ellipses \\
%     (with eigenvectors of A as principal axes)
%         % \vspace{0.25\baselineskip}
    
%         % \begin{footnotesize}
%         % Look at $q(\xv) = \xv^T \A \xv + \bv^T \xv + c$ \\
%         % Now use $\yv = \xv - \wv = \xv + \frac{1}{2} \A^{-1} \bv$\\
%         % This already gives us the general form of an ellipse:\\
%         % $\yv^T \A \yv = (\xv-\wv)^T \A (\xv-\wv) = q(\xv) + const$\\
%         % If we use $\zv = \vv^T y$ we obtain it in standard form\\
%         % $\sumin \lambda_i z_i^2 = \zv^T \bm{\Lambda} \zv = y^T \vv \bm{\Lambda} \vv^T y = \yv^T \A \yv = q(\xv) + const $
%         % \end{footnotesize}

        
%         %     \textbf{Proof:} With $\vv=\V^T\xv$:

%         %     \vspace{-0.5\baselineskip}

%         %     \begin{equation*}
%         %         q(\xv) = \xv^T \H \xv + \bv^T\xv + c = \vv^T\Lambda\vv + \bv^T V \vv + c =: \tilde{q}(\vv)
%         %     \end{equation*}

%         %     Now:
%         %     \begin{equation*}
%         %         q(\vv_j) = \mathbf{e}_j^T\Lambda\mathbf{e}_j + \bv^T V \mathbf{e}_j + c = \tilde{q}(\mathbf{e}_j)
%         %     \end{equation*}

%         %     Especially: $q(\textcolor{magenta}{\vv_\text{max}}) = \textcolor{magenta}{\lambda_\text{max}} = \tilde{q}(\mathbf{e}_1) \quad\text{and}\quad q(\textcolor{orange}{\vv_\text{min}}) = \textcolor{orange}{\lambda_\text{min}} = \tilde{q}(\mathbf{e}_d)$
%         % \end{footnotesize}
% \end{itemize}

% Recall: \textbf{Second order condition for optimality} is \textbf{sufficient}.

% \medskip

% We skipped the \textbf{proof} at first, but can now catch up on it.


%     \footnotesize
%      If $H(\xv^\ast) \succ 0$ at stationary point~$\xv^\ast$, then $\xv^\ast$ is local minimum ($\prec$ for maximum).

%      \medskip

%     \textbf{Proof:}
%     Let $\textcolor{orange}{\lambda_\text{min}}>0$ denote the smallest eigenvalue of $H(\xv^\ast)$.
%     Then:
    
%     %\vspace{-1.25\baselineskip}
    
%     \begin{equation*}
%         f(\xv) = f(\xv^\ast) + \underbrace{\nabla f(\xv^\ast)}_{=0}{}(\xv-\xv^\ast) + \frac{1}{2}\underbrace{(\xv-\xv^\ast)^T H(\xv^\ast)(\xv-\xv^\ast)}_{\geq \textcolor{orange}{\lambda_\text{min}} \|\xv-\xv^\ast\|^2 \text{ (see above)}} + \underbrace{R_2(\xv,\xv^\ast)}_{=o(\|\xv-\xv^\ast\|^2)}.
%     \end{equation*}

%     Choose~$\eps>0$ s.t. $|R_2(\xv,\xv^\ast)| < \frac{1}{2} \textcolor{orange}{\lambda_\text{min}} \|\xv-\xv^\ast\|^2$ for each~$\xv \neq \xv^\ast$ with $\|\xv-\xv^\ast\|<\eps$.
%     Then:

%     \vspace{-1.25\baselineskip}

%     \begin{equation*}
%         f(\xv) \geq f(\xv^\ast) + \underbrace{\frac{1}{2} \textcolor{orange}{\lambda_\text{min}} \|\xv-\xv^\ast\|^2 + R_2(\xv,\xv^\ast)}_{>0} > f(\xv^\ast) \quad\text{for each $\xv \neq \xv^\ast$ with $\|\xv-\xv^\ast\|<\eps$}.
%     \end{equation*}


% \framebreak

% If spectrum of~$\Amat$ is known, also that of $\H = 2\Amat$ is known.

% \begin{itemize}
%     \item If \textbf{all} eigenvalues of $\H$ $\overset{(>)}{\geq} 0$ ($\Leftrightarrow$ $\H \overset{(\succ)}{\succcurlyeq} 0$):
%         \begin{itemize} 
%             \item $q$ (strictly) convex,
%             \item there is a (unique) global minimum. 
%         \end{itemize}
%     \item If \textbf{all} eigenvalues of $\H$ $\overset{(<)}{\leq} 0$ ($\Leftrightarrow$ $\H \overset{(\prec)}{\preccurlyeq} 0$):
%         \begin{itemize} 
%             \item $q$ (strictly) concave,
%             \item there is a (unique) global maximum. 
%         \end{itemize}
%     \item If~$\H$ has both positive and negative eigenvalues ($\Leftrightarrow$ $\H$ indefinite):
%         \begin{itemize}
%             \item $q$ neither convex nor concave,
%             \item there is a saddle point.
%         \end{itemize}
% \end{itemize}

% % \begin{figure}
% %     \includegraphics[width=0.3\textwidth, keepaspectratio]{figure_man/minmaxsaddle_2.png}
% % \end{figure}

% \begin{figure}
%     \centering
%     \includegraphics[width=0.67\textwidth]{figure_man/minmaxsaddle.png}
% \end{figure}

% \end{vbframe}

\begin{framei}{Geometry of quadratic functions}
\item Example: $\Amat = \begin{pmatrix} 2 & -1 \\ -1 & 2\end{pmatrix}$ $\Rightarrow$ $\H = 2\Amat = \begin{pmatrix} 4 & -2 \\ -2 & 4\end{pmatrix}$
\item Since $\H$ symmetric: eigendecomposition $\H = \V\Lambda\V^T$
\begin{equation*}
\V = \begin{pmatrix}
| & | \\
\textcolor{magenta}{v_\text{max}} & \textcolor{orange}{v_\text{min}} \\
| & |
\end{pmatrix}
= \frac{1}{\sqrt{2}} \begin{pmatrix}
1 & 1 \\
-1 & 1
\end{pmatrix}\;\text{orthogonal}
\end{equation*}
\begin{equation*}
\Lambda = \begin{pmatrix}
\textcolor{magenta}{\lambda_\text{max}} & 0 \\
0 & \textcolor{orange}{\lambda_\text{min}}
\end{pmatrix}
= \begin{pmatrix}6 & 0 \\ 0 & 2\end{pmatrix}
\end{equation*}
\imageC[0.35]{figure_man/quadr-eigenv.png}
\end{framei}

\begin{framei}[fs=footnotesize]{Spectrum and curvature}
\item $\vvmax$ direction of highest curvature, with curvature value $\lammax$
$$
\vv^T \H \vv = \vv^T \V \Lambda \V^T \vv = \wv^T \Lambda \wv = 
\sumid \lambda_i w_i^2 \le \lammax \sumid w_i^2 = \lammax || \wv ||^2
$$
\item Since $|| \vv || = || \xv ||$ ($\V$ orthogonal): 
$\max_{|| \vv || = 1} \vv^T \H \vv \le \lammax$
\item For $\vvmax$ we obtain this upper bound:
$\vvmax^T \H \vvmax = \mathbf{e}_1^T \Lambda \mathbf{e}_1 = \lammax$ 
\item Analogously, $\vvmin$ direction of lowest curvature, with curvature value $\lammin$
\item Contour lines of any quadratic function are ellipses
% \item[] Look at $q(\xv) = \xv^T \A \xv + \bv^T \xv + c$, set $\yv = \xv - \wv = \xv + \tfrac{1}{2} \A^{-1} \bv$
% \item[] Then $(\xv-\wv)^T \A (\xv-\wv) = \yv^T \A \yv = q(\xv) + \text{const}$
% \item[] With $\zv = \vv^T \yv$ we obtain standard form: $\sumin \lambda_i z_i^2 = \zv^T \bm{\Lambda} \zv = \yv^T \A \yv = q(\xv) + \text{const}$
\end{framei}

\begin{framei}[fs=footnotesize]{Second order condition}
\item Recall: Second order condition for optimality is sufficient
\item If $H(\xv^\ast) \succ 0$ at stationary point $\xv^\ast$, then $\xv^\ast$ local minimum ($\prec$ for maximum)
\item[] \begin{equation*}
f(\xv) = f(\xv^\ast) + \underbrace{\nabla f(\xv^\ast)}_{=0}(\xv-\xv^\ast) + \tfrac{1}{2}\underbrace{(\xv-\xv^\ast)^T H(\xv^\ast)(\xv-\xv^\ast)}_{\ge \textcolor{orange}{\lambda_\text{min}} \|\xv-\xv^\ast\|^2} + \underbrace{R_2(\xv,\xv^\ast)}_{=o(\|\xv-\xv^\ast\|^2)}
\end{equation*}
\item Choose $\eps>0$ s.t. $|R_2(\xv,\xv^\ast)| < \tfrac{1}{2} \textcolor{orange}{\lambda_\text{min}} \|\xv-\xv^\ast\|^2$ for $\xv \neq \xv^\ast$, $\|\xv-\xv^\ast\|<\eps$
\item[] \begin{equation*}
f(\xv) \ge f(\xv^\ast) + \underbrace{\tfrac{1}{2} \textcolor{orange}{\lambda_\text{min}} \|\xv-\xv^\ast\|^2 + R_2(\xv,\xv^\ast)}_{>0} > f(\xv^\ast)
\end{equation*}
\end{framei}

\begin{framei}{Eigenvalues and shape}
\item If spectrum of $\Amat$ is known, also that of $\H = 2\Amat$ is known
\item If all eigenvalues of $\H$ $\overset{(>)}{\ge 0}$ ($\Leftrightarrow$ $\H \overset{(\succ)}{\succcurlyeq} 0$):
\begin{itemize}
\item $q$ (strictly) convex
\item (Unique) global minimum
\end{itemize}
\item If all eigenvalues of $\H$ $\overset{(<)}{\le 0}$ ($\Leftrightarrow$ $\H \overset{(\prec)}{\preccurlyeq} 0$):
\begin{itemize}
\item $q$ (strictly) concave
\item (Unique) global maximum
\end{itemize}
\item If $\H$ has both positive and negative eigenvalues ($\Leftrightarrow$ $\H$ indefinite):
\begin{itemize}
\item $q$ neither convex nor concave
\item there is a saddle point
\end{itemize}
\imageC[0.67]{figure_man/minmaxsaddle.png}
\end{framei}

\begin{framei}{Condition and curvature}
\item $\kappa(\H) = \kappa(\Amat) = |\textcolor{magenta}{\lambda_\text{max}}| / |\textcolor{orange}{\lambda_\text{min}}|$
\item High condition means
\begin{itemize}
\item $|\textcolor{magenta}{\lambda_\text{max}}| \gg |\textcolor{orange}{\lambda_\text{min}}|$
\item Curvature along $\textcolor{magenta}{\vv_\text{max}} \gg$ along $\textcolor{orange}{\vv_\text{min}}$
\item Problem for algorithms like gradient descent
\end{itemize}
\imageC[1]{figure_man/quadr-conds.png}
{\footnotesize Left: Excellent condition. Middle: Good condition. Right: Bad condition.}
\end{framei}

\begin{framei}{Approximation of smooth functions}
\item Any $f \in \mathcal{C}^2$ can be locally approximated by quadratic function (second order Taylor)
\item[] \begin{equation*}
f(\xv) \approx f(\bm{\tilde{x}}) + \nabla f(\bm{\tilde{x}})(\xv-\bm{\tilde{x}}) + \tfrac 12(\xv-\bm{\tilde{x}})^T\nabla^2 f(\bm{\tilde{x}})(\xv-\bm{\tilde{x}})
\end{equation*}
\imageC[0.3]{figure_man/taylor_2D_quadratic.png}
{\footnotesize $f$ and second order approximation: dark vs bright grid. (Source: \url{daniloroccatano.blog})}
\item $\implies$ Hessians provide information about \textbf{local} geometry of a function
\end{framei}
  
\endlecture

\end{document}
  
  