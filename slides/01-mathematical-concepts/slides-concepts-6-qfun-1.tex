\documentclass[11pt,compress,t,notes=noshow, xcolor=table]{beamer}

\input{../../style/preamble}
\input{../../latex-math/basic-math}

% sets
\newcommand{\CC}[1]{\mathcal{C}^{#1}}         % class C^k
\newcommand{\CCinf}{\mathcal{C}^{\infty}}     % class C^infty


% objectives, domains, points
\renewcommand{\S}{\mathcal{S}}              % domain of f
\newcommand{\xv}{\bm{x}}                  % vector x (bold)
\newcommand{\yv}{\bm{y}}                  % vector y (bold)
\newcommand{\vv}{\bm{v}}                  % vector v (bold)
\newcommand{\wv}{\bm{w}}                  % vector w (bold)
\newcommand{\xvs}{\bm{x}^\ast}            % x-star,  theoretical optimum
\newcommand{\xs}{x^\ast}                  % x-star,  theoretical optimum
\newcommand{\fxvs}{f(\xvs)}               % f at x-star, theoretical optimum
\renewcommand{\H}{\mathbf{H}}                     % Hessian matrix



% some other stuff
\providecommand{\eps}{\epsilon}               % epsilon
\newcommand{\A}{\mathbf{A}}                     % Hessian matrix
\newcommand{\bv}{\mathbf{b}}                     % Hessian matrix
\newcommand{\zv}{\mathbf{z}}                     % Hessian matrix
\newcommand{\V}{\mathbf{V}}                     % Hessian matrix


\newcommand{\vvmax}{\bm{v}_\text{max}}                  % vector v (bold)
\newcommand{\vvmin}{\bm{v}_\text{min}}                  % vector v (bold)
\newcommand{\lammax}{\lambda_\text{max}}                  % vector v (bold)
\newcommand{\lammin}{\lambda_\text{min}}                  % vector v (bold)

\newcommand{\sumid}{\sum\limits_{i=1}^d} % summation from i=1 to n


\title{Optimization in Machine Learning}

\begin{document}

\titlemeta{
Mathematical Concepts 
}{
Quadratic forms I
}{
figure_man/quadratic_functions_2D_example_1_1.png
}{
\item Definition of quadratic functions
\item Gradient, Hessian
\item Optima
}

\begin{framei}{Univariate quadratic}
\item Quadratic function $q: \R \to \R$
$$ q(x) = a x^2 + b x + c,\quad a \ne 0 $$
\imageC[0.35]{figure_man/quadratic_functions_1D.png}
\item Left: $q_1(x) = x^2$. Right: $q_2(x) = - x^2$
\end{framei}

\begin{framei}{Univariate: basic properties}
\item Slope at $(x, q(x))$: $$ q'(x) = 2 a x + b $$
\imageC[0.25]{figure_man/quadratic_functions_1D_derivative.png}
\item Curvature: $$ q''(x) = 2 a $$
\imageC[0.25]{figure_man/quadratic_functions_1D_curvature.png}
%\item $q_1 = x^2$ (orange), $q_2 = 2 x^2$ (green), $q_3 = - x^2$ (blue), $q_4 = - 3 x^2$ (magenta)
\item $a > 0$: $q$ convex, bounded from below, unique global minimum
\item $a < 0$: $q$ concave, bounded from above, unique global maximum
\item Optimum $\xs$
$$ q'(x) = 0 \;\Leftrightarrow\; 2 a x + b = 0 \;\Rightarrow\; \xs = \frac{-b}{2 a} $$\\
as 2nd derivative: $q''(\xs) = 2 a \ne 0$ 
\end{framei}
  
\begin{framei}{Multivariate quadratic}
\item $q: \R^d \to \R$
$$ q(\xv) = \xv^T \A \xv + \bv^T \xv + c $$
\item with $\A \in \R^{d \times d}$ full rank, $\bv \in \R^d$, $c \in \R$
\vfill
\splitV{
\imageC[0.9]{figure_man/quadratic_functions_2D_example_1_1.png}
}{
\imageC[0.9]{figure_man/quadratic_functions_2D_example_1_2.png}
}
\end{framei}

\begin{framei}{Symmetrization}
\item W.l.o.g. assume $\A$ symmetric, i.e., $\A^T = \A$
\item If $\A$ not symmetric, there exists symmetric $\tilde \A$ with $$ q(\xv) = \xv^T \A \xv = \xv^T \tilde \A \xv =: \tilde q(\xv) $$
\item Justification
$$ q(\xv) = \xv^T \A \xv = \tfrac12 \xv^T \underbrace{(\A + \A^T)}_{\tilde \A_1} \xv + \tfrac12 \xv^T \underbrace{(\A - \A^T)}_{\tilde \A_2} \xv $$
\item $\tilde \A_1$ symmetric, $\tilde \A_2$ anti-symmetric ($\tilde \A_2^T = - \tilde \A_2$)
\item Since $\xv^T \A^T \xv$ is a scalar, equal to its transpose
\begin{align*}
\xv^T (\A - \A^T) \xv &= \xv^T \A \xv - \xv^T \A^T \xv = \xv^T \A \xv - (\xv^T \A^T \xv)^T \\
                      &= \xv^T \A \xv - \xv^T \A \xv = 0
\end{align*}
\item Therefore $\tilde q(\xv) = \xv^T \tilde \A \xv$ with $\tilde \A = \tilde \A_1/2$
\end{framei}

\begin{framei}{Gradient and Hessian}
\item $q: \R^d \to \R$
$$ q(\xv) = \xv^T \A \xv + \bv^T \xv + c $$
\item Gradient
$$ \nabla q(\xv) = ((\A^T + \A) \xv + \bv)^T $$
\item Under assumed symmetry: $\nabla q(\xv) = (2 \A \xv + \bv)^T$
\item Directional derivative: $\nabla q(\xv)\, \vv $
\item Hessian
$$ \nabla^2 q(\xv) = (\A^T + \A) = 2 \A =: \H \in \R^{d \times d} $$
\item Under assumed symmetry: $\H = 2 \A$
\item Directional curvature: $\vv^T \H \vv $
\end{framei}


\begin{framei}{Optimum}
\item $q: \R^d \to \R$
$$ q(\xv) = \xv^T \A \xv + \bv^T \xv + c $$
\item Since $\A$ full rank, unique stationary point $\xvs$ (min, max, or saddle)
\begin{align*}
\nabla q(\xvs)      &= \boldsymbol{0}^T \\
(2 \A \xvs + \bv)^T &= \boldsymbol{0}^T \\
\xvs                &= -\tfrac{1}{2} \A^{-1} \bv
\end{align*}
\item $q(\xvs) = c - \frac12 \bv^T \A^{-1} \bv$
\vfill
\imageC[1]{figure_man/minmaxsaddle.png}
\item Left: $\A$ pos. def. \quad Middle: $\A$ neg. def. \quad Right: $\A$ indefinite
\end{framei}

\begin{framei}[fs=footnotesize]{Optima: rank-deficient case}
\item $q: \R^d \to \R$
$$ q(\xv) = \xv^T \A \xv + \bv^T \xv + c $$
\item Assume $\A$ symmetric now
\item For stationary points to exist, we need : $\nabla q(\xv) = 2 \A \xv + \bv = 0$
\item This implies we need $\bv \in range(\A)$, let's assume this is the case
\item Let $\xv_p$ be stationary, so $2 \A \xv_p = -\bv$
\item Then any point in affine space $\xv_p + ker(\A)$ is also stationary, with same function value and same Hessian (as it is constant)
\imageC[0.35]{figure_man/convex-example.png}
\end{framei}0


\begin{framei}[fs=footnotesize]{Optima: rank-deficient case}
\item All affine spaces of form $\xv_p + ker(\A)$ for diff. valid $\xv_p$ are the same
\item Any stationary point must be in $\xv_p + ker(\A)$
\item So $\xv_p + ker(\A)$ are all the stationary points, with same curvature
\item If $\A \succeq 0$, they are all minima
\item If $\A \preceq 0$, they are all maxima
\item If $\A$ is indefinite, they are all saddle points
\imageC[0.35]{figure_man/convex-example.png}
\end{framei}
  

\endlecture

\end{document}
