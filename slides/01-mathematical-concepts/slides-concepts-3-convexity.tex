\documentclass[11pt,compress,t,notes=noshow, xcolor=table]{beamer}

\input{../../style/preamble}
\input{../../latex-math/basic-math}
\input{../../latex-math/basic-ml}
\usepackage[export]{adjustbox}


\newcommand{\titlefigure}{figure_man/convex.png}
\newcommand{\learninggoals}{
\item Convex sets
\item Convex functions
}


%\usepackage{animate} % only use if you want the animation for Taylor2D

\title{Optimization}
%\author{Bernd Bischl}
\date{}

\begin{document}

\lecturechapter{Convexity}
\lecture{Optimization}
\sloppy




\begin{vbframe}{Convex sets}

A set of $\mathcal{S}$ is \textbf{convex}, if for all $\xv, \mathbf{y} \in \mathcal{S}$ and all $t \in [0, 1]$ the following holds:

$$
\xv + t (\mathbf{y} - \xv) \in \mathcal{S}
$$

Intuitively: Connecting line between any $\xv, \bm{y} \in \mathcal{S}$ lies completely in $\mathcal{S}$.

\begin{center}
\includegraphics[width = 0.2\textwidth]{figure_man/convex.png}~~~\includegraphics[width = 0.2\textwidth]{figure_man/concave.png} \\
\footnotesize{Left: convex set; right: not convex. Source: Wikipedia. 
}
\end{center}

\end{vbframe}

\begin{vbframe}{Convex functions}

Consider $f: \mathcal{S} \to \R$, $\mathcal{S}$ convex.\\
$f$ is \textbf{convex} if for all $\xv, \mathbf{y} \in \mathcal{S}$ and all $t \in [0, 1]$

\vspace*{-0.2cm}

$$
f(\xv + t(\mathbf{y} - \xv)) \le \fx + t(f(\mathbf{y}) - \fx).
$$

Intuitively: Connecting line lies above function.

\begin{center}
\includegraphics[width = 0.4\textwidth, keepaspectratio]{figure_man/convexity_1.pdf}~~~\includegraphics[width = 0.40\textwidth, keepaspectratio]{figure_man/convexity_2.pdf} \\
\footnotesize{Left: Strictly convex function. Right: Convex, but not strictly. }
\end{center}

\textbf{Strictly convex} if \enquote{$<$} instead of \enquote{$\le$}. \textbf{Concave} (strictly) if the equation holds with \enquote{$\ge$} (\enquote{$>$}), respectively. 

\vspace*{0.2cm}

\textbf{Note:} $f$ (strictly) concave $\Leftrightarrow$ $-f$ (strictly) convex.



\end{vbframe}

\begin{vbframe}{Examples}

\textbf{Convex function:} $f(x) = |x|$. \\
\begin{footnotesize}

\vspace*{-0.5cm}

\begin{eqnarray*}
\textbf{Proof: } f\left(\xv + t(\mathbf{y} - \xv)\right) &=& |\xv + t(\mathbf{y} - \xv)| = |(1 - t) \xv + t\cdot\mathbf{y}| \le |(1 - t) \xv| + |t\cdot\mathbf{y}| \\ &=& (1 - t) |\xv| + t |\mathbf{y}| = |\xv| + t \cdot (|\mathbf{y}| - |\xv|) \\ &=& f(\xv) + t \cdot (f(\mathbf{y}) - f(\xv))
\end{eqnarray*}
% https://sboyles.github.io/teaching/ce377k/convexity.pdf
\end{footnotesize}

\textbf{Concave function}: $f(x) = \log(x)$. 

\vspace*{0.2cm}

\textbf{Neither nor}: $f(x) = \exp(-x^2)$ (but log-concave)

\begin{figure}
    \centering
    \includegraphics[width=0.45\textwidth]{figure_man/conv_conc_functions.png}
\end{figure}

\end{vbframe}




\begin{vbframe}{Prove convexity via Hessian}

Let $f \in \mathcal{C}^2$ and $H(\xv)$ its Hessian.

\lz 

$f$ is \textbf{convex iff} $H(\xv)$ is positive semidefinite (p.s.d.) for all $\xv \in \mathcal{S}$, i.e. if for all points $\xv$ and all vectors $\mathbf{d} \ne 0$:

$$
\mathbf{d}^{\top} \nabla^2\fx\mathbf{d} \ge 0.
$$

If $H(\xv)$ positive definite (strict \enquote{$>$}), $f$ is strictly convex.

\lz

\textbf{Alternatively:} Matrix p.s.d. $\Leftrightarrow$ all eigenvalues $\ge 0$. 


\framebreak

% Weiterhin können folgende Implikationen für convexe Funktionen festgestellt werden:
% \medskip
% \begin{itemize}
% \item $f$ convex $\Rightarrow$ Subniveaumenge $S_1 = \{x|f(x) < a\}$ und $S_2 = \{x|f(x) \leq a \}$ bilden convexe Mengen, $a\in \R$.
% \item umgekehrte Implikation \textbf{nicht} notwendigerweise erfüllt.

% \end{itemize}
% \framebreak

\begin{footnotesize}
\textbf{Example:} $f(\xv) = x_1^2 + x_2^2 - 2x_1x_2$, $\nabla f(\xv) = \begin{pmatrix}2x_1 - 2x_2 \\ 2x_2 - 2x_1\end{pmatrix}$, $H(\xv) = \begin{pmatrix} 2 & -2 \\ -2 & 2 \end{pmatrix}. 
$

\begin{center}
  \includegraphics[width = 0.4\textwidth]{figure_man/convex-example.png}
\end{center}

%<<echo=FALSE, size = "footnotesize">>=
%foo = function(x, y) {
%  x^2 + y^2 - 2 * x * y
%}
%x = seq(-4, 6, length = 40); y = x
%z = outer(x, y, FUN = foo)
%persp2(x, y, z, theta = 30, phi = 30)
%@

$f$ is convex since $H(\xv)$ is p.s.d. for all $\xv$:

\begin{eqnarray*}
\mathbf{d}^{\top}\begin{pmatrix} 2 & -2 \\ -2 & 2 \end{pmatrix}\mathbf{d} &=& \mathbf{d}^{\top} \begin{pmatrix} 2d_1 - 2d_2 \\ -2d_1 + 2d_2\end{pmatrix} = 2d_1^2 - 2d_1d_2 -2d_1d_2 + 2d_2^2 \\ &=& 2d_1^2 - 4d_1d_2 + 2d_2^2 = 2 (d_1 - d_2)^2 \ge 0.
\end{eqnarray*}

% So the function $f$ is convex and every local minimum is also a global minimum.

\end{footnotesize}

\end{vbframe}


\begin{vbframe}{Convex functions in optimization}
  

  \begin{itemize}
    \item For a convex function, every local optimum is a global one  
    \item A strictly convex function at most one optimal point
  \end{itemize}
  
  \begin{center}
  \includegraphics[width = 0.9\textwidth]{figure_man/convexity_3.pdf} \\
  \begin{footnotesize}
  Left: Strictly convex; exactly one local minimum, which is also global. Middle: Convex, but not strictly; all local optima are global ones, but not unique. Right: Not convex.
  \end{footnotesize} 
  \end{center}
  
\framebreak 

\vspace*{2cm}

\begin{center}
\Large{\enquote{...in fact, the great watershed in optimization isn't between linearity and nonlinearity, but convexity and nonconvexity.}}\\
\normalsize - R. Tyrrell Rockafellar, in SIAM Review, 1993
\end{center}

	
\end{vbframe}

\endlecture
\end{document}