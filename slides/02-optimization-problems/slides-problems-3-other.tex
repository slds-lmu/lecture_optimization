\documentclass[11pt,compress,t,notes=noshow, xcolor=table]{beamer}

\input{../../style/preamble}
\input{../../latex-math/basic-math}
\input{../../latex-math/basic-ml}

\title{Optimization in Machine Learning}

\begin{document}

\titlemeta{
Optimization Problems
}{
Other optimization problems
}{
figure_man/landscapes2.jpg
}{
\item Discrete / feature selection
\item Black-box / hyperparameter optimization
\item Noisy
\item Multi-objective    
}

\begin{frame2}{Other classes of optimization problems}
\textbf{So far}: \enquote{nice} (un)constrained problems:
\begin{itemize}
\item Problem defined on continuous domain $\mathcal{S}$
\item Analytical objectives (and constraints)
\end{itemize}
\textbf{Other characteristics}:
\begin{itemize}
\item Discrete domain $\mathcal{S}$
\item $f$ \textbf{black-box}: Objective not available in analytical form\\computer program to evaluate
\item $f$ \textbf{noisy}: Objective can be queried but evaluations are noisy $\fx = f_{\text{true}}(\xv) + \eps, \quad \eps \sim F$
\item $f$ \textbf{expensive}: Single query takes time / resources
\item $f$ multi-objective: $\fx: \mathcal{S} \to \R^m$, $\fx = \left(f_1(\xv), ..., f_m(\xv)\right)$
\end{itemize}
These make the problem typically much harder to solve!
\end{frame2}


\begin{frame2}{Example 1: Best subset selection}
Let {\footnotesize $\D = \left(\left(\xi, \yi\right)\right)_{1 \le i \le n}$, $\xi \in \R^p$}. Fit LM based on best feature subset.
$$
\min_{\thetav \in \Theta} \left(\yi - \thetav^\top \xi\right)^2, ||\thetav||_{0} \leq k
$$
\spacer
\splitV[0.55]{
\textbf{Problem characteristics}:
\begin{itemizeL}
\item White-box: Objective available in analytical form
\item Discrete: $\mathcal{S}$ is mixed continuous and discrete
\item Constrained
\end{itemizeL}
\spacer
The problem is even \textbf{NP-hard} (Bin et al., 1997, The Minimum Feature Subset Selection Problem)!
}{
\image[1][https://rpubs.com/davoodastaraky/subset]{figure_man/subset_selection.png}
}
\end{frame2}


\begin{frame2}{Example 2: Wrapper Feature selection}
Subset sel. can be generalized to any learner $\ind$ using only features $\bm{s}$:
$$
\min_{\textbf{s} \in \{0, 1\}^p} \widehat{\text{GE}}(\mathcal{I}, \mathcal{J}, \rho, \bm{s}),
$$
$\widehat{\text{GE}}$ general. err. with metric $\rho$ and estim. with resampling splits $\mathcal{J}$
\spacer
\splitV[0.55]{
\textbf{Problem characteristics}:
\begin{itemize}
\item black box\\{\footnotesize eval by program}
\item $\mathcal{S}$ is discrete / binary
\item expensive\\{\footnotesize 1 eval: 1 or multiple ERM(s)!}
\item noisy\\{\footnotesize uses data / resampling}
\item NB: Less features can be better in prediction (overfitting)
\end{itemize}
}{
\image{figure_man/feature_subset_selection.png}
}
\end{frame2}


% \begin{vbframe}{Example 3: Feature sel. (constrained)}
% With prior knowledge / preferences on max. number of features $k$:
% \begin{eqnarray*}
% \min_{\textbf{s} \in \{0, 1\}^p} \widehat{\text{GE}}(\mathcal{I}, \mathcal{J}, \rho, \bm{s}), \quad \sum\nolimits_{i = 1}^p s_i \le k, 
% \end{eqnarray*}
% \begin{figure}[!tbp]
%   \centering
%   \begin{minipage}[b]{0.5\textwidth}
%     \includegraphics[width =\textwidth]{figure_man/pipeline.png}
%   \end{minipage}
%   \hfill
%   \begin{minipage}[b]{0.4\textwidth}
%     \includegraphics[width =\textwidth]{figure_man/subsetsize.png}
%   \end{minipage}
% \end{figure}
% \framebreak
% \lz 
% \textbf{Problem characteristics}:
% \begin{itemize}
% 	\item Black-box: Objective not available in analytical form
% 	\item Discrete: $\mathcal{S}$ is discrete
% 	\item Constrained 
% 	\item Noisy: Function evaluations depend on resampling	
% 	\item Expensive (depending on training time of learner $\ind$)
% \end{itemize}
% \lz 
% Complexity of the problem is reduced with $k$: Lower $k$ $\to$ less possible values for $\bm{s}$. 
% \end{vbframe}


\begin{frame2}{Example 3: Feature sel. (multiobjective)}
Feature selection is usually inherently multi-objective,
with model sparsity as a 2nd trade-off target:
$$
\min_{\textbf{s} \in \{0, 1\}^p} \left(\widehat{\text{GE}}(\mathcal{I}, \mathcal{J}, \rho, \bm{s}), \sum\nolimits_{i = 1}^p s_i\right).
$$

$\widehat{\text{GE}}$ general. err. with metric $\rho$ and estim. with resampling splits $\mathcal{J}$
\spacer
\splitV[0.45]{
\begin{itemize}
\item Multiobjective
\item black box\\{\footnotesize eval by program}
\item S is discrete / binary
\item expensive\\{\footnotesize 1 eval: 1 or multiple ERM(s)!}
\item noisy\\{\footnotesize uses data / resampling}
\end{itemize}
}{
\image{figure_man/pareto.png}
}
\end{frame2}


\begin{framei}{Example 4: Hyperparameter optimization}
\item Learner $\ind$ usually configurable by hyperparameters $\lamv \in \Lambda$.
\item Find best HP configuration $\lamv^\ast$
$$
\lamv^\ast \in \argmin\nolimits_{\lamv \in \Lambda} c(\lamv) = \argmin \widehat{\text{GE}}(\mathcal{I}, \mathcal{J}, \rho, \lamv)
$$
$\widehat{\text{GE}}$ general. err. with metric $\rho$ and estim. with resampling splits $\mathcal{J}$
\spacer
\splitV[0.6]{
\image{figure_man/hpo_loop_1.png}
}{
\image[1][https://ceur-ws.org/Vol-2846/paper22.pdf]{figure_man/landscapes2.jpg}
}
\end{framei}


\begin{frame2}{Example 4: Hyperparameter optimization}
Solving
$$
\lamv^\ast \in \argmin_{\lamv \in \Lambda} c(\lamv)
$$
is very challenging:
\splitV{
\begin{itemize}
\item $c$ black box\\{\footnotesize eval by progrmm}
\item expensive\\{\footnotesize 1 eval: 1 or multiple ERM(s)!}
\item noisy\\{\footnotesize uses data / resampling}
\item the search space $\Lambda$ might be mixed\\{\footnotesize continuous, integer, categ. or hierarchical}
\end{itemize}
}{
\image[1][https://ceur-ws.org/Vol-2846/paper22.pdf]{figure_man/landscapes2.jpg}
}
\end{frame2}


\begin{frame2}{More black-box problems}
Black-box problems from engineering: \textbf{oil well placement}
\spacer
\splitV[0.65]{
\begin{itemize}
\item The goal is to determine the optimal locations and operation parameters for wells in oil reservoirs
\item Basic premise: achieving maximum revenue from oil while minimizing operating costs
\item In addition, the objective function is subject to complex combinations of geological, economical, petrophysical and fluiddynamical constraints
\item Each function evaluation requires several computationally expensive reservoir simulations while taking uncertainty in the reservoir description into account
\end{itemize}
}{
\image[0.9][https://doi.org/10.1007/s13202-019-0710-1]{figure_man/oil_well_problem.jpg}\\
{\footnotesize Oil saturation at various depths with possible location of wells.}
}
\end{frame2}

\endlecture
\end{document}
