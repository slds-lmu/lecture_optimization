\documentclass[11pt,compress,t,notes=noshow, xcolor=table]{beamer}

\usepackage[]{graphicx}
\usepackage[]{color}
% maxwidth is the original width if it is less than linewidth
% otherwise use linewidth (to make sure the graphics do not exceed the margin)
\makeatletter
\def\maxwidth{ %
  \ifdim\Gin@nat@width>\linewidth
    \linewidth
  \else
    \Gin@nat@width
  \fi
}
\makeatother

% ---------------------------------%
% latex-math dependencies, do not remove:
% - mathtools
% - bm
% - siunitx
% - dsfont
% - xspace
% ---------------------------------%

%--------------------------------------------------------%
%       Language, encoding, typography
%--------------------------------------------------------%

\usepackage[english]{babel}
\usepackage[utf8]{inputenc} % Enables inputting UTF-8 symbols
% Standard AMS suite
\usepackage{amsmath,amsfonts,amssymb}

% Font for double-stroke / blackboard letters for sets of numbers (N, R, ...)
% Distribution name is "doublestroke"
% According to https://mirror.physik.tu-berlin.de/pub/CTAN/fonts/doublestroke/dsdoc.pdf
% the "bbm" package does a similar thing and may be superfluous.
% Required for latex-math
\usepackage{dsfont}

% bbm – "Blackboard-style" cm fonts (https://www.ctan.org/pkg/bbm)
% Used to be in common.tex, loaded directly after this file
% Maybe superfluous given dsfont is loaded
% TODO: Check if really unused?
% \usepackage{bbm}

% bm – Access bold symbols in maths mode - https://ctan.org/pkg/bm
% Required for latex-math
% https://tex.stackexchange.com/questions/3238/bm-package-versus-boldsymbol
\usepackage{bm}

% pifont – Access to PostScript standard Symbol and Dingbats fonts
% Used for \newcommand{\xmark}{\ding{55}, which is never used
% aside from lecture_advml/attic/xx-automl/slides.Rnw
% \usepackage{pifont}

% Quotes (inline and display), provdes \enquote
% https://ctan.org/pkg/csquotes
\usepackage{csquotes}

% Adds arg to enumerate env, technically superseded by enumitem according
% to https://ctan.org/pkg/enumerate
% Replace with https://ctan.org/pkg/enumitem ?
\usepackage{enumerate}

% Line spacing - provides \singlespacing \doublespacing \onehalfspacing
% https://ctan.org/pkg/setspace
% \usepackage{setspace}

% mathtools – Mathematical tools to use with amsmath
% https://ctan.org/pkg/mathtools?lang=en
% latex-math dependency according to latex-math repo
\usepackage{mathtools}

% Maybe not great to use this https://tex.stackexchange.com/a/197/19093
% Use align instead -- TODO: Global search & replace to check, eqnarray is used a lot
% $ rg -f -u "\begin{eqnarray" -l | grep -v attic | awk -F '/' '{print $1}' | sort | uniq -c
%   13 lecture_advml
%   14 lecture_i2ml
%    2 lecture_iml
%   27 lecture_optimization
%   45 lecture_sl
\usepackage{eqnarray}

% For shaded regions / boxes
% Used sometimes in optim
% https://www.ctan.org/pkg/framed
\usepackage{framed}

%--------------------------------------------------------%
%       Cite button (version 2024-05)
%--------------------------------------------------------%
% Note this requires biber to be in $PATH when running,
% telltale error in log would be e.g. Package biblatex Info: ... file 'authoryear.dbx' not found
% aside from obvious "biber: command not found" or similar.

\usepackage{hyperref}
\usepackage{usebib}
\usepackage[backend=biber, style=authoryear]{biblatex}

% Only try adding a references file if it exists, otherwise
% this would compile error when references.bib is not found
\IfFileExists{references.bib} {
  \addbibresource{references.bib}
  \bibinput{references}
}

\newcommand{\citelink}[1]{%
\NoCaseChange{\resizebox{!}{9pt}{\protect\beamergotobutton{\href{\usebibentry{\NoCaseChange{#1}}{url}}{\begin{NoHyper}\cite{#1}\end{NoHyper}}}}}%
}

%--------------------------------------------------------%
%       Displaying code and algorithms
%--------------------------------------------------------%

% Reimplements verbatim environments: https://ctan.org/pkg/verbatim
% verbatim used sed at least once in
% supervised-classification/slides-classification-tasks.tex
\usepackage{verbatim}

% Both used together for algorithm typesetting, see also overleaf: https://www.overleaf.com/learn/latex/Algorithms
% algorithmic env is also used, but part of the bundle:
%   "algpseudocode is part of the algorithmicx bundle, it gives you an improved version of algorithmic besides providing some other features"
% According to https://tex.stackexchange.com/questions/229355/algorithm-algorithmic-algorithmicx-algorithm2e-algpseudocode-confused
\usepackage{algorithm}
\usepackage{algpseudocode}

%--------------------------------------------------------%
%       Tables
%--------------------------------------------------------%

% multi-row table cells: https://www.namsu.de/Extra/pakete/Multirow.html
% Provides \multirow
% Used e.g. in evaluation/slides-evaluation-measures-classification.tex
\usepackage{multirow}

% colortbl: https://ctan.org/pkg/colortbl
% "The package allows rows and columns to be coloured, and even individual cells." well.
% Provides \columncolor and \rowcolor
% \rowcolor is used multiple times, e.g. in knn/slides-knn.tex
\usepackage{colortbl}

% long/multi-page tables: https://texdoc.org/serve/longtable.pdf/0
% Not used in slides
% \usepackage{longtable}

% pretty table env: https://ctan.org/pkg/booktabs
% Is used
% Defines \toprule
\usepackage{booktabs}

%--------------------------------------------------------%
%       Figures: Creating, placing, verbing
%--------------------------------------------------------%

% wrapfig - Wrapping text around figures https://de.overleaf.com/learn/latex/Wrapping_text_around_figures
% Provides wrapfigure environment -used in lecture_optimization
\usepackage{wrapfig}

% Sub figures in figures and tables
% https://ctan.org/pkg/subfig -- supersedes subfigure package
% Provides \subfigure
% \subfigure not used in slides but slides-tuning-practical.pdf errors without this pkg, error due to \captionsetup undefined
\usepackage{subfig}

% Actually it's pronounced PGF https://en.wikibooks.org/wiki/LaTeX/PGF/TikZ
\usepackage{tikz}

% No idea what/why these settings are what they are but I assume they're there on purpose
\usetikzlibrary{shapes,arrows,automata,positioning,calc,chains,trees, shadows}
\tikzset{
  %Define standard arrow tip
  >=stealth',
  %Define style for boxes
  punkt/.style={
    rectangle,
    rounded corners,
    draw=black, very thick,
    text width=6.5em,
    minimum height=2em,
    text centered},
  % Define arrow style
  pil/.style={
    ->,
    thick,
    shorten <=2pt,
    shorten >=2pt,}
}

% Defines macros and environments
\usepackage{../../style/lmu-lecture}

\let\code=\texttt     % Used regularly

% Not sure what/why this does
\setkeys{Gin}{width=0.9\textwidth}

% -- knitr leftovers --
% Used often in conjunction with \definecolor{shadecolor}{rgb}{0.969, 0.969, 0.969}
% Removing definitions requires chaning _many many_ slides, which then need checking to see if output still ok
\definecolor{fgcolor}{rgb}{0.345, 0.345, 0.345}
\definecolor{shadecolor}{rgb}{0.969, 0.969, 0.969}
\newenvironment{knitrout}{}{} % an empty environment to be redefined in TeX

%-------------------------------------------------------------------------------------------------------%
%  Unused stuff that needs to go but is kept here currently juuuust in case it was important after all  %
%-------------------------------------------------------------------------------------------------------%

% \newcommand{\hlnum}[1]{\textcolor[rgb]{0.686,0.059,0.569}{#1}}%
% \newcommand{\hlstr}[1]{\textcolor[rgb]{0.192,0.494,0.8}{#1}}%
% \newcommand{\hlcom}[1]{\textcolor[rgb]{0.678,0.584,0.686}{\textit{#1}}}%
% \newcommand{\hlopt}[1]{\textcolor[rgb]{0,0,0}{#1}}%
% \newcommand{\hlstd}[1]{\textcolor[rgb]{0.345,0.345,0.345}{#1}}%
% \newcommand{\hlkwa}[1]{\textcolor[rgb]{0.161,0.373,0.58}{\textbf{#1}}}%
% \newcommand{\hlkwb}[1]{\textcolor[rgb]{0.69,0.353,0.396}{#1}}%
% \newcommand{\hlkwc}[1]{\textcolor[rgb]{0.333,0.667,0.333}{#1}}%
% \newcommand{\hlkwd}[1]{\textcolor[rgb]{0.737,0.353,0.396}{\textbf{#1}}}%
% \let\hlipl\hlkwb

% \makeatletter
% \newenvironment{kframe}{%
%  \def\at@end@of@kframe{}%
%  \ifinner\ifhmode%
%   \def\at@end@of@kframe{\end{minipage}}%
%   \begin{minipage}{\columnwidth}%
%  \fi\fi%
%  \def\FrameCommand##1{\hskip\@totalleftmargin \hskip-\fboxsep
%  \colorbox{shadecolor}{##1}\hskip-\fboxsep
%      % There is no \\@totalrightmargin, so:
%      \hskip-\linewidth \hskip-\@totalleftmargin \hskip\columnwidth}%
%  \MakeFramed {\advance\hsize-\width
%    \@totalleftmargin\z@ \linewidth\hsize
%    \@setminipage}}%
%  {\par\unskip\endMakeFramed%
%  \at@end@of@kframe}
% \makeatother

% \definecolor{shadecolor}{rgb}{.97, .97, .97}
% \definecolor{messagecolor}{rgb}{0, 0, 0}
% \definecolor{warningcolor}{rgb}{1, 0, 1}
% \definecolor{errorcolor}{rgb}{1, 0, 0}
% \newenvironment{knitrout}{}{} % an empty environment to be redefined in TeX

% \usepackage{alltt}
% \newcommand{\SweaveOpts}[1]{}  % do not interfere with LaTeX
% \newcommand{\SweaveInput}[1]{} % because they are not real TeX commands
% \newcommand{\Sexpr}[1]{}       % will only be parsed by R
% \newcommand{\xmark}{\ding{55}}%

% textpos – Place boxes at arbitrary positions on the LATEX page
% https://ctan.org/pkg/textpos
% Provides \begin{textblock}
% TODO: Check if really unused?
% \usepackage[absolute,overlay]{textpos}

% -----------------------%
% Likely knitr leftovers %
% -----------------------%

% psfrag – Replace strings in encapsulated PostScript figures
% https://www.overleaf.com/latex/examples/psfrag-example/tggxhgzwrzhn
% https://ftp.mpi-inf.mpg.de/pub/tex/mirror/ftp.dante.de/pub/tex/macros/latex/contrib/psfrag/pfgguide.pdf
% Can't tell if this is needed
% TODO: Check if really unused?
% \usepackage{psfrag}

% arydshln – Draw dash-lines in array/tabular
% https://www.ctan.org/pkg/arydshln
% !! "arydshln has to be loaded after array, longtable, colortab and/or colortbl"
% Provides \hdashline and \cdashline
% Not used in slides
% \usepackage{arydshln}

% tabularx – Tabulars with adjustable-width columns
% https://ctan.org/pkg/tabularx
% Provides \begin{tabularx}
% Not used in slides
% \usepackage{tabularx}

% placeins – Control float placement
% https://ctan.org/pkg/placeins
% Defines a \FloatBarrier command
% TODO: Check if really unused?
% \usepackage{placeins}

% Can't find a reason why common.tex is not just part of this file?
% This file is included in slides and exercises

% Rarely used fontstyle for R packages, used only in 
% - forests/slides-forests-benchmark.tex
% - exercises/single-exercises/methods_l_1.Rnw
% - slides/cart/attic/slides_extra_trees.Rnw
\newcommand{\pkg}[1]{{\fontseries{b}\selectfont #1}}

% Spacing helpers, used often (mostly in exercises for \dlz)
\newcommand{\lz}{\vspace{0.5cm}} % vertical space (used often in slides)
\newcommand{\dlz}{\vspace{1cm}}  % double vertical space (used often in exercises, never in slides)

% Don't know if this is used or needed, remove?
% textcolor that works in mathmode
% https://tex.stackexchange.com/a/261480
% Used e.g. in forests/slides-forests-bagging.tex
% [...] \textcolor{blue}{\tfrac{1}{M}\sum^M_{m} [...]
% \makeatletter
% \renewcommand*{\@textcolor}[3]{%
%   \protect\leavevmode
%   \begingroup
%     \color#1{#2}#3%
%   \endgroup
% }
% \makeatother


% dependencies: amsmath, amssymb, dsfont
% math spaces
\ifdefined\N
\renewcommand{\N}{\mathds{N}} % N, naturals
\else \newcommand{\N}{\mathds{N}} \fi
\newcommand{\Z}{\mathds{Z}} % Z, integers
\newcommand{\Q}{\mathds{Q}} % Q, rationals
\newcommand{\R}{\mathds{R}} % R, reals
\ifdefined\C
\renewcommand{\C}{\mathds{C}} % C, complex
\else \newcommand{\C}{\mathds{C}} \fi
\newcommand{\continuous}{\mathcal{C}} % C, space of continuous functions
\newcommand{\M}{\mathcal{M}} % machine numbers
\newcommand{\epsm}{\epsilon_m} % maximum error

% counting / finite sets
\newcommand{\setzo}{\{0, 1\}} % set 0, 1
\newcommand{\setmp}{\{-1, +1\}} % set -1, 1
\newcommand{\unitint}{[0, 1]} % unit interval

% basic math stuff
\newcommand{\xt}{\tilde x} % x tilde
\DeclareMathOperator*{\argmax}{arg\,max} % argmax
\DeclareMathOperator*{\argmin}{arg\,min} % argmin
\newcommand{\argminlim}{\mathop{\mathrm{arg\,min}}\limits} % argmax with limits
\newcommand{\argmaxlim}{\mathop{\mathrm{arg\,max}}\limits} % argmin with limits
\newcommand{\sign}{\operatorname{sign}} % sign, signum
\newcommand{\I}{\mathbb{I}} % I, indicator
\newcommand{\order}{\mathcal{O}} % O, order
\newcommand{\bigO}{\mathcal{O}} % Big-O Landau
\newcommand{\littleo}{{o}} % Little-o Landau
\newcommand{\pd}[2]{\frac{\partial{#1}}{\partial #2}} % partial derivative
\newcommand{\floorlr}[1]{\left\lfloor #1 \right\rfloor} % floor
\newcommand{\ceillr}[1]{\left\lceil #1 \right\rceil} % ceiling
\newcommand{\indep}{\perp \!\!\! \perp} % independence symbol

% sums and products
\newcommand{\sumin}{\sum\limits_{i=1}^n} % summation from i=1 to n
\newcommand{\sumim}{\sum\limits_{i=1}^m} % summation from i=1 to m
\newcommand{\sumjn}{\sum\limits_{j=1}^n} % summation from j=1 to p
\newcommand{\sumjp}{\sum\limits_{j=1}^p} % summation from j=1 to p
\newcommand{\sumik}{\sum\limits_{i=1}^k} % summation from i=1 to k
\newcommand{\sumkg}{\sum\limits_{k=1}^g} % summation from k=1 to g
\newcommand{\sumjg}{\sum\limits_{j=1}^g} % summation from j=1 to g
\newcommand{\summM}{\sum\limits_{m=1}^M} % summation from m=1 to M
\newcommand{\meanin}{\frac{1}{n} \sum\limits_{i=1}^n} % mean from i=1 to n
\newcommand{\meanim}{\frac{1}{m} \sum\limits_{i=1}^m} % mean from i=1 to n
\newcommand{\meankg}{\frac{1}{g} \sum\limits_{k=1}^g} % mean from k=1 to g
\newcommand{\meanmM}{\frac{1}{M} \sum\limits_{m=1}^M} % mean from m=1 to M
\newcommand{\prodin}{\prod\limits_{i=1}^n} % product from i=1 to n
\newcommand{\prodkg}{\prod\limits_{k=1}^g} % product from k=1 to g
\newcommand{\prodjp}{\prod\limits_{j=1}^p} % product from j=1 to p

% linear algebra
\newcommand{\one}{\bm{1}} % 1, unitvector
\newcommand{\zero}{\mathbf{0}} % 0-vector
\newcommand{\id}{\bm{I}} % I, identity
\newcommand{\diag}{\operatorname{diag}} % diag, diagonal
\newcommand{\trace}{\operatorname{tr}} % tr, trace
\newcommand{\spn}{\operatorname{span}} % span
\newcommand{\scp}[2]{\left\langle #1, #2 \right\rangle} % <.,.>, scalarproduct
\newcommand{\mat}[1]{\begin{pmatrix} #1 \end{pmatrix}} % short pmatrix command
\newcommand{\Amat}{\mathbf{A}} % matrix A
\newcommand{\Deltab}{\mathbf{\Delta}} % error term for vectors

% basic probability + stats
\renewcommand{\P}{\mathds{P}} % P, probability
\newcommand{\E}{\mathds{E}} % E, expectation
\newcommand{\var}{\mathsf{Var}} % Var, variance
\newcommand{\cov}{\mathsf{Cov}} % Cov, covariance
\newcommand{\corr}{\mathsf{Corr}} % Corr, correlation
\newcommand{\normal}{\mathcal{N}} % N of the normal distribution
\newcommand{\iid}{\overset{i.i.d}{\sim}} % dist with i.i.d superscript
\newcommand{\distas}[1]{\overset{#1}{\sim}} % ... is distributed as ...

% machine learning
\newcommand{\Xspace}{\mathcal{X}} % X, input space
\newcommand{\Yspace}{\mathcal{Y}} % Y, output space
\newcommand{\Zspace}{\mathcal{Z}} % Z, space of sampled datapoints
\newcommand{\nset}{\{1, \ldots, n\}} % set from 1 to n
\newcommand{\pset}{\{1, \ldots, p\}} % set from 1 to p
\newcommand{\gset}{\{1, \ldots, g\}} % set from 1 to g
\newcommand{\Pxy}{\mathbb{P}_{xy}} % P_xy
\newcommand{\Exy}{\mathbb{E}_{xy}} % E_xy: Expectation over random variables xy
\newcommand{\xv}{\mathbf{x}} % vector x (bold)
\newcommand{\xtil}{\tilde{\mathbf{x}}} % vector x-tilde (bold)
\newcommand{\yv}{\mathbf{y}} % vector y (bold)
\newcommand{\xy}{(\xv, y)} % observation (x, y)
\newcommand{\xvec}{\left(x_1, \ldots, x_p\right)^\top} % (x1, ..., xp)
\newcommand{\Xmat}{\mathbf{X}} % Design matrix
\newcommand{\allDatasets}{\mathds{D}} % The set of all datasets
\newcommand{\allDatasetsn}{\mathds{D}_n}  % The set of all datasets of size n
\newcommand{\D}{\mathcal{D}} % D, data
\newcommand{\Dn}{\D_n} % D_n, data of size n
\newcommand{\Dtrain}{\mathcal{D}_{\text{train}}} % D_train, training set
\newcommand{\Dtest}{\mathcal{D}_{\text{test}}} % D_test, test set
\newcommand{\xyi}[1][i]{\left(\xv^{(#1)}, y^{(#1)}\right)} % (x^i, y^i), i-th observation
\newcommand{\Dset}{\left( \xyi[1], \ldots, \xyi[n]\right)} % {(x1,y1)), ..., (xn,yn)}, data
\newcommand{\defAllDatasetsn}{(\Xspace \times \Yspace)^n} % Def. of the set of all datasets of size n
\newcommand{\defAllDatasets}{\bigcup_{n \in \N}(\Xspace \times \Yspace)^n} % Def. of the set of all datasets
\newcommand{\xdat}{\left\{ \xv^{(1)}, \ldots, \xv^{(n)}\right\}} % {x1, ..., xn}, input data
\newcommand{\ydat}{\left\{ \yv^{(1)}, \ldots, \yv^{(n)}\right\}} % {y1, ..., yn}, input data
\newcommand{\yvec}{\left(y^{(1)}, \hdots, y^{(n)}\right)^\top} % (y1, ..., yn), vector of outcomes
\newcommand{\greekxi}{\xi} % Greek letter xi
\renewcommand{\xi}[1][i]{\xv^{(#1)}} % x^i, i-th observed value of x
\newcommand{\yi}[1][i]{y^{(#1)}} % y^i, i-th observed value of y
\newcommand{\xivec}{\left(x^{(i)}_1, \ldots, x^{(i)}_p\right)^\top} % (x1^i, ..., xp^i), i-th observation vector
\newcommand{\xj}{\xv_j} % x_j, j-th feature
\newcommand{\xjvec}{\left(x^{(1)}_j, \ldots, x^{(n)}_j\right)^\top} % (x^1_j, ..., x^n_j), j-th feature vector
\newcommand{\phiv}{\mathbf{\phi}} % Basis transformation function phi
\newcommand{\phixi}{\mathbf{\phi}^{(i)}} % Basis transformation of xi: phi^i := phi(xi)

%%%%%% ml - models general
\newcommand{\lamv}{\bm{\lambda}} % lambda vector, hyperconfiguration vector
\newcommand{\Lam}{\bm{\Lambda}}	 % Lambda, space of all hpos
% Inducer / Inducing algorithm
\newcommand{\preimageInducer}{\left(\defAllDatasets\right)\times\Lam} % Set of all datasets times the hyperparameter space
\newcommand{\preimageInducerShort}{\allDatasets\times\Lam} % Set of all datasets times the hyperparameter space
% Inducer / Inducing algorithm
\newcommand{\ind}{\mathcal{I}} % Inducer, inducing algorithm, learning algorithm

% continuous prediction function f
\newcommand{\ftrue}{f_{\text{true}}}  % True underlying function (if a statistical model is assumed)
\newcommand{\ftruex}{\ftrue(\xv)} % True underlying function (if a statistical model is assumed)
\newcommand{\fx}{f(\xv)} % f(x), continuous prediction function
\newcommand{\fdomains}{f: \Xspace \rightarrow \R^g} % f with domain and co-domain
\newcommand{\Hspace}{\mathcal{H}} % hypothesis space where f is from
\newcommand{\fbayes}{f^{\ast}} % Bayes-optimal model
\newcommand{\fxbayes}{f^{\ast}(\xv)} % Bayes-optimal model
\newcommand{\fkx}[1][k]{f_{#1}(\xv)} % f_j(x), discriminant component function
\newcommand{\fh}{\hat{f}} % f hat, estimated prediction function
\newcommand{\fxh}{\fh(\xv)} % fhat(x)
\newcommand{\fxt}{f(\xv ~|~ \thetav)} % f(x | theta)
\newcommand{\fxi}{f\left(\xv^{(i)}\right)} % f(x^(i))
\newcommand{\fxih}{\hat{f}\left(\xv^{(i)}\right)} % f(x^(i))
\newcommand{\fxit}{f\left(\xv^{(i)} ~|~ \thetav\right)} % f(x^(i) | theta)
\newcommand{\fhD}{\fh_{\D}} % fhat_D, estimate of f based on D
\newcommand{\fhDtrain}{\fh_{\Dtrain}} % fhat_Dtrain, estimate of f based on D
\newcommand{\fhDnlam}{\fh_{\Dn, \lamv}} %model learned on Dn with hp lambda
\newcommand{\fhDlam}{\fh_{\D, \lamv}} %model learned on D with hp lambda
\newcommand{\fhDnlams}{\fh_{\Dn, \lamv^\ast}} %model learned on Dn with optimal hp lambda
\newcommand{\fhDlams}{\fh_{\D, \lamv^\ast}} %model learned on D with optimal hp lambda

% discrete prediction function h
\newcommand{\hx}{h(\xv)} % h(x), discrete prediction function
\newcommand{\hh}{\hat{h}} % h hat
\newcommand{\hxh}{\hat{h}(\xv)} % hhat(x)
\newcommand{\hxt}{h(\xv | \thetav)} % h(x | theta)
\newcommand{\hxi}{h\left(\xi\right)} % h(x^(i))
\newcommand{\hxit}{h\left(\xi ~|~ \thetav\right)} % h(x^(i) | theta)
\newcommand{\hbayes}{h^{\ast}} % Bayes-optimal classification model
\newcommand{\hxbayes}{h^{\ast}(\xv)} % Bayes-optimal classification model

% yhat
\newcommand{\yh}{\hat{y}} % yhat for prediction of target
\newcommand{\yih}{\hat{y}^{(i)}} % yhat^(i) for prediction of ith targiet
\newcommand{\resi}{\yi- \yih}

% theta
\newcommand{\thetah}{\hat{\theta}} % theta hat
\newcommand{\thetav}{\bm{\theta}} % theta vector
\newcommand{\thetavh}{\bm{\hat\theta}} % theta vector hat
\newcommand{\thetat}[1][t]{\thetav^{[#1]}} % theta^[t] in optimization
\newcommand{\thetatn}[1][t]{\thetav^{[#1 +1]}} % theta^[t+1] in optimization
\newcommand{\thetahDnlam}{\thetavh_{\Dn, \lamv}} %theta learned on Dn with hp lambda
\newcommand{\thetahDlam}{\thetavh_{\D, \lamv}} %theta learned on D with hp lambda
\newcommand{\mint}{\min_{\thetav \in \Theta}} % min problem theta
\newcommand{\argmint}{\argmin_{\thetav \in \Theta}} % argmin theta
% LS 29.10.2024 addin thetab back for now because apparently this broke and nobody updated slides to reflect thetab -> thetav changes?
\newcommand{\thetab}{\bm{\theta}} % theta vector


% densities + probabilities
% pdf of x
\newcommand{\pdf}{p} % p
\newcommand{\pdfx}{p(\xv)} % p(x)
\newcommand{\pixt}{\pi(\xv~|~ \thetav)} % pi(x|theta), pdf of x given theta
\newcommand{\pixit}[1][i]{\pi\left(\xi[#1] ~|~ \thetav\right)} % pi(x^i|theta), pdf of x given theta
\newcommand{\pixii}[1][i]{\pi\left(\xi[#1]\right)} % pi(x^i), pdf of i-th x

% pdf of (x, y)
\newcommand{\pdfxy}{p(\xv,y)} % p(x, y)
\newcommand{\pdfxyt}{p(\xv, y ~|~ \thetav)} % p(x, y | theta)
\newcommand{\pdfxyit}{p\left(\xi, \yi ~|~ \thetav\right)} % p(x^(i), y^(i) | theta)

% pdf of x given y
\newcommand{\pdfxyk}[1][k]{p(\xv | y= #1)} % p(x | y = k)
\newcommand{\lpdfxyk}[1][k]{\log p(\xv | y= #1)} % log p(x | y = k)
\newcommand{\pdfxiyk}[1][k]{p\left(\xi | y= #1 \right)} % p(x^i | y = k)

% prior probabilities
\newcommand{\pik}[1][k]{\pi_{#1}} % pi_k, prior
\newcommand{\lpik}[1][k]{\log \pi_{#1}} % log pi_k, log of the prior
\newcommand{\pit}{\pi(\thetav)} % Prior probability of parameter theta

% posterior probabilities
\newcommand{\post}{\P(y = 1 ~|~ \xv)} % P(y = 1 | x), post. prob for y=1
\newcommand{\postk}[1][k]{\P(y = #1 ~|~ \xv)} % P(y = k | y), post. prob for y=k
\newcommand{\pidomains}{\pi: \Xspace \rightarrow \unitint} % pi with domain and co-domain
\newcommand{\pibayes}{\pi^{\ast}} % Bayes-optimal classification model
\newcommand{\pixbayes}{\pi^{\ast}(\xv)} % Bayes-optimal classification model
\newcommand{\pix}{\pi(\xv)} % pi(x), P(y = 1 | x)
\newcommand{\piv}{\bm{\pi}} % pi, bold, as vector
\newcommand{\pikx}[1][k]{\pi_{#1}(\xv)} % pi_k(x), P(y = k | x)
\newcommand{\pikxt}[1][k]{\pi_{#1}(\xv ~|~ \thetav)} % pi_k(x | theta), P(y = k | x, theta)
\newcommand{\pixh}{\hat \pi(\xv)} % pi(x) hat, P(y = 1 | x) hat
\newcommand{\pikxh}[1][k]{\hat \pi_{#1}(\xv)} % pi_k(x) hat, P(y = k | x) hat
\newcommand{\pixih}{\hat \pi(\xi)} % pi(x^(i)) with hat
\newcommand{\pikxih}[1][k]{\hat \pi_{#1}(\xi)} % pi_k(x^(i)) with hat
\newcommand{\pdfygxt}{p(y ~|~\xv, \thetav)} % p(y | x, theta)
\newcommand{\pdfyigxit}{p\left(\yi ~|~\xi, \thetav\right)} % p(y^i |x^i, theta)
\newcommand{\lpdfygxt}{\log \pdfygxt } % log p(y | x, theta)
\newcommand{\lpdfyigxit}{\log \pdfyigxit} % log p(y^i |x^i, theta)

% probababilistic
\newcommand{\bayesrulek}[1][k]{\frac{\P(\xv | y= #1) \P(y= #1)}{\P(\xv)}} % Bayes rule
\newcommand{\muk}{\bm{\mu_k}} % mean vector of class-k Gaussian (discr analysis)

% residual and margin
\newcommand{\eps}{\epsilon} % residual, stochastic
\newcommand{\epsv}{\bm{\epsilon}} % residual, stochastic, as vector
\newcommand{\epsi}{\epsilon^{(i)}} % epsilon^i, residual, stochastic
\newcommand{\epsh}{\hat{\epsilon}} % residual, estimated
\newcommand{\epsvh}{\hat{\epsv}} % residual, estimated, vector
\newcommand{\yf}{y \fx} % y f(x), margin
\newcommand{\yfi}{\yi \fxi} % y^i f(x^i), margin
\newcommand{\Sigmah}{\hat \Sigma} % estimated covariance matrix
\newcommand{\Sigmahj}{\hat \Sigma_j} % estimated covariance matrix for the j-th class

% ml - loss, risk, likelihood
\newcommand{\Lyf}{L\left(y, f\right)} % L(y, f), loss function
\newcommand{\Lypi}{L\left(y, \pi\right)} % L(y, pi), loss function
\newcommand{\Lxy}{L\left(y, \fx\right)} % L(y, f(x)), loss function
\newcommand{\Lxyi}{L\left(\yi, \fxi\right)} % loss of observation
\newcommand{\Lxyt}{L\left(y, \fxt\right)} % loss with f parameterized
\newcommand{\Lxyit}{L\left(\yi, \fxit\right)} % loss of observation with f parameterized
\newcommand{\Lxym}{L\left(\yi, f\left(\bm{\tilde{x}}^{(i)} ~|~ \thetav\right)\right)} % loss of observation with f parameterized
\newcommand{\Lpixy}{L\left(y, \pix\right)} % loss in classification
\newcommand{\Lpiy}{L\left(y, \pi\right)} % loss in classification
\newcommand{\Lpiv}{L\left(y, \piv\right)} % loss in classification
\newcommand{\Lpixyi}{L\left(\yi, \pixii\right)} % loss of observation in classification
\newcommand{\Lpixyt}{L\left(y, \pixt\right)} % loss with pi parameterized
\newcommand{\Lpixyit}{L\left(\yi, \pixit\right)} % loss of observation with pi parameterized
\newcommand{\Lhy}{L\left(y, h\right)} % L(y, h), loss function on discrete classes
\newcommand{\Lhxy}{L\left(y, \hx\right)} % L(y, h(x)), loss function on discrete classes
\newcommand{\Lr}{L\left(r\right)} % L(r), loss defined on residual (reg) / margin (classif)
\newcommand{\lone}{|y - \fx|} % L1 loss
\newcommand{\ltwo}{\left(y - \fx\right)^2} % L2 loss
\newcommand{\lbernoullimp}{\ln(1 + \exp(-y \cdot \fx))} % Bernoulli loss for -1, +1 encoding
\newcommand{\lbernoullizo}{- y \cdot \fx + \log(1 + \exp(\fx))} % Bernoulli loss for 0, 1 encoding
\newcommand{\lcrossent}{- y \log \left(\pix\right) - (1 - y) \log \left(1 - \pix\right)} % cross-entropy loss
\newcommand{\lbrier}{\left(\pix - y \right)^2} % Brier score
\newcommand{\risk}{\mathcal{R}} % R, risk
\newcommand{\riskbayes}{\mathcal{R}^\ast}
\newcommand{\riskf}{\risk(f)} % R(f), risk
\newcommand{\riskdef}{\E_{y|\xv}\left(\Lxy \right)} % risk def (expected loss)
\newcommand{\riskt}{\mathcal{R}(\thetav)} % R(theta), risk
\newcommand{\riske}{\mathcal{R}_{\text{emp}}} % R_emp, empirical risk w/o factor 1 / n
\newcommand{\riskeb}{\bar{\mathcal{R}}_{\text{emp}}} % R_emp, empirical risk w/ factor 1 / n
\newcommand{\riskef}{\riske(f)} % R_emp(f)
\newcommand{\risket}{\mathcal{R}_{\text{emp}}(\thetav)} % R_emp(theta)
\newcommand{\riskr}{\mathcal{R}_{\text{reg}}} % R_reg, regularized risk
\newcommand{\riskrt}{\mathcal{R}_{\text{reg}}(\thetav)} % R_reg(theta)
\newcommand{\riskrf}{\riskr(f)} % R_reg(f)
\newcommand{\riskrth}{\hat{\mathcal{R}}_{\text{reg}}(\thetav)} % hat R_reg(theta)
\newcommand{\risketh}{\hat{\mathcal{R}}_{\text{emp}}(\thetav)} % hat R_emp(theta)
\newcommand{\LL}{\mathcal{L}} % L, likelihood
\newcommand{\LLt}{\mathcal{L}(\thetav)} % L(theta), likelihood
\newcommand{\LLtx}{\mathcal{L}(\thetav | \xv)} % L(theta|x), likelihood
\newcommand{\logl}{\ell} % l, log-likelihood
\newcommand{\loglt}{\logl(\thetav)} % l(theta), log-likelihood
\newcommand{\logltx}{\logl(\thetav | \xv)} % l(theta|x), log-likelihood
\newcommand{\errtrain}{\text{err}_{\text{train}}} % training error
\newcommand{\errtest}{\text{err}_{\text{test}}} % test error
\newcommand{\errexp}{\overline{\text{err}_{\text{test}}}} % avg training error

% lm
\newcommand{\thx}{\thetav^\top \xv} % linear model
\newcommand{\olsest}{(\Xmat^\top \Xmat)^{-1} \Xmat^\top \yv} % OLS estimator in LM


\title{Optimization in Machine Learning}

\begin{document}

\titlemeta{% Chunk title (example: CART, Forests, Boosting, ...), can be empty
  Evolutionary Algorithms
  }{% Lecture title  
  CMA-ES Algorithm
  }{% Relative path to title page image: Can be empty but must not start with slides/
  figure_man/cmaes/cmaes_generations.png
  }{
    \item CMA-ES strategy
    \item Estimation of distribution
    \item Step size control
}
%%%%%%%%%%%%%%%%%%%%%%%%%%%%%%%%%%%%%%%%%%%%%%%%%%%%%%%%%%%%%%%%%%%%%%%%%%%%%%%%%%%

% \section{Covariance Matrix Adaptation Evolution Strategy (CMA-ES)}

%\begin{vbframe}{CMA-ES as part of many EDAs}
%\textbf{Estimation of Distribution Algorithms} (EDAs) widely used class of algorithms designed to solve optimization problems of the form
%
%\vspace{-10pt}
%\begin{eqnarray*}
%\xv^* = \argmax_{\xv \in \mathcal{S}} f(\xv), \quad \text{where } f:\mathcal{S}\rightarrow \R.
%\end{eqnarray*}
%
%
%Instead of solving above objective directly, EDAs solve related objective:
%\begin{eqnarray*}
%\thetab^\ast = \argmax_{\thetab} \E_{p(\xv|\thetab)} f(\xv),
%\end{eqnarray*}
%\vspace{-10pt}
%
%where $p(\xv|\thetab)$ is a probability density over $\mathcal{S}$, parameterized by $p$ parameters $\thetab \in \R^p$.
%
%\lz
%
%Reason for later formulation: convenience of derivative-free optimization, enabling of rigorous analysis and leveraging of a probabilistic formulation to incorporate auxiliary information.
%\end{vbframe}
%% \framebreak
\begin{frame}{Estimation of Distribution Algorithm}

\begin{minipage}{0.62\textwidth}
\begin{itemize}
    % \item General algorithmic template
    \item Instead of population, maintain distribution to sample offspring from
\end{itemize}

\vspace{\baselineskip}

\begin{enumerate}
    \item Draw $\lambda$ offsprings $\xv^{(i)}$ from $p(\cdot|\thetab^{[t]})$
    \item Evaluate fitness $f(\xv^{(i)})$ 
    %\item Reduce to $\mu$ best offspring
    %where $W(\cdot)$ gives weights for each $\xv^{(i)}$, typically $0$ or $1$ (order-preserving fitness transformation)
    \item Update $\thetab^{[t+1]}$ with $\mu$ best offsprings
\end{enumerate}

%This core algorithm is often modified in a variety of ways to improve performance via \textbf{Covariance Matrix Adaptation (CMA-ES)}.
\end{minipage}\hfill
\begin{minipage}{0.35\textwidth}\raggedleft
\begin{figure}
  \includegraphics[width=1\textwidth, height=0.9\textheight]{figure_man/cmaes/cmaes_eda.png}
\end{figure}
\end{minipage}

\end{frame}


% \begin{vbframe}{CMA-ES}

% Covariance Matrix Adaptation Evolution Strategy (CMA-ES) is

% \begin{itemize}
% \item A state-of-the-art tool in evolutionary computation 
% \item Stochastic/randomized black box optimization algorithm
% \item For usage in continuous domain
% \item For non-linear, non-convex optimization problems
% \item Particularly effective in \enquote{hard}/ill-conditioned settings
% \end{itemize}
% \vspace{0.3cm}
% Detailed information on CMA-ES can be found in

% \begin{enumerate}
% \item Nikolaus Hansen. The CMA Evolution Strategy. 2016
% \item A. Auger, N. Hansen: Tutorial CMA-ES: Evolution Strategies and Covariance Matrix Adaptation. 2012.
% \end{enumerate}

%\end{vbframe}


\begin{vbframe}{Covariance Matrix Adaptation}

Sample distribution is multivariate Gaussian
$$
    \xv^{[t+1](i)} \sim \mathbf{m}^{[t]} + \sigma^{[t]} \normal (\bm{0}, \mathbf{C}^{[t]}) \quad \text{for } i = 1, \dots, \lambda
$$
\vspace{-20pt}
\begin{itemize}
\item $\xv^{[t+1](i)} \in \R^d$ $i$-th offspring; $\lambda \geq 2$ number of offspring
%\item $\normal(\bm{0}, \mathbf{C}^{[t]})$ is multivariate normal distribution with zero mean, covariance matrix $\mathbf{C}^{[t]}$. \textit{Note}: $\mathbf{m}^{[t]} + \sigma^{[t]} \normal(\bm{0}, \mathbf{C}^{[t]}) \sim \normal(\mathbf{m}^{[t]}, (\sigma^{[t]})^2 \mathbf{C}^{[t]})$.
\item $\mathbf{m}^{[t]} \in \R^d$ mean value and $\mathbf{C}^{[t]} \in \R^{d \times d}$ covariance matrix
\item $\sigma^{[t]} \in \R_{+}$ \enquote{overall} standard deviation/step size
% Up to the scalar factor $\sigma^{(g)^2}$, $\mathbf{C}^{(g)}$ is the covariance matrix of the search distribution.
\end{itemize}

\begin{figure}
  \includegraphics[width=0.5\textwidth]{figure_man/cmaes/cmaes_generations.png}
\end{figure}


\textbf{Question:} How to adapt $\mathbf{m}^{[t+1]}$, $\mathbf{C}^{[t+1]}$, $\sigma^{[t+1]}$ for next generation $t+1$?

\end{vbframe}


% \begin{vbframe}{Recall: Evolution Strategies (ES)}

% New search points are sampled normally distributed as perturbations of $\mathbf{m}$:
% \begin{eqnarray*}
% \xv_k \sim \mathbf{m} + \sigma \normal_k (\bm{0}, \mathbf{C}) \quad \text{for } i = 1, \dots, \lambda
% \end{eqnarray*}

% where $\xv_k$, $\mathbf{m} \in \R^{n}$, $\sigma \in \R_{+}$, $\mathbf{C} \in \R^{n \times n}$

% \begin{itemize}
% \item Mean vector $\mathbf{m} \in \R^d$ represents the favorite solution
% \item Step-size $\sigma \in \R_{+}$ controls the step length
% \item Covariance matrix $\mathbf{C} \in \R^{n \times n}$ determines the shape of the distribution ellipsoid.
% \end{itemize}

% Remaining question: How to update $\mathbf{m}$, $\mathbf{C}$ and $\sigma$?
% \end{vbframe}



% \begin{vbframe}{CMA-ES: Basic Method}
% \begin{enumerate}
% \item \textbf{Sample maximum entropy} distribution
% \item[] $x_i = m + \sigma \normal_i(\bm{0}, \mathbf{C})$ multivariate normal distribution
% \item \textbf{Ranking} solutions according to their fitness
% \item[] Invariance to order-preserving transformations
% \item \textbf{Update mean and covariance matrix} by natural gradient ascend, improving the \enquote{expected fitness} and the likelihood for good steps
% \item[] PCA $\rightarrow$ variable metric, new problem representation, invariant under changes of the coordinate system
% \item \textbf{Update step-size} based on non-local information
% \item[] Exploit correlations in the history of steps.
% \end{enumerate}
% \end{vbframe}



\begin{vbframe}{CMA-ES: Basic Method - Iteration 1}
% \begin{eqnarray*}
% \mathbf{m} \leftarrow \mathbf{m} + \sigma \yv_w, \quad \yv_w = \sum_{i=1}^{\mu} w_i \yv_{i:\lambda}, \quad \yv_i\sim \normal_i(\bm{0}, \mathbf{C})
% \end{eqnarray*}

\begin{enumerate}
    \addtocounter{enumi}{-1}
    \item Initialize $\mathbf{m}^{[0]},\sigma^{[0]}$ problem-dependent and $\mathbf{C}^{[0]}=\mathbf{I}_{d}$
        \scalebox{0.95}{\includegraphics{figure_man/cmaes/cmaes_update_1.png}}

    \framebreak

    \item \textbf{Sample} $\lambda$ offsprings from distribution
    $$\xv^{[1](i)} = \mathbf{m}^{[0]} + \sigma^{[0]} \normal(\bm{0}, \mathbf{C}^{[0]})$$
        \scalebox{0.95}{\includegraphics{figure_man/cmaes/cmaes_update_2.png}}

    \item \textbf{Selection and recombination} of $\mu<\lambda$ best-performing offspring using fixed weights $w_1\geq\ldots\geq w_{\mu}>0,\sum_{i=1}^{\mu} w_i = 1$. %solutions according to their fitness (\textit{Selection} of $\mu$ best)
    
    $\xv_{i:\lambda}$ is $i$-th ranked solution, ranked by $f(\xv_{i:\lambda})$.

        \scalebox{0.95}{\includegraphics{figure_man/cmaes/cmaes_update_21.png}}
        \vspace{-0.5cm}\\
        Calculation of auxiliary variables ($\mu=3$ points) $\yv_w^{[1]} := \sum_{i=1}^{\mu} w_i (\xv_{i:\lambda}^{[1]}-\mathbf{m}^{[0]})/\sigma^{[0]} := \sum_{i=1}^{\mu} w_i \yv_{i:\lambda}^{[1]}$%, using $\mu = 3$ points. %(high fitness $\rightarrow$ high weights)

%Movement to new population mean $\mathbf{m}^{[1]}$ (disregarding $\sigma$) of the $\mu = 3$ selected points (high fitness $\rightarrow$ high weights, $\yv_w := \sum_{i=1}^{\mu} w_i \xv_{i:\lambda}$).

    \item \textbf{Update mean}
        \scalebox{0.95}{\includegraphics{figure_man/cmaes/cmaes_update_3.png}}
        Movement towards the new distribution with mean
        
        $\mathbf{m}^{[1]} = \mathbf{m}^{[0]} + \sigma^{[0]} \yv_{w}^{[1]}$.

    \item \textbf{Update covariance matrix} %(\textit{Recombination}),
    
        Roughly: elongate density ellipsoid in direction of successful steps.
        
        $\mathbf{C}^{[1]}$ reproduces successful points with higher probability than $\mathbf{C}^{[0]}$. %Improving \enquote{expected fitness} and likelihood for successful steps.
        
        \scalebox{0.95}{\includegraphics{figure_man/cmaes/cmaes_update_4.png}}
        \vspace{-0.5cm}

        \begin{small}
            Update $\mathbf{C}^{[0]}$ using sum of outer products and parameter $c_{\mu}$: %(simplified):
            $\mathbf{C}^{[1]} = (1-c_{\mu}) \mathbf{C}^{[0]} + c_{\mu} \sum_{i=1}^{\mu} w_i \yv_{i:\lambda}^{[1]}(\yv_{i:\lambda}^{[1]})^{\top}$ (rank-$\mu$ update).
            %\yv_w^{[1]} (\yv_w^{[1]})^\top$ (Rank 1 update).
        \end{small}

    %New Distribution $\normal^{[1]}\sim (\mathbf{m}^{[1]}, \mathbf{C}^{[1]})$ (disregarding $\sigma$) \\
    %of generation $t=1$, $\lambda = 6$.
\end{enumerate}
\end{vbframe}

\begin{vbframe}{CMA-ES: Basic Method - Iteration 2}
\begin{enumerate}
    \item \textbf{Sample} from distribution for new generation
        \scalebox{0.95}{\includegraphics{figure_man/cmaes/cmaes_update_5.png}}

    \framebreak
    
    \item \textbf{Selection and recombination} of $\mu<\lambda$ best-performing offspring
\item \textbf{Update mean}
    \scalebox{0.95}{\includegraphics{figure_man/cmaes/cmaes_update_6.png}}

    \item \textbf{Update covariance matrix} %(\textit{Recombination})
        \scalebox{0.95}{\includegraphics{figure_man/cmaes/cmaes_update_7.png}}

    \item \textbf{Update step-size} exploiting correlation in history of steps.
    
        steps point in similar direction $\implies$ increase step-size
        
        steps cancel out $\implies$ decrease step-size
        
        \scalebox{0.95}{\includegraphics{figure_man/cmaes/cmaes_update_8.png}}

\end{enumerate}
\end{vbframe}

%\begin{vbframe}{Updating \MakeLowercase{$\mathbf{m}$}: The $(\mu/\mu_W, \lambda)$-ES}
%
%$(\mu/\mu_W, \lambda)$-ES marks the \textbf{E}volution \textbf{S}trategy with $\bm{\mu}$ parents, \textbf{W}eighted recombination of all $\bm{\mu}$ parents and $\bm{\lambda}$ offspring.
%
%\lz
%
%Let $\xv_{i:\lambda}$ be the $i$-th ranked solution point, the new mean vector is
%% such that $f(\xv_{1:\lambda}) \leq \dots \leq f(\xv_{\lambda:\lambda})$.
%
%\vspace{-10pt}
%\begin{eqnarray*}
%\mathbf{m}^{[t+1]} = \mathbf{m}^{[t]} + \sigma^{[t]} \underbrace{\sum_{i=1}^{\mu} w_i \xv_{i:\lambda}^{[t]}}_{=: \yv_w^{[t]}}
%\end{eqnarray*}
%
%where $w_1 \geq \dots \geq w_{\mu} > 0$, $\sum_{i=1}^{\mu} w_i = 1$ and $\frac{1}{\sum_{i=1}^{\mu} w_i^2} =: \mu_w \approx \frac{\lambda}{4}$.
%
%\lz
%
%The best $\mu$ points are selected from the new solutions (non-elitistic) and weighted intermediate recombination is applied.
%
%\lz
%
%If $w_{i=1:\mu} = 1/\mu$ then $\yv_w$ is equal to the mean of the $\mu$ best points.
%\end{vbframe}


%\begin{vbframe}{Updating \MakeLowercase{$\mathbf{m}$}: The $(\mu/\mu_W, \lambda)$-ES}
%
%Remarks on weights $w_i$
%
%\begin{itemize}
%\item $w_{i=1:\mu} \in \R_{>0}$ are positive weight coefficients for recombination
%\item Typically chosen as weighted average of $\mu$ selected points $w_{i=1:\mu} = 1/\mu$
%\item Assigning different weights $w_i$ should be interpreted as a selection mechanism
%\item Approaches exist, which give the remaining $\lambda-\mu$ points negative weights, such that $\lambda$ weights in total are used (e.g active covariance matrix adaptation)
%\item Weights depend only on the ranking, not on the function values directly $\rightarrow$ renders the algorithm invariant under order-preserving transformation of the objective function
%\end{itemize}
%\end{vbframe}

\begin{vbframe}{Updating $\mathbf{C}$: Full Update}

Full CMA update of $\mathbf{C}$ combines rank-$\mu$ update with a rank-$1$ update using exponentially smoothed evolution path $\mathbf{p}_c \in \mathbb{R}^{d}$ of successive steps and learning rate $c_1$:
$$\mathbf{p}_{c}^{[0]}=\bm{0}, \quad \mathbf{p}_{c}^{[t+1]} = (1-c_1)\mathbf{p}_{c}^{[t]} + \sqrt{\frac{c_1(2-c_1)}{\sum_{i=1}^{\mu}w_i^2}}\yv_w$$
Final update of $\mathbf{C}$ is
$$\mathbf{C}^{[t+1]}=(1-c_1-c_{\mu}{\scriptstyle\sum_j} w_j)\mathbf{C}^{[t]}+c_1 \underbrace{\mathbf{p}_{c}^{[t+1]}(\mathbf{p}_{c}^{[t+1]})^{\top}}_{\text{rank-$1$}}+c_{\mu}\underbrace{\sum_{i=1}^{\mu}w_i \yv_{i:\lambda}^{[t+1]}(\yv_{i:\lambda}^{[t+1]})^{\top}}_{\text{rank-$\mu$}}$$
\vspace{-0.4cm}
\begin{itemize}
    \item Correlation between generations used in rank-$1$ update
    \item Information from entire population is used in rank-$\mu$ update
\end{itemize}

\end{vbframe}

%\begin{vbframe}{Updating $C$: CMA - Rank-One Update}
%Initialize $\mathbf{m} \in \R^d$ and $\mathbf{C} = \bm{\I}$, set $\sigma = 1$, learning rate $c_{cov} \approx 2/d^2$. While not terminate

%\begin{align*}
%\xv^{(i)} &= \mathbf{m} + \sigma \normal_i(\bm{0}, \mathbf{C}) \\
%\mathbf{m} &\leftarrow \mathbf{m} + \sigma \yv_w, \quad \text{where } \yv_w = \sum_{i=1}^\mu \bm{w}_i(\xv_{i:\lambda}-\mathbf{m})/\sigma \\
%\mathbf{C} &\leftarrow (1-c_{cov}) \mathbf{C} + c_{cov}\mu_w \underbrace{\yv_w\yv_w^\top}_{\text{rank-one}}, \quad \text{where } \mu_w = \frac{1}{\sum_{i=1}^\mu w_i^2}
%\end{align*}

%The rank-one update was developed in several domains independently, conducting a \textbf{principle component analysis} (PCA) of steps $\yv_w$ sequentially in time and space.

% \lz

% \textit{In principle}: the adaptation increases the likelihood of successful steps $\yv_w$ to appear again.

% \textit{Different viewpoint}: the adaptation follows a natural gradient approximation of the expected fitness.

% \framebreak

% \begin{eqnarray*}
% \mathbf{C} &\leftarrow (1-c_{cov}) \mathbf{C} + c_{cov}\mu_w \yv_w\yv_w^\top
% \end{eqnarray*}

% \begin{itemize}
% \item Conducting a \textbf{principle component analysis} (PCA) of steps $\yv_w$ sequentially in time and space
% \item Approximation of the \textbf{inverse Hessian} on quadratic functions
% \item Learning of a new \textbf{rotated problem representation}
% \item[] Components only independent in the new representation
% \item Learning of all \textbf{pairwise dependencies} between variables
% \item[] Dependencies reflected by off-diagonal entries in the covariance matrix
% \item Learning of a \textbf{new} (Mahalanobis) \textbf{metric}
% \item For $\mu = 1$: Conducting a \textbf{natural gradient ascent} on the normal distribution $\normal$ (independent of the given coordinate system).
% \end{itemize}

%\end{vbframe}

%\begin{vbframe}{Updating $C$: CMA - Cumulation}
%\enquote{Cumulation} as a widely used technique and known under various names (\textit{exponential smoothing} in forecasting and time series, exponentially weighted \textit{moving average}, \textit{iterate averaging} in stochastic approximation, etc.).
%
%\lz
%
%Using cumulation / an evolution path for the rank-one update of the covariance matrix reduces the number of function evaluations to adapt to a straight ridge from about $\order(d^2)$ to $\order(d)$.
%
%\lz
%
%For the evolution/search path taken over a number of generation steps an exponentially weighted sum of steps $\yv_w$ is used:
%
%\begin{eqnarray*}
%\mathbf{p}_c \propto \sum_{t = 0}^T \underbrace{(1-c_{\mathbf{C}})^{T-i}}_{\substack{\text{exponentially} \\ \text{fading weights}}} \yv_w^{[t]}
%\end{eqnarray*}
%
%\framebreak
%
%Cumulation as \textit{recursive construction of the evolution path}:
%
%\begin{eqnarray*}
%\mathbf{p}_c^{[t+1]} = \underbrace{(1- c_{\mathbf{C}})}_{\text{decay factor}} \mathbf{p}_c^{[t]} + \underbrace{\sqrt{1-(1-c_{\mathbf{C}})^2}}_{\text{normalization factor}} \sqrt{\mu_w} \underbrace{\yv_w^{[t]}}_{input},
%\end{eqnarray*}
%
%where $\yv_{w}^{[t]} = \frac{\mathbf{m}^{[t+1]} - \mathbf{m}^{[t]}}{\sigma^{[t]}}$ and $\mu_w = \frac{1}{\sum w_i^2}$, $c_{C} << 1$. %FIXME: \ll
%History information is accumulated in the evolution path.
%
%\begin{figure}
%  \includegraphics[width=0.8\textwidth, height=0.4\textheight]{figure_man/cmaes/cmaes_cumulation_1.png}
%\end{figure}
%
%\framebreak
%
%
%$\yv_w \yv_w^\top$ was used for updating $\mathbf{C}$ and because $\yv_w \yv_w^\top = -\yv_w (-\yv_w)^\top$ the sign of $\yv_w$ is lost.
%
%\lz
%
%The \textbf{sign information} (signifying correlation between steps) is (re-)introduced by using the \textit{evolution path}.
%
%\begin{align*}
%\mathbf{p}_c^{[t+1]} &= \underbrace{(1- c_{\mathbf{C}})}_{\text{decay factor}} \mathbf{p}_{\mathbf{C}}^{[t]} + \underbrace{\sqrt{1-(1-c_{\mathbf{C}})^2}}_{\text{normalization factor}} \sqrt{\mu_w} \yv_w^{[t]} \\
%\mathbf{C}^{[t+1]} &= (1- c_{cov}) \mathbf{C}^{[t]} + c_{cov} \underbrace{\mathbf{p}_c^{[t+1]} (\mathbf{p}_c^{[t+1]})^\top}_{\text{rank-one}},
%\end{align*}
%
%where $\mu_w = \frac{1}{\sum w_i^2}, c_{cov} << c_{\mathbf{C}} << 1$, such that $1/{c_{\mathbf{C}}}$ is the \enquote{backward time horizon}. %FIXME: \ll
%
%\framebreak
%
%\begin{align*}
%\textcolor{green}{\mathbf{p}_c^{[t+1]}} &= \underbrace{(1- \textcolor{blue}{c_{\mathbf{C}}})}_{\text{decay factor}} \textcolor{green}{\mathbf{p}_{\mathbf{C}}^{[t]}} + \underbrace{\sqrt{1-(1-\textcolor{blue}{c_{\mathbf{C}}})^2}}_{\text{normalization factor}} \sqrt{\mu_w} \yv_w^{[t]} \\
%\textcolor{green}{\mathbf{C}^{[t+1]}} &= (1- \textcolor{blue}{c_{cov}})\textcolor{green}{\mathbf{C}^{[t]}} + \textcolor{blue}{c_{cov}} \underbrace{\textcolor{green}{\mathbf{p}_c^{[t+1]} (\mathbf{p}_c^{[t+1]}})^\top}_{\text{rank-one}},
%\end{align*}
%\vspace{-10pt}
%
%where $\mu_w = \frac{1}{\sum \textcolor{blue}{w_i}^2}, \textcolor{blue}{c_{cov}} << \textcolor{blue}{c_{\mathbf{C}}} << 1$, such that $1/{c_{\mathbf{C}}}$ is the \enquote{backward time horizon}. %FIXME: \ll
%
%\begin{figure}
%  \includegraphics[width=0.8\textwidth, height=0.37\textheight]{figure_man/cmaes/cmaes_cumulation_2.png}
%\end{figure}
%
%\end{vbframe}

%
%\begin{vbframe}{Updating $C$: CMA - Rank-$\mu$ Update}
%In case of \textit{large population sizes $\lambda$} the \textbf{rank-$\mu$ update} extends the update rule using $\mu > 1$ vectors to update$\mathbf{C}$ at each generation step.
%
%\lz
%
%The weighted empirical covariance matrix computes a weighted mean of the outer products of the best $\mu$ steps and has rank $\min(\mu, d)$ with probability 1.
%
%\vspace{-10pt}
%
%\begin{eqnarray*}
%\mathbf{C}_\mu^{[t+1]} = \sum_{i=1}^\mu \bm{w}_i \xv_{i:\lambda}^{[t+1]} (\xv_{i:\lambda}^{[t+1]})^\top
%\end{eqnarray*}
%
%The rank-$\mu$-update then reads
%\begin{eqnarray*}
%\mathbf{C}^{[t+1]} = (1-c_{cov}) \mathbf{C}^{[t]} + c_{cov} \mathbf{C}_{\mu}^{[t+1]},
%\end{eqnarray*}
%
%where $c_{cov} \approx \mu_w/d^2$ and $c_{cov} \leq 1$.
%
%\framebreak
%
%\begin{figure}
%  \includegraphics[width=1\textwidth, height=0.3\textheight]{figure_man/cmaes/cmaes_rankmu.png}
%\end{figure}
%
%\begin{enumerate}
%  \item Sampling $\lambda = 150$ solutions, where $\mathbf{C}^{[t]} = \bm{\I}$ and $\sigma^{[t]} = 1$.
%  \item[] $\xv^{[t+1](i)} = \mathbf{m}^{[t]} + \sigma^{[t]} \normal(\bm{0}, \mathbf{C}^{[t]})$
%  \item Calculation $\mathbf{C}$, where $\mu = 50$, $w_1 = \dots = w_{\mu} = 1/\mu$ and $c_{cov}=1$
%  \item[] $\mathbf{C}_{\mu}^{[t+1]} = 1/\mu \sum \xv_{1:\lambda}^{[t]} (\xv_{1:\lambda}^{[t]})^\top$ and $\mathbf{C}^{[t+1]} = (1-1) \times \mathbf{C}^{[t]} + 1 \times \mathbf{C}_{\mu}^{[t+1]}$
%  \item New distribution
%  \item[] $\mathbf{m}^{[t+1]} = \mathbf{m}^{[t]} + 1/\mu \sum \xv_{1:\lambda}^{[t]}$.
%\end{enumerate}
%
%\end{vbframe}
%
%
%\begin{vbframe}{Updating $C$: Rank-One and Rank-$\mu$ Update}
%\textbf{Rank-one update}
%
%\begin{itemize}
%\item Uses the evolution path
%\item Can reduce the number of \textit{function evaluations} to adapt to straight ridges from about $\order(d^2)$ to $\order(d)$.
%\end{itemize}
%
%\textbf{Rank-$\mu$ update}
%
%\begin{itemize}
%\item Increases the learning rate in large populations and therefore the primary mechanism for large populations (thumb rule: $\lambda \geq 3d + 10$)
%\item Can reduce the number of \textit{generations} from about $\order(d^2)$ to $\order(d)^{(12)}$, given $\mu_w\propto \lambda \propto d$.
%\end{itemize}
%
%
%\textbf{Hybrid Update}: rank-one and rank-$\mu$ update can be combined.
%\end{vbframe}



\begin{vbframe}{Updating $\sigma$: Methods Step-Size Control}
\begin{itemize}
\setlength\itemsep{1.0em}
\item \textbf{$1/5$-th success rule}: increases the step-size if more than 20 \% of the new solutions are successful, decrease otherwise
\item \textbf{$\sigma$-self-adaptation}: mutation is applied to the step-size and the better - according to the objective function value - is selected
\item \textbf{Path length control via cumulative step-size adaptation (CSA)}\\ Intuition:
\begin{itemize}
    \item Short cumulative step-size $\triangleq$ steps cancel $\to$ decrease $\sigma^{[t+1]}$ 
    \item Long cumulative step-size $\triangleq$ corr. steps $\to$ increase $\sigma^{[t+1]}$ 
\end{itemize}
\vspace{0.3cm}
\begin{center}
\includegraphics[width=0.7\textwidth]{figure_man/cmaes/cumulative-step-size.png}
\end{center}

%\item Alternative step-size adaptation mechanism: two-point step-size adaptation, median success rule, population success rule.
\end{itemize}
\end{vbframe}

%\begin{vbframe}{Updating $\sigma$: Path Length Control (CSA)}
%Measure the length of the evolution path with informal steps:
%\begin{itemize}
%\item perpendicular under random selection (in expectation)
%\item perpendicular in the desired solution (to be most effective)
%\end{itemize}
%
%\begin{figure}
%  \includegraphics[width=1\textwidth, height=0.33\textheight]{figure_man/cmaes/cmaes_path.png}
%\end{figure}
%
%Pathway of mean vector $\mathbf{m}$ in the generation sequence of above pictures
%\begin{enumerate}
%\item decreases $\sigma$ as single steps cancel each other off
%\item ideal case as single steps are uncorrelated
%\item increases $\sigma$ as single steps point in same direction
%\end{enumerate}
%
%\framebreak
%
%Initialize $\mathbf{m} \in \R^d$, $\sigma \in \R_+$, evolution path $p_\sigma = \bm{0}$.
%
%Set $c_\sigma \approx 4/d$, $d_\sigma \approx 1$.
%
%\begin{align*}
%\mathbf{m}^{[t+1]} &= \mathbf{m}^{[t]} + \sigma^{[t]} \yv_w^{[t]}\\
%\mathbf{p}_\sigma^{[t+1]} &= (1- c_\sigma) \mathbf{p}_\sigma^{[t]} + \underbrace{\sqrt{1-(1-c_\sigma)^2}}_{\text{accounts for } 1-c_\sigma} \underbrace{\sqrt{\mu_w}}_{\text{account for} w_i} \yv_w^{[t]} \\
%\sigma^{[t+1]} &= \sigma^{[t]} \times \underbrace{\exp\biggl( \frac{c_\sigma}{d_\sigma}\Bigl(\frac{||\mathbf{p}_\sigma^{[t+1]}||}{\E||\normal(\bm{0}, \bm{\I})||} - 1 \Bigl)\biggl)}_{>1 \Longleftrightarrow ||\mathbf{p}_\sigma|| \text{ is greater than its expectation}}
%\end{align*}
%
%\framebreak



% \begin{vbframe}{Updating $\sigma$: Path Length Control (CSA)}
% \begin{figure}
%   \includegraphics[width=1\textwidth, height=0.7\textheight]{figure_man/cmaes/cmaes_step-size.png}
% \end{figure}

% CSA effective and robust for $\lambda \leq n$.

% \end{vbframe}


% \begin{frame}{Updating $C$: CMA - Rank-One Update}
% \begin{eqnarray*}
% \mathbf{m} \leftarrow \mathbf{m} + \sigma \yv_w, \quad \yv_w = \sum_{i=1}^{\mu} w_i \yv_{i:\lambda}, \quad \yv_i\sim \normal_i(\bm{0}, \mathbf{C})
% \end{eqnarray*}

% \begin{figure}
% \begin{overprint}
% \centering
% \only<1>{\scalebox{0.6}{\includegraphics[width=1.5\textwidth, height=0.75\textheight]{figure_man/cmaes_rankone_1.png}}

% Initial distribution with $\mathbf{C} = 1$.}
% \only<2>{\scalebox{0.6}{\includegraphics[width=1.5\textwidth, height=0.75\textheight]{figure_man/cmaes_rankone_2.png}}

% Initial distribution with $\mathbf{C} = 1$.}
% \only<3>{\scalebox{0.6}{\includegraphics[width=1.5\textwidth, height=0.75\textheight]{figure_man/cmaes_rankone_3.png}}

% Movement to the population mean $m$ (disregarding $\sigma$) with $\yv_w$.}
% \only<4>{\scalebox{0.6}{\includegraphics[width=1.5\textwidth, height=0.75\textheight]{figure_man/cmaes_rankone_4.png}}

% Blue circle as a mixture of $\mathbf{C}$ and step $\yv_w$: $\mathbf{C} \leftarrow 0.8\times \mathbf{C} + 0.2 \times \yv_w \yv^\top$.}
% \only<5>{\scalebox{0.6}{\includegraphics[width=1.5\textwidth, height=0.75\textheight]{figure_man/cmaes_rankone_5.png}}

% Movement towards the new distribution (disregarding $\sigma$).}
% \only<6>{\scalebox{0.6}{\includegraphics[width=1.5\textwidth, height=0.75\textheight]{figure_man/cmaes_rankone_6.png}}

% New Distribution (disregarding $\sigma$).}
% \only<7>{\scalebox{0.6}{\includegraphics[width=1.5\textwidth, height=0.75\textheight]{figure_man/cmaes_rankone_7.png}}

% Movement to the population mean $\mathbf{m}$.}
% \only<8>{\scalebox{0.6}{\includegraphics[width=1.5\textwidth, height=0.75\textheight]{figure_man/cmaes_rankone_8.png}}

% Green circle as a mixture of $\mathbf{C}$ and step $\yv_w$: $\mathbf{C} \leftarrow 0.8\times \mathbf{C} + 0.2 \times \yv_w \yv^\top$.}
% \only<9>{\scalebox{0.6}{\includegraphics[width=1.5\textwidth, height=0.75\textheight]{figure_man/cmaes_rankone_9.png}}

% Movement towards the new distribution (disregarding $\sigma$).}
% \end{overprint}
% \end{figure}
% \end{frame}


\endlecture
\end{document}

