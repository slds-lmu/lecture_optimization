\documentclass[11pt,compress,t,notes=noshow, xcolor=table]{beamer}

\input{../../style/preamble}
\input{../../latex-math/basic-math}
\input{../../latex-math/basic-ml}


\newcommand{\titlefigure}{figure_man/1dim-ackley-func-selection.png}
\newcommand{\learninggoals}{
\item Recombination 
\item Mutation 
\item A few simple examples
}


%\usepackage{animate} % only use if you want the animation for Taylor2D

\title{Optimization in Machine Learning}
%\author{Bernd Bischl}
\date{}

\begin{document}

\lecturechapter{Evolutionary Algorithms - \\ES / Numerical Encodings}
\lecture{\inserttitle}
\sloppy
%%%%%%%%%%%%%%%%%%%%%%%%%%%%%%%%%%%%%%%%%%%%%%%%%%%%%%%%%%%%%%%%%%%%%%%%%%%%%%%%%%%

\begin{vbframe}{Recombination for numeric}

Options for recombination of two individuals $\xv, \tilde{\xv} \in \R^d$: 
\begin{itemize}
    \item \textbf{Uniform crossover}: Choose gene $j$ of parent 1 with probability $p$ and of parent 2 with probability $1-p$
    \item \textbf{Intermediate recombination}: Offspring is created from mean of two parents: $\frac{1}{2}(\xv + \tilde{\xv})$
    \item \textbf{Simulated Binary Crossover (SBX)}: generate \textbf{two offspring}
        $$
            \bar{\xv} \pm \frac{1}{2} \beta (\tilde{\xv} - \xv), \; \bar{\xv} = \frac{1}{2} (\xv + \tilde{\xv}), \; \text{$\beta \in [0,1]$ uniformly at random}
        $$
\end{itemize}

\vspace*{-0.3cm}
\begin{figure}
  \vspace*{-0.3cm}
  \includegraphics[width=0.5\textwidth]{figure_man/ea_recombination_uniform.png}  
  % https://docs.google.com/presentation/d/12v81ZaLxJUgXVUBy3VdW0y6q7OKHS_6lFcQym3yeYsg/edit#slide=id.p
  \includegraphics[width=0.3\textwidth]{figure_man/ea_recombination_numeric.pdf}
\end{figure}

\end{vbframe}

\begin{vbframe}{Mutation for numeric}
\footnotesize

\textbf{Mutation:} Individuals get modified

\medskip

\textbf{Example} for $\xv \in \R^d$:

\begin{itemize}
    \item \textbf{Uniform mutation:} Select random gene $x_j$ and replace it by uniformly distributed value (within feasible range).
    \item \textbf{Gauss mutation:} $\xv \pm \mathcal{N}(0, \sigma \mathbf{I})$
    \item \textbf{Polynomial mutation:} Use a different distribution instead of normal distribution
\end{itemize}
\begin{center}

\begin{figure}
  \includegraphics[height = 3.5cm, width = 4cm]{figure_man/polynomial_mutation.png}\\
  \scriptsize{\textbf{Source:} K. Deb, D. Deb. Analysing mutation schemes for real-parameter genetic algorithms, 2014}
\end{figure}
\end{center}

\end{vbframe}

\begin{vbframe}{Example of an evolutionary algorithm}
\small

(Simple) EA on 1-dim Ackley function on $[-30, 30]$.
Usually, for optimizing a function $f : \R^d \to \R$, individuals are encoded as real vectors $\xv \in \R^d$.

\medskip

\begin{center}
\begin{figure}
  \includegraphics[width=0.6\textwidth]{figure_man/1dim-ackley-func.png}
\end{figure}
\end{center}

\end{vbframe}

\begin{vbframe}{Example of an evolutionary algorithm}

Random initial population with size $\mu = 10$

\medskip

\begin{center}
\begin{figure}
  \includegraphics[width=0.6\textwidth]{figure_man/1dim-ackley-func-2.png}
\end{figure}
\end{center}

\end{vbframe}

% \begin{vbframe}{Example 1:}

% \begin{itemize}
% \item Let the fitness function be a 1-dim Ackley function, aiming to be optimized on the interval $[-30, 30]$. That is, $-30$ and $+30$ are the lower and upper boundaries, respectively.
% \item Consider an initial population with size $\mu=30$ and $\lambda=5$ as the number of offspring per iteration. Besides, let $\sigma=2$ represent a Gaussian mutation. We want $5$ iterations for this algorithm.
% \item Step 1: Randomly initialize the population and evaluate it by the fitness function.
% \item Step 2: As the first iterative step, choose parents by neutral selection.
% \item Step 3: As the second iterative step, choose the Gaussian mutation as variation and evaluate the fitness function.
% \item Step 4: As the final iterative step, use $(\mu + \lambda)$-selection strategy as the survival selection.
% \end{itemize}

% \framebreak

% \vspace{1cm}
% \begin{center}
% \begin{figure}
%   \includegraphics[height = 5cm, width = 8cm]{figure_man/example.png}
% \end{figure}
% \end{center}

% \end{vbframe}


\begin{vbframe}{Example 1: Ackley function}
We choose $\lambda = 5$ offsprings by neutral selection (red individuals).

\vspace{0.5cm}

\begin{center}
\begin{figure}
  \includegraphics[width=0.6\textwidth]{figure_man/1dim-ackley-func-neutral-selec.png}
\end{figure}
\end{center}

\end{vbframe}

\begin{vbframe}{Example 1: Ackley function}

%We use a Gaussian mutation with $\sigma = 2$ and don't apply a recombination.
Use Gaussian mutation with $\sigma=2$, but without recombination.

\medskip

\begin{center}
\begin{figure}
  \includegraphics[width=0.6\textwidth]{figure_man/1dim-ackley-func-gaussian-mutation.png}
\end{figure}
\end{center}

\end{vbframe}


\begin{vbframe}{Example 1: Ackley function}
Use $(\mu + \lambda)$ selection.
Selected individuals are marked in green.

\medskip

\begin{center}
\begin{figure}
  \includegraphics[width=0.6\textwidth]{figure_man/1dim-ackley-func-selection.png}
\end{figure}
\end{center}


\end{vbframe}

\begin{vbframe}{Example 1: Ackley function}

After $50$ iterations: 

\vspace{0.5cm}

\begin{center}
\begin{figure}
  \includegraphics[width=1\textwidth]{figure_man/1dim-ackley-func-final.png}
\end{figure}
\end{center}

\end{vbframe}

\begin{vbframe}{Example 2: Grid of balls}

Consider a grid in which $n$ balls with random radius are placed.
\begin{center}
\begin{figure}
  \includegraphics[height = 4.25cm, width = 5.25cm]{figure_man/grid.png}
\end{figure}
\end{center}


\textbf{Aim:} Find the circle with the largest possible radius in the grid that does \textbf{not} intersect with the other existing circles.

\begin{itemize}
\item What is the fitness function?
\item How is the population defined?
\end{itemize}

Implementation: \url{https://juliambr.shinyapps.io/balls/}

\framebreak

In our example, the chromosome of an individual is the center of a circle, so the chromosomes are encoded as 2-dimensional real vectors $\xv = (x_1, x_2) \in \R^2$.

\lz

The population $P \subset \R^2$ is given as a set of circle centers.

\lz

The fitness function evaluates an individual $\xv \in P$ based on the distance to the nearest neighboring gray circle $k$.

$$
f(\xv) = \min_{k \in \text{Grid}} \text{distance} (k, \xv),
$$

where the distance is defined as $0$ if a circle center is within the radius of a circle of the grid.

This function is to be maximized: we are looking for the largest circle that does not touch any of the gray circles.

\end{vbframe}



\endlecture
\end{document}