\documentclass[11pt,compress,t,notes=noshow, xcolor=table]{beamer}

\input{../../style/preamble}
\input{../../latex-math/basic-math}
\input{../../latex-math/basic-ml}

% sets
\newcommand{\CC}[1]{\mathcal{C}^{#1}}         % class C^k
\newcommand{\CCinf}{\mathcal{C}^{\infty}}     % class C^infty


% objectives, domains, points
\renewcommand{\S}{\mathcal{S}}              % domain of f
\newcommand{\xv}{\bm{x}}                  % vector x (bold)
\newcommand{\yv}{\bm{y}}                  % vector y (bold)
\newcommand{\vv}{\bm{v}}                  % vector v (bold)
\newcommand{\wv}{\bm{w}}                  % vector w (bold)
\newcommand{\xvs}{\bm{x}^\ast}            % x-star,  theoretical optimum
\newcommand{\xs}{x^\ast}                  % x-star,  theoretical optimum
\newcommand{\fxvs}{f(\xvs)}               % f at x-star, theoretical optimum
\renewcommand{\H}{\mathbf{H}}                     % Hessian matrix



% some other stuff
\providecommand{\eps}{\epsilon}               % epsilon
\newcommand{\A}{\mathbf{A}}                     % Hessian matrix
\newcommand{\bv}{\mathbf{b}}                     % Hessian matrix
\newcommand{\zv}{\mathbf{z}}                     % Hessian matrix
\newcommand{\V}{\mathbf{V}}                     % Hessian matrix


\newcommand{\vvmax}{\bm{v}_\text{max}}                  % vector v (bold)
\newcommand{\vvmin}{\bm{v}_\text{min}}                  % vector v (bold)
\newcommand{\lammax}{\lambda_\text{max}}                  % vector v (bold)
\newcommand{\lammin}{\lambda_\text{min}}                  % vector v (bold)

\newcommand{\sumid}{\sum\limits_{i=1}^d} % summation from i=1 to n


\title{Optimization in Machine Learning}

\begin{document}

\titlemeta{
Evolutionary Algorithms
}{
ES / Numerical Encodings
}{
figure_man/1dim-ackley-func-selection.png
}{
\item Recombination 
\item Mutation 
\item A few simple examples
}

\begin{framev}{Recombination for numeric}
Options for recombination of two individuals $\xv, \tilde{\xv} \in \R^d$: 
\begin{itemize}
\item \textbf{Uniform crossover}: Choose gene $j$ of parent 1 with probability $p$ and of parent 2 with probability $1-p$
\item \textbf{Intermediate recombination}: Offspring is created from mean of two parents: $\frac{1}{2}(\xv + \tilde{\xv})$
\item \textbf{Simulated Binary Crossover (SBX)}: generate \textbf{two offspring}
$$
\bar{\xv} \pm \frac{1}{2} \beta (\tilde{\xv} - \xv), \; \bar{\xv} = \frac{1}{2} (\xv + \tilde{\xv}), \; \text{$\beta \in \unitint$ uniformly at random}
$$
\end{itemize}
\vfill
\splitV[0.62]{
\imageC[1]{figure_man/ea_recombination_uniform.png} % https://docs.google.com/presentation/d/12v81ZaLxJUgXVUBy3VdW0y6q7OKHS_6lFcQym3yeYsg/edit#slide=id.p
}{
\imageC[1]{figure_man/ea_recombination_numeric.pdf}
}
\end{framev}


\begin{framev}[fs=footnotesize]{Mutation for numeric}
\textbf{Mutation:} Individuals get modified
\medskip
\textbf{Example} for $\xv \in \R^d$:
\begin{itemize}
\item \textbf{Uniform mutation:} Select random gene $x_j$ and replace it by uniformly distributed value (within feasible range).
\item \textbf{Gauss mutation:} $\xv \pm \normal(0, \sigma \mathbf{I})$
\item \textbf{Polynomial mutation:} Use a different distribution instead of normal distribution
\end{itemize}
\vfill
\imageC[0.45]{figure_man/polynomial_mutation.png}
\scriptsize{\textbf{Source:} K. Deb, D. Deb. Analysing mutation schemes for real-parameter genetic algorithms, 2014}
\end{framev}


\begin{framev}[fs=small]{Example of an evolutionary algorithm}
(Simple) EA on 1-dim Ackley function on $[-30, 30]$.
Usually, for optimizing a function $f : \R^d \to \R$, individuals are encoded as real vectors $\xv \in \R^d$.
\medskip
\vfill
\imageC[0.6]{figure_man/1dim-ackley-func.png}
\end{framev}


\begin{framev}{Example of an evolutionary algorithm}
Random initial population with size $\mu = 10$
\medskip
\vfill
\imageC[0.6]{figure_man/1dim-ackley-func-2.png}
\end{framev}


\begin{framev}{Example 1: Ackley function}
We choose $\lambda = 5$ offspring by neutral selection (red individuals).
\vfill
\imageC[0.6]{figure_man/1dim-ackley-func-neutral-selec.png}
\end{framev}


\begin{framev}{Example 1: Ackley function}
Use Gaussian mutation with $\sigma=2$, but without recombination.
\medskip
\vfill
\imageC[0.6]{figure_man/1dim-ackley-func-gaussian-mutation.png}
\end{framev}


\begin{framev}{Example 1: Ackley function}
Use $(\mu + \lambda)$ selection.
Selected individuals are marked in green.
\medskip
\vfill
\imageC[0.6]{figure_man/1dim-ackley-func-selection.png}
\end{framev}


\begin{framev}{Example 1: Ackley function}
After $50$ iterations: 
\vfill
\imageC{figure_man/1dim-ackley-func-final.png}
\end{framev}


\begin{framev}{Example 2: Grid of balls}
Consider a grid in which $n$ balls with random radius are placed.
\vfill
\imageC[0.55]{figure_man/grid.png}
\vfill
\textbf{Aim:} Find the circle with the largest possible radius in the grid that does \textbf{not} intersect with the other existing circles.
\begin{itemize}
\item What is the fitness function?
\item How is the population defined?
\end{itemize}
Implementation: \url{https://juliambr.shinyapps.io/balls/}
\end{framev}


\begin{framev}{Example 2: Grid of balls}
In our example, the chromosome of an individual is the center of a circle, so the chromosomes are encoded as 2-dimensional real vectors $\xv = (x_1, x_2) \in \R^2$.
\lz
The population $P \subset \R^2$ is given as a set of circle centers.
\lz
The fitness function evaluates an individual $\xv \in P$ based on the distance to the nearest neighboring gray circle $k$.
$$
f(\xv) = \min_{k \in \text{Grid}} \text{distance} (k, \xv),
$$
where the distance is defined as $0$ if a circle center is within the radius of a circle of the grid.
This function is to be maximized: we are looking for the largest circle that does not touch any of the gray circles.
\end{framev}

\endlecture
\end{document}
