\documentclass[11pt,compress,t,notes=noshow, xcolor=table]{beamer}

\input{../../style/preamble}
\input{../../latex-math/basic-math}
\input{../../latex-math/basic-ml}


\newcommand{\titlefigure}{figure_man/1dim-ackley-func-selection.png}
\newcommand{\learninggoals}{
\item Recombination operators
\item Mutation operators
}


%\usepackage{animate} % only use if you want the animation for Taylor2D

\title{Optimization in Machine Learning}
%\author{Bernd Bischl}
\date{}

\begin{document}

\lecturechapter{Evolutionary Algorithms - Numeric}
\lecture{Optimization in Machine Learning}
\sloppy
%%%%%%%%%%%%%%%%%%%%%%%%%%%%%%%%%%%%%%%%%%%%%%%%%%%%%%%%%%%%%%%%%%%%%%%%%%%%%%%%%%%

\begin{vbframe}{Recombination for numeric}

Options for recombination of two individuals $\bm{x}, \bm{\tilde x} \in \R^d$: 
\begin{itemize}
\item \textbf{Uniform crossover}: choose gene $j$ with probability $p$ of 1st parent and probability $1-p$ of 2nd parent.
\item \textbf{Intermediate recombination}: new individual is created from the mean value of two parents $\frac{1}{2}(\bm{x} + \bm{\tilde x})$.
\item \textbf{Simulated Binary Crossover (SBX)}: generate \textbf{two offspring}

$$
\bm{\bar x} \pm \frac{1}{2} \beta (\bm{\tilde x} - \bm{x}), \bm{\bar x} = \frac{1}{2} (\bm{x} + \bm{\tilde x}), \beta \text{ randomly sampled}
$$
\end{itemize}

\vspace*{-0.3cm}
\begin{figure}
  \vspace*{-0.3cm}
  \includegraphics[width=0.5\textwidth]{figure_man/ea_recombination_uniform.png}  
  % https://docs.google.com/presentation/d/12v81ZaLxJUgXVUBy3VdW0y6q7OKHS_6lFcQym3yeYsg/edit#slide=id.p
  \includegraphics[width=0.3\textwidth]{figure_man/ea_recombination_numeric.pdf}
\end{figure}

\end{vbframe}

\begin{vbframe}{Mutation for numeric}

\footnotesize
\textbf{Mutation:} individuals are changed, for example for $\bm{x} \in \R^d$
\begin{itemize}
\item \textbf{Uniform mutation:} choose a random gene $\bm{x}_j$ and replace it with a value uniformly distributed (within the feasible range).
\item \textbf{Gauss mutation:} $\bm{\tilde{x}} = \bm{x} \pm \sigma \mathcal{N}(0, \boldsymbol{I})$
\item \textbf{Polynomial mutation:} polynomial distribution instead of normal distribution
\end{itemize}
\begin{center}
\begin{figure}
  \includegraphics[height = 3.5cm, width = 4cm]{figure_man/polynomial_mutation.png}\\
  \scriptsize{Source: K. Deb, Analysing mutation schemes for real-parameter genetic algorithms, 2014}
\end{figure}
\end{center}

\end{vbframe}





\begin{vbframe}{Example of an evolutionary algorithm}
\footnotesize
In the following, a (simple) EA is shown on the 1-dim Ackley function, optimized on $[-30, 30]$. Usually, for the optimization of a function $f:\R^d \to \R$ individuals are coded as real vectors $\bm{x}_j \in \R^d$, so here we use simply one real number as a chromosome.

\begin{center}
\begin{figure}
  \includegraphics[width=0.6\textwidth]{figure_man/1dim-ackley-func.png}
\end{figure}
\end{center}

\end{vbframe}

\begin{vbframe}{Example of an evolutionary algorithm}

Randomly init population with size $\mu = 10$.
\vspace{0.5cm}

\begin{center}
\begin{figure}
  \includegraphics[width=0.6\textwidth]{figure_man/1dim-ackley-func-2.png}
\end{figure}
\end{center}

\end{vbframe}

% \begin{vbframe}{Example 1:}

% \begin{itemize}
% \item Let the fitness function be a 1-dim Ackley function, aiming to be optimized on the interval $[-30, 30]$. That is, $-30$ and $+30$ are the lower and upper boundaries, respectively.
% \item Consider an initial population with size $\mu=30$ and $\lambda=5$ as the number of offspring per iteration. Besides, let $\sigma=2$ represent a Gaussian mutation. We want $5$ iterations for this algorithm.
% \item Step 1: Randomly initialize the population and evaluate it by the fitness function.
% \item Step 2: As the first iterative step, choose parents by neutral selection.
% \item Step 3: As the second iterative step, choose the Gaussian mutation as variation and evaluate the fitness function.
% \item Step 4: As the final iterative step, use $(\mu + \lambda)$-selection strategy as the survival selection.
% \end{itemize}

% \framebreak

% \vspace{1cm}
% \begin{center}
% \begin{figure}
%   \includegraphics[height = 5cm, width = 8cm]{figure_man/example.png}
% \end{figure}
% \end{center}

% \end{vbframe}


\begin{vbframe}{Example 1: Ackley function}
We choose $\lambda = 5$ offspring by neutral selection (red individuals).

\vspace{0.5cm}

\begin{center}
\begin{figure}
  \includegraphics[width=0.6\textwidth]{figure_man/1dim-ackley-func-neutral-selec.png}
\end{figure}
\end{center}

\end{vbframe}

\begin{vbframe}{Example 1: Ackley function}

%We use a Gaussian mutation with $\sigma = 2$ and don't apply a recombination.
Using a Gaussian mutation with $\sigma=2$, without a recombination.
\vspace{0.5cm}

\begin{center}
\begin{figure}
  \includegraphics[width=0.6\textwidth]{figure_man/1dim-ackley-func-gaussian-mutation.png}
\end{figure}
\end{center}

\end{vbframe}


\begin{vbframe}{Example 1: Ackley function}
We use a $(\mu + \lambda)$ selection. The selected individuals are green.

\vspace{0.5cm}

\begin{center}
\begin{figure}
  \includegraphics[width=0.6\textwidth]{figure_man/1dim-ackley-func-selection.png}
\end{figure}
\end{center}


\end{vbframe}

\begin{vbframe}{Example 1: Ackley function}

After $50$ iterations: 

\vspace{0.5cm}

\begin{center}
\begin{figure}
  \includegraphics[width=1\textwidth]{figure_man/1dim-ackley-func-final.png}
\end{figure}
\end{center}


\end{vbframe}

\begin{vbframe}{Example 2: Grid of balls}

Consider a grid in which $n$ balls with random radius are placed.
\begin{center}
\begin{figure}
  \includegraphics[height = 4.25cm, width = 5.25cm]{figure_man/grid.png}
\end{figure}
\end{center}


\textbf{Aim:} Find the circle with the largest possible radius in the grid that does \textbf{not} intersect with the other existing circles.

\begin{itemize}
\item What is the fitness function?
\item How is the population defined?
\end{itemize}

Implementation: \url{https://juliambr.shinyapps.io/balls/}

\framebreak

In our example, the chromosome of an individual is the center of a circle, so the chromosomes are encoded as 2-dimensional real vectors $\bm{x} = (x_1, x_2) \in \R^2$.

\lz

The population $P \subset \R^2$ is given as a set of circle centers.

\lz

The fitness function evaluates an individual $\bm{x} \in P$ based on the distance to the nearest neighboring gray circle $k$.

$$
f(\bm{x}) = \min_{k \in \text{Grid}} \text{distance} (k, \bm{x}),
$$

where the distance is defined as $0$ if a circle center is within the radius of a circle of the grid.

This function is to be maximized: we are looking for the largest circle that does not touch any of the gray circles.

\end{vbframe}



\endlecture
\end{document}