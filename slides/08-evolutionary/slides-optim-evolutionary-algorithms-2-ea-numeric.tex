\documentclass[11pt,compress,t,notes=noshow, xcolor=table]{beamer}

\input{../../style/preamble}
\input{../../latex-math/basic-math}
\input{../../latex-math/basic-ml}


\newcommand{\titlefigure}{figure_man/selection.png}
\newcommand{\learninggoals}{
\item Recombination for Numeric
\item Mutation for Numeric
\item Example
}


%\usepackage{animate} % only use if you want the animation for Taylor2D

\title{Optimization in Machine Learning}
%\author{Bernd Bischl}
\date{}

\begin{document}

\lecturechapter{Evolutionary Algorithms - Numeric}
\lecture{Optimization in Machine Learning}
\sloppy
%%%%%%%%%%%%%%%%%%%%%%%%%%%%%%%%%%%%%%%%%%%%%%%%%%%%%%%%%%%%%%%%%%%%%%%%%%%%%%%%%%%

\begin{vbframe}{Recombination for numeric}

Options for recombination of two individuals $\bm{x}, \bm{\tilde x} \in \R^d$: 
\begin{itemize}
\item \textbf{Uniform crossover}: choose gene $j$ with probability $p$ of 1st parent and probability $1-p$ of 2nd parent.
\item \textbf{Intermediate recombination}: new individual is created from the mean value of two parents $\frac{1}{2}(\bm{x} + \bm{\tilde x})$.
\item \textbf{Simulated Binary Crossover (SBX)}: generate \textbf{two offspring}

$$
\bm{\bar x} \pm \frac{1}{2} \beta (\bm{\tilde x} - \bm{x}), \bm{\bar x} = \frac{1}{2} (\bm{x} + \bm{\tilde x}), \beta \text{ randomly sampled}
$$
\end{itemize}

\vspace*{-0.3cm}
\begin{figure}
  \includegraphics[width=0.3\textwidth]{figure_man/ea_recombination_numeric.pdf}
\end{figure}

\end{vbframe}

\begin{vbframe}{Mutation for numeric}

\footnotesize
\textbf{Mutation:} individuals are changed, for example for $\bm{x} \in \R^d$
\begin{itemize}
\item \textbf{Uniform mutation:} choose a random gene $\bm{x}_j$ and replace it with a value uniformly distributed (within the feasible range).
\item \textbf{Gauss mutation:} $\bm{\tilde{x}} = \bm{x} \pm \sigma \mathcal{N}(0, \boldsymbol{I})$
\item \textbf{Polynomial mutation:} polynomial distribution instead of normal distribution
\end{itemize}
\begin{center}
\begin{figure}
  \includegraphics[height = 3.5cm, width = 4cm]{figure_man/polynomial_mutation.png}\\
  \scriptsize{Source: K. Deb, Analysing mutation schemes for real-parameter genetic algorithms, 2014}
\end{figure}
\end{center}

\end{vbframe}





\begin{vbframe}{Example of an evolutionary algorithm}
\footnotesize
In the following, a (simple) EA is shown on the 1-dim Ackley function, optimized on $[-30, 30]$. Usually, for the optimization of a function $f:\R^d \to \R$ individuals are coded as real vectors $\bm{x}_j \in \R^d$, so here we use simply one real number as a chromosome.

\begin{center}
\begin{figure}
  \includegraphics[height = 6cm, width = 7cm]{figure_man/1dim-ackley-func.png}
\end{figure}
\end{center}

\end{vbframe}

\begin{vbframe}{Example of an evolutionary algorithm}

Randomly init population with size $\mu = 20$.
\vspace{0.5cm}

\begin{center}
\begin{figure}
  \includegraphics[height = 6cm, width = 7cm]{figure_man/1dim-ackley-func-2.png}
\end{figure}
\end{center}

\end{vbframe}

\begin{vbframe}{Example of an evolutionary algorithm}
We choose $\lambda = 5$ offspring by neutral selection (red individuals).

\vspace{0.5cm}

\begin{center}
\begin{figure}
  \includegraphics[height = 6cm, width = 7cm]{figure_man/neutral-selec.png}
\end{figure}
\end{center}

\end{vbframe}


\begin{vbframe}{Example of an evolutionary algorithm}
%We use a Gaussian mutation with $\sigma = 2$ and don't apply a recombination.
Using a Gaussian mutation with $\sigma=2$, without a recombination.
\vspace{0.5cm}

\begin{center}
\begin{figure}
  \includegraphics[height = 6cm, width = 7cm]{figure_man/gaussian-mutation.png}
\end{figure}
\end{center}

\end{vbframe}


\begin{vbframe}{Example of an evolutionary algorithm}
We use a $(\mu + \lambda)$ selection. The selected individuals are green.

\vspace{0.5cm}

\begin{center}
\begin{figure}
  \includegraphics[height = 6cm, width = 7cm]{figure_man/selection.png}
\end{figure}
\end{center}


\end{vbframe}



\endlecture
\end{document}