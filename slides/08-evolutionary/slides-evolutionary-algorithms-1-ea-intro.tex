\documentclass[11pt,compress,t,notes=noshow, xcolor=table]{beamer}

\input{../../style/preamble}
\input{../../latex-math/basic-math}
\input{../../latex-math/basic-ml}

% sets
\newcommand{\CC}[1]{\mathcal{C}^{#1}}         % class C^k
\newcommand{\CCinf}{\mathcal{C}^{\infty}}     % class C^infty


% objectives, domains, points
\renewcommand{\S}{\mathcal{S}}              % domain of f
\newcommand{\xv}{\bm{x}}                  % vector x (bold)
\newcommand{\yv}{\bm{y}}                  % vector y (bold)
\newcommand{\vv}{\bm{v}}                  % vector v (bold)
\newcommand{\wv}{\bm{w}}                  % vector w (bold)
\newcommand{\xvs}{\bm{x}^\ast}            % x-star,  theoretical optimum
\newcommand{\xs}{x^\ast}                  % x-star,  theoretical optimum
\newcommand{\fxvs}{f(\xvs)}               % f at x-star, theoretical optimum
\renewcommand{\H}{\mathbf{H}}                     % Hessian matrix



% some other stuff
\providecommand{\eps}{\epsilon}               % epsilon
\newcommand{\A}{\mathbf{A}}                     % Hessian matrix
\newcommand{\bv}{\mathbf{b}}                     % Hessian matrix
\newcommand{\zv}{\mathbf{z}}                     % Hessian matrix
\newcommand{\V}{\mathbf{V}}                     % Hessian matrix


\newcommand{\vvmax}{\bm{v}_\text{max}}                  % vector v (bold)
\newcommand{\vvmin}{\bm{v}_\text{min}}                  % vector v (bold)
\newcommand{\lammax}{\lambda_\text{max}}                  % vector v (bold)
\newcommand{\lammin}{\lambda_\text{min}}                  % vector v (bold)

\newcommand{\sumid}{\sum\limits_{i=1}^d} % summation from i=1 to n


\title{Optimization in Machine Learning}

\begin{document}

\titlemeta{
Evolutionary Algorithms
}{
Introduction
}{
figure_man/ea.eps
}{
\item Evolutionary algorithms
\item Encoding
\item Parent selection, variation, survival selection
}

\begin{framev}{Evolutionary Algorithms}
\textbf{Evolutionary algorithms} (EA) are a class of stochastic, metaheuristic optimization techniques whose mode of operation is inspired by the evolution of natural organisms.
\spacer
{\footnotesize
History of evolutionary algorithms:
\begin{itemize}
\item \textbf{Genetic algorithms}: Use binary problem representation, therefore closest to the biological model of evolution.
\item \textbf{Evolution strategies}: Use direct problem representation, e.g., vector of real numbers.
\item \textbf{Genetic programming}: Create structures that convert an input into a fixed output (e.g. computer programs); solution candidates are represented as trees.
\item \textbf{Evolutionary programming}: Similar to genetic programming, but solution candidates are not represented by trees, but by finite state machines.
\end{itemize}
The boundaries between the terms become increasingly blurred and are often used synonymously.
}
\end{framev}


\begin{framev}{Structure of an evolutionary algorithm}
\imageC[0.9]{figure_man/ea.eps}
\end{framev}


\begin{framev}[fs=small]{Notation and Terminology}
\begin{itemize}
\item A chromosome is a set of parameters which encodes a proposed solution to the problem that the genetic algorithm is trying to solve. The chromosome is often represented as a binary string, although a wide variety of other data structures are also used.\\
\item The set of all solutions is known as the population.
\end{itemize}
\spacer
\begin{center}
\begin{tabular}{c|c}
\textbf{Symbols} & \textbf{EA Terminology} \\
\hline\hline \\
solution candidate $\xv \in \S$ & chromosome of an individual \\ [0.2em]
$x_j$  & $j$-th gene of chromosome \\[0.2em]
set of candidates $P$ with $\mu = |P|$ & population and size \\[0.2em]
$\lambda$ & number of generated offsprings \\[0.2em]
$f: \S \to \R$ & fitness function
\end{tabular}
\end{center}
\textbf{Note}: Unintuitively, we are minimizing fitness because we always minimize $f$ by convention.
\end{framev}


\begin{framev}{Encoding}
Encoding of chromosomes is the first step of solving a problem with EAs.
Technically: Mapping from \textbf{genotype} to \textbf{phenotype}.
Encoding depends on the problem, and eventually decides performance of problem solving.
\textbf{Encoding methods}:
\begin{itemize}
\item Binary encoding: Strings of 0s and 1s
\item Real value encoding: Real values
\end{itemize}
\imageC[0.5]{figure_man/encoding_binary_numeric.png}% https://docs.google.com/presentation/d/12v81ZaLxJUgXVUBy3VdW0y6q7OKHS_6lFcQym3yeYsg/edit#slide=id.p
\end{framev}


\begin{framev}{Encoding}
\begin{itemize}
\item Tree encoding: Tree objects
\end{itemize}
\imageC[0.5]{figure_man/encoding_tree_1.png}
\imageC[0.5][https://www.scribd.com/document/430241958/genetic-algorithm]{figure_man/encoding_tree_2.png}
\begin{small}
Floor planning problem. Given are $n$ circuits of different area requirements. Goal: arrange them into a floor layout so that all circuits are placed in a minimum layout. Each solution candidate can be represented by a tree.
\end{small}
\end{framev}


\begin{framei}{Step 1: Initialize population}
\item Evolutionary algorithms start with generating initial population $P = \{\xv^{(1)}, ..., \xv^{(\mu)}\}$.
\item Usually: Initialize uniformly at random.
\item Introducing prior knowledge possible.
\item Population is evaluated: objective function is computed for each initial individual.
\item Initialization influences quality of solution, so many EAs employ \textit{restarts} with new randomly generated initial populations.
\end{framei}


\begin{framev}[fs=small]{Step 2: parent selection}
Choose a number of $\lambda$ parents pairs creating $\lambda$ offsprings.
\begin{itemize}
\item \textbf{Neutral selection:} Draw parents uniformly at random.
\item \textbf{Fitness-proportional / Roulette wheel selection:} Draw individuals with probability proportional to their fitness.
\item \textbf{Tournament selection:} Randomly select $k$ individuals for a "tournament group" and pick the best one (according to fitness value).
\end{itemize}
\vfill
\splitVCC[0.35]{
\imageC[1]{figure_man/ea_parent_selection.pdf}
}{
\imageC[1]{figure_man/tournament_selection.png}
}
\begin{footnotesize}\begin{center}
\textbf{Left:} Fitness-proportional selection.
Fitness values of $\mu = 10$ individuals are converted into probabilities.
\textbf{Right:} Tournament selection.
\end{center}\end{footnotesize}
\end{framev}


\begin{framev}{Step 3: variation}
New individuals (offsprings) are generated from parents.
\begin{itemize}
\item Recombination/Crossover: Combine two parents into offspring.
\item Mutation: Modify the offspring locally.
\end{itemize}
Sometimes only one of both operations is performed.
\spacer
\imageC[0.9]{figure_man/rec-and-mut.png}
\textbf{Note: } Particular operation depends on encoding.
Examples for binary and numeric encodings follow later.
\end{framev}


\begin{framev}{Step 4: Survival selection}
Choosing surviving individuals.
Two common strategies are:
\spacer
\begin{itemizeL}
\item \textbf{$(\mu, \lambda)$-selection:}
Select $\mu$ best individuals \textit{only from set of offsprings} ($\lambda \ge \mu$ necessary).
\textbf{But:} Best individual can get lost!
\item \textbf{$(\mu + \lambda)$-selection:}
Select $\mu$ best individuals from set of $\mu$ parents and $\lambda$ offsprings
\textbf{Now:} Best individual certainly survives.
\end{itemizeL}
\end{framev}


\begin{framei}{Evolutionary Algorithms}
\item \textbf{Advantages}
\item Simple but enough to solve complex problems
\item All parameter types possible in general
\item Highly parallelizable
\item Flexible through different variation operations
\spacer
\item \textbf{Disadvantages}
\item Little mathematical rigor (for realistic, complex EAs)
\item Hard to find balance between exploration and exploitation
\item Quite some parameters, hard to determine them
\item Customization necessary for complex problems
\item Not suitable for expensive problems like HPO as large number of function evaluations necessary
\end{framei}

\endlecture
\end{document}
