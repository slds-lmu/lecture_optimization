\documentclass[11pt,compress,t,notes=noshow, xcolor=table]{beamer}

\input{../../style/preamble}
\input{../../latex-math/basic-math}
\input{../../latex-math/basic-ml}
\input{../../latex-math/ml-mbo}

\title{Optimization in Machine Learning}

\begin{document}

\titlemeta{
Bayesian Optimization
}{
Important Surrogate Models
}{
figure_man/surrogate_1.png
}{
\item Search space / input data peculiarities in black box problems
\item Gaussian process 
\item Random forest
}

\begin{framev}{Surrogate Models}
Desiderata:

\begin{itemize}
\item Regression model (there are also classification approaches)
\item Non-linear local model
\item Accurate predictions (especially for small sample sizes)
\item Often: uncertainty estimates
\item Robust, works often well without human modeler intervention
\end{itemize}
\spacer
Depending on the application:
\begin{itemize}
\item Can handle different types of inputs (numerical and categorical)
\item Can handle dependencies (i.e., hierarchical input)
\end{itemize}
\end{framev}


\begin{framev}[fs=small]{Gaussian Process}
Posterior predictive distribution for test point $\xv \in \mathcal{S}$ under zero mean:
$$
Y(\xv) ~|~ \xv, \Dt \sim \mathcal{N}\left(\fh(\xv), \sh^2(\xv)\right)
$$
with
$$
\begin{aligned}
\fh(\xv) &= k(\xv)^\top \bm{K}^{-1} \bm{y} \\
\sh^2(\xv) &= k(\xv, \xv) - k(\xv)^\top \bm{K}^{-1} k(\xv)
\end{aligned}
$$
\vfill
\imageC[0.3]{figure_man/bayesian_loop_sm_normal.png}
\spacer
\begin{footnotesize}
Note: $\xv$ here denotes the test input.
$k(\xv) \coloneqq (k(\xv, \xvsi[1]), \ldots, k(\xv, \xvsi[t]))^\top$.
$\bm{y} \coloneqq (y^{[1]}, \ldots, y^{[t]})^\top$. \\
Kernel / Gram matrix $\bm{K} \coloneqq \left(k(\xvsi, \xvsi[j])\right)_{i,j}$ where $i, j \in \{1, \ldots, t\}.$
\end{footnotesize}
\end{framev}


\begin{framev}[fs=small]{Gaussian Process}
Example kernel functions:
\begin{itemize}
\item Radial basis function kernel (also known as Gauss kernel):
$k(\xv, \xv^{'})=\exp \left(-\frac{d(\xv, \xv^{'})^2}{2 l^2}\right)$
\begin{itemize}
\item $l$ length scale; $d(\cdot, \cdot)$ Euclidean distance
\item infinitely differentiable - very \enquote{smooth}
\end{itemize}
\item Matérn kernels:
$k(\xv, \xv^{'})=\frac{1}{\Gamma(\nu) 2^{\nu-1}}\left(\frac{\sqrt{2 \nu}}{l} d(\xv, \xv^{'})\right)^\nu K_\nu\left(\frac{\sqrt{2 \nu}}{l} d(\xv, \xv^{'})\right)$
\begin{itemize}
\item $l$ length scale; $d(\cdot, \cdot)$ Euclidean distance; $K_\nu(\cdot)$ modified Bessel function; $\Gamma(\cdot)$ Gamma function
\item for $\nu = 3/2$ once differentiable, for $\nu = 5/2$ twice differentiable
\item Popular choice as a kernel function when using a GP as SM
\end{itemize}
\end{itemize}
\end{framev}


\begin{framei}{Gaussian Process}
\item[Pros:]
\item Smooth, local, powerful estimator, also for small data
\item GPs yield well-calibrated uncertainty estimates
\item The posterior predictive distribution under a GP is normal
\spacer
\item[Cons:]
\item Vanilla GPs scale cubic in the number of data points
\item Can natively only handle numeric features
Mixed inputs / dependencies require special kernels
\item GPs aren't that robust; numerical problems can occur
\item Can be sensitive to the choice of kernel and hyperparameters
\end{framei}


\begin{framev}{Random Forest}
\begin{itemize}
\item Bagging ensemble
\item Fit $B$ decision trees on bootstrap samples
\item Feature subsampling
\end{itemize}
\vfill
\imageC[0.7]{figure_man/random_forests.jpg}
\spacer
\enquote{extratrees} / random splits:
\begin{itemize}
\item Choose split location uniformly at random
\item Results in a \enquote{smoother} mean prediction
\end{itemize}
\end{framev}


\begin{framev}[fs=small]{Random Forest - Mean and Variance}
\begin{itemize}
\item Let $\fh_b: \mathcal{S} \to \R$ be the mean prediction of a decision tree $b$ (mean of all data points in the same node as observation $\xv \in \mathcal{S}$)
\item Let $\sh_{b}^2: \mathcal{S} \to \R$ be the variance prediction (variance of all data points in the same node as observation $\xv \in \mathcal{S}$)
\item Mean prediction of forest: $\fh: \mathcal{S} \to \R$,
$$
\xv \mapsto \frac{1}{B} \sum_{b = 1}^{B} \fh_{b}(\xv)
$$
\item Variance prediction of forest: $\sh^{2}: \mathcal{S} \to \R$,
$$
\xv \mapsto  \left( \frac{1}{B} \sum_{b = 1}^{B} \sh_{b}^2(\xv) + \fh_{b}(\xv)^{2} \right) - \fh(\xv)^{2}
$$
(law of total variance assuming a mixture of $B$ models)
\item Alternative variance estimator:
\begin{itemize}
\item (infinitesimal) Jackknife
\end{itemize}
\item Variance prediction derived from randomness of individual trees
\begin{itemize}
\item Bagging / boostrap samples
\item Features sampled at random
\item (randomized split locations in the case of \enquote{extratrees})
\end{itemize}
\end{itemize}
\end{framev}


\begin{framev}{Random Forest - Different Choices}
% we always use all data (fraction = 1)
% no bootstrap means we do subsampling (i.e., simply use all data once)
% no random split means we split by variance criterion
% random split means we choose the split point in a feature uniformly at random between feature bounds
\imageC[0.9]{figure_man/surrogate_1.png}
\end{framev}


\begin{framei}{Random Forest}
\item[Pros:]
\item Cheap(er) to train
\item Scales well with the number of data points
\item Scales well with the number of dimensions
\item Can easily handle hierarchical mixed spaces. Either via imputation or directly respecting dependencies in the tree structure
\item Robust
\spacer
\item[Cons:]
\item Suboptimal uncertainty estimates
\item Not really Bayesian (no real posterior predictive distribution)
\item Poor extrapolation
\end{framei}


\begin{framev}[fs=small]{Example}
% BO_RF is the RF top right on slide 9
% BO_RF_ET is the RF top bottom on slide 9
Minimize the 2D Ackley Function using BO\_GP (GP with Matérn 3/2, EI), BO\_RF (standard Random Forest, EI), BO\_RF\_ET (Random Forest with extratrees, EI) or a random search:
\vfill
\imageC[0.5]{figure_man/surrogate_2.png}
\spacer
\begin{footnotesize}
Strong BO\_GP performance. BO\_RF and BO\_RF\_ET not too bad either. BO\_RF\_ET maybe slightly better final performance than BO\_RF.
\end{footnotesize}
\end{framev}

\endlecture
\end{document}
