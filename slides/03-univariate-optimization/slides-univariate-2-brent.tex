\documentclass[11pt,compress,t,notes=noshow, xcolor=table]{beamer}

\input{../../style/preamble}
\input{../../latex-math/basic-math}
\input{../../latex-math/basic-ml}

\newcommand{\titlefigure}{figure_man/quadratic-title.png}
\newcommand{\learninggoals}{
\item Quadratic interpolation
\item Brent's procedure
}

%\usepackage{animate} % only use if you want the animation for Taylor2D

\title{Optimization in Machine Learning}
%\author{Bernd Bischl}
\date{}

\begin{document}

\lecturechapter{Univariate optimization: Brent's method}
\lecture{\inserttitle}
\sloppy



\begin{frame}{Quadratic Interpolation}

Similar to golden ratio procedure but select $x^{\text{new}}$ differently: $x^{\text{new}}$ as minimum of a parabola fitted through 

$$(x^{\text{left}}, f^{\text{left}}), (x^{\text{best}}, f^{\text{best}}), (x^{\text{right}}, f^{\text{right}}).$$ 


\foreach \i in {1, 2, 3, 4, 5}{
  \only<\i>{
  \begin{center}
  \includegraphics{figure_man/quadratic\i.pdf} \\
  \begin{footnotesize}
    Left: Fit parabola (dashed) and propose minimum (red) as new point. Middle: Switch / not switch with $x^{\text{best}}$. Right: New interval. 
  \end{footnotesize}
  \end{center}
  }
}

\end{frame}


%Code für Erzeugung der Bilder in Ordner figure_man/quadr_int für animation
% <<echo = FALSE>>=
% min_func = function(x) cos(x) + 3 *  x^2 + 3*exp(x)
%
% plot_quad = function(minfunc, left = -2.5, right = 2.5, best = -1, max_it = 3, ...) {
%   l = left
%   r = right
%
%   parab = function(x) A * x^2 + B * x + C
%
%   iter = 1
%   curve(minfunc, from = l-0.5, to = r+0.5, ylab = "f(x)", ...)
%   abline(v = c(left, best, right), lty = 2)
%   points(c(left, best, right), minfunc(c(left, best, right)), pch = 15)
%   axis(3, at = c(left, best, right),
%          labels = c(expression("x"[left]), expression("x"[best]),expression("x"[right])), las = 3, cex.axis = 1.5)
%   while (iter <= max_it + 1) {
%     curve(minfunc, from = l-0.5, to = r+0.5, ylab = "f(x)", ...)
%     fl = min_func(left)
%     fr = min_func(right)
%     fb = min_func(best)
%
%     denom = (left - best) * (left - right) * (best - right)
%     A = (right * (fb - fl) + best * (fl - fr) + left * (fr - fb)) / denom
%     B = (right^2 * (fl - fb) + best^2 * (fr - fl) + left^2 * (fb - fr)) / denom
%     C = (best * right * (best - right) * fl + right * left * (right - left) * fb +
%            left * best * (left - best) * fr) / denom
%     xnew = - B / (2 * A)
%
%     points(seq(l - 0.5, r + 0.5, length.out=100),
%            parab(seq(l - 0.5, r + 0.5, length.out=100)), type = "l", lty = 2)
%     abline(v = c(left, best, xnew, right), lty = 2)
%     points(c(left, best, xnew, right), minfunc(c(left, best, xnew, right)),
%            pch = 15)
%     axis(3, at = c(left, best, xnew, right),
%          labels=c(expression("x"[left]), expression("x"[best]),
%                   expression("x"[new]), expression("x"[right])), las = 3, cex.axis = 1.5)
%
%     if (minfunc(xnew) < minfunc(best)) {
%       store = xnew
%       xnew = best
%       best = store
%     }
%     if (best < xnew) {right = xnew}
%     if (best > xnew) {left = xnew}
%
%     curve(minfunc, from = l-0.5, to = r+0.5, ylab = "f(x)", ...)
%     abline(v = c(left, best, right), lty = 2)
%     points(c(left, best, right), minfunc(c(left, best, right)),
%            pch = 15)
%     axis(3, at = c(left, best, right),
%          labels=c(expression("x'"[left]), expression("x'"[best]),
%                   expression("x'"[right])), las = 3, cex.axis = 1.5)
%     iter = iter + 1
%   }
% }
% plot_quad(min_func, l = -2.5, r = 2.5, best = 0, max_it = 3, ylim = c(-10, 90))
% @

% \begin{frame}[t,fragile]{Quadratische Interpolation Beispiel}
% \begin{center}
%   \only<1>{\includegraphics[width=0.8\textwidth]{figure_man/quadr_int/pic-1.png}}
%   \only<2>{\includegraphics[width=0.8\textwidth]{figure_man/quadr_int/pic-2.png}}
%   \only<3>{\includegraphics[width=0.8\textwidth]{figure_man/quadr_int/pic-3.png}}
%   \only<4>{\includegraphics[width=0.8\textwidth]{figure_man/quadr_int/pic-4.png}}
%   \only<5>{\includegraphics[width=0.8\textwidth]{figure_man/quadr_int/pic-5.png}}
% \end{center}
%
%   % \animategraphics[poster=first,autoplay,loop,width=0.65\textwidth]{1}{figure_man/quadr_int/pic-}{1}{9}
% \end{frame}


\begin{vbframe}{Quadratic interpolation comments}
\begin{itemize}
\item Quadratic interpolation \textbf{not robust}. The following may happen:
\begin{itemize}
\item Algorithm suggests the same $x^{\text{new}}$ in each step,
\item $x^{\text{new}}$ outside of search interval,
\item Parabola degenerates to line and no real minimum exists 
%(if 3 points are collinear).
\end{itemize}
\item Algorithm must then abort, finding a global minimum is not guaranteed.
% \item In practice, extensively tested numerical libraries should be used.
\end{itemize}
\end{vbframe}

\begin{vbframe}{Brent's method}

\begin{itemize}
  \item Brent proposed an algorithm (1973) that alternates between golden ratio search and quadratic interpolation as follows: 
  \begin{itemize}
    \item Quadratic interpolation step acceptable if: (i) $x^{\text{new}}$ falls within $[x^{\text{left}}, x^{\text{right}}]$ (ii) $x^{\text{new}}$ sufficiently far away from  $x^{\text{best}}$ \\
    \begin{footnotesize}(Heuristic: Less than half of movement of step before last) 
    \end{footnotesize}
    % https://mil.ufl.edu/nechyba/www/__eel6825.f2003/course_materials/t11.neural_networks/papers/nric_ch10.1-2.pdf
    \item Otherwise: Proposal via golden ratio
  \end{itemize}
  \item Benefit: Fast convergence (quadratic interpolation), unstable steps (e.g. parabola degenerated) stabilized by golden ratio search
  \item Convergence guaranteed if the function $f$ has a local minimum
  \item Used in R-function \texttt{optimize()}
\end{itemize}


% \begin{itemize}
% \item Zunächst wird Suchintervall durch Goldenen Schnitt solange verkleinert, bis Zielfunktion durch Parabel an den letzten 3 Evaluationspunkten gut approximiert wird.
% \item Dann weitere Suche durch Quadratische Interpolation, Ausnutzen der schnelleren Konvergenz.
% \item Bei frühzeitiger Terminierung, erneut Goldener Schnitt.
% \end{itemize}
% Dieses Verfahren wird \textbf{Brent'sche Methode} genannt. Funktioniert sehr gut, da sich alle (zweimal stetig diff'baren) Funktionen in einer $\epsilon$ Umgebung um das globale Minimum gut durch eine quadratische Funktion approximieren lassen (vgl. Taylorreihe).
\end{vbframe}


\begin{vbframe}{Example: MLE Poisson}

\begin{itemize}
    \item Poisson density: $f(k ~|~ \lambda) := \P(x = k) = \frac{\lambda^k \cdot \exp(-\lambda)}{k!}$
    \item Negative log-likelihood for $n$ observations:
        \begin{equation*}
        - \ell(\lambda, \mathcal{D}) = - \log \prod_{i = 1}^n  f\left(x^{(i)} ~|~ \lambda\right) =  - \sum_{i = 1}^n \log f\left(x^{(i)} ~|~ \lambda\right) 
        \end{equation*}
\end{itemize}

\vspace{-\baselineskip}

\begin{figure}
    \centering
    \includegraphics[width=0.7\textwidth]{figure_man/poisson.pdf}
    \caption*{\footnotesize GR and Brent converge to minimum at $x^\ast \approx 1$. \\
        \textbf{But:} GR needs $\approx 45$ it., Brent only needs $\approx 15$ it. for same tolerance.}
\end{figure}

\end{vbframe}

\endlecture

\end{document}


