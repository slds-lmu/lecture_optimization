\documentclass[11pt,compress,t,notes=noshow, xcolor=table]{beamer}

\input{../../style/preamble}\input{../../latex-math/basic-math}
\input{../../latex-math/basic-ml}


\newcommand{\titlefigure}{figure_man/golden-ratio-6.png}
\newcommand{\learninggoals}{
\item TODO
}


%\usepackage{animate} % only use if you want the animation for Taylor2D

\title{Optimization}
\author{Bernd Bischl}
\date{}

\begin{document}

\lecturechapter{Unconstrained problems}
\lecture{Optimization}
\sloppy

\begin{vbframe}{Univariate Optimization}

Let $f: \R \to \R$. 

\lz 

\textbf{Goal}: Construct efficient optimization procedure to find optimum. 

\vspace*{0.2cm} 

\begin{figure}
  \includegraphics{figure_man/golden-ratio-0.png}
\end{figure}

\end{vbframe}

\begin{frame}{Simple nesting procedure}

Let $f: \R \to \R$. 

\lz 

\textbf{Goal}: Construct efficient optimization procedure to find optimum. 

\vspace*{0.2cm} 


\only<1>{
  \begin{figure}
    \includegraphics{figure_man/golden-ratio-1.png}
  \end{figure}
}

\only<2>{
  \begin{figure}
    \includegraphics{figure_man/golden-ratio-2.png}
  \end{figure}
}

\only<3>{
  \begin{figure}
    \includegraphics{figure_man/golden-ratio-3.png}
  \end{figure}
}

\only<4>{
  \begin{figure}
    \includegraphics{figure_man/golden-ratio-4.png}
  \end{figure}
}

\only<5>{
  \begin{figure}
    \includegraphics{figure_man/golden-ratio-5.png}
  \end{figure}
}

\only<6>{
  \begin{figure}
    \includegraphics{figure_man/golden-ratio-6.png}
  \end{figure}
}

\only<7>{
  \begin{figure}
    \includegraphics{figure_man/golden-ratio-7.png}
  \end{figure}
}

\only<8>{
  \begin{figure}
    \includegraphics{figure_man/golden-ratio-8.png}
  \end{figure}
}

\only<9>{
  \begin{figure}
    \includegraphics{figure_man/golden-ratio-9.png}
  \end{figure}
}


\only<10>{
  \begin{figure}
    \includegraphics{figure_man/golden-ratio-10.png}
  \end{figure}
}

\only<11>{
  \begin{figure}
    \includegraphics{figure_man/golden-ratio-11.png}
  \end{figure}
}

\only<12>{
  \begin{figure}
    \includegraphics{figure_man/golden-ratio-12.png}
  \end{figure}
}

\only<13>{
  \begin{figure}
    \includegraphics{figure_man/golden-ratio-13.png}
  \end{figure}
}


\end{frame}

\begin{vbframe}{Simple nesting procedure}

\begin{itemize}
\item\textbf{Initialization}: Search interval  $(x^{\text{left}}, x^{\text{right}})$, $x^{\text{left}} < x^{\text{right}}$
\item Choose $x^{\text{best}}$ randomly.
\item For $t = 0, 1, 2, ...$
\begin{itemize}
    \item Choose $x^{\text{new}}$ randomly in $[x^{\text{left}}, x^{\text{right}}]$
    \item If $f(x^{\text{new}}) < f(x^{\text{new}})$:
    \begin{itemize}
        \item $x^{\text{best}} \leftarrow x^{\text{new}}$
    \end{itemize}
    \item New interval: Points around $x^{\text{best}}$
\end{itemize}
\end{itemize}

\lz 

\begin{figure}
    \centering
    \includegraphics[width=0.24\textwidth]{figure_man/golden-ratio-2.png}     \includegraphics[width=0.24\textwidth]{figure_man/golden-ratio-3.png}
    \includegraphics[width=0.24\textwidth]{figure_man/golden-ratio-4.png}
    \includegraphics[width=0.24\textwidth]{figure_man/golden-ratio-5.png}
\end{figure}

\end{vbframe}


\begin{vbframe}{Golden ratio}

\textbf{Key question:} How choose $x^{\text{new}}$ most efficiently in every iteration? 

\begin{itemize}
    \item \textbf{Insight 1: } It should always be in the bigger interval.
    \item \textbf{Insight 2: } $x^{new}$ must be chosen symmetrically to $x^{best}$ so that potential new intervals $(x^{left}, x^{new}]$, $[x^{best}, x^{right}]$  are of the same length.
\end{itemize}

\lz 

\begin{figure}
\includegraphics[width=0.4\textwidth]{figure_man/goldensec.png}\\
\end{figure}

W.l.o.g. consider two hypothetical outcomes $x^{\text{new}}$: $f_{new, a}$ and $f_{new, b}$. 

\framebreak 

If $f_{new, a}$ is the outcome, $x_{best}$ will stay our best observed point and we would search in the interval around $x_{best}$ : 

$$
    (x_{left}, x_{new}) 
$$

\begin{figure}
\includegraphics[width=0.8\textwidth]{figure_man/goldensec-1.png}\\
\end{figure}

\framebreak 

If $f_{new, b}$ is the outcome, $x_{new}$ will stay our best observed point and we would search in the interval around $x_{new}$ : 

$$
    (x_{best}, x_{right}) 
$$

\begin{figure}
\includegraphics[width=0.8\textwidth]{figure_man/goldensec-2.png}\\
\end{figure}

\framebreak 

We require those two potential intervals to be of equal size: 

\begin{eqnarray*}
    b := x_{right} - x_{best} = x_{new} - x_{left}
\end{eqnarray*}

\begin{figure}
\includegraphics[width=0.8\textwidth]{figure_man/goldensec-3.png}\\
\end{figure}

\framebreak 

Thinking one iteration further, we also want the intervals be of same size. We want 

\begin{eqnarray*}
    c := x_{best} - x_{left} = x_{right} - x_{new}
\end{eqnarray*}

\begin{figure}
\includegraphics[width=0.8\textwidth]{figure_man/goldensec-4.png}\\
\end{figure}

\framebreak 

\begin{figure}
\includegraphics[width=0.4\textwidth]{figure_man/goldensec.png}\\
\end{figure}

\end{vbframe}

\begin{vbframe}{Golden ratio idea}

The golden ratio ensures that these two potential intervals are of equal size. Thus it is prevented that intervals are chosen \enquote{unluckily}.

\lz

\textbf{Aim} is to arrange $x^{[t, left]}$, $x^{[t, new]}$, $x^{[t, best]}$ and $x^{[t, right]}$ such that the ratio of the length of two consecutive intervals is constant.

\framebreak

Without loss of generality $x^{[t, left]}< x^{[t, best]}<x^{[t, new]}<x^{[t, right]}$. \\

\vspace*{-0.3cm}

\begin{center}
\includegraphics[width = 0.3\textwidth]{figure_man/goldensec.png}
\end{center}

\vspace*{-0.5cm}

\begin{itemize}

% \framebreak
% Sei $f$ eine unimodale Funktion, $a<x<y<b$.
% \medskip


% \textbf{Ziel:} Finde Algorithmus zur Eingrenzung das Intervall von $[a, b]$, in dem das Minimum liegt.
% \medskip

% \textbf{Forderungen an den Algorithmus:}

% \begin{itemize}
% \item $x$ und $y$ sind symmetrisch im Restintervall zu wählen
% \item Der Reduktionsfaktor $\sigma$ für die Intervallänge sei konstant, $\sigma \in (0.5, 1)$ d.h.Der Quotient aus der neuen und alten Intervallänge ist konstant gleich $\sigma = \frac{l_{neu}}{l_{alt}}$.
% \end{itemize}

% \medskip
% \textbf{Ziel:} Bestimme den von der Funktion $f$ unabhängigen Reduktionsfaktor $\sigma$ mit Hilfe des Goldenen Schnitts.

% \framebreak

% Sei $f:[0,1]\rightarrow \R$ streng monoton wachsend.
% \medskip

% O.B.d.A. sei $0<x<y<1$, $ \Rightarrow y = \sigma$ und $x = 1-\sigma$.
% \medskip

% \scriptsize
% \textbf{Schritte zur Bestimmung des Reduktionsfaktors:} $\mathbf{\sigma}$
% \medskip

% 1. Intervall $[0, 1]$ wird auf Restintervall $[0, y]$ reduziert.\\
% 2. Wähle im Restintervall einen Punkt $z$ der symmetrisch zu $x$ ist. In Abhängigkeit der Wahl von $y$ gilt (i) $z<x$ oder (ii) $z \geq x$
% \textbf{Fallunterscheidung:}
% \medskip

% (i) $\sigma < \frac{2}{3}$: Es gilt $x=1-\sigma >\frac{1}{3}$ und somit: \\

% $$
% z = y-x = \sigma - (1- \sigma) < \frac{1}{3} < x
% $$
% $\sigma = const.$ fordert dann:
% $$
% \frac{x}{y} = \frac{y}{1} \Rightarrow 0 = y^{2}-x = \sigma^{2} + \sigma -1.
% $$
% Lösen der quadratischen Gleichung liefert:
% $$
% \sigma = \frac{1}{2}(\sqrt{5}-1) \approx 0.618 < \frac{2}{3}.
% $$

% \medskip
% (ii) $\sigma \geq \frac{2}{3}$: Analog zu (i), jedoch keine Lösung.
% \normalsize
% \framebreak

\item $x^{[t, new]}$ must be chosen symmetrically to $x^{[t, best]}$ so that potential new intervals $[x^{[t, left]}, x^{[t, new]}]$, $[x^{[t, best]}, x^{[t, right]}]$  are of the same length.
\item Define
\vspace*{-0.3cm}
\begin{footnotesize}
\begin{eqnarray*}
a &:=& x^{[t, right]}-x^{[t, left]}, \\
b &:=& x^{[t, right]}-x^{[t, best]} = x^{[t, new]}-x^{[t, left]},\\
c &:=& x^{[t, best]}-x^{[t, left]} = x^{[t, right]}-x^{[t, new]}.\\
\end{eqnarray*}
\end{footnotesize}
\item $a$ is the length of the current search interval, $b$ is the length of the next search interval, $c$ is the length of the search interval after $b$, such that $a = b + c$.
\item According to definition it applies:
$$
\varphi := \frac{b}{a} = \frac{c}{b}
$$
$$
\varphi^2 = \frac{c}{a}
$$

\item Divide $a = b + c$ by $a$:
\begin{eqnarray*}
\frac{a}{a} &=& \frac{b}{a} + \frac{c}{a} \\
1 &=& \varphi + \varphi^2 \\
0 &=& \varphi^2 + \varphi - 1
\end{eqnarray*}
\item Unique positive solution is $\varphi = \frac{\sqrt{5}-1}{2} \approx 0.618$.

\framebreak

\item Given $x^{[t, left]}$ and $x^{[t, right]}$ it follows
\begin{eqnarray*}
x^{[t, best]}&=&x^{[t, right]}-\varphi(x^{[t, right]}-x^{[t, left]})\\
&=&x^{[t, left]}+(1-\varphi)(x^{[t, right]}-x^{[t, left]})
\end{eqnarray*}
and due to symmetry
\begin{eqnarray*}
x^{[t, new]}&=& x^{[t, left]}+\varphi(x^{[t, right]}-x^{[t, left]})\\ &=& x^{[t, right]}-(1-\varphi)(x^{[t, right]}-x^{[t, left]}).
\end{eqnarray*}
\end{itemize}

\framebreak

\textbf{Iteration-step cut} \\
\begin{enumerate}
\item If $f(x^{[t, new]}) < f(x^{[t, best]})$, swap $x^{[t, new]}$ and $x^{[t, best]}$
\item Then choose
$$x^{[t + 1, left]} = \begin{cases}
x^{[t, left]} & \text{if } x^{[t, best]}<x^{[t, new]} \\
x^{[t, new]} & \text{otherwise}
\end{cases}$$
and
$$x^{[t + 1, right]} = \begin{cases}
x^{[t, new]} & \text{if } x^{[t, best]}<x^{[t, new]} \\
x^{[t, right]} & \text{otherwise}
\end{cases}$$
and set $x^{[t + 1, best]}=x^{[t, best]}$.
\item $x^{[t + 1, new]}$ due to symmetry
$$
x^{[t + 1, new]}=x^{[t + 1, left]}-(x^{[t + 1, right]}-x^{[t + 1, best]}).
$$
\end{enumerate}
%
% \begin{algorithm}[H]
%   \caption{Goldener Schnitt}
%   \begin{algorithmic}[1]
%   \For {$j = 1 \text{ to } n$}
%     \State $l_{jj} = \left(a_{jj} - \sum_{k = 1}^{j - 1}l_{jk}^2\right)^{\frac{1}{2}}$
%     \For {$i=j+1 \text{ to } n$}
%       \State $l_{ij} =
%   \frac{1}{l_{jj}}\left(a_{ij} - \sum_{k = 1}^{j - 1}l_{ik}l_{jk}\right)$
%     \EndFor
%   \EndFor
%   \end{algorithmic}
% \end{algorithm}

\end{vbframe}

\begin{frame}{Golden Ratio}

\foreach \i in {0, 1, 2, 3}{
  \only<\i>{
  \vspace*{1cm}
  \begin{center}
  \includegraphics{figure_man/golden_ratio\i.pdf} 
  \end{center}
  }
}

\end{frame}


\begin{vbframe}{Golden Ratio}

Note that in every iteration, 3/4 of the values from the previous iteration can be reused: \\

\begin{center}
  \includegraphics[width=0.65\textwidth]{figure_man/Diagram_of_a_golden_section_search.jpg}
\end{center}

\framebreak

Termination criterion:

  \begin{itemize}
    \item A reasonable choice is the absolute error, i.e. the width of the last interval:
    $$
    |x^{[t, best]}-x^{[t, new]}| < \tau
    $$
    \item In practice, more complicated termination criteria are usually applied, for example in \emph{Numerical Recipes in C, 2017}

    $$
    |x^{[t, right]}-x^{[t, left]}| \le \tau (|x^{[t, best]}| + |x^{[t, new]}|)
    $$

    is proposed as a termination criterion.

% (Schranke für Breite des Suchintervalls in Abhängigkeit der Größenordnung des Minimums):
% $$
% |x^{[t, right]}-x^{[t, left]}| < \tau(1+|x^{[t, best]}|)
% $$
\end{itemize}

\end{vbframe}


\end{document}


