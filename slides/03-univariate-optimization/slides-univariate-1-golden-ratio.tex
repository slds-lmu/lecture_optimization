\documentclass[11pt,compress,t,notes=noshow, xcolor=table]{beamer}

\input{../../style/preamble}\input{../../latex-math/basic-math}
\input{../../latex-math/basic-ml}


\newcommand{\titlefigure}{figure_man/golden-ratio-6.png}
\newcommand{\learninggoals}{
\item Simple nesting procedure
\item Golden ratio
}


%\usepackage{animate} % only use if you want the animation for Taylor2D

\title{Optimization in Machine Learning}
\author{Bernd Bischl}
\date{}

\begin{document}

\lecturechapter{Univariate optimization: Golden ratio}
\lecture{Optimization in Machine Learning}
\sloppy

\begin{vbframe}{Univariate Optimization}

Let $f: \R \to \R$. 

\lz 

\textbf{Goal}: Construct efficient optimization procedure to find optimum. 

\vspace*{0.2cm} 

\begin{figure}
  \includegraphics{figure_man/golden-ratio-0.png}
\end{figure}

\end{vbframe}

\begin{frame}{Simple nesting procedure}

Let $f: \R \to \R$. 

\lz 


\only<1>{
  Always maintain three points: left, right, and current best.

  \vspace*{0.2cm} 
  \begin{figure}
    \includegraphics{figure_man/golden-ratio-1.png}
  \end{figure}
}

\only<2>{
  Propose a random point. 
  \begin{figure}
    \includegraphics{figure_man/golden-ratio-2.png}
  \end{figure}

  \vfill

  \begin{footnotesize}
    NB: Later we will define the optimal choice for a new proposal.
  \end{footnotesize}  
}

\only<3>{
  Compare proposal against current best. 

  \begin{figure}
    \includegraphics{figure_man/golden-ratio-3.png}
  \end{figure}
}

\only<4>{
  If it is better: proposal becomes current best.
  \begin{figure}
    \includegraphics{figure_man/golden-ratio-4.png}
  \end{figure}
}

\only<5>{
  New search interval: around current best. 
  \begin{figure}
    \includegraphics{figure_man/golden-ratio-5.png}
  \end{figure}
}

\only<6>{
  Propose a random point.   
  \begin{figure}
    \includegraphics{figure_man/golden-ratio-6.png}
  \end{figure}
}

\only<7>{
  Compare proposal against current best. 

  \begin{figure}
    \includegraphics{figure_man/golden-ratio-7.png}
  \end{figure}
}

\only<8>{
  If it is better: proposal becomes current best.

  \begin{figure}
    \includegraphics{figure_man/golden-ratio-8.png}
  \end{figure}
}

\only<9>{
  New search interval: around current best. 

  \begin{figure}
    \includegraphics{figure_man/golden-ratio-9.png}
  \end{figure}
}


\only<10>{
  Propose a random point.   

  \begin{figure}
    \includegraphics{figure_man/golden-ratio-10.png}
  \end{figure}
}

\only<11>{
  Compare proposal against current best. 

  \begin{figure}
    \includegraphics{figure_man/golden-ratio-11.png}
  \end{figure}
}

\only<12>{
  If it is better: proposal becomes current best.

  \begin{figure}
    \includegraphics{figure_man/golden-ratio-12.png}
  \end{figure}
}

\only<13>{
  New search interval: around current best. 

  \begin{figure}
    \includegraphics{figure_man/golden-ratio-13.png}
  \end{figure}
}


\end{frame}

\begin{vbframe}{Simple nesting procedure}

\begin{itemize}
\item\textbf{Initialization}: Search interval  $(x^{\text{left}}, x^{\text{right}})$, $x^{\text{left}} < x^{\text{right}}$
\item Choose $x^{\text{best}}$ randomly.
\item For $t = 0, 1, 2, ...$
\begin{itemize}
    \item Choose $x^{\text{new}}$ randomly in $[x^{\text{left}}, x^{\text{right}}]$
    \item If $f(x^{\text{new}}) < f(x^{\text{new}})$:
    \begin{itemize}
        \item $x^{\text{best}} \leftarrow x^{\text{new}}$
    \end{itemize}
    \item New interval: Points around $x^{\text{best}}$
\end{itemize}
\end{itemize}

\lz 

\begin{figure}
    \centering
    \includegraphics[width=0.7\textwidth]{figure_man/golden-ratio-summary.png}  
\end{figure}

\end{vbframe}


\begin{vbframe}{Golden ratio}

\textbf{Key question:} How can $x^{\text{new}}$ be chosen better than randomly? 

\begin{itemize}
    \item \textbf{Insight 1: } Always in the bigger interval to maximize reduction.
    \item \textbf{Insight 2: } $x^{new}$ symetrically to $x^{best}$ for uniform reduction. 
\end{itemize}

\begin{figure}
\includegraphics[width=0.8\textwidth]{figure_man/goldensec-0.png}
\end{figure}

\vspace*{-0.5cm}

Consider two hypothetical outcomes $x^{\text{new}}$: $f_{new, a}$ and $f_{new, b}$. 

\framebreak

If $f_{new, a}$ is the outcome, $x_{best}$ stays best and we search around $x_{best}$ : 

$$
    [x_{left}, x_{new}]
$$

\begin{figure}
\includegraphics[width=0.8\textwidth]{figure_man/goldensec-1.png}\\
\end{figure}

\framebreak 

If $f_{new, b}$ is outcome, $x_{new}$ becomes best point and search around $x_{new}$ : 

$$
    [x_{best}, x_{right}] 
$$

\begin{figure}
\includegraphics[width=0.8\textwidth]{figure_man/goldensec-2.png}\\
\end{figure}

\framebreak 

For uniform reduction, require the two potential intervals equal sized: 

\begin{eqnarray*}
    b := x_{right} - x_{best} = x_{new} - x_{left}
\end{eqnarray*}

\begin{figure}
\includegraphics[width=0.8\textwidth]{figure_man/goldensec-3.png}\\
\end{figure}

\framebreak 

One iteration ahead: require again the intervals to be of same size. 

\begin{eqnarray*}
    c := x_{best} - x_{left} = x_{right} - x_{new}
\end{eqnarray*}

\begin{figure}
\includegraphics[width=0.8\textwidth]{figure_man/goldensec-4.png}\\
\end{figure}

\framebreak 

To summarize, we require: 

\begin{eqnarray*}
    a &=& x^{right}-x^{left}, \\
    b &=& x_{right} - x_{best} = x_{new} - x_{left} \\
    c &=& x_{best} - x_{left} = x_{right} - x_{new}
\end{eqnarray*}

\vspace*{-0.3cm}

\begin{figure}
\includegraphics[width=0.3\textwidth]{figure_man/goldensec.png}\\
\end{figure}

\framebreak 

\begin{itemize}
\item We require the same percentage improvement in each iteration
\item For $\varphi$ reduction factor of interval sizes ($a$ to $b$, and $b$ to $c$)

$$
\varphi := \frac{b}{a} = \frac{c}{b}
$$
$$
\varphi^2 = \frac{b}{a} \cdot \frac{c}{b} = \frac{c}{a}
$$

\item Divide $a = b + c$ by $a$:
\begin{eqnarray*}
\frac{a}{a} &=& \frac{b}{a} + \frac{c}{a} \\
1 &=& \varphi + \varphi^2 \\
0 &=& \varphi^2 + \varphi - 1
\end{eqnarray*}
\item Unique positive solution is $\varphi = \frac{\sqrt{5}-1}{2} \approx 0.618$.

\framebreak

\item With $x^{\text{new}}$ we always go $\varphi$ percentage points into the interval. 
\item Given $x^{left}$ and $x^{right}$ it follows
\begin{eqnarray*}
x^{best}&=&x^{right}-\varphi(x^{right}-x^{left})\\
&=&x^{left}+(1-\varphi)(x^{right}-x^{left})
\end{eqnarray*}
and due to symmetry
\begin{eqnarray*}
x^{new}&=& x^{left}+\varphi(x^{right}-x^{left})\\ &=& x^{right}-(1-\varphi)(x^{right}-x^{left}).
\end{eqnarray*}
\end{itemize}

\framebreak 

Termination criterion:

  \begin{itemize}
    \item A reasonable choice is the absolute error, i.e. the width of the last interval:
    $$
    |x^{best}-x^{new}| < \tau
    $$
    \item In practice, more complicated termination criteria are usually applied, for example in \emph{Numerical Recipes in C, 2017}

    $$
    |x^{right}-x^{left}| \le \tau (|x^{best}| + |x^{new}|)
    $$

    is proposed as a termination criterion.

% (Schranke für Breite des Suchintervalls in Abhängigkeit der Größenordnung des Minimums):
% $$
% |x^{right}-x^{left}| < \tau(1+|x^{best}|)
% $$
\end{itemize}

\end{vbframe}

\endlecture

\end{document}


