\documentclass[11pt,compress,t,notes=noshow, xcolor=table]{beamer}

\input{../../style/preamble}
\input{../../latex-math/basic-math}
\input{../../latex-math/basic-ml}


\title{Optimization in Machine Learning}

\begin{document}

\titlemeta{% Chunk title (example: CART, Forests, Boosting, ...), can be empty
  Second order methods
  }{% Lecture title  
  Quasi-Newton
  }{% Relative path to title page image: Can be empty but must not start with slides/
  figure_man/NR_2.png
  }{
    \item Newton-Raphson vs. Quasi-Newton
    \item SR1
    \item BFGS
}
%%%%%%%%%%%%%%%%%%%%%%%%%%%%%%%%%%%%%%%%%%%%%%%%%%%%%%%%%%%%%%%%%%%%%%%%%%%%%%%%%%%

\begin{vbframe}{Quasi-Newton: Idea}

% \framebreak

% Die Aktualisierungsrichtung $\mathbf{d}_{i}$ ist eine Abstiegsrichtung für $f$ in $\mathbf{x}_{i}$
% genau dann, wenn
% $$
% \mathbf{d}^\top \nabla f(\mathbf{x}_{i}) < 0.
% $$
% Sei $\mathbf{A}_{i}$ eine positiv definite Matrix, dann ist
% $$
% \mathbf{d}_{i} = - (\mathbf{A}_{i})^{-1} \nabla f(\mathbf{x}_{i})
% $$
% eine Abstiegsrichtung:
% $$
% (\mathbf{d}_{i})^\top \nabla f(\mathbf{x}_{i}) = - (\nabla f(\mathbf{x}_{i}))^\top (\mathbf{A}_{i})^{-1}
% \nabla f(\mathbf{x}_{i}) < 0
% $$
% $\mathbf{A}_{i} = \nabla^2 f(\mathbf{x}_{i})$ ist für quadratische Probleme optimal, bei allgemeinen
% Problemen muss die Hessematrix weit weg vom Optimum noch nicht einmal positiv definit sein.

Start point of \textbf{QN method} is (as with NR) a Taylor approximation of the gradient, except that H is replaced by a \textbf{pd} matrix $\bm{A}^{[t]}$:

\vspace*{-0.2cm}
\begin{footnotesize}
\begin{alignat*}{4}
\nabla f(\mathbf{x}) &\approx  \nabla f(\mathbf{x}^{[t]}) + \nabla^2 f(\mathbf{x}^{[t]}) & (\mathbf{x} - \mathbf{x}^{[t]}) ~ &=& ~\mathbf{0}  &\qquad& \text{ NR} \\
\nabla f(\mathbf{x}) &\approx \nabla f(\mathbf{x}^{[t]}) + \bm{A}^{[t]} & (\mathbf{x} - \mathbf{x}^{[t]}) ~ &=& ~ \mathbf{0} &\qquad& \text{ QN}
\end{alignat*}
\end{footnotesize}

The update direction:

\begin{footnotesize}
\begin{alignat*}{3}
\bm{d}^{[t]} &= - \nabla^2 f(\mathbf{x}^{[t]})^{-1} & \nabla f(\mathbf{x}^{[t]}) &\qquad& \text{ NR} \\
\bm{d}^{[t]} &= - (\bm{A}^{[t]})^{-1} & \nabla f(\mathbf{x}^{[t]}) &\qquad& \text{ QN} \\
\end{alignat*}
\end{footnotesize}

\framebreak

%\textbf{Quasi-Newton method}:


\begin{enumerate}
\item Select a starting point $\mathbf{x}^{[0]}$ and initialize pd matrix
$\mathbf{A}^{[0]}$ (can also be a diagonal matrix - a very rough approximation of Hessian).
\item Calculate update direction by solving

$$
\bm{A}^{[t]} \bm{d}^{[t]} = - \nabla f(\mathbf{x}^{[t]})
$$

and set $\bm{x}^{[t+1]} = \bm{x}^{[t]} + \alpha^{[t]} \bm{d}^{[t]}$ (Step size through backtracking)
\item Calculate an efficient update $\mathbf{A}^{[t+1]}$,\\based on $\mathbf{x}^{[t]}$,
$\mathbf{x}^{[t+1]}$, $\nabla f(\mathbf{x}^{[t]})$, $\nabla f(\mathbf{x}^{[t+1]})$ and
$\mathbf{A}^{[t]}$.
\end{enumerate}

\framebreak

Usually the matrices $\bm{A}^{[t]}$ are calculated recursively by performing an additive update 

$$
\bm{A}^{[t+1]} = \bm{A}^{[t]} + \bm{B}^{[t]}. 
$$

How $ \bm{B}^{[t]}$ is constructed is shown on the next slides. \\
\textbf{Requirements} for the matrix sequence $\bm{A}^{[t]}$:
\begin{enumerate}
\item Symmetric pd, so that $\bm{d}^{[t]}$ are descent directions.
\item Low computational effort when solving LES

$$
\bm{A}^{[t]} \bm{d}^{[t]} = - \nabla f(\mathbf{x}^{[t]})
$$

\item Good approximation of Hessian: The \enquote{modified} Taylor series for $\nabla f(\bm{x})$ (especially for $t \to \infty$) should provide a good approximation

$$
\nabla f(\mathbf{x}) \approx \nabla f(\mathbf{x}^{[t]}) +
\bm{A}^{[t]}(\mathbf{x} - \mathbf{x}^{[t]})
$$
\end{enumerate}



\end{vbframe}

\begin{vbframe}{Symmetric rank 1 update (SR1)}

Simplest approach: symmetric rank 1 updates (\textbf{SR1}) of form

$$
\bm{A}^{[t+1]} \leftarrow \bm{A}^{[t]} + \bm{B}^{[t]} = \bm{A}^{[t]} + \beta \bm{u}^{[t]}(\bm{u}^{[t]})^{\top}
$$

with appropriate vector $\bm{u}^{[t]} \in \R^n$, $\beta \in \R$.


\framebreak

\textbf{Choice of} $\bm{u}^{[t]}$:

Vectors should be chosen so that the \enquote{modified} Taylor series corresponds to the gradient:

\begin{eqnarray*}
\nabla f(\mathbf{x}) &\overset{!}{=}& \nabla f(\mathbf{x}^{[t+1]}) +
\bm{A}^{[t+1]}(\mathbf{x} - \mathbf{x}^{[t+1]}) \\
\nabla f(\mathbf{x}) &=& \nabla f(\mathbf{x}^{[t+1]}) +  \left(\bm{A}^{[t]} +
\beta \bm{u}^{[t]}(\bm{u}^{[t]})^\top\right)\underbrace{(\mathbf{x} - \mathbf{x}^{[t+1]})}_{:= \bm{s}^{[t+1]}} \\
\underbrace{\nabla f(\mathbf{x}) - \nabla f(\mathbf{x}^{[t+1]})}_{\bm{y}^{[t+1]}} &=& \left(\bm{A}^{[t]} + \beta \bm{u}^{[t]} (\bm{u}^{[t]})^{\top}\right) \bm{s}^{[t+1]} \\
\bm{y}^{[t+1]} - \bm{A}^{[t]} \bm{s}^{[t+1]} &=& \left(\beta (\bm{u}^{[t]})^{\top} \bm{s}^{[t+1]}\right) \bm{u}^{[t]}
\end{eqnarray*}

For $\bm{u}^{[t]} = \bm{y}^{[t+1]} - \bm{A}^{[t]} \bm{s}^{[t+1]}$ and $\beta = \frac{1}{\left(\bm{y}^{[t+1]} - \bm{A}^{[t]}\bm{s}^{[t+1]}\right)^\top\bm{s}^{[t+1]}}$ the equation is satisfied.

\framebreak

\textbf{Advantage}
\begin{itemize}
\item Provides a sequence of \textbf{symmetric pd} matrices
\item Matrices can be inverted efficiently and stable using Sherman-Morrison:
$$
(\bm{A} + \beta \bm{u}\bm{u}^{\top})^{-1} = \bm{A} + \beta \frac{\bm{u}\bm{u}^{\top}}{1 + \beta\bm{u}^\top\bm{u}}.
$$
\end{itemize}

\textbf{Disadvantage}
\begin{itemize}
\item The constructed matrices are not necessarily pd, and the update directions $\bm{d}^{[t]}$ are therefore not necessarily descent directions
\end{itemize}

\end{vbframe}

\begin{vbframe}{BFGS algorithm}

Instead of Rank 1 updates, the \textbf{BFGS} procedure (published simultaneously in 1970 by Broyden, Fletcher, Goldfarb and Shanno) uses rank 2 modifications of the form

$$
\bm{A}^{[t]} + \beta \bm{u}^{[t]}(\bm{u}^{[t]})^{\top} + \beta \bm{v}^{[t]}(\bm{v}^{[t]})^{\top}
$$

with $\bm{s}^{[t]} := \bm{x}^{[t+1]} - \bm{x}^{[t]}$

\begin{itemize}
  \item $\bm{u}^{[t]} = \nabla f(\bm{x}^{[t+1]}) - \nabla f(\bm{x}^{[t]})$
  \item $\bm{v}^{[t]} = \bm{A}^{[t]} \bm{s}^{[t]}$
  \item $\beta = \frac{1}{(\bm{u}^{[t]})^\top (\bm{s}^{[t]})}$
  \item $\beta = - \frac{1}{(\bm{s}^{[t]})^\top \bm{A}^{[t]} \bm{s}^{[t]}}$
\end{itemize}

The resulting matrices $\bm{A}^{[t]}$ are positive definite and the corresponding quasi-newton update directions $\bm{d}^{[t]}$ are actual descent directions.


% \begin{itemize}
% \item Basiert auf einer endlichen Differenzenapproximation der zweiten Ableitung.
% \item Verwendet eine additive Aktualisierung $\mathbf{A}^{(i + 1)} = \mathbf{A}^{(i)} + \mathbf{B}^{(i)}$.
% \item Invertierung von $\mathbf{A}^{(i)}$ lässt sich mit Hilfe der Sherman-Morrison-Formel
% vergleichsweise effizient durchführen.
% \end{itemize}

\end{vbframe}


\endlecture
\end{document}



