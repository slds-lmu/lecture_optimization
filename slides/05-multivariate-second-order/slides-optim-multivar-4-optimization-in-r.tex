\documentclass[11pt,compress,t,notes=noshow, xcolor=table]{beamer}

\input{../../style/preamble}
\input{../../latex-math/basic-math}
\input{../../latex-math/basic-ml}


\newcommand{\titlefigure}{}
\newcommand{\learninggoals}{
\item optim()
}


%\usepackage{animate} % only use if you want the animation for Taylor2D

\title{Optimization in Machine Learning}
%\author{Bernd Bischl}
\date{}

\begin{document}

\lecturechapter{Second order methods: Optimization in R}
\lecture{Optimization in Machine Learning}
\sloppy
%%%%%%%%%%%%%%%%%%%%%%%%%%%%%%%%%%%%%%%%%%%%%%%%%%%%%%%%%%%%%%%%%%%%%%%%%%%%%%%%%%%

\begin{vbframe}{Optimization in R}
Function \pkg{optim()} from base R provides algorithms for general optimization problems: \\[0.15cm]
\begin{itemize}
\item \textbf{Brent:} Only for one-dimensional functions. Use the function \pkg{optimize()}.
      Can be useful if \pkg{optim()} is called within another function.
\item \textbf{CG:} conjugated Gradient Methods
% \item \textbf{Nelder-Mead Simplex:} Gut für nicht-dif'bare Funktionen, basiert nur auf Funktionsauswertungen (default)
\item \textbf{BFGS, Quasi-Newton}
% \item \textbf{SANN:} Stochastisches Simulated Annealing
\end{itemize}
\framebreak
\lz

\begin{verbbox}
# General Call:
optim(par, fn, gr, method, lower, upper, control)
\end{verbbox}
\col

\lz
\begin{itemize}
\item \textbf{par} starting values of the parameters to be optimized
\item \textbf{fn} (objective) function, to be optimized (default: minimized)
\item \textbf{gr} gradient / derivative with corresponding method
\item \textbf{method} optimization method (see above)
\item \textbf{lower/upper} boundaries for optimization (L-BFGS-B)
\item \textbf{control} List of control parameters
\end{itemize}
\end{vbframe}

\endlecture
\end{document}



