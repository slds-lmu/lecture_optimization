\documentclass[11pt,compress,t,notes=noshow, xcolor=table]{beamer}

\input{../../style/preamble}
\input{../../latex-math/basic-math}
\input{../../latex-math/basic-ml}

\title{Optimization in Machine Learning}

\begin{document}

\titlemeta{
  Second order methods
  }{ 
  Fisher Scoring
  }{
  figure_man/NR_2.png
  }{
    \item Fisher Scoring
    \item Newton-Raphson vs. Fisher scoring
    \item Logistic regression
}
%%%%%%%%%%%%%%%%%%%%%%%%%%%%%%%%%%%%%%%%%%%%%%%%%%%%%%%%%%%%%%%%%%%%%%%%%%%%%%%%%%%

\begin{vbframe}{Recap of Newton's method}
Second-order Taylor expansion of log-likelihood around the current iterate $\thetav^{(t)}$:
$$\loglt \approx \logl(\thetav^{(t)}) +
\nabla\logl(\thetav^{(t)})^\top(\thetav-\thetav^{(t)}) + 
\frac{1}{2}(\thetav-\thetav^{(t)})^\top[\nabla^{2}\logl(\thetav^{(t)})](\thetav-\thetav^{(t)})$$
We then differentiate w.r.t.~$\thetav$ and set the gradient to zero:
$$\nabla\logl(\thetav^{(t)}) + [\nabla^{2}\logl(\thetav^{(t)})](\thetav-\thetav^{(t)}) = \zero$$
Solving for $\thetav^{(t)}$ yields the pure Newton-Raphson update:
$$\thetav^{(t+1)}=\thetav^{(t)} + [-\nabla^{2}\logl(\thetav^{(t)})]^{-1}\nabla\logl(\thetav^{(t)})$$

\lz

\textbf{Potential stability issue}: pure Newton-Raphson updates do not always converge. Its quadratic convergence rate is ``local'' in the sense that it requires starting close to a solution.  %are not guaranteed to be an ascent algorithm. 
%It's equally happy to head uphill or downhill.
\end{vbframe}

\begin{vbframe}{Fisher scoring}
Fisher’s scoring method replaces the negative \textit{observed Hessian} $-\nabla^{2}\loglt$ 
by the Fisher information matrix, i.e., the variance of $\nabla \loglt$, which, under weak regularity conditions, equals the negative \textit{expected Hessian}
$$\E[\nabla\loglt\nabla\loglt^\top]=\E[-\nabla^{2}\loglt],$$
and is positive semi-definite under exchangeability of expectation and differentiation.\\

\textbf{NB}: it can be shown that $\E[\nabla \ell(\thetav)]=\mathbf{0}$, which provides the expression of the variance of $\nabla \ell(\thetav)$ as the expected outer product of the gradients.

\lz

Therefore the Fisher scoring iterates are given by
$$\thetav^{(t+1)}=\thetav^{(t)}+\E[-\nabla^{2}\logl(\thetav^{(t)})]^{-1}\nabla\logl(\thetav^{(t)})$$
\end{vbframe}

\begin{vbframe}{Newton-Raphson vs. Fisher scoring}
  \begin{table}[h!]
    \centering
    \begin{tabular}{|p{2.5cm}|p{4cm}|p{4cm}|}
    \hline
    \textbf{Aspect} & \textbf{Newton-Raphson} & \textbf{Fisher scoring} \\ \hline
    Second-order Matrix & Exact negative \newline Hessian matrix & Fisher information matrix \\ \hline
    Curvature & Exact & Approximated \\ \hline
    Computational\newline Cost & Higher & Lower (often has a \newline simpler structure) \\ \hline
    Convergence & Fast but potentially \newline unstable & Slower but more stable \\ \hline
    Positive\newline Definite & Not guaranteed & Yes with\newline Fisher information \\ \hline
    Use Case & General non-linear\newline optimization & Likelihood-based models,\newline especially GLMs \\ \hline
    \end{tabular}
\end{table}
In many cases Newton-Raphson and Fisher scoring are equivalent (see below).
\end{vbframe}

\begin{vbframe}{Logistic regression}
The goal of logistic regression is to predict a binary event.
Given $n$ observations $\left(\xi, \yi\right) \in \R^{p+1} \times \{0, 1\}$,
$\yi |\, \xi \sim Bernoulli(\pi^{(i)})$.\\
\lz
We want to minimize the following risk 
\begin{eqnarray*}
  \riske(\thetav)  & = & 
  -\sum^n_{i=1} \yi\log(\pi^{(i)}) + \left(1-\yi\log(1-\pi^{(i)})\right)
\end{eqnarray*}

with respect to $\thetav$, 
where the probabilistic classifier $\pi^{(i)} = \pixit = s\left(\fxit\right)$,
the sigmoid function $s(f) = \frac{1}{1 + \exp(-f)}$ and the score $\fxit = \thx.$\\
\lz

\textbf{NB}: Note that $\frac{\partial}{\partial f} s(f) = s(f)(1-s(f))$ and $\frac{\partial \fxit}{\partial \thetav} = \left(\xi\right)^\top.$\\
\vspace{0.3cm}
For more details we refer to the \href{https://slds-lmu.github.io/i2ml/chapters/11_advriskmin/}{\color{blue}{i2ml}} lecture.

\framebreak
Partial derivative of empirical risk using chain rule:

{\small
\begin{align*}
  \frac{\partial}{\partial\thetav}\riske(\thetav) 
& =  
 -\sumin \frac{\partial}{\partial\pi^{(i)} }(\yi\log(\pi^{(i)})+(1-\yi)\log(1-\pi^{(i)}))\frac{\partial\pi^{(i)}}{\partial \thetav} \\
& =  
  -\sumin \left(\frac{\yi}{\pi^{(i)}} -  \frac{1-\yi}{1-\pi^{(i)}}\right)\frac{\partial s(\fxit)}{\partial  \fxit}\frac{\partial  \fxit}{\partial\thetav}\\
& = 
  \sumin \left(\pi^{(i)} - \yi\right)\left(\xi\right)^\top\\
& = 
  \left(\pi(\mathbf{X}\vert\;\thetav) - \mathbf{y}\right)^\top\mathbf{X}\\
\end{align*}
}

where  $\mathbf{X} = \left(
    {\xi[1]}^{\top}, \dots, 
    {\xi[n]}^{\top}\right)^\top \in \R^{n\times (p+1)}, \mathbf{y} = \left(
    \yi[1], \dots,
    \yi[n]
\right)^\top,$ \\ $\pi(\mathbf{X}\vert\;\thetav) = \left(
    \pi^{(1)}, \dots,
    \pi^{(n)}
\right)^\top \in \R^{n}$.

$\nabla_{\thetav}\riske = \left(\frac{\partial}{\partial\thetav}\riske\right)^\top$

\framebreak

The Hessian of logistic regression:

{\small
\begin{align*}
  \nabla^2_{\thetav}\riske  = \frac{\partial^2}{\partial{\thetav^\top}\partial\thetav}\riske  & =  
 \frac{\partial}{\partial{\thetav^\top}} \sumin \left(\pi^{(i)} - \yi\right)\left(\xi\right)^\top\\
 & =  
  \sum^n_{i=1}\xi \left(\pi^{(i)}\left(1-\pi^{(i)}\right)\right)\left(\xi\right)^\top\\
  & =  
\mathbf{X}^\top \mathbf{D} \mathbf{X}\\
\end{align*}

where $\mathbf{D} \in \mathbb{R}^{n\times n}$ is a diagonal matrix containing the variances of $\yi$ on the diagonals 
$$\mathbf{D}=\text{diag}\left(\pi^{(1)}(1-\pi^{(1)}), \dots, \pi^{(n)}(1-\pi^{(n)})\right).$$
}

\framebreak
We now have
$$\nabla_{\thetav}\riske =  \mathbf{X}^\top\left(\pi(\mathbf{X}\vert\;\thetav) - \mathbf{y}\right) $$
$$\nabla^2_{\thetav}\riske =  \mathbf{X}^\top \mathbf{D} \mathbf{X}$$

\lz

Newton-Raphson: $$\thetav^{(t+1)}=\thetav^{(t)} - [\mathbf{X}^\top \mathbf{D} \mathbf{X}]^{-1}\nabla_{\thetav^{(t)}}\riske $$
Fisher scoring: $$\thetav^{(t+1)}=\thetav^{(t)} - \E[\mathbf{X}^\top \mathbf{D} \mathbf{X}]^{-1}\nabla_{\thetav^{(t)}}\riske $$

Note that the Hessian does not depend on the $y^{(i)}$ explicitly but only depends on $\E[y^{(i)}]=\pi^{(i)}$. Thus the expectation of the observed Hessian w.r.t. $y^{(i)}\sim P(y^{(i)}|\xv^{(i)},\thetav)$ coincides with $\nabla^2_{\thetav}\riske(\thetav)$ itself.
\end{vbframe}


\begin{vbframe}{Generalized linear models}
$y|\xv$ belongs to an \textbf{exponential family} with density:
$$ p(y|\delta, \phi) = exp \left\{ \frac{y\delta-b(\delta)}{a(\phi)} + c(y,\phi) \right\}, $$
where $\delta$ is the natural parameter and $\phi > 0$ is the dispersion parameter.\\
We often take $a_i(\phi) = \frac{\phi}{w_i}$, with $\phi$ a pos. constant, and $w_i$ is a weight.

\lz

Generalized linear models (GLMs) relate the conditional mean $\mu(\xv)=\E[y|\xv]$ of $y$ to a linear predictor $\eta$ via a 
strictly increasing link function $g(\mu)=\eta=\xv^\top\theta$.\\

\lz

One can show that mean $\mu = \mu(\xv)=b'(\delta)=g^{-1}(\eta)$, variance $Var(y|\xv)=a(\phi)b''(\delta)$, where\\
$$\frac{\partial{b(\delta)}}{\partial{\theta}}
=\frac{\partial{b(\delta)}}{\partial{\delta}}\frac{\partial{\delta}}{\partial{\mu}}\frac{\partial{\mu}}{\partial{\eta}}\frac{\partial{\eta}}{\partial{\theta}}
=\mu \frac{1}{b''(\delta)} \frac{1}{g'(\mu)} \xv$$

\framebreak
We can estimate $\delta$ using MLE with sample $(\xv^{(i)},y^{(i)})$ for $i=1,\ldots,n$.\\
Take $a^{(i)}(\phi) = \frac{\phi}{w^{(i)}}$, $\phi$ is a positive constant, 
we could ignore it since the goal is to maximize the function:\\
\begin{align*}
\nabla \logl_{\theta}(\delta, \phi) &= \sumin \frac{w_i(y^{(i)}-\mu^{(i)})}{b''(\delta)g'(\mu^{(i)})}\xv^{(i)}\\
 &= \sumin \frac{w^{(i)}(y^{(i)}-\mu^{(i)})g'(\mu^{(i)})}{b''(\delta)[g'(\mu^{(i)})]^2}\xv^{(i)}\\
 &= \mathbf{X}^\top \mathbf{W} \mathbf{G} (\mathbf{Y}-\bm{\mu})
\end{align*}
\lz
$\mathbf{W}$ is a diagonal matrix with element $\frac{w^{(i)}}{b''(\delta)[g'(\mu^{(i)})]^2}$.\\
$\mathbf{G}$ is a diagonal matrix with element $g'(\mu^{(i)})$.\\

\framebreak

\begin{align*}
-\nabla^2 \logl_{\theta}(\delta, \phi)
  & = \sumin \frac{w^{(i)}}{b''(\delta)[g'(\mu^{(i)})]^2}\xv^{(i)}{\xv^{(i)}}^\top \\
  & - \sumin \frac{w^{(i)}(y^{(i)}-\mu^{(i)})(g''(\mu^{(i)})/g'(\mu^{(i)}))}{b''(\delta)[g'(\mu^{(i)})]^2}\xv^{(i)}{\xv^{(i)}}^\top \\ 
  & + \sumin \frac{w^{(i)}(y^{(i)}-\mu^{(i)})(\partial{b''(\delta)[g'(\mu^{(i)})]^2}/\partial{\mu^{(i)}})}{[b''(\delta)]^2[g'(\mu^{(i)})]^4}\xv^{(i)}{\xv^{(i)}}^\top
\end{align*}

$$\E[-\nabla^2 \logl_{\theta}(\delta, \phi)] = \sumin \frac{w^{(i)}}{b''(\delta)[g'(\mu^{(i)})]^2}\xv^{(i)}{\xv^{(i)}}^\top = \mathbf{X}^\top \mathbf{W} \mathbf{X}$$
\lz
Iteratively Reweighted Least Squares (IRLS) with weights $\frac{w^{(i)}}{b''(\delta)[g'(\mu^{(i)})]^2}$\\

\framebreak

Fisher scoring:
\begin{align*}
\thetav^{(t+1)} &=\thetav^{(t)} + (\mathbf{X}^\top \mathbf{W} \mathbf{X})^{-1}\mathbf{X}^\top \mathbf{W} \mathbf{G} (\mathbf{Y}-\bm{\mu})\\
& =(\mathbf{X}^\top \mathbf{W} \mathbf{X})^{-1}\mathbf{X}^\top \mathbf{W} \left(\mathbf{G} (\mathbf{Y}-\bm{\mu})+\mathbf{X}\thetav^{(t)}\right)
\end{align*}

\lz

For canonical link where $\eta = g(\mu) = \xv^\top\theta$, the second and third term of Hessian vanishes and Hessian coincides with Fisher information matrix.\\
\lz
This will now be a convex problem with Fisher scoring equal to Newton's method.\\
\lz
There are also hybrid algorithms that start out with IRLS which is easier to initialize, and switch over to Newton-Raphson after some iterations.

\end{vbframe}

\endlecture
\end{document}
