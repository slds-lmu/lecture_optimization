\documentclass[11pt,compress,t,notes=noshow, xcolor=table]{beamer}

\input{../../style/preamble}
\input{../../latex-math/basic-math}
\input{../../latex-math/basic-ml}

\title{Optimization in Machine Learning}

\begin{document}

\titlemeta{
Second order methods
}{
Optimization in R
}{% Relative path to title page image: Can be empty but must not start with slides/
}{
\item \code{optim()}
}

\begin{framei}{Optimization in R}
\item[] Function \pkg{optim()} from base R provides algorithms for general optimization problems: \\[0.15cm]
\spacer
\item \textbf{Brent:} Only for one-dimensional functions. Use the function \pkg{optimize()}.
Can be useful if \pkg{optim()} is called within another function.
\item \textbf{CG:} conjugated Gradient Methods
\item \textbf{BFGS, Quasi-Newton}
\end{framei}


\begin{frame}[fragile]{Optimization in R}
General Call:
\begin{verbatim}
optim(par, fn, gr, method, lower, upper, control)
\end{verbatim}
\spacer
\begin{itemize}
\item \textbf{par} starting values of the parameters to be optimized
\item \textbf{fn} (objective) function, to be optimized (default: minimized)
\item \textbf{gr} gradient / derivative with corresponding method
\item \textbf{method} optimization method (see above)
\item \textbf{lower/upper} boundaries for optimization (L-BFGS-B)
\item \textbf{control} List of control parameters
\end{itemize}
\end{frame}

\endlecture
\end{document}
