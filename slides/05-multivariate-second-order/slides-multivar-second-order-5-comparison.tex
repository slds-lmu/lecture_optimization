\documentclass[11pt,compress,t,notes=noshow, xcolor=table]{beamer}

\input{../../style/preamble}
\input{../../latex-math/basic-math}
\input{../../latex-math/basic-ml}

% sets
\newcommand{\CC}[1]{\mathcal{C}^{#1}}         % class C^k
\newcommand{\CCinf}{\mathcal{C}^{\infty}}     % class C^infty


% objectives, domains, points
\renewcommand{\S}{\mathcal{S}}              % domain of f
\newcommand{\xv}{\bm{x}}                  % vector x (bold)
\newcommand{\yv}{\bm{y}}                  % vector y (bold)
\newcommand{\vv}{\bm{v}}                  % vector v (bold)
\newcommand{\wv}{\bm{w}}                  % vector w (bold)
\newcommand{\xvs}{\bm{x}^\ast}            % x-star,  theoretical optimum
\newcommand{\xs}{x^\ast}                  % x-star,  theoretical optimum
\newcommand{\fxvs}{f(\xvs)}               % f at x-star, theoretical optimum
\renewcommand{\H}{\mathbf{H}}                     % Hessian matrix



% some other stuff
\providecommand{\eps}{\epsilon}               % epsilon
\newcommand{\A}{\mathbf{A}}                     % Hessian matrix
\newcommand{\bv}{\mathbf{b}}                     % Hessian matrix
\newcommand{\zv}{\mathbf{z}}                     % Hessian matrix
\newcommand{\V}{\mathbf{V}}                     % Hessian matrix


\newcommand{\vvmax}{\bm{v}_\text{max}}                  % vector v (bold)
\newcommand{\vvmin}{\bm{v}_\text{min}}                  % vector v (bold)
\newcommand{\lammax}{\lambda_\text{max}}                  % vector v (bold)
\newcommand{\lammin}{\lambda_\text{min}}                  % vector v (bold)

\newcommand{\sumid}{\sum\limits_{i=1}^d} % summation from i=1 to n


\title{Optimization in Machine Learning}

\begin{document}

\titlemeta{
Second order methods
}{
Newton-Raphson vs Gradient Descent
}{
figure_man/NR_2.png
}{
\item Comparison of Newton-Raphson and Gradient Descent
\item Pure Newton vs relaxed Newton with step size
}

\begin{framei}{Newton-Raphson and GD (Recap)}
\item Gradient Descent: \textbf{first order method} \\
$\Rightarrow$ \textit{Gradient} information, i.e., first derivatives 
\item Newton-Raphson: \textbf{second order method} \\
$\Rightarrow$ \textit{Hessian} information, i.e., second derivatives
\vfill
\item[] \textbf{Gradient Descent:}
$$
\thetatn = \thetat - \alpha \nabla f(\thetat)
$$
\item[] \textbf{Pure Newton-Raphson:}
$$
\thetatn = \thetat - (\nabla^2 f(\thetat))^{-1}\nabla f(\thetat)
$$
\item[] \textbf{Relaxed/Damped Newton-Raphson:}
$$
\thetatn = \thetat -\alpha (\nabla^2 f(\thetat))^{-1}\nabla f(\thetat)
$$
\end{framei}


\begin{framei}{Comparison simulation set-up}
\item Comparison of Newton-Raphson, relaxed NR and GD+momentum:
\item \textbf{Logistic regression} (log loss) simulation with $n=500$ samples and $p=11$ features, where $\thetav^{\ast}=(-5,-4,\ldots0,\ldots,4,5)^T$, and $\Xmat \sim \normal(\zero, \Sigma)$ for $\Sigma=\id$ (i.i.d.) or $\Sigma_{i,j}=0.99^{|i-j|}$ (corr. features)
\item To simulate response, we set $\yi \sim \mathcal{B}(\pi^{(i)}), \pi^{(i)} = \frac{1}{1+e^{-(\xv^{(i)})^T\thetav^{\ast}}}$
\item Indep. features result in a condition number of $\approx 2.9$ while corr. features yield (moderately) bad condition number $\approx 600$
\item ERM has unique global minimizer (convexity) but no closed-form solution. We can approximate $\thetavh$ using \texttt{glm} solution
\item We also track the optimization error $|| \thetav - \thetavh ||_2$
\item For relaxed NR we use $\alpha=0.7$ and for GD we set $\alpha=1$, momentum to $0.8$ and use no step size control
\end{framei}


\begin{framei}{Logistic Regression (GD variants recap)}
\item Recall comparison of GD variants on log. reg. in last chapter:
\imageC[0.8]{figure_man/GD_log_med_lr_iters.pdf}
\imageC[0.8]{figure_man/GD_log_coef_med.pdf}
{\footnotesize Dotted lines indicate global minimizers.}
\item \textbf{GD+momentum} was fastest $\Rightarrow$ now compare w/ Newton-Raphson
\item \textbf{NB}: GD+momentum only converges after several thousand steps
\end{framei}


\begin{framei}[fs=footnotesize]{Logistic Regression (GD vs. NR)}
\item Let's run GD vs. NR for \textbf{$1000$ steps} (independent features):
\imageC[0.7]{figure_man/simu-newton/NR_GD_log_indep_1000iters.pdf}
\imageC[0.7]{figure_man/simu-newton/NR_GD_log_coef_1000indep.pdf}
Dotted lines indicate global minimizers.
\item \textbf{NR} and \textbf{relaxed NR} $\Rightarrow$ almost instantaneous convergence (see optimization error)
\item Using $\alpha<1$ slightly slows down \textbf{relaxed NR}
\item \textbf{GD+mom} several orders of magnitude slower than NR
\end{framei}


\begin{framei}[fs=footnotesize]{Logistic Regression (GD vs. NR)}
\item Let's run the same configuration only for \textbf{$50$ steps} to see clearer picture:
\imageC[0.8]{figure_man/simu-newton/NR_GD_log_indep_50iters.pdf}
\imageC[0.8]{figure_man/simu-newton/NR_GD_log_coef_50indep.pdf}
Dotted lines indicate global minimizers.
\item \textbf{NR} takes $\approx 10$ steps to reach same optimization error as \textbf{GD+mom} after $20,000$ steps!
\item \textbf{Relaxed NR} with $\alpha<1$ shows no advantage here
\end{framei}


\begin{framei}[fs=footnotesize]{Runtime comparison (indep.)}
\item Clearly, NR makes more progress than GD per iteration
\item OTOH Newton steps are much more expensive than GD updates
\item $\Rightarrow$ How do NR and GD compare wrt runtime instead of iterations (50 steps)?
\imageC[0.9]{figure_man/simu-newton/NR_GD_runtime_comparison.pdf}
\item[] \textbf{Observations:}
\item \textbf{NR} steps are indeed slower than \textbf{GD} steps ($\approx 3\times$ here)
\item But each NR step is so much better than GD ($\approx 2000 \times$) that per-iteration runtime advantage of GD becomes \textbf{irrelevant}
\end{framei}


\begin{framei}{Logistic Regression (corr.)}
\item In case of correlated features the results are very similar:
\imageC[0.8]{figure_man/simu-newton/NR_GD_log_indep_50iters_corr.pdf}
\imageC[0.8]{figure_man/simu-newton/NR_GD_log_coef_50indep_corr.pdf}
{\footnotesize Dotted lines indicate global minimizers.}
\item \textbf{NR}'s performance is unaffected by feature correlation
\item \textbf{GD} iterates become ``warped'' compared to before
\end{framei}


\begin{framei}{Runtime comparison (corr.)}
\item Previous conclusions on runtime comparison for independent features carry over to correlated feature case:
\imageC[0.95]{figure_man/simu-newton/NR_GD_runtime_comparison_corr.pdf}
\item[] \textbf{Observations:}
\item \textbf{NR} steps are indeed slower than \textbf{GD} steps ($\approx 4\times$ here)
\item Overall \textbf{NR} is strongly superior to \textbf{GD} wrt optim error and speed
\end{framei}

\endlecture
\end{document}
