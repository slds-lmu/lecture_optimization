\documentclass[11pt,compress,t,notes=noshow, xcolor=table]{beamer}

\input{../../style/preamble}
\input{../../latex-math/basic-math}
\input{../../latex-math/basic-ml}


\newcommand{\titlefigure}{figure_man/NR_2.png}
\newcommand{\learninggoals}{
\item Newton-Raphson
\item Limitations
}


%\usepackage{animate} % only use if you want the animation for Taylor2D

\title{Optimization in Machine Learning}
%\author{Bernd Bischl}
\date{}

\begin{document}

\lecturechapter{Second order methods: Newton-Raphson}
\lecture{Optimization in Machine Learning}
\sloppy
%%%%%%%%%%%%%%%%%%%%%%%%%%%%%%%%%%%%%%%%%%%%%%%%%%%%%%%%%%%%%%%%%%%%%%%%%%%%%%%%%%%

\begin{vbframe}{From first to second order methods}

\begin{itemize}
    \setlength{\itemsep}{1em}
    \item So far: \textbf{First order methods} \\
        $\Rightarrow$ \textit{Gradient} information, i.e., first derivatives 
    \item Now: \textbf{Second order methods} \\
        $\Rightarrow$ \textit{Hessian} information, i.e., second derivatives
\end{itemize}

\end{vbframe}

\begin{vbframe}{Newton-Raphson}

\textbf{Assumption:} $f \in \mathcal{C}^2$

\medskip

\textbf{Aim:} Find stationary point $\xv^\ast$, i.e., $\nabla f(\xv^\ast) = \mathbf{0}$

\medskip

\textbf{Idea:} Find root of first order Taylor approximation of $\nabla f(\xv)$:
\begin{align*}
    \nabla f(\xv) \approx \nabla f(\xv^{[t]}) +
    \nabla^2 f(\xv^{[t]})(\xv - \xv^{[t]}) &= \mathbf{0} \\
    \nabla^2 f(\xv^{[t]})(\xv - \xv^{[t]}) &= - \nabla f(\xv^{[t]}) \\
    \xv^{[t+1]} &= \xv^{[t]} - \left(\nabla^2 f(\xv^{[t]})\right)^{-1}\nabla f(\xv^{[t]})
\end{align*}

\textbf{Update scheme:}
\begin{equation*}
    \xv^{[t+1]} = \xv^{[t]} + \mathbf{d}^{[t]}
\end{equation*}
with $\mathbf{d}^{[t]} = - \left(\nabla^2 f(\mathbf{ x}^{[t]})\right)^{-1}\nabla f(\xv^{[t]})$

\framebreak

\textbf{Note: } In practice, we get $\mathbf{d}^{[t]}$ by solving the linear system
\begin{equation*}
    \nabla^2 f(\xv^{[t]})\mathbf{d}^{[t]} = - \nabla f(\xv^{[t]})
\end{equation*}
with direct (matrix decompositions) or iterative methods. 

\medskip

\textbf{Relaxed/Damped Newton-Raphson:} Use step size $\alpha > 0$ with
\begin{equation*}
    \xv^{[t+1]} = \xv^{[t]} + \alpha \mathbf{d}^{[t]}
\end{equation*}
to satisfy Wolfe conditions (or just Armijo rule)

% \begin{vbframe}{Newton-Raphson}
% Der Newton-Raphson-Algorithmus benutzt als \textbf{Abstiegsrichtung}
% $$
% \mathbf{d}_{i} = -(\nabla^2 f(\xv_{i}))^{-1} \nabla f(\xv_{i})
% $$
% Dies kann auf mehrere Arten motiviert werden:
% \begin{itemize}
% \item Man löst die Gleichung $\nabla f(\xv) = \mathbf{0}$, indem man den Gradienten durch
% eine Taylorreihe erster Ordnung approximiert:
% $$
% \nabla f(\xv) \approx \nabla f(\xv_{i}) +
% \nabla^2 f(\xv_{i})(\xv - \xv_{i}) = \mathbf{0}.
% $$
% \item Man adjustiert die Richtung $-\nabla f(\xv_{i})$ vom ``steilsten Abstieg'' an die lokale
% Krümmung $\nabla^2 f(\xv_{i})$.
% \end{itemize}
% Im Vergleich zum \enquote{steilsten Abstieg}: Newton-Raphson divergiert ebenfalls leicht, hat aber quadratische
% Konvergenz nahe beim Minimum.

\end{vbframe}

\begin{vbframe}{Analytical example with quadratic form}

\begin{equation*}
    f(x_1, x_2) = x_1^2 + \frac{x_2^2}{2}
\end{equation*}

Update direction: $\mathbf{d}^{[t]} = -\left( \nabla^2 f(x_1^{[t]}, x_2^{[t]}) \right)^{-1} \nabla f(x_1^{[t]}, x_2^{[t]})$

\begin{equation*}
    \nabla f(x_1, x_2) = \begin{pmatrix}2x_1 \\ x_2\end{pmatrix}, \quad
    \nabla^2 f(x_1, x_2) = \begin{pmatrix}2 & 0 \\ 0 & 1\end{pmatrix}
\end{equation*}

First step:
\begin{align*}
    \begin{pmatrix}
        x_1^{[1]} \\ x_2^{[1]}
    \end{pmatrix} &= \begin{pmatrix}
        x_1^{[0]} \\ x_2^{[0]}
    \end{pmatrix} + \mathbf{d}^{[0]} = \begin{pmatrix}
        x_1^{[0]} \\ x_2^{[0]}
    \end{pmatrix} - \begin{pmatrix}
    1/2 &  0 \\ 0 & 1
    \end{pmatrix} \begin{pmatrix}
        2x_1^{[0]} \\ x_2^{[0]}
    \end{pmatrix} \\
    &= \begin{pmatrix}
        x_1^{[0]} \\ x_2^{[0]}
    \end{pmatrix} + \begin{pmatrix}
        -x_1^{[0]} \\
        -x_2^{[0]}
    \end{pmatrix} = \mathbf{0}
\end{align*}

\textbf{Note:} Newton-Raphson only needs one iteration for quadratic forms

\end{vbframe}

\begin{vbframe}{Newton-Raphson vs. GD on Branin function}

\begin{figure}
    \centering
    \includegraphics[width=0.45\textwidth]{slides/05-multivariate-second-order/figure_man/NR_1.png} ~~ \includegraphics[width=0.45\textwidth]{slides/05-multivariate-second-order/figure_man/NR_2.png}
    \caption*{\centering \textcolor{red}{Red}: Newton-Raphson.
        \textcolor{green}{Green}: Gradient descent.
        
        Newton-Raphson has much better convergence speed here.}
\end{figure}

\end{vbframe}

\begin{vbframe}{Discussion}

\textbf{Advantage:}

\begin{itemize}
    \item For $f$ sufficiently smooth:
        \begin{framed}
            \centering
            Newton-Raphson converges \textit{locally} quadratically \\
            (i.e., for starting points close enough to stationary point)
        \end{framed}
\end{itemize}

\lz

\textbf{Disadvantage:}

\begin{itemize}
    \setlength{\itemsep}{1em}
    \item For \enquote{bad} starting points:
        \begin{framed}
            \centering
            Newton-Raphson may diverge
        \end{framed}
\end{itemize}

\end{vbframe}

\begin{vbframe}{Limitations}

\textbf{Problem 1:} In general, $\mathbf{d}^{[t]}$ is not a descent direction

\vspace*{-0.75\baselineskip}

\begin{figure}
    \centering
    \includegraphics[width=0.5\textwidth]{slides/05-multivariate-second-order/figure_man/NR_2.png}
\end{figure}

\begin{footnotesize}
\textbf{But}: If Hessian is positive definite, $\mathbf{d}^{[t]}$ is descent direction:

\begin{equation*}
    \nabla f(\xv^{[t]})^\top \mathbf{d}^{[t]} = - \nabla f(\xv^{[t]})^\top \left(\nabla^2 f(\xv^{[t]})\right)^{-1} \nabla f(\xv^{[t]}) < 0
\end{equation*}

Near minimum, Hessian is positive definite.
For initial steps, Hessian is often not positive definite and Newton-Raphson may give non-descending update directions
\end{footnotesize}

\framebreak

\textbf{Problem 2:} Hessian can be \textbf{computationally expensive} to calculate, since descent direction $\mathbf{d}^{[t]}$ is the solution of the linear system
\begin{equation*}
    \nabla^2 f(\xv^{[t]}) \mathbf{d}^{[t]} = - \nabla f(\xv^{[t]}).
\end{equation*}

%can be numerically unstable.

\lz

\textbf{Aim}: Find quasi-second order methods not relying on exact Hessians
\begin{itemize}
    \item Quasi-Newton method
    \item Gauss-Newton algorithm (for least squares)
\end{itemize}

\end{vbframe}


% \begin{vbframe}{Newton-Raphson mit Backtracking}
% Das Hauptproblem bei der globalen Divergenz ist, dass man auch in
% Abstiegsrichtung das nächstgelegene Minimum überspringen kann und
% dann $f(\xv^{(i + 1)}) > f(\xv^{(i)})$ gilt.
% Eine zuverlässige Methode um dieses Problem zu beheben ist die {\em
% Richtung} von $\mathbf{d}$ zu behalten, aber die Schrittweite zu reduzieren:
% \begin{enumerate}
% \item $\mathbf{d}$ berechnen und den vollen Schritt $\xv^{(i + 1)} = \xv^{(i)} + \mathbf{d}$
% versuchen. Den Punkt behalten, falls $f(\xv^{(i + 1)}) < f(\xv^{(i)})$.
% \item Falls nicht, teste Aktualisierungen mit
% \[\xv^{(i + 1)} = \xv^{(i)} + \lambda \mathbf{d} \ \ \ \ \ \ \ \ \ \ \ 0 < \lambda < 1\]
% mit abnehmender Schrittgröße $\lambda$ bis
% \[f(\xv^{(i + 1)}) < f(\xv^{(i)}) - \epsilon\]
% mit vorgegebenem $\epsilon$.
% \end{enumerate}
% Möglichkeiten für die Wahl von $\lambda$:
% \begin{itemize}
% \item Reduziere $\lambda$ um einen konstanten Faktor $c < 1$ bis Verbesserung der Lösung erzielt wurde.
% \item Löse das eindimensionale Optimierungsproblem
% $$
%   \min_\lambda g(\lambda)   \mbox{ wobei }  g(\lambda) = f(\xv^{(i)} + \lambda \mathbf{d}).
% $$
% % wird in der Praxis eher nicht gemacht (zu zeitaufwendig).
% \item  Minimiere quadratische Funktion basierend auf
% $
% g(0), g(1) \mbox{ und } g'(0)
% $
% (die drei Werte sind ja bekannt).
% \end{itemize}
% \end{vbframe}

\endlecture
\end{document}



