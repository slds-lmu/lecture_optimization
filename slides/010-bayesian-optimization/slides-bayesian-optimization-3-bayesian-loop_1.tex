\documentclass[11pt,compress,t,notes=noshow, xcolor=table]{beamer}

\input{../../style/preamble}
\input{../../latex-math/basic-math}
\input{../../latex-math/basic-ml}
\input{../../latex-math/ml-mbo}

\newcommand{\titlefigure}{figure_man/bayesian_loop_sm_normal.png}
\newcommand{\learninggoals}{
\item Bayesian surrogate modeling
\item Acquisition functions
\item Lower confidence bound
}

\title{Optimization in Machine Learning}
%\author{Bernd Bischl}
\date{}

\begin{document}

\lecturechapter{Bayesian Optimization:\\ Posterior Uncertainty and Acquisition Functions I}
\lecture{Optimization in Machine Learning}

\begin{vbframe}{Bayesian Surrogate Modeling}

\textbf{Goal:}

Find trade-off between \textbf{exploration} (areas we have not visited yet) and \textbf{exploitation} (search around good design points)

\vspace{+.45cm}

\begin{center}
  \includegraphics[width = 0.5\textwidth]{figure_man/bayesian_loop_ee.png}
\end{center}

\framebreak 

\begin{itemize}
\item \textbf{Idea}: Use a \textbf{Bayesian approach} to build SM that yields estimates for the posterior mean $\fh(\xv)$ and the posterior variance $\sh^2(\xv)$
\item  $\sh^2(\xv)$ expresses \enquote{confidence}/\enquote{certainty} in prediction
% \item The High the more observations there are in region
\end{itemize}
\vspace{+.45cm}
\begin{minipage}[b]{0.45\textwidth}
  \includegraphics[width = \textwidth]{figure_man/bayesian_loop_ee.png}
\end{minipage}
\hfill
\begin{minipage}[b]{0.45\textwidth}
  \includegraphics[width = \textwidth]{figure_man/bayesian_loop_sm.png}
\end{minipage}

\framebreak

\begin{itemize}
\item Denote by $Y~|~\xv, \Dt$ the (conditional) RV associated with the posterior predictive distribution of a new point $\xv$ under a SM;\\will abbreviate it as $Y(\xv)$
\item Most prominent choice for a SM is a \textbf{Gaussian process}, here $Y(\xv) \sim \mathcal{N}\left(\fh(\xv), \sh^2(\xv)\right)$
\end{itemize}
\vspace{-1em}
\begin{center}
  \includegraphics[width = 0.5\textwidth]{figure_man/bayesian_loop_sm_normal.png}
\end{center}
\begin{footnotesize}
    For now we assume an interpolating SM; $\fh(\xv) = f(\xv)$ and $\sh(\xv) = 0$ for training points\\
\end{footnotesize}

\end{vbframe} 

\begin{vbframe}{Acquisition Functions}

To sequentially propose new points based on the SM, we make use of so-called acquisition functions $a: \mathcal{S} \to \R$\\

\vspace{1em}

Let $\fmin \coloneqq \min \left\{f(\xvsi[1]), \ldots, f(\xvsi[t])\right\}$ denote the best observed value so far (visualized in green - we will need this later!)
\vspace*{-0.2cm}

\begin{center}
  \includegraphics[width = 0.5\textwidth]{figure_man/bayesian_loop_sm_fmin.png}
\end{center}

\vspace*{-0.3cm}

In the examples before we simply used the posterior mean $a(\xv) = \fh(\xv)$ as acquisition function - ignoring uncertainty

\end{vbframe}

\begin{vbframe}{Lower Confidence Bound}

\textbf{Goal}: Find $\xvsi[t+1]$ that minimizes the \textbf{Lower Confidence Bound} (LCB):

$$
  a_{\text{LCB}}(\xv) = \fh(\xv) - \tau \sh(\xv)
$$

where $\tau > 0$ is a constant that controls the \enquote{mean vs. uncertainty} trade-off\\

\vspace{1em}

The LCB is conceptually very simple and does \textbf{not} rely on distributional assumptions of the posterior predictive distribution under a SM

\framebreak

$\tau = 1$

\begin{center}
  \includegraphics[width = 0.6\textwidth]{figure_man/bayesian_loop_lcb_0.png}
\end{center}

Top: Design points and SM showing $\fh(\xv)$ (blue) and $\fh(\xv) - \tau \sh(\xv)$ (red)\\
Bottom: the red point depicts $\argmin_{\xv \in \mathcal{S}} a_{\text{LCB}}(\xv)$

\framebreak

$\tau = 5$

\begin{center}
  \includegraphics[width = 0.6\textwidth]{figure_man/bayesian_loop_lcb_1.png}
\end{center}

\framebreak

$\tau = 10$

\begin{center}
  \includegraphics[width = 0.6\textwidth]{figure_man/bayesian_loop_lcb_2.png}
\end{center}

\end{vbframe}

\endlecture

\end{document}
