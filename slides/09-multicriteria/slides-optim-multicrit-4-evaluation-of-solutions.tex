\documentclass[11pt,compress,t,notes=noshow, xcolor=table]{beamer}

\input{../../style/preamble}
\input{../../latex-math/basic-math}
\input{../../latex-math/basic-ml}


\newcommand{\titlefigure}{figure_man/dominated_hypervolume.png}
\newcommand{\learninggoals}{
\item Quality indicators
\item $\epsilon$-indicator
\item Hypervolume indicator
}


%\usepackage{animate} % only use if you want the animation for Taylor2D

\title{Optimization in Machine Learning}
%\author{Bernd Bischl}
\date{}

\begin{document}

\lecturechapter{Evaluation of solutions}
\lecture{Optimization in Machine Learning}
\sloppy

%%%%%%%%%%%%%%%%%%%%%%%%%%%%%%%%%%%%%%%%%%%%%%%%%%%%%%%%%%%%%%%%%%%%%%%%%%%%%%%%%%%


\begin{vbframe}{Evaluation of solutions}


The result of a multi-objective algorithm is a set of candidates $\mathcal{P}$.

\lz

To evaluate the quality of this set or compare it to other sets, one needs a quantitative measure. This is usually achieved by the so-called \textbf{quality indicators}.

\lz

How well a single solution set represents the Pareto front can be divided into four qualities:

\begin{itemize}
\item \textbf{Convergence:} The proximity to the true Pareto front
\item \textbf{Spread:} The coverage of the Pareto front
\item \textbf{Uniformity:} The evenness of the distribution of the solutions
\item \textbf{Cardinality:} The number of solutions
\end{itemize}

\framebreak
Many quality indicators focus on one of the four mentioned qualities by mapping the approximation of the Pareto front to a real number representing the quality of a set of solutions.
\lz

Nonetheless, some quality indicators focus on all the four qualities. These quality indicators are divided into two groups:

\begin{itemize}
\item \textbf{Distance-based:} requiring the knowledge of the true Pareto front
\item \textbf{Volume-based:} measuring the space implied by a solution
\end{itemize}

\lz

\textbf{$\epsilon$-indicator} and \textbf{hypervolume indicator} are common examples for the distance-based and the volume-based approaches, respectively.


%\framebreak
\end{vbframe}


\begin{vbframe}{$\epsilon$-indicator}

\begin{itemize}
\item \textbf{$\epsilon$-indicator} is a binary, distance-based quality indicator, defined as: 
$$
\epsilon (\mathcal{P}, \mathcal{Q}):= \max_{q \in \mathcal{Q}} \min_{p \in \mathcal{P}} \max_{1 \leq i \leq m} \frac{p_{i}}{q_{i}}
$$
\small
where $\mathcal{P}$ and $\mathcal{Q}$ are two solution sets and $m$ is the number of objectives. 
\vspace{0.2cm}
\normalsize
\item $\epsilon$ denotes the factor needed to scale $\mathcal{Q}$, such that $\mathcal{Q}$ weakly dominates $\mathcal{P}$:
$$
\epsilon(\mathcal{P}, \mathcal{Q}) \leq 1 \Rightarrow \mathcal{P} \preceq \mathcal{Q}
$$
\small
where $\mathcal{P}$ is said to weakly dominate $\mathcal{Q}$, if every solution $p \in \mathcal{P}$ is weakly Pareto optimal regarding the combined solution set $\mathcal{P} \cup \mathcal{Q}$.

\vspace{0.2cm}
\normalsize
\item Note that $\epsilon$-indicator considers only the objective with the maximum difference. Hence, it might output similar results for substantially different approximations.

\end{itemize}

\end{vbframe}

\begin{vbframe}{Hypervolume indicator}

\begin{minipage}[t]{1\textwidth}
\begin{itemize}
\small
\item \textbf{Hypervolume indicator} is a volume-based quality indicator, defined as:
\footnotesize

\vspace{-0.2cm}
$$S(\mathcal{P}, R) = \lambda\left(\bigcup_{\tilde{\bm{x}} \in \mathcal{P}}\left\{\bm{x} | \tilde{\bm{x}} \prec \bm{x} \prec R\right\}\right),$$ 

\vspace{-0.1cm} 
where $\mathcal{P} \subset \Xspace$ and $\lambda$ is the Lebesgue measure.
\end{itemize}
\end{minipage}


\vspace{0.4cm}
\begin{minipage}[t]{0.65\textwidth}
\footnotesize
\begin{itemize}
            \item HV is calculated w.r.t the reference point $R$, which often reflects in each component the natural maximum of the respective objective -- if possible
            \item The dominated hypervolume is also often called \textbf{S-Metric}.
            \item Computation of HV scales exponentially in the number of objective functions $\mathcal{O}(n^{m-1})$.
            \item Fast approximations exist for small values of $m$, and especially for machine learning applications we rarely optimize $m > 3$ objectives.
    \end{itemize}
\end{minipage}
\begin{minipage}[t]{0.3\textwidth}
\begin{figure}
\includegraphics[width = 1\linewidth]{figure_man/dominated_hypervolume.png}
\end{figure}
\end{minipage}


%\vspace*{-0.4cm}

%The dominated hypervolume of the set of points $\mathcal{P} \subset \Xspace$ (here: 5 black points) is the area in the target function space (regarding a reference point $R$) which is dominated by points $\mathcal{P}$.

\end{vbframe}

\endlecture
\end{document}