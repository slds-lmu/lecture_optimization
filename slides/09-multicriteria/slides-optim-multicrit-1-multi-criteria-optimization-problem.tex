\documentclass[11pt,compress,t,notes=noshow, xcolor=table]{beamer}

\input{../../style/preamble}
\input{../../latex-math/basic-math}
\input{../../latex-math/basic-ml}


\newcommand{\titlefigure}{figure_man/price-per-night.png}
\newcommand{\learninggoals}{
\item Introduction
\item Definition
}


%\usepackage{animate} % only use if you want the animation for Taylor2D

\title{Optimization in Machine Learning}
%\author{Bernd Bischl}
\date{}

\begin{document}

\lecturechapter{Multi-criteria optimization problem}
\lecture{Optimization in Machine Learning}
\sloppy

%%%%%%%%%%%%%%%%%%%%%%%%%%%%%%%%%%%%%%%%%%%%%%%%%%%%%%%%%%%%%%%%%%%%%%%%%%%%%%%%%%%

\begin{vbframe}{Introductory example}

Often we want to solve optimization problems concerning several goals.

\vspace{0.5cm}
    \textbf{General applications:}
\begin{itemize}
\item Medicine: maximum effect, but minimum side effect of a drug.
\item Finances: maximum return, but minimum risk of an equity portfolio.
\item Production planning: maximum revenue, but minimum costs.
\item Booking a hotel: maximum rating, but minimum costs.
\end{itemize}

\vspace{0.5cm}
    \textbf{In machine learning:}
\begin{itemize}
\item Sparse models: maximum predictive performance, but minimal number of features.
\item Fast models: maximum predictive performance, but short prediction time.
\item ...
\end{itemize}

\framebreak

\textbf{Example}:

Choose the best hotel to stay at by maximizing ratings subject to a maximum price per night.

\vspace{0.5cm}
 \textbf{Problems}:

\footnotesize
\begin{itemize}
 \item The result depends on how we select the maximum price and usually returns different solutions for different maximum price values.
 \item We could also choose a minimum rating and optimize the price per night.
 \item The more objectives we optimize, the more difficult such a definition becomes.
\end{itemize}

\normalsize
\vspace{0.5cm}
\textbf{Goal}:\\
\footnotesize
Find a more general approach to solve multi-criteria problems.\\

\framebreak

\normalsize

\begin{center}
\includegraphics[width = 0.45\linewidth]{figure_man/booking1.png} ~~~ \includegraphics[width = 0.45\linewidth]{figure_man/booking2.png}
\end{center}

When booking a hotel: find the hotel with

\begin{itemize}
\item minimum price per night (\textbf{costs}) and
\item maximum user rating (\textbf{performance}).
\end{itemize}

\vfill

\begin{footnotesize}
Since our standard is to minimize objectives, we minimize negative ratings.
\end{footnotesize}

\framebreak


The objectives often conflict with each other:

\begin{itemize}
\item Lower price $\to$ usually lower hotel rating.
\item Better rating $\to$ usually higher price.
\end{itemize}

Example: (negative) average rating by hotel guests (1 - 5) vs. average price per night (excerpt).

%\begin{center}
%\includegraphics[height = 4cm, width = 5cm]{figure_man/price-per-night.png}
%\end{center}


\vspace*{0.2cm}

\begin{center}
\includegraphics[width = 4.5cm, height = 4cm ]{figure_man/price-per-night.png}
\end{center}
%<<echo = F, fig.width = 3, fig.height=1.5, fig.align='center'>>=
%df = readRDS("rsrc/expedia_example.rds")
%
%p = ggplot(data = df, aes(x = mean_price, y = - mean_rating)) + geom_point(size = 1.5)
%p = p + theme_bw()
%p = p + ylim(c(- 5.5, -2))
%p = p + xlab("Price per night") + ylab("Rating")
%p
%@

\framebreak
\footnotesize
Often, objectives are not directly comparable as they are measured on different scales:
\begin{itemize}
\item Left: A hotel with rating $4$ for $89$ Euro ($\textcolor{green}{y^{(1)}} = \left(89, - 4.0\right)$) would be preferred to a hotel for $108$ Euro with the same rating ($\textcolor{red}{y^{(2)}} = \left(108, - 4.0\right)$).
\item Right: How to decide if $\textcolor{orange}{y^{(1)}} = \left(89, - 4.0\right)$ or $\textcolor{orange}{y^{(2)}} = \left(95, - 4.5\right)$ is preferred?
\item How much is one \textit{rating point} worth?
\end{itemize}

\vspace{0.5cm}
\normalsize

\begin{center}
\includegraphics[height=3.5cm, width =0.35\linewidth]{figure_man/Example1.png} ~~~ \includegraphics[height=3.5cm, width =0.35\linewidth]{figure_man/Example2.png}
\end{center}

%<<echo = F, fig.width = 5, fig.height=1.5, fig.align='center'>>=
%p1 = ggplot(data = df, aes(x = mean_price, y = - mean_rating)) + geom_point(size = 1.5)
%p1 = p1 + geom_point(data = df[16:17, ], aes(x = mean_price, y = - mean_rating), size = 2, colour = c("green", "red"))
%p1 = p1 + theme_bw()
%p1 = p1 + ylim(c(- 5.5, -2))
%p1 = p1 + xlab("Price per night") + ylab("Rating")
%
%p2 = ggplot(data = df, aes(x = mean_price, y = - mean_rating)) + geom_point(size = 2)
%p2 = p2 + geom_point(data = df[c(10, 16), ], aes(x = mean_price, y = - mean_rating), size = 2, colour = "orange")
%p2 = p2 + theme_bw()
%p2 = p2 + ylim(c(-5.5, -2))
%p2 = p2 + xlab("Price per night") + ylab("Rating")
%
%grid.arrange(p1, p2, ncol = 2)
%@

\end{vbframe}

\begin{vbframe}{multi-criteria optimization problem}
A \textbf{multi-criteria optimization problem} is defined by\\

$$
\min_{\bm{x} \in \Xspace} f(\bm{x}) \Leftrightarrow \min_{\bm{x} \in \Xspace} \left(f_1(\bm{x}),  f_2(\bm{x}), ..., f_m(\bm{x})\right),
$$

%Be $\Xspace \subset \R^n$ and $f: \Xspace \to \R^m$, $m \ge 2$. A \textbf{multi-criteria optimization problem} is defined by

%\begin{eqnarray*}
%\min_{\bm{x} \in \Xspace} & & f(\bm{x}) \\ \Leftrightarrow \min_{\bm{x} \in \Xspace} \left(f_1(\bm{x}), f_2(\bm{x}), ..., f_m(\bm{x})\right).
%\end{eqnarray*}
with $\Xspace \subset \R ^ n$ and multi-criteria objective function $f: \Xspace \to \R ^ m$, $m \ge 2$.

\begin{itemize}
\item \textbf{Goal:} minimize multiple target functions simultaneously.
\item $\left(f_1(\bm{x}), ..., f_m(\bm{x})\right)^\top$ maps each candidate $\bm{x}$ into the objecive space $\R^m$.
\item Often no clear best solution, as objective are usually conflicting and we cannot totally order in $\R^m$. 
% \item Objective functions are often conflicting.
\item W.l.o.g. we always minimize.
\item Alternative names: multi-criteria optimization, multi-objective optimization, Pareto optimization.
\end{itemize}

\end{vbframe}


\endlecture
\end{document}

