\documentclass[11pt,compress,t,notes=noshow, xcolor=table]{beamer}

\input{../../style/preamble}
\input{../../latex-math/basic-math}
\input{../../latex-math/basic-ml}


\newcommand{\titlefigure}{figure_man/posteriori1.png}
\newcommand{\learninggoals}{
\item Approaches
\item Procedures
}


%\usepackage{animate} % only use if you want the animation for Taylor2D

\title{Optimization in Machine Learning}
%\author{Bernd Bischl}
\date{}

\begin{document}

\lecturechapter{A-priori vs. A-posteriori}
\lecture{Optimization in Machine Learning}
\sloppy

%%%%%%%%%%%%%%%%%%%%%%%%%%%%%%%%%%%%%%%%%%%%%%%%%%%%%%%%%%%%%%%%%%%%%%%%%%%%%%%%%%%


\begin{vbframe}{A-priori vs. A-posteriori}

\begin{itemize}
\item The Pareto set is a set of equally optimal solutions.
\item In many applications, one is often interested in a \textbf{single} optimal solution.
\item Without further information, no unambiguous optimal solution can be determined. \\
$\to$ The decision must be based on other criteria.
\end{itemize}

\vspace{0.5cm}

There are two possible approaches:
\begin{itemize}
\item \textbf{A-priori approach}: User preferences are considered \textbf{before} the optimization process
\item \textbf{A-posteriori approach}: User preferences are considered \textbf{after} the optimization process
\end{itemize}

\end{vbframe}


\begin{vbframe}{A-priori procedure}

\textbf{Example: weighted total}\\
\textbf{Prior knowledge:} One rating point is worth $50$ Euro to a customer. \\
    $\to$ We optimize the weighted sum:

$$
\min_\text{Hotel} \text{(Price / Night)} - 50 \cdot \text{Rating}
$$


\begin{center}
\includegraphics[height=3.5cm, width =0.3\linewidth]{figure_man/priori1.png} ~~~ \includegraphics[height=3.5cm, width =0.35\linewidth]{figure_man/priori2.png}
\end{center}

%<<echo = F, fig.width = 4, fig.height=2, fig.align='center', out.height = '40%', out.width = '70%'>>=
%df$apriori = df$mean_price + 50 * df$mean_rating
%
%p1 = ggplot()
%p1 = p1 + geom_point(data = df, aes(x = apriori, y = 0), size = 2)
%p1 = p1 + geom_point(data = df[which.min(df$apriori), ], aes(x = apriori, y = 0), colour = "green", size = 2)
%p1 = p1 + theme_bw()
%p1 = p1 + xlab("Weighted sum")
%p1 = p1 + theme(axis.title.y = element_blank(),
%                axis.text.y = element_blank(),
%                axis.ticks.y = element_blank())
%
%p2 = ggplot(data = df, aes(x = mean_price, y = mean_rating)) + geom_point(size = 2)
%p2 = p2 + geom_point(data = df[which.min(df$apriori), ], aes(x = mean_price, y = mean_rating), size = 2, colour = "green")
%p2 = p2 + theme_bw()
%p2 = p2 + ylim(c(-5, -2))
%p2 = p2 + xlab("Price per night") + ylab("Rating")
%
%grid.arrange(p1, p2, ncol = 2)
%@

Alternative a weighted sum: $\min_{\bm{x} \in \Xspace} \sum_{i = 1}^m w_i f_i(\bm{x}) \qquad \text{with} \quad w_i \ge 0 $

\framebreak

\textbf{Example: Lexicographic method}\\
\textbf{Prior knowledge:} Customer prioritizes rating over price. \\
$\to$ Optimize target functions one after the other.

\lz

\begin{center}
\includegraphics[height=4cm, width =0.35\linewidth]{figure_man/priori3.png} ~~~ \includegraphics[height=4cm, width =0.35\linewidth]{figure_man/priori4.png}
\end{center}

%<<echo = F, fig.width = 5, fig.height=2, fig.align='center', out.height = '40%', out.width = '80%'>>=
%p1 = ggplot(data = df, aes(x = mean_price, y = mean_rating)) + geom_point(size = 2)
%p1 = p1 + geom_point(data = df[df$mean_rating == -5, ], aes(x = mean_price, y = mean_rating), size = 2, colour = "orange")
%p1 = p1 + theme_bw()
%p1 = p1 + ylim(c(-5, -2))
%p1 = p1 + ggtitle("1) max. rating")
%p1 = p1 + xlab("Price per night") + ylab("Rating")
%
%p2 = p1 + geom_point(data = df[(df$mean_rating == - 5.0 & df$mean_price < 150), ], colour = "green", size = 2)
%p2 = p2 + ggtitle("2) min. price")
%
%grid.arrange(p1, p2, ncol = 2)
%@

\framebreak

A-priori approach: lexicographic method

\begin{eqnarray*}
y_1^* &=& \min_{\bm{x} \in \Xspace} f_1(\bm{x})\\
y_2^* &=& \min_{\bm{x} \in \{\bm{x} ~|~ f_1(\bm{x}) = y_1^*\}} f_2(\bm{x}) \\
y_3^* &=& \min_{\bm{x} \in \{\bm{x} ~|~ f_1(\bm{x}) = y_1^* \land f_2(\bm{x}) = y_2^*\}} f_3(\bm{x}) \\
&\vdots&
\end{eqnarray*}

\textbf{But:} Different sequences provide different solutions.

\framebreak

\textbf{Summary a-priori approach:}
\vspace{0.5cm}
\begin{itemize}
\item Implicit assumption: Single-objective optimization is \emph{easy}.
\item Only one solution is obtained, which depends on a-priori weights, order, etc.
\item Several solutions can be obtained if weights, order, etc. are systematically varied.
\item Usually, not all non-dominated candidates can be found by these methods.
\end{itemize}

\end{vbframe}


\begin{vbframe}{A-posteriori procedure}

A-posteriori methods try to

\begin{itemize}
\item find the set of \textbf{all} optimal candidates (the Pareto set),
\item select (if necessary) an optimal candidate based on prior knowledge or individual preferences.
\item Implicit assumption: Specifying your hidden preferences / making a selection from a pool of candidates is easier if you see the non-dominated solutions.
\end{itemize}

A-posteriori methods are therefore the more generic approach to solving a multi-criteria optimization problem.

\framebreak

\textbf{Example:} A user is displayed all Pareto optimal hotels (left) and chooses an optimal candidate (right) based on his hidden preferences or additional criteria (e.g. location of the hotel).

\vspace*{0.3cm}

\lz
\begin{center}
\includegraphics[height=4cm, width =0.35\linewidth]{figure_man/posteriori1.png} ~~~ \includegraphics[height=4cm, width =0.35\linewidth]{figure_man/posteriori2.png}
\end{center}

\end{vbframe}


\endlecture
\end{document}


