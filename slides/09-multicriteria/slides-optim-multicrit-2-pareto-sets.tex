\documentclass[11pt,compress,t,notes=noshow, xcolor=table]{beamer}

\input{../../style/preamble}
\input{../../latex-math/basic-math}
\input{../../latex-math/basic-ml}


\newcommand{\titlefigure}{figure_man/pareto-front.png}
\newcommand{\learninggoals}{
\item Definitions of optimality
\item Objective/Target functions
}


%\usepackage{animate} % only use if you want the animation for Taylor2D

\title{Optimization in Machine Learning}
%\author{Bernd Bischl}
\date{}

\begin{document}

\lecturechapter{Pareto sets und Pareto optimality}
\lecture{Optimization in Machine Learning}
\sloppy

%%%%%%%%%%%%%%%%%%%%%%%%%%%%%%%%%%%%%%%%%%%%%%%%%%%%%%%%%%%%%%%%%%%%%%%%%%%%%%%%%%%

\begin{vbframe}{Pareto sets und Pareto optimality}

\textbf{Definition:}

Given a multicriteria optimization problem $$\min_{\bm{x} \in \Xspace} f(\bm{x}) = \left(f_1(\bm{x}), ..., f_m(\bm{x})\right), \quad f_i: \Xspace \to \R^m.$$

\begin{itemize}
\item A candidate $\bm{x}$ \textbf{(Pareto-) dominates} $\bm{\tilde{x}}$, if $f(\bm{x}) \prec f( \tilde{\bm x})$, i.e.
\begin{enumerate}
\item $f_i(\bm{x}) \le f_i(\tilde{\bm x})$ for all $i \in \{1, 2, ..., m\}$ and
\item $f_j(\bm{x}) < f_j(\tilde{\bm x})$ for at least one $j \in \{1, 2, ..., m\}$
\end{enumerate}
\vspace*{0.1cm}
\item A candidate $\bm{x}^*$ that is not dominated by any other candidates is called \textbf{Pareto optimal}.
\vspace*{0.1cm}
\item The set of all Pareto optimal candidates is called \textbf{Pareto set} $\mathcal{P} := \{\bm{x} \in \Xspace |\not \exists ~\tilde{\bm{x}} \text{ with } f(\tilde{\bm{x}}) \prec f(\bm{x})\}$
\item $\mathcal{F} = f(\mathcal{P}) = \{f(\bm{x}) | \bm{x} \in \mathcal{P}\}$ is called \textbf{Pareto front}.
\end{itemize}

\end{vbframe}

\begin{vbframe}{How to define optimality?}
Let $y = (\text{price}, - \text{rating})$. For some cases it is \textit{clear} which point is the better one:
\begin{itemize}
    \item The candidate $\textcolor{green}{y^{(1)}} = \left(89, - 4.0\right)$ dominates $\textcolor{red}{y^{(2)}} = \left(108, - 4.0\right)$: $\textcolor{green}{y^{(1)}}$ is not worse in any dimension and is better in one dimension. Therefore, $\textcolor{red}{y^{(2)}}$ gets \textbf{dominated} by $\textcolor{green}{y^{(1)}}$
$$
\textcolor{red}{y^{(2)}} \prec \textcolor{green}{y^{(1)}}.
$$
\end{itemize}

\begin{center}
\includegraphics[width = 4.5cm, height = 4cm ]{figure_man/Example1.png}
\end{center}

\framebreak

For the points $\textcolor{orange}{y^{(1)}} = \left(89, - 4.0\right)$ and $\textcolor{orange}{y^{(2)}} = \left(95, - 4.5\right)$ we cannot say which one is better.

\begin{itemize}
\item We define the points as \textbf{equivalent} and write
$$
\textcolor{orange}{y^{(1)}} \not\prec \textcolor{orange}{y^{(2)}} \text{ and } \textcolor{orange}{y^{(2)}} \not\prec \textcolor{orange}{y^{(1)}}.
$$

\end{itemize}

\begin{center}
\includegraphics[width = 4.5cm, height = 4cm ]{figure_man/Example2.png}
\end{center}

\framebreak

\begin{itemize}
\item The set of all equivalent points that are not dominated by another point is called the \textbf{Pareto front}.



\vspace*{0.3cm}

\lz
\begin{center}
\includegraphics[width = 5cm, height = 5cm ]{figure_man/pareto.png}
\end{center}

\end{itemize}

\end{vbframe}


\begin{vbframe}{Example: One objective function}

We consider the minimization problem

$$
\min_{0 \le \bm{x} \le 3} f(\bm{x}) = (\bm{x} - 1)^2, \qquad 0 \le \bm{x} \le 3.
$$

The optimum is at $\bm{x}^* = 1$.

\lz

\begin{center}
\includegraphics[width = 4cm, height = 4cm ]{figure_man/one-obj-func.png}
\end{center}

\end{vbframe}


\begin{vbframe}{Example: Two target functions}

We extend the above problem to two objective functions $f_1(\bm{x}) = (\bm{x} - 1)^2$ and $f_2(\bm{x}) = 3(\bm{x} - 2)^2$, thus

$$
\min_{0 \le \bm{x} \le 3} f(\bm{x}) := \left(f_1(\bm{x}), f_2(\bm{x})\right), \qquad 0 \le \bm{x} \le 3.
$$

\vspace*{0.1cm}

\begin{center}
\includegraphics[width = 4cm, height = 4cm ]{figure_man/two-target-func.png}
\end{center}


\framebreak

We consider the functions in the objective function space $f(\Xspace)$ by drawing the objective function values $\left(f_1(\bm{x}), f_2(\bm{x})\right)$ for all $0 \le \bm{x} \le 3$.

\vspace*{0.1cm}

\begin{center}
\includegraphics[width = 4cm, height = 4cm ]{figure_man/pareto-front.png}
\end{center}

\vspace*{-0.3cm}

The Pareto front is shown in green.
The Pareto front cannot be \emph{left} without getting worse in at least one objective function.

\end{vbframe}


% \begin{vbframe}{Zusammenfassung}
%
% Bisher:
% \begin{itemize}
% \item Multikriterielle Optimierungsprobleme sind Probleme, in denen mehrere Funktionen gleichzeitig optimiert werden sollen.
% \item Zielfunktionen sind häufig konkurrierend und lassen sich nicht auf derselben Skala vergleichen
% \item $\to$ Einführung eines geeigneten Optimalitätsbegriffs: Pareto-Optimalität
% \end{itemize}
%
% \begin{itemize}
% \item Wie findet man eine Pareto-Front?
% \item $\to$ Lösungsverfahren
% \end{itemize}
%
% \end{vbframe}

%\section{Solvers}



\endlecture
\end{document}


