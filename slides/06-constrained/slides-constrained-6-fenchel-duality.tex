\documentclass[11pt,compress,t,notes=noshow, xcolor=table]{beamer}

\input{../../style/preamble}
\input{../../latex-math/basic-math}
\input{../../latex-math/basic-ml}

\title{Optimization in Machine Learning}

\begin{document}

\titlemeta{
Constrained Optimization
}{
Other forms of duality
}{
figure_man/fenchel.png
}{
\item Dual norms
\item Conjugate functions
\item Fenchel duality
\item Examples in statistics 
}

\begin{framei}{Constrained minimization and dual norms}
\item Consider norm minimization under linear constraints in its primal form:
$$
\min_{\xv \in \R^d} \| \xv \| \quad \text{s.t. } \bm{G} \xv = \bm{h}
$$
\item $\| \cdot \|$ is some norm function
\item If the norm is the $L_1$ norm, this is the famous \href{https://en.wikipedia.org/wiki/Basis_pursuit}{\beamergotobutton{basis pursuit}} problem
\vfill
\item \textbf{Question:} Is there a more straightforward way to solve constrained optimization problems involving norms?
\end{framei}


\begin{framei}{Constrained minimization and dual norms}
\item The concept of the \textbf{dual norm} from functional analysis can help
\item[Definition] Let $\| \xv \|$ be the norm of $\xv$. The dual norm $\| \xv \|_*$ is defined as
$$
\| \xv \|_* = \max_{\|\bm{z}\| \le 1} \bm{z}^\top \xv
$$
\item If $\| \xv \|$ is a norm and $\| \xv \|_*$ its dual norm, then
$$
| \bm{z}^\top \xv | \le \| \bm{z} \| \| \xv \|_*
$$
\vfill
\item[Example] The dual norm of the $L_p$ norm $\| \cdot \|_p$ is the $L_q$ norm $\| \cdot \|_q$ where $1/p + 1/q = 1$
\end{framei}

\endlecture
\end{document}
