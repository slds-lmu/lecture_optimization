\documentclass[11pt,compress,t,notes=noshow, xcolor=table]{beamer}

\input{../../style/preamble}
\input{../../latex-math/basic-math}
\input{../../latex-math/basic-ml}


\newcommand{\titlefigure}{figure_man/fenchel.png}
\newcommand{\learninggoals}{
\item Dual norms
\item Conjugate functions
\item Fenchel duality
\item Examples in statistics
}


%\usepackage{animate} % only use if you want the animation for Taylor2D

\title{Optimization in Machine Learning}
%\author{Bernd Bischl}
\date{}

\begin{document}

\lecturechapter{Other forms of duality}
\lecture{\inserttitle}
\sloppy
%%%%%%%%%%%%%%%%%%%%%%%%%%%%%%%%%%%%%%%%%%%%%%%%%%%%%%%%%%%%%%%%%%%%%%%%%%%%%%%%%%%
\begin{vbframe}{Constrained minimization and dual norms}

Consider the problem of norm minimization under linear constraints
in its primal form:

\begin{eqnarray*}
 \min_{\xv \in \R^d} && \| \bm{x} \| \\
\text{s.t. } &&  \bm{G} \bm{x} = \bm{h}\,,\\
\end{eqnarray*}

where $\| \cdot \|$ is some norm function. For instance, if the norm is the $L1$ norm, this problem is the famous  \href{https://en.wikipedia.org/wiki/Basis_pursuit
}{\beamergotobutton{basis pursuit}} problem.

\lz

\textbf{Question:} Is there a more straightforward way to solve constrained optimization problems involving norms?

\framebreak

Here, the concept of the dual norm from functional analysis can be helpful.


\textbf{Definition:}  Let $\| \bm{x} \|$ be the norm of $\bm{x}$. Then the dual norm $\| \bm{x} \|_*$ is defined as 

\begin{equation*}
 \| \bm{x} \|_* = \max_{\|\bm{z}\| \le 1} \bm{z}^T\bm{x}  
\end{equation*}

Using this definition, one can show that if $\| \bm{x} \|$ is a norm and $\| \bm{x} \|_*$ is the dual norm
of it, $\| \bm{z}^T \bm{x} \| \le \| \bm{z} \| \| \bm{x} \|_{*}$ holds.

\lz

\textbf{Examples:} The dual norm of the Lp norm $\| \cdot \|_p$ is the
Lq norm $\| \cdot \|_q$ where it holds that $1/p + 1/q = 1$. 


\framebreak

\end{vbframe}

\begin{vbframe}{Constrained problems and conjugate functions}



\end{vbframe}


\endlecture
\end{document}
