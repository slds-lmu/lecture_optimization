\documentclass[11pt,compress,t,notes=noshow, xcolor=table]{beamer}

\input{../../style/preamble}
\input{../../latex-math/basic-math}
\input{../../latex-math/basic-ml}

\title{Optimization in Machine Learning}

\begin{document}

\titlemeta{
Constrained Optimization
}{
Algorithms for linear programs
}{
figure_man/convex_programs.png
}{
\item Understand the Simplex algorithm for solving LPs
\item Know the two-phase approach for finding a starting point
\item Understand how Simplex traverses polytope corners
}

\begin{framei}{Simplex algorithm}
\item Most important method for solving linear programs
\item Published in 1947 by Georg Dantzig
\vfill
\item[Basic idea:] Start from an arbitrary corner of the polytope.\\
Run along edges as long as the solution improves. Find a new edge, ...
\vfill
\item[Output:] A path along corners of the polytope ending at the optimum
\vfill
\item Since LP is a \textbf{convex} optimization problem, the optimal corner found is also a global optimum
\end{framei}


\begin{framev}{Simplex algorithm}
\imageC[0.6]{figure_man/simplex.png}
\end{framev}


\begin{framei}{Simplex algorithm}
\item The Simplex algorithm consists of two phases:
\begin{itemize}
\item \textbf{Phase I:} Determination of a \textbf{starting point}
\item \textbf{Phase II:} Determination of the \textbf{optimal solution}
\end{itemize}
\vfill
\item In \textbf{Phase I}, a feasible corner $\bm{x}_0$ must be found first
\vfill
\item In \textbf{Phase II}, this solution is iteratively improved by searching for an edge that improves the objective and running along it to the next corner
\end{framei}


\begin{framei}[fs=small]{Simplex -- Phase I}
\item Find starting point $\bm{x}_0$ by solving auxiliary LP with artificial variables $\bm{\epsilon}$:
$$
\min_{\epsilon_1, \ldots, \epsilon_m} \sum_{i = 1}^m \epsilon_i \quad \text{s.t. } \bm{Ax} + \bm{\epsilon} \ge \bm{b}, \; \epsilon_1, \ldots, \epsilon_m \ge 0, \; \bm{x} \ge 0
$$
\item A feasible starting point for the auxiliary problem is $\bm{x} = \bm{0}$ and $\epsilon_i = \begin{cases} 0 & \text{if } b_i < 0 \\
b_i & \text{if } b_i \ge 0
\end{cases}$
\item We then apply Phase II of Simplex to the auxiliary problem
\item If the original problem is feasible, the optimal solution \textbf{must} be $\bm{\epsilon} = (0, \ldots, 0)$ (all artificial variables disappear)
\item If we find $\bm{\epsilon} = \bm{0}$, we have a valid starting point
\item Otherwise, the original problem has no feasible solution
\end{framei}


\begin{framei}{Simplex -- Example}
\item Consider the following LP:
$$
\min_{\bm{x} \in \R^2} -x_1 - x_2
$$
$$
\text{s.t. } x_1 - x_2 \ge -0.5, \; -x_1 - 2x_2 \ge -2, \; -2x_1 - x_2 \ge -2, \; -x_1 + x_2 \ge -0.5, \; \bm{x} \ge 0
$$
\item Starting point is the corner $\bm{(0, 0)}$
\end{framei}


\begin{framev}{Simplex -- Example}
\imageC[0.6]{figure_man/simplex_implementation/iter1.png}
\end{framev}


\begin{framev}{Simplex -- Example}
\imageC[0.6]{figure_man/simplex_implementation/iter2.png}
\end{framev}


\begin{framev}{Simplex -- Example}
\imageC[0.6]{figure_man/simplex_implementation/iter3.png}
\end{framev}


\begin{framev}{Simplex -- Example}
\imageC[0.6]{figure_man/simplex_implementation/iter4.png}
\end{framev}

\endlecture
\end{document}
