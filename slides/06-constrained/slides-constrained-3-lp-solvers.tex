\documentclass[11pt,compress,t,notes=noshow, xcolor=table]{beamer}

\input{../../style/preamble}
\input{../../latex-math/basic-math}
\input{../../latex-math/basic-ml}


\newcommand{\titlefigure}{figure_man/convex_programs.png}
\newcommand{\learninggoals}{
\item Definition
\item Max. Likelihood 
\item Normal regression
\item Risk Minimization
}


%\usepackage{animate} % only use if you want the animation for Taylor2D

\title{Optimization in Machine Learning}
%\author{Bernd Bischl}
\date{}

\begin{document}

\lecturechapter{Algorithms for linear programs}
\lecture{\inserttitle}
\sloppy
%%%%%%%%%%%%%%%%%%%%%%%%%%%%%%%%%%%%%%%%%%%%%%%%%%%%%%%%%%%%%%%%%%%%%%%%%%%%%%%%%%%
\begin{vbframe}{Simplex algorithm}

The Simplex algorithm is the most important method for solving Linear programming. It was published in 1947 by Georg Dantzig.

\lz

\textbf{Basic idea:} start from an arbitrary corner of the polytope. Run along this edge as long as the solution improves. Find a new edge, ...

\lz

\textbf{Output:} a path along the corners of the polytope that ends at the optimal point of the polytope.

\lz

Since linear programming is a \textbf{convex} optimization problem, the optimal corner found in this way is also a global optimum.

\framebreak

\begin{center}
\includegraphics[width = 0.6\textwidth]{figure_man/simplex.png}
\end{center}

\framebreak

The simplex algorithm can be divided into two steps:

\begin{itemize}
\item \textbf{Phase I:} determination of a \textbf{starting point}
\item \textbf{Phase II:} determination of the \textbf{optimal solution}
\end{itemize}

To be able to start, a starting point must first be found in \textbf{Phase I}, i.e. a feasible corner $\bm{x}_0$.

\lz

In \textbf{phase II} this solution is iteratively improved by searching for an edge that improves the solution and running along it to the next corner.

\framebreak

\textbf{Phase I}:

One way to find a starting point $\bm{x}_0$ is to solve a auxiliary linear problem with artificial variables $\bm{\epsilon}$:

\begin{eqnarray*}
\min_{\epsilon_1, ..., \epsilon_m} && \sum_{i = 1}^m \epsilon_i \\
\text{s.t. } && \bm{Ax} + \bm{\epsilon} \ge \bm{b} \\
&& \epsilon_1, ..., \epsilon_m \ge 0\\
&& \bm{x} \ge 0
\end{eqnarray*}

\begin{itemize}
\item A feasible starting point for the auxiliary problem is $\bm{x} = \bm{0}$ and $\epsilon_i = \begin{cases} 0 & \text{if } b_i < 0 \\
b_i & \text{if } b_i \ge 0
\end{cases}$
\item We then apply phase II of the simplex algorithm to the auxiliary problem.
\item If the original problem has a feasible solution, then the optimal solution of the auxiliary problem \textbf{must} be $\bm{\epsilon} = (0, ..., 0)$ (all artificial variables disappear) and the objective function is $0$.
\item If we find a solution with $\bm{\epsilon} = \bm{0}$, then we have found a valid starting point.
\item If we do not find a solution with $\bm{\epsilon} = \bm{0}$, the problem can not be solved.
\end{itemize}

\framebreak

\textbf{Example:}

\begin{eqnarray*}
\min_{\bm{x} \in \R^2} && -x_1 - x_2 \\
\text{s.t. } && x_1 - x_2 \ge - 0.5 \\
&& - x_1 - 2 x_2 \ge - 2 \\
&& - 2x_1 - x_2 \ge - 2 \\
&& - x_1 + x_2 \ge - 0.5 \\
&& \bm{x} \ge 0
\end{eqnarray*}

A starting point is the corner $\bm{(0, 0)}$.

\framebreak

\begin{center}
\includegraphics[width = 0.6\textwidth]{figure_man/simplex_implementation/iter1.png}
\end{center}

\framebreak

\begin{center}
\includegraphics[width = 0.6\textwidth]{figure_man/simplex_implementation/iter2.png}
\end{center}

\framebreak

\begin{center}
\includegraphics[width = 0.6\textwidth]{figure_man/simplex_implementation/iter3.png}
\end{center}

\framebreak

\begin{center}
\includegraphics[width = 0.6\textwidth]{figure_man/simplex_implementation/iter4.png}
\end{center}

\end{vbframe}

% \begin{vbframe}{Komplexität des Simplex Algorithmus}
%
% \end{vbframe}

\endlecture
\end{document}


