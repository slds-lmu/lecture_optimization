\documentclass[11pt,compress,t,notes=noshow, xcolor=table]{beamer}

\input{../../style/preamble}
\input{../../latex-math/basic-math}
\input{../../latex-math/basic-ml}


\title{Optimization in Machine Learning}

\begin{document}

\titlemeta{% Chunk title (example: CART, Forests, Boosting, ...), can be empty
  }{% Lecture title  
  Linear Programming
  }{% Relative path to title page image: Can be empty but must not start with slides/
  figure_man/convex_programs.png
  }{
    \item Definition and different forms of an LP
    \item Geometric intuition of LPs
    \item Chracteristics of vertices
    \item Simplex algorithms
}
%%%%%%%%%%%%%%%%%%%%%%%%%%%%%%%%%%%%%%%%%%%%%%%%%%%%%%%%%%%%%%%%%%%%%%%%%%%%%%%%%%%

\begin{vbframe}{Linear programming}

\textbf{Linear program} (LP):
\begin{center}
    optimization problem with \textbf{linear} objective function + \textbf{linear} constraints
\end{center}

\textbf{General form}
\vspace*{-1cm}

\begin{eqnarray*}
\min_{\xv} && \mathbf{c}^\top\xv \\
\text{s.t. } && \Amat_1\xv \le \mathbf{b}_1\\
&& \Amat_2\xv \ge \mathbf{b}_2 \\
&& \Amat_3\xv = \mathbf{b}_3
\end{eqnarray*}

\textbf{Examples:}\\
\begin{minipage}{0.45\textwidth}
    \begin{eqnarray*}
        \min_{\xv} && \|\Amat\xv-\mathbf{b}\|_1 \Leftrightarrow\\
        \min_{\xv,\mathbf{s}} && \one^\top\mathbf{s} \\
        \text{s.t. } && \Amat\xv - \mathbf{b} \le \mathbf{s}\\
        && \Amat\xv - \mathbf{b} \ge -\mathbf{s}
    \end{eqnarray*}
\end{minipage}
\hfill
\begin{minipage}{0.45\textwidth}
    \begin{eqnarray*}
        \min_{\xv} && \|\Amat\xv-\mathbf{b}\|_\infty \Leftrightarrow\\
        \min_{\xv,t} && t \\
        \text{s.t. } && \Amat\xv - \mathbf{b} \le t\one\\
        && \Amat\xv - \mathbf{b} \ge -t\one
    \end{eqnarray*}
\end{minipage}

\framebreak

\textbf{Standard Form}:
\vspace{-\baselineskip}

\begin{eqnarray*}
\min && \mathbf{c}^\top\xv \\
\text{s.t. } && \Amat\xv \le \mathbf{b}\\
&& \xv \ge \zero \\
\end{eqnarray*}

$\Amat\xv \ge \mathbf{b} \Leftrightarrow -\Amat\xv \le -\mathbf{b}$.\\
$\Amat\xv = \mathbf{b} \Leftrightarrow \Amat\xv \le \mathbf{b},\ -\Amat\xv \le -\mathbf{b}$.\\
$\xv = \xv^+ - \xv^-$, where $\xv^+ \ge \zero,\ \xv^- \ge \zero$.\\
$\min_{\xv} \mathbf{c}^\top\xv \Leftrightarrow \min_{\xv^+, \xv^-} \begin{bmatrix} \mathbf{c}^\top & -\mathbf{c}^\top \end{bmatrix} \begin{bmatrix} \xv^+ \\ \xv^- \end{bmatrix}$.

\framebreak
\textbf{Equality Form}:
\vspace{-\baselineskip}

\begin{eqnarray*}
\min && \mathbf{c}^\top\xv \\
\text{s.t. } && \Amat\xv = \mathbf{b}\\
&& \xv \ge \zero \\
\end{eqnarray*}

By introducing slack variables $\mathbf{s}$:
$\Amat\xv \le \mathbf{b} \Leftrightarrow \begin{bmatrix} \Amat & \id \end{bmatrix} \begin{bmatrix} \xv \\ \mathbf{s} \end{bmatrix} = \Amat\xv + \mathbf{s} = \mathbf{b},\ \mathbf{s} \ge 0$.\\
$\min_{\xv} \mathbf{c}^\top\xv \Leftrightarrow \min_{\xv,\mathbf{s}} \begin{bmatrix} \mathbf{c}^\top & \zero^\top \end{bmatrix} \begin{bmatrix} \xv \\ \mathbf{s} \end{bmatrix}$.

\end{vbframe}

\begin{vbframe}{Geometric interpretation}
    
\textbf{Feasible set:}
\begin{itemize}
    \setlength{\itemsep}{1em}
    \item Points $\{\xv: \Amat_i^\top \xv = b_i\}$ form a hyperplane in $\R^n$.\\
          $\Amat_i$ is perpendicular to the hyperplane and called \textbf{normal vector}.
    \item Points $\{\xv: \Amat_i^\top \xv \le b_i\}$ lie on one side of the hyperplane, which form a convex half-space.
    \item Points satisfying \textbf{all} inequalities form a \textbf{convex polytope}.\\
          The intersection of convex sets is still convex, also easy to prove using definition of convex set.
\end{itemize}

\lz

Polytope $\{\xv: \Amat \xv \le \mathbf{b},\ \Amat \in \R^{m \times n}\}$ is an $n$-\textbf{simplex}, i.e., 
convex hull of $n + 1$ \textbf{affinely independent} points, which we call vertices.

\framebreak

There are 3 conditions for solving linear programming:
\begin{enumerate}
\item Feasible set is \textbf{empty} $\Rightarrow$ LP is infeasible.
\item Feasible set is \textbf{\enquote{unbounded}}.
\item Feasible set is \textbf{\enquote{bounded}} $\Rightarrow$ LP has at least one solution.
\end{enumerate}

\begin{center}
    \includegraphics[width=0.9\textwidth]{figure_man/solutions-lp.png}
\end{center}

\framebreak

Case 3 when LP is feasible and bounded:
\begin{itemize}
\item Points on the interior: never optimal, can be improved by moving along $−\mathbf{c}$.
\item Points on faces/edges: can be optimal only if the face/edge is perpendicular to $\mathbf{c}$. 
\item Points on faces/edges not perpendicular to $\mathbf{c}$: can be improved by moving along $-\mathbf{c}$.
\item Vertices: can also be optimal.
\end{itemize}

\begin{center}
    \includegraphics[width=0.3\textwidth]{figure_man/opposite-direction.png}
\end{center}

\end{vbframe}

\begin{vbframe}{Vertices}
We assume that the rows of $\Amat \in \R^{m \times n}$ are linearly independent and $m \le n$ to form a bounded unempty feasible set.\\
\lz
$\Amat\xv = \mathbf{b}$ imposes $m$ equality constraints:
\begin{itemize}
    \item Each equality constraint reduces the dimension of the feasible set by 1.
    \item Starting with $n$-dimensional space, applying $m$ independent equality constraints leaves a solution space of dimension $n-m$.
\end{itemize}
\lz
$\xv \ge \zero$ imposes $n$ non-negativity.\\

\framebreak

While statisfying $\Amat\xv = \mathbf{b}$, the indices of a vertex vector can thus be participated into two sets:
\begin{itemize}
    \item $\mathcal{V}$ with $n-m$ elements: $i \in \mathcal{V} \Rightarrow x_i = 0$ (active constraints).
    \item $\mathcal{B}$ with $m$ elements: $i \in \mathcal{B} \Rightarrow x_i \ge 0$.
\end{itemize}

\lz

We have $\Amat_{\mathcal{B}}^{m\times m}\xv_{\mathcal{B}} = \mathbf{b} \Rightarrow \xv_{\mathcal{B}} = \Amat_{\mathcal{B}}^{-1}\mathbf{b}$.

\lz
\lz

\textbf{Note:} While every vertex has an associated partition $(\mathcal{B}, \mathcal{V})$, not every partition corresponds to a vertex.
\end{vbframe}

\begin{vbframe}{Simplex algorithms}
The \textbf{simplex algorithm} solves linear programs by moving from vertex to vertex of the feasible set, 
and produces an optimal vertex.\\

\lz

It operates on equality-form linear programs $\Amat\xv = \mathbf{b}, \xv \ge \zero$. 
We still assume that the rows of $\Amat \in \R^{m \times n}$ are linearly independent and $m \le n$.\\

\lz

The method is guaranteed to arrive at an optimal solution so long as the linear program is feasible and bounded.\\

\lz

The simplex algorithm operates in two phases:
\begin{itemize}
    \item \textbf{Initialization} phase: identifies a vertex partition.
    \item \textbf{Optimization} phase: transitions between vertex partitions toward a partition corresponding to an optimal vertex.
\end{itemize}

\framebreak

%The \textbf{first-order necessary conditions (FONCs)} for optimality are used to determine when a vertex is optimal, 
%and to inform how to transition to a more favorable vertex.\\
%\lz
We construct a Lagrangian for the equality form of the linear program:
$$L(\xv, \bm{\mu}, \bm{\lambda}) = \mathbf{c}^\top\xv - \bm{\mu}^\top\xv - \bm{\lambda}^\top(\Amat\xv - \mathbf{b})$$
with $\bm{\mu} \ge \zero$.

The optimal solution satisfies $\frac{\partial L}{\partial \xv}=0$, i.e., $\Amat^\top \bm{\lambda} + \bm{\mu} = \mathbf{c}$.
%with the following FONCs:

%\begin{itemize}
%    \item \textbf{feasibility:} $\Amat\xv = \mathbf{b},\ \xv \ge \zero$.
%    \item \textbf{dual feasibility:} $\bm{\mu} \ge \zero$.
%    \item \textbf{complementary slackness:} $\bm{\mu} \odot \xv = 0$
%    \item \textbf{stationarity:} $\Amat^\top \bm{\lambda} + \bm{\mu} = \mathbf{c}$
%\end{itemize}

%\framebreak

%The FONCs are sufficient conditions for optimality for linear programs.\\
%\lz
%Thus, if $\bm{\mu}$ and $\bm{\lambda}$ can be computed for a given vertex and all four FONC equations are satisfied, then the vertex is optimal.\\
%\lz
%\lz
We can decompose this stationarity condition into $\mathcal{B}$ and $\mathcal{V}$ components:
$$
\Amat^\top \bm{\lambda} + \bm{\mu} = \mathbf{c} 
\quad \implies \quad
\begin{cases}
    \Amat_{\mathcal{B}}^\top \bm{\lambda} + \bm{\mu}_{\mathcal{B}} = \mathbf{c}_{\mathcal{B}} \\
    \Amat_{\mathcal{V}}^\top \bm{\lambda} + \bm{\mu}_{\mathcal{V}} = \mathbf{c}_{\mathcal{V}}
\end{cases}
$$

\framebreak

We can choose $ \bm{\mu}_{\mathcal{B}} = 0 $ to satisfy $\bm{\mu} \odot \xv = 0$, since for optimality, we need $\mu_i=0$ when $x_i > 0$. 
The value of $ \bm{\lambda} $ can be computed from $ \mathcal{B} $:
% complementary slackness
$$
\Amat_{\mathcal{B}}^\top \bm{\lambda} + \underbrace{\bm{\mu}_{\mathcal{B}}}_{= 0} = \mathbf{c}_{\mathcal{B}} \quad \implies \quad \bm{\lambda} = \Amat_{\mathcal{B}}^{-\top} \mathbf{c}_{\mathcal{B}}
$$

We can use this to obtain:
\begin{eqnarray*}
\Amat_{\mathcal{V}}^\top \bm{\lambda} + \bm{\mu}_{\mathcal{V}} &&=\quad \mathbf{c}_{\mathcal{V}}\\
\bm{\mu}_{\mathcal{V}} &&=\quad \mathbf{c}_{\mathcal{V}} - \Amat_{\mathcal{V}}^\top \bm{\lambda}\\
\bm{\mu}_{\mathcal{V}} &&=\quad \mathbf{c}_{\mathcal{V}} - \left( \Amat_{\mathcal{B}}^{-1} \Amat_{\mathcal{V}} \right)^\top \mathbf{c}_{\mathcal{B}}
\end{eqnarray*}

\lz

Knowing $ \bm{\mu}_{\mathcal{V}} $ allows us to assess the optimality of the vertices. 
If $ \bm{\mu}_{\mathcal{V}} $ contains negative components, then $\bm{\mu} \ge \zero$ is not satisfied and the vertex is suboptimal.

\framebreak

Now to the optimization phase.\\
The partition can be updated by swapping indices between $\mathcal{B}$ and $\mathcal{V}$.
Such a swap equates to moving from one vertex along an edge of the feasible set polytope to another vertex.\\
\lz
A transition $ \xv \to \xv' $ between vertices must satisfy $ \Amat \xv' = \mathbf{b} $. 
Starting with a partition defined by $ \mathcal{B} $, 
we choose an \textbf{entering index} $ q \in \mathcal{V} $ that is to enter $ \mathcal{B} $ using one of the heuristics described near the end of this section. 
The new vertex $ \xv' $ must satisfy:
$$
\Amat \xv' = \Amat_{\mathcal{B}} \xv'_{\mathcal{B}} + \Amat_{\{q\}} x'_q = \Amat_{\mathcal{B}} \xv_{\mathcal{B}} = \Amat \xv = \mathbf{b}.
$$
\lz
One \textbf{leaving index} $ p \in \mathcal{B} $ in $ \xv_{\mathcal{B}} $ becomes zero during the transition, and is replaced by the column of $ \Amat $ corresponding to index $ q $.\\
This action is referred to as \textbf{pivoting}.

\framebreak

We can solve for the new design point:
$$
\xv'_{\mathcal{B}} = \xv_{\mathcal{B}} - \Amat_{\mathcal{B}}^{-1} \Amat_{\{q\}} x'_q.
$$
\lz
A particular \textbf{leaving index} $ p \in \mathcal{B} $ becomes active when:
$$
\left( \xv'_{\mathcal{B}} \right)_p = 0 = \left( \xv_{\mathcal{B}} \right)_p - \left( \Amat_{\mathcal{B}}^{-1} \Amat_{\{q\}} \right)_p x'_q.
$$

and is thus obtained by increasing $ x_q = 0 $ to $ x'_q $ with:
$$x'_q = \frac{\left( \xv_{\mathcal{B}} \right)_p}{\left( \Amat_{\mathcal{B}}^{-1} \Amat_{\{q\}} \right)_p}.$$
\lz
The leaving index is obtained using the \textbf{minimum ratio test}, which computes for each potential leaving index and selects the one with minimum $ x'_q $. We then swap $ p $ and $ q $ between $ \mathcal{B} $ and $ \mathcal{V} $.

\framebreak


The effect that a transition has on the objective function can be computed using $x'_q$.\\
\lz

The objective function value at the new vertex is:
\begin{eqnarray*}
    \mathbf{c}^\top \xv' &&=\quad \mathbf{c}_{\mathcal{B}}^\top \xv'_{\mathcal{B}} + c_q x'_q \\
    &&=\quad \mathbf{c}_{\mathcal{B}}^\top \left( \xv_{\mathcal{B}} - \Amat_{\mathcal{B}}^{-1} \Amat_{\{q\}} x'_q \right) + c_q x'_q \\
    &&=\quad \mathbf{c}_{\mathcal{B}}^\top \xv_{\mathcal{B}} - \mathbf{c}_{\mathcal{B}}^\top \Amat_{\mathcal{B}}^{-1} \Amat_{\{q\}} x'_q + c_q x'_q \\
    &&=\quad \mathbf{c}_{\mathcal{B}}^\top \xv_{\mathcal{B}} - (c_q - \mu_q) x'_q + c_q x'_q \quad (\bm{\lambda} = \Amat_{\mathcal{B}}^{-\top} \mathbf{c}_{\mathcal{B}}, \Amat_{\{q\}}^\top\bm{\lambda}=c_q-\mu_q)\\
    &&=\quad \mathbf{c}^\top \xv + \mu_q x'_q.
\end{eqnarray*}

\framebreak

Choosing an entering index $q$ decreases the objective function value by
$$
\mathbf{c}^\top \xv' - \mathbf{c}^\top \xv = \mu_q x'_q.
$$

\lz
\lz
The objective function decreases only if $\mu_q$ is negative.\\
\lz
In order to progress toward optimality, we must choose an index $q \in \mathcal{V}$ such that $\mu_q$ is negative.\\
\lz
If all components of $\boldsymbol{\mu}_{\mathcal{V}}$ are non-negative, we have found a global optimum.

\framebreak

Since there can be multiple negative entries in $\boldsymbol{\mu}_{\mathcal{V}}$, different heuristics can be used to select an entering index:

\begin{itemize}
    \item \textbf{Greedy heuristic}: choose a $q$ that maximally reduces $\mathbf{c}^\top \xv$.
    \item \textbf{Dantzig's rule}: choose the $q$ with the most negative entry in $\boldsymbol{\mu}$. 
    This rule is easy to calculate, but it does not guarantee the maximum reduction in $\mathbf{c}^\top \xv$. 
    It is also sensitive to scaling of the constraints.
    \item \textbf{Bland's rule}: choose the first $q$ with a negative entry in $\boldsymbol{\mu}$. 
    When used on its own, Bland's rule tends to result in poor performance in practice. 
    However, this rule can help us prevent cycles, which occur when we return to a vertex we have visited before without decreasing the objective function. 
    This rule is usually used only after no improvements have been made for several iterations of a different rule to break out of a cycle and ensure convergence.
\end{itemize}

\end{vbframe}

\begin{vbframe}{Examlpes}
$\Amat_{\mathcal{V}} =
\begin{pmatrix}
2 & 1 & 1 & 0 \\
-4 & 2 & 0 & 1
\end{pmatrix},$
$\mathbf{b} = (9, 2)^\top, \mathbf{c} = (3, -1, 0, 0)^\top$\\
\lz
\textbf{Solution:}
$\mathcal{V}=\{1,2\}, \mathcal{B}=\{3, 4\}$\\
$\xv_{\mathcal{B}} = \Amat_{\mathcal{B}}^{-1}\mathbf{b} = (9, 2)^\top$\\
$\bm{\lambda} =  \Amat_{\mathcal{B}}^{-1}\mathbf{c}_{\mathcal{B}} = \zero$\\
$\bm{\mu}_{\mathcal{V}} = \mathbf{c}_{\mathcal{V}} - \left( \Amat_{\mathcal{B}}^{-1} \Amat_{\mathcal{V}} \right)^\top \mathbf{c}_{\mathcal{B}} = (3, -1)^\top$\\
$\bm{\mu}_{\mathcal{V}}$ contains negative elements, so our current $\mathcal{B}$ is suboptimal.\\
We will pivot on the negative one with $q=2, - \Amat_{\mathcal{B}}^{-1} \Amat_{\{q\}}=(1,2)^\top$.\\
This causes $x_4 = 0$, so updated $\mathcal{B}=\{2, 3\}$.\\
In the second iteration, we find $\xv_{\mathcal{B}}=(1,8)^\top, \bm{\lambda} = (0, -\frac{1}{2})^\top, \bm{\mu}_{\mathcal{V}} = (1, \frac{1}{2})^\top.$\\
This is optimal with no negative entry, thus we have $x^*=(0,1,8,0)^\top$
%Calculate $x'_q = \frac{\left( \xv_{\mathcal{B}} \right)_p}{\left( \Amat_{\mathcal{B}}^{-1} \Amat_{\{q\}} \right)_p}$ 
%and choose the minumum one with $p = 5, q = 1, x'_1 = 10$.
%$\xv'_{\mathcal{B}} = \xv_{\mathcal{B}} - \Amat_{\mathcal{B}}^{-1} \Amat_{\{1\}}x'_1 = (10, 0, 0)$\\
%Now new $\mathcal{B}' = \{4, 1, 6\}, \mathcal{V}' = \{5, 2, 3\}, \xv' = (10,0,0,10,0,0)$
\end{vbframe}

\endlecture
\end{document}


