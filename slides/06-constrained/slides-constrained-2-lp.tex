
\input{../../style/preamble}
\input{../../latex-math/basic-math}
\input{../../latex-math/basic-ml}

%\newcommand{\titlefigure}{figure_man/}
\newcommand{\learninggoals}{
\item \textcolor{red}{LEARNING GOAL 1}
\item \textcolor{red}{LEARNING GOAL 1}}

%\usepackage{animate} % only use if you want the animation for Taylor2D

\title{Optimization}
%\author{}
%\date{}

\begin{document}

\lecturechapter{Linear Programming}
\lecture{Optimization}
\sloppy

% --------------------------------------------------------------------------------------------

\begin{vbframe}{Linear programming}

\textbf{Special case:} \textbf{Linear programming} (LP)

objective function and constraints are \textbf{linear functions}.

\lz

\textbf{Example:}

\vspace*{-1cm}
\begin{footnotesize}
\begin{eqnarray*}
\min && - x_1 - x_2 \\
\text{s.t. } && x_1 + 2x_2 \le 1\\
&& 2x_1 + x_2 \le 1 \\
&& x_1, x_2 \ge 0
\end{eqnarray*}
\end{footnotesize}

\begin{center}
\includegraphics[width = 0.3\textwidth]{figure_man/linear-pro-example.png}
\end{center}

%<<echo = F>>=
%# plot polytope
%plotPoly = function(A, b) {

%  p = ggplot(data = data.frame(x = 0), mapping = aes(x = x))
%  p = p + geom_point(x = 0, y = 0, size = 2, color = "green")

%  for (i in 1:dim(A)[1]) {
%    p = p + geom_abline(intercept = b[i] / A[i, 2], slope = - A[i, 1] / A[i, 2], color = "green")
%  }

%  p = p + geom_segment(aes(x = 0, y = -Inf, xend = 0, yend = Inf), color = "green")
%  p = p + geom_segment(aes(x = -Inf, y = 0, xend = Inf, yend = 0), color = "green")

%  combs = combn(1:dim(A)[1], 2)

%  V = matrix(0, 0, 2)

%  for (i in 1:dim(combs)[2]) {

%    ind = combs[, i]

%    if (det(A[ind, ]) != 0) {
%      sol = solve(A[ind, ], b[ind])

%      if (all(A %*% sol <= b)) {
%        V = rbind(V, solve(A[ind, ], b[ind]))
%        p = p + geom_point(x = sol[1], y = sol[2], color = "green", size = 2)
%      }
%    }
%  }

%  V = as.data.frame(V)

%  center = apply(V, 2, mean)
%  Vdiff = V - center

%  ordering = vector(length = 0)

%  for (i in 1:dim(Vdiff)[1]) {
%    ordering = cbind(ordering, atan2(Vdiff[i, 1], Vdiff[i, 2]))
%  }

%  V = V[order(ordering), ]

%  p = p + geom_polygon(data = V, aes(x = V1, y = V2), fill = "green", alpha = 0.1)

%  p = p + coord_equal() + xlab(expression(x[1])) + ylab(expression(x[2]))


%  p = p + xlim(c(0, 1)) + ylim(c(0, 1)) + theme_bw()

%  p
%}
%@

%\vspace*{-0.5cm}

%<<echo = F, out.width = '50%', fig.align='center', warning = F>>=
%A = matrix(c(1, 2, -1, 0, 2, 1, 0, -1), ncol = 2)
%b = c(1, 1, 0, 0)

%plotPoly(A, b)
%@


\framebreak

W.l.o.g. Linear programming is specified using the so-called \textbf{standard form}.

\vspace*{-0.5cm}

\begin{eqnarray*}
\min_{\bm{x} \in \R^n} && \bm{c}^T\bm{x}\\
\text{s.t. } && \bm{A}\bm{x} \ge \bm{b} \\
&& \bm{x} \ge 0
\end{eqnarray*}

The inequalities $\bm{Ax} \ge \bm{b}$ ($\bm{A} \in \R^{m\times n}$, $\bm{b} \in \R^m$) and $\bm{x} \ge 0$ are to be understood componentwise.

\lz

The condition $\bm{x} \ge 0$ is known as the \textbf{non-negativity constraint} and the vector $\bm{c}$ is known as the \textbf{cost vector}.

%
% \begin{eqnarray*}
% &&\min_{\bm{x} \in \R^n} f(\bm{x})\\
% \text{u. d. N. } && g_i(\bm{x}) \le 0, h_j(\bm{x}) = 0,
% \end{eqnarray*}
%
% wobei $f: \R^n \to \R$, $g_i: \R^n \to \R$, $h_j: \R^n \to \R$ lineare Funktionen sind.

\framebreak

Linear optimization problems can be converted to the standard form by the following operations:

\begin{itemize}
\item Maximization instead of minimization: multiplication of the cost vector $\bm{c}$ by $-1$
\item Less than or equal instead of greater than or equal: multiply the inequality by $-1$
\item Equality instead of inequality: replace $\bm{a}_i\bm{x}=b_i$ with two conditions of inequality $\bm{a}_i\bm{x}\ge b_i$ and $\bm{a}_i\bm{x}\le b_i$
\item Variable without non-negativity constraint: replace $x_i$ with $x_i^+ - x_i^-$ with $x_i^+, x_i^- \ge 0$ (positive or negative part).
\end{itemize}

In the following we assume that the LP is given in standard form.

\framebreak

\textbf{Example:}

\lz

The example above

\vspace*{-1cm}
\begin{eqnarray*}
\min && - x_1 - x_2 \\
\text{s.t. } && x_1 + 2x_2 \le 1\\
&& 2x_1 + x_2 \le 1 \\
&& x_1, x_2 \ge 0
\end{eqnarray*}

can also be formulated as

\vspace*{-0.2cm}

\begin{eqnarray*}
\min && (-1, -1) \begin{pmatrix} x_1 \\ x_2 \end{pmatrix} \\
\text{s.t. } &&  \begin{pmatrix} - 1 & - 2 \\ - 2 & - 1 \end{pmatrix} \bm{x} \ge \begin{pmatrix} - 1 \\ - 1
\end{pmatrix} \\
&& \bm{x} \ge 0
\end{eqnarray*}

\end{vbframe}


\begin{vbframe}{Geometric interpretation}

Linear programming can be interpreted geometrically.

\lz

\textbf{Feasible set:}
\begin{itemize}
\item Let $\bm{a}_i \bm{x} \ge b_i$ be the i-th line of the inequality conditions.
\item The points that satisfy the linear system $\bm{a}_i \bm{x} = b_i$ form a hyperplane in $n$-dimensional space.
\item The vector $\bm{a}_i$ is perpendicular to the plane and is called the \textbf{normal vector}.
\framebreak
\item The set of points $\{\bm{x}: \bm{a}_i \bm{x} \ge b_i\}$ consists of points on the side of the hyperplane into which the normal vector points (\textbf{half-space}).

\vspace{0.2cm}
\begin{center}
\includegraphics[width = 0.5\textwidth]{figure_man/geo-interpretation.png}
\end{center}


%<<fig.align='center', out.width='80%', echo = F, warning = F>>=
%f = function(x) b[1] / A[1, 2] - b[1] / A[1, 2] * x

%p = ggplot(data = data.frame(x = 0))
%p = p + geom_abline(slope = - b[1] / A[1, 2], intercept =  b[1] / A[1, 2], color = "green")
%p = p + geom_polygon(data = data.frame(x = c(-5, 5, 5, -5), y = c(-5, -5, f(5), f(-5))), aes(x = x, y = y), fill = "green", alpha = 0.1)
%p = p + geom_segment(aes(x = 0, y = f(0), xend = 0 - A[1, 1], yend = f(0) - A[1, 2]), arrow = arrow(length = unit(0.03, "npc")), color = "green")

%p = p + xlab(expression(x[1])) + ylab(expression(x[2]))

%p = p + theme_bw() + coord_equal()

%p = p + xlim(c(-5, 5))
%p = p + ylim(c(-5, 5))

%p
%@

\framebreak

\item Each of the $m$ inequalities divides the $n$-dimensional space into two halves.
\item \textbf{Claim:} The points that satisfy \textbf{all} inequalities form a \textbf{convex polytope}.
\end{itemize}

\begin{center}
\includegraphics[width = 0.5\textwidth]{figure_man/linear-pro-example.png}
\end{center}


%<<echo = F, out.width = '60%', fig.align='center', warning = F>>=
%plotPoly(A, b)
%@


\framebreak

In geometry, a \textbf{polytope} denotes a generalized polygon in an arbitrary dimension. A polytope consists of several subpolytopes.

\begin{itemize}
\item A $0$ polytope is a single point,
\item A $1$ polytope is a line,
\item A $2$ polytope is a polygon, ...
\end{itemize}

In general, a $d$ polytope is formed from several $(d-1)$ polytopes (so-called facets) which in turn can have a $(d-2)$ polytope in common. Thus a $3$ polytope (e.g. a cube) has several sides / facets, some of which have common edges, etc.

\framebreak

The points for which $\bm{a}_i \bm{x} = b_i$ applies lie on the \textbf{facet} of the polytope.


\lz

The polytope which is defined by the inequalities $\bm{Ax} \ge \bm{b}$ is convex. For two points $\bm{x}_1, \bm{x}_2$, which lie in the polytope, any point that results from the convex combination of the two points, is again inside the polytope:

\begin{eqnarray*}
\bm{A}(\bm{x}_1 + t(\bm{x}_2 - \bm{x}_1)) &=& \bm{A}\bm{x}_1 + t(\bm{A}\bm{x}_2 - \bm{A}\bm{x}_1) \\ &=& (1 -t)\underbrace{\bm{A}\bm{x}_1}_{\ge \bm{b}} + t\underbrace{\bm{A}\bm{x}_2}_{\ge \bm{b}} \\ &\ge& (1 - t) \bm{b} + t \bm{b} = \bm{b} \qquad \text{for }t \in [0, 1])
\end{eqnarray*}

A polytope formed by the convex hull of $(n + 1)$ affin independent points in $\R^n$ is also called $n$-\textbf{simplex}.

\framebreak

\textbf{Objective function:}
\begin{itemize}
\item For the objective function, we consider the \textbf{contour lines}. Points that lie on the same contour line have the same objective function value.
\item In the linear case, the contours $y = \bm{c}^T \bm{x}$ are also a hyperplane for fixed $y$.
\item The vector $\bm{c}$ is again perpendicular to the respective contour lines.
\item The vector $\bm{c}$ can also be interpreted as a gradient. The \textbf{negative} gradient $- \bm{c}$ points in the direction of the \enquote{steepest} descent.
\framebreak
\item The value of the function becomes smaller when we go in the direction of the \textbf{negative gradient} $- \bm{c}$.

\begin{center}
\includegraphics[width = 0.5\textwidth]{figure_man/negative-gradient.png}
\end{center}


%<<echo = F, out.width = '80%', fig.align='center', warning = F>>=
%f =  function(x, z) - z - x
%dd = expand.grid(x = seq(-2, 2, by = 0.01), z = seq(-2, 2, by = 1 / 6))
%dd %>% rowwise %>% mutate(y = f(x = x, z = z)) -> dd2

%p = ggplot() + geom_path(data = dd2, aes(x, y, col = z, group = z))
%p = p + scale_colour_gradientn(colours = terrain.colors(10))
%p = p + theme_bw() + xlim(c(0, 1)) + ylim(c(0, 1))
%p = p + xlab("x1") + ylab("x2") + labs(colour = "y")
%p = p + geom_segment(aes(x = 0.5, y = 0.5, xend = 0.625, yend = 0.625), arrow = arrow(length = unit(0.03, "npc")))
%p = p + geom_text(aes(x = 0.5, y = 0.5, label = "-c"), hjust = -1.5, vjust = -0.4, size = 5)

%p = p + coord_equal()

%p
%@

\framebreak
\item So we move the objective function \textbf{in the opposite direction} of the normal vector until the line just touches the polygon.

\begin{center}
\includegraphics[width = 0.5\textwidth]{figure_man/opposite-direction.png}
\end{center}

%<<echo = F, out.width = '80%', fig.align='center', warning = FALSE, message = FALSE>>=
%p = plotPoly(A, b)

%p = p + geom_path(data = dd2, aes(x, y, col = z, group = z))

%p = p + scale_colour_gradientn(colours = terrain.colors(5))
%p = p + xlab("x1") + ylab("x2") + labs(colour = "y") + theme_bw()

%p = p + geom_abline(intercept = 2 / 3, slope = - 1, color = "black")
%p = p + geom_segment(aes(x = 1 / 3, y = 1 / 3, xend = 1 / 3 + 0.1, yend = 1 / 3 + 0.1), arrow = arrow(length = unit(0.03, "npc")))
%p = p + geom_text(aes(x = 1 / 3, y = 1 / 3, label = "-c"), hjust = -1.5, vjust = -0.4, size = 5)

%p = p + coord_equal()

%p
%@

\end{itemize}

\end{vbframe}

\begin{vbframe}{Solutions to LP}

There are three ways to solve Linear programming:

\begin{enumerate}
\item LP is \textbf{infeasible}, the feasible set is empty ($\mathcal{S} = \emptyset$)
\item LP is unconstrained
\item LP has at least one optimal solution
\end{enumerate}


\begin{center}
\includegraphics[width = 0.8\textwidth]{figure_man/solutions-lp.png}
\end{center}


\framebreak

\begin{itemize}
\item If LP is solvable and constrained (neither case 1 nor case 2), there is always an optimal point that can  \textbf{not} be convexly combined from other points in the polytope.
\item The optimal solution is then a corner, edge or side of the polytope.
\end{itemize}


\end{vbframe}

% \section{Algorithms for LP}

% \begin{vbframe}{Simplex algorithm}

% The Simplex algorithm is the most important method for solving Linear programming. It was published in 1947 by Georg Dantzig.

% \lz

% \textbf{Basic idea:} start from an arbitrary corner of the polytope. Run along this edge as long as the solution improves. Find a new edge, ...

% \lz

% \textbf{Output:} a path along the corners of the polytope that ends at the optimal point of the polytope.

% \lz

% Since linear programming is a \textbf{convex} optimization problem, the optimal corner found in this way is also a global optimum.

% \framebreak

% \begin{center}
% \includegraphics[width = 0.6\textwidth]{figure_man/simplex.png}
% \end{center}

% \framebreak

% The simplex algorithm can be divided into two steps:

% \begin{itemize}
% \item \textbf{Phase I:} determination of a \textbf{starting point}
% \item \textbf{Phase II:} determination of the \textbf{optimal solution}
% \end{itemize}

% To be able to start, a starting point must first be found in \textbf{Phase I}, i.e. a feasible corner $\bm{x}_0$.

% \lz

% In \textbf{phase II} this solution is iteratively improved by searching for an edge that improves the solution and running along it to the next corner.

% \framebreak

% \textbf{Phase I}:

% One way to find a starting point $\bm{x}_0$ is to solve a auxiliary linear problem with artificial variables $\bm{\epsilon}$:

% \begin{eqnarray*}
% \min_{\epsilon_1, ..., \epsilon_m} && \sum_{i = 1}^m \epsilon_i \\
% \text{s.t. } && \bm{Ax} + \bm{\epsilon} \ge \bm{b} \\
% && \epsilon_1, ..., \epsilon_m \ge 0\\
% && \bm{x} \ge 0
% \end{eqnarray*}

% \begin{itemize}
% \item A feasible starting point for the auxiliary problem is $\bm{x} = \bm{0}$ and $\epsilon_i = \begin{cases} 0 & \text{if } b_i < 0 \\
% b_i & \text{if } b_i \ge 0
% \end{cases}$
% \item We then apply phase II of the simplex algorithm to the auxiliary problem.
% \item If the original problem has a feasible solution, then the optimal solution of the auxiliary problem \textbf{must} be $\bm{\epsilon} = (0, ..., 0)$ (all artificial variables disappear) and the objective function is $0$.
% \item If we find a solution with $\bm{\epsilon} = \bm{0}$, then we have found a valid starting point.
% \item If we do not find a solution with $\bm{\epsilon} = \bm{0}$, the problem can not be solved.
% \end{itemize}

% \framebreak

% \textbf{Example:}

% \begin{eqnarray*}
% \min_{\bm{x} \in \R^2} && -x_1 - x_2 \\
% \text{s.t. } && x_1 - x_2 \ge - 0.5 \\
% && - x_1 - 2 x_2 \ge - 2 \\
% && - 2x_1 - x_2 \ge - 2 \\
% && - x_1 + x_2 \ge - 0.5 \\
% && \bm{x} \ge 0
% \end{eqnarray*}

% A starting point is the corner $\bm{(0, 0)}$.

% \framebreak

% \begin{center}
% \includegraphics[width = 0.6\textwidth]{figure_man/simplex_implementation/iter1.png}
% \end{center}

% \framebreak

% \begin{center}
% \includegraphics[width = 0.6\textwidth]{figure_man/simplex_implementation/iter2.png}
% \end{center}

% \framebreak

% \begin{center}
% \includegraphics[width = 0.6\textwidth]{figure_man/simplex_implementation/iter3.png}
% \end{center}

% \framebreak

% \begin{center}
% \includegraphics[width = 0.6\textwidth]{figure_man/simplex_implementation/iter4.png}
% \end{center}

% \end{vbframe}

% % \begin{vbframe}{Komplexität des Simplex Algorithmus}
% %
% % \end{vbframe}

% \section{Duality}

% \begin{vbframe}{Duality: introductory example}

% \textbf{Example:}

% A bakery sells brownies for $50$ ct and mini cheesecakes for $80$ ct each. The two products contain the following ingredients

% \begin{center}
% \begin{tabular}{ r c c c}
%     & \text{Chocolate} & \text{Sugar} & \text{Creamcheese} \\
%     \hline
%   \text{Brownie} & 3 & 2 & 2 \\
%   \text{Cheesecake} & 0 & 4 & 5
% \end{tabular}
% \end{center}

% A student wants to minimize his expenses, but at the same time eat at least $6$ units of chocolate, $10$ units of sugar and $8$ units of creamcheese.

% \framebreak

% He is therefore confronted with the following optimization problem:

% \begin{eqnarray*}
% \min_{\bm{x}\in \R^2} && 50x_1 + 80x_2 \\
% \text{s.t. } && 3x_1 \ge 6 \\
% && 2x_1 + 4x_2 \ge 10 \\
% && 2x_1 + 5x_2 \ge 8 \\
% && \bm{x} \ge 0
% \end{eqnarray*}

% \framebreak

% The solution of the Simplex algorithm:
% \vspace{0.3cm}

% \footnotesize
% \begin{verbbox}
% res = solveLP(cvec = c, bvec = b, Amat = A)
% summary(res)
% ##
% ##
% ## Results of Linear Programming / Linear Optimization
% ##
% ## Objective function (Minimum): 220
% ##
% ## Solution
% ## opt
% ## 1 2.0
% ## 2 1.5
% \end{verbbox}
% \col

% %<<echo = F>>=
% %A = - matrix(c(3, 2, 2, 0, 4, 5), ncol = 2)
% %b = - c(6, 10, 8)
% %c = c(50, 80)
% %@

% %<<>>=
% %res = solveLP(cvec = c, bvec = b, Amat = A)
% %summary(res)
% %@

% \framebreak
% \normalsize
% The baker informs the supplier that he needs at least $6$ units of chocolate, $10$ units of sugar and $8$ units of creamcheese to meet the student's requirements.

% \lz

% The supplier asks himself how he must set the prices for chocolate, sugar and creamcheese ($\alpha_1, \alpha_2, \alpha_3$) such that he can

% \begin{itemize}
% \item maximize his revenue
% $$
% \max_{\bm{\alpha} \in \R^3}6 \alpha_1 + 10 \alpha_2 + 8 \alpha_3
% $$
% \item and at the same time ensure that the baker buys from him (purchase cost $\le$ selling price)
% \begin{eqnarray*}
% 3\alpha_1 + 2\alpha_2 + 2\alpha_3 &\le& 50 \qquad \text{Brownie} \\
% 4\alpha_2 + 5\alpha_3 &\le& 80 \qquad \text{Cheesecake}
% \end{eqnarray*}

% \end{itemize}

% \framebreak

% The presented example is known as a \textbf{dual problem}. The variables $\alpha_i$ are called \textbf{dual variables}.

% \lz

% In an economic context, dual variables can often be interpreted as \textbf{shadow prices} for certain goods.

% \lz

% If we solve the dual problem, we see that the dual problem has the same objective function value as the primal problem. This is later referred to as \textbf{strong duality}.

% % \lz
% %
% % Der Lieferant sollte eine Einheit Schokolade für mindestens $\alpha_1$ verkaufen, der Bäcker aber höchstens zu $\alpha_1$ einkaufen.
% \framebreak
% \footnotesize
% \begin{verbbox}
% res = solveLP(cvec = c, bvec = b, Amat = A, maximum = T)
% summary(res)
% ##
% ##
% ## Results of Linear Programming / Linear Optimization
% ##
% ## Objective function (Maximum): 220
% ##
% ## Solution
% ## opt
% ## 1 3.333333
% ## 2 20.000000
% ## 3 0.000000
% \end{verbbox}
% \col

% %<<echo = F>>=
% %A = matrix(c(3, 0, 2, 4, 2, 5), ncol = 3)
% %b = c(50, 80)
% %c = c(6, 10, 8)
% %@

% %<<>>=
% %res = solveLP(cvec = c, bvec = b, Amat = A, maximum = T)
% %summary(res)
% %@


% % \textbf{Beispiel:} Eine Firma stellt Tische und Stühle her und verkauft diese für $30$ Euro (Tisch),  $20$ Euro (Stuhl). Hierfür gibt es $3$ Maschinen, die jeweils nur eine gewisse Zeit zur Verfügung stehen.
% %
% % \lz
% %
% % In der Tabelle ist aufgelistet, wie viel Stunden die jeweilige Maschine für Stuhl bzw. Tisch benötigt.
% %
% % \begin{center}
% % \begin{tabular}{ r | c c | c}
% %   \text{Maschine}  & \text{Stuhl} & \text{Tisch} & \text{Verfügbarkeit (in h)} \\
% %     \hline
% %   M_1 & 3 & 2 & 2 \\
% %   M_2 & 0 & 4 & 5 \\
% %   M_3 & 0 & 4 & 5
% %
% % \end{tabular}
% % \end{center}

% \end{vbframe}

% \normalsize
% \begin{vbframe}{Mathematical intuition}

% The example explained duality from an economic point of view. But what is the mathematical intuition behind duality?

% \lz

% \textbf{Idea: } In minimization problems one is often interested in \textbf{lower bounds} of the objective function. How could we derive a lower bound for the problem above?

% \lz

% If we \enquote{skilfully} multiply the three inequalities by factors and add factors (similar to a linear system), we can find a lower bound.

% \framebreak
% \vspace*{-1.6cm}

% \begin{eqnarray*}
% \min_{\bm{x}\in \R^2} && 50x_1 + 80x_2  \\
% \text{s.t. } && 3x_1 \ge 6 \quad \vert\textcolor{green}{\cdot 5}\\
% && 2x_1 + 4x_2 \ge 10 \quad \vert \textcolor{green}{\cdot 5} \\
% && 2x_1 + 5x_2 \ge 8 \quad \vert \textcolor{green}{\cdot 12} \\
% && \bm{x} \ge 0
% \end{eqnarray*}

% \vspace*{-0.2cm}

% If we add up the constraints we obtain

% \vspace*{-0.5cm}
% \begin{eqnarray*}
% & & \textcolor{green}{5} \cdot (3x_1) + \textcolor{green}{5} \cdot (2x_1 + 4x_2) + \textcolor{green}{12} \cdot (2x_1 + 5x_2)
% \\ &=& 15x_1 + 10x_1 + 24x_1 + 20x_2 + 60x_2 = 49 x_1 + 80 x_2 \\
% &\ge& 30 + 50 + 96 = 176
% \end{eqnarray*}

% Since $x_1 \ge 0$ we found a lower bound because

% $$
% 50x_1 + 80 x_2 \ge 49 x_1 + 80 x_2 \ge 176.
% $$

% \framebreak

% Maybe we could have transformed the equations more cleverly and then found a \textbf{better} lower bound. We replace the multipliers $5, 5, 12$ with $\alpha_1, \alpha_2, \alpha_3$ and demand (as a condition for a lower bound).

% % \begin{eqnarray*}
% % \min_{\bm{x}\in \R^2} && 50x_1 + 80x_2  \\
% % \text{u. d. N. } && 3x_1 \ge 6 \quad \vert \textcolor{green}{\cdot \alpha_1}\\
% % && 2x_1 + 4x_2 \ge 10 \quad \vert \textcolor{green}{\cdot \alpha_2} \\
% % && 2x_1 + 5x_2 \ge 8 \quad \vert \textcolor{green}{\cdot \alpha_3} \\
% % && \bm{x} \ge 0
% % \end{eqnarray*}
% %
% % \framebreak

% \vspace*{-0.5cm}

% \begin{eqnarray*}
% && 50x_1 + 80x_2 \ge \textcolor{green}{\alpha_1} (3x_1) + \textcolor{green}{\alpha_2} (2x_1 + 4x_2) + \textcolor{green}{\alpha_3} (2x_1 + 5x_2) \\ &=& (3 \textcolor{green}{\alpha_1} + 2  \textcolor{green}{\alpha_2} + 2 \textcolor{green}{\alpha_3}) x_1 + (4  \textcolor{green}{\alpha_2} + 5 \textcolor{green}{\alpha_3}) x_2
% \end{eqnarray*}

% or equivalently

% \vspace*{-0.5cm}
% \begin{eqnarray*}
% 3 \textcolor{green}{\alpha_1} + 2  \textcolor{green}{\alpha_2} + 2 \textcolor{green}{\alpha_3} \le 50, \\
% 4  \textcolor{green}{\alpha_2} + 5 \textcolor{green}{\alpha_3} \le 80.
% \end{eqnarray*}

% The lower bound (right-hand side of the inequality) is given by

% $$
% \textcolor{green}{\alpha_1} \cdot 6 + \textcolor{green}{\alpha_2} \cdot 10 + \textcolor{green}{\alpha_3} \cdot 8.
% $$

% \framebreak

% We are interested in a \textbf{largest possible} lower bound, because it gives us the most information. As an optimization problem

% \begin{eqnarray*}
% \max_{\bm{\alpha} \in \R^3} && 6 \alpha_1 + 10 \alpha_2 + 8 \alpha_3 \\
% \text{s.t. } && 3\alpha_1 + 2\alpha_2 + 2\alpha_3 \le 50 \\
% && 4\alpha_2 + 5\alpha_3 \le 80\\
% && \bm{\alpha} \ge 0.
% \end{eqnarray*}


% \end{vbframe}



% \begin{vbframe}{Duality}

% We denote

% \vspace*{-0.2cm}
% \begin{eqnarray*}
% \max_{\bm{\alpha} \in \R^m} && g(\bm{\alpha}) := \bm{\alpha}^T\bm{b}\\
% \text{s.t. } && \bm{\alpha}^T\bm{A} \le \bm{c}^T \\
% && \bm{\alpha} \ge 0
% \end{eqnarray*}

% as the \textbf{dual problem} of

% \begin{eqnarray*}
% \min_{\bm{x} \in \R^n} && f(\bm{x}) := \bm{c}^T\bm{x}\\
% \text{s.t. } && \bm{A}\bm{x} \ge \bm{b} \\
% && \bm{x} \ge 0
% \end{eqnarray*}

% (\textbf{primal problem}).

% \framebreak

% Duality overview:

% \begin{footnotesize}
% \begin{center}
%     \begin{tabular}{c | c | c | c}
%     & $\begin{array}{c} \text{Primal} \\ \text{(minimize)} \end{array}$ & $\begin{array}{c} \text{Dual} \\ \text{(maximize)} \end{array}$ \\
%     \hline
%     \multirow{3}{*}{$\begin{array}{c} \text{condition} \end{array}$} & $\le$ & $\le 0$ & \multirow{3}{*}{variable} \\
%   & $\ge$ & $\ge 0$ & \\
%     & $=$ & unconstrained & \\
%     \hline
%     \multirow{3}{*}{variable} & $\ge 0 $ & $\le$ & \multirow{3}{*}{$\begin{array}{c} \text{condition} \end{array}$}\\
%     & $\le 0$ &  $\ge$ & \\
%     & unconstrained & $=$ & \\
% \hline
%     \end{tabular}
% \end{center}
% \end{footnotesize}
% \end{vbframe}

% \begin{vbframe}{Duality theorem}

% In general, the \textbf{weak duality theorem}$^{(*)}$ applies to all feasible $\bm{x}, \bm{\alpha}$

% $$
% g(\bm{\alpha}) = \bm{\alpha}^T\bm{b} \le \bm{c}^T\bm{x}  = f(\bm{x})
% $$

% The value of the dual function is therefore \textbf{always} a lower bound for the objective function value of the primal problem.

% \lz

% \textbf{Proof:}
% $\bm{\alpha}^T\bm{b} \overset{\bm{Ax} \ge b}{\le}\bm{\alpha}^T\bm{Ax} \overset{\bm{\alpha}^T\bm{A} \le \bm{c}^T}{\le}\bm{c}^T\bm{x}$

% \framebreak

% The \textbf{strong duality theorem} states that if one of the two problems has a constrained solution, then the other also has a constrained solution. The objective function values are the same in this case.

% $$
% g(\bm{\alpha}^*) = (\bm{\alpha}^*)^T\bm{b} = \bm{c}^T\bm{x}^* = f(\bm{x}^*).
% $$

% In this case, the dual problem can be solved instead of the primal problem, which can lead to enormous runtime advantages, especially with many constraints and few variables.

% \lz

% The \textbf{dual simplex algorithm}, which has emerged as a standard procedure for Linear programming, is based on this idea.


% \end{vbframe}

\endlecture
\end{document}


