\documentclass[11pt,compress,t,notes=noshow, xcolor=table]{beamer}

\input{../../style/preamble}
\input{../../latex-math/basic-math}
\input{../../latex-math/basic-ml}

\title{Optimization in Machine Learning}

\begin{document}

\titlemeta{
Constrained Optimization
}{
Linear Programming
}{
figure_man/convex_programs.png
}{
\item Instances of LPs underlying statistical estimation
\item Definition of an LP
\item Geometric intuition of LPs
}

\begin{framei}{Linear programming}
\item \textbf{Linear problems} (LP): linear objective + linear constraints
\item \textbf{Example:}
$$
\min - x_1 - x_2 \quad \text{s.t. } x_1 + 2x_2 \le 1, \; 2x_1 + x_2 \le 1, \; x_1, x_2 \ge 0
$$
\imageC[0.35]{figure_man/cons-linear-pro-example.png}
\end{framei}


\begin{framei}[fs=small]{LP examples: Quantile regression}
\item \textbf{(Sparse) Quantile regression}:
$$
\min_{\beta_0,\bm{\beta}} \frac{1}{n}\sum_{i = 1}^n \rho_\tau \left(y^{(i)} - \beta_0 -  \bm{\beta}^\top\xv^{(i)}\right) \quad \text{s.t. } \|\bm{\beta}\|_1 \le t
$$
where $\beta_0 \in \R$ and $\bm{\beta} \in \R^p$ are coefficients, and $\rho_\tau$, $\tau \in [0,1]$, is the check function:
$$
\rho_\tau(s) =
\begin{cases}
\tau \cdot s & \text{if } s>0, \\
-(1-\tau)\cdot s & \text{if } s\le0.
\end{cases}
$$
\item \textbf{Case $\tau = 1/2$:} Median regression (a.k.a.\ least absolute errors (LAE), least absolute deviations (LAD))
\item Parameter $t \geq 0$ determines regularization
\end{framei}


\begin{framei}[fs=small]{LP examples: Dantzig selector}
\item \textbf{Dantzig selector}:
$$
\min_{\bm{\beta}\in \R^p} \|\bm{\beta}\|_1 \quad \text{s.t. } \| \Xmat^\top (\Xmat \bm{\beta} - \bm{y})\|_\infty \le \lambda
$$
where $\bm{y} \in \R^n$, $\Xmat \in \R^{n \times p}$, and $\lambda > 0$ is a tuning parameter
\item The infinity norm is defined as $\| x \|_\infty = \max\{|x_1|, \dots, |x_n|\}$
\item Similar to (and behaves similar to) the Lasso
\item Introduced for variable selection by Tao and Cand\`es
\item Details about LPs in statistical estimation: see \href{https://etd.ohiolink.edu/apexprod/rws_etd/send_file/send?accession=osu1222035715&disposition=inline}{\beamergotobutton{PhD thesis of Yonggong Gao}}
\end{framei}


\begin{framei}{LP: Standard form}
\item LPs can be formulated in \textbf{standard form}:
$$
\max_{\xv \in \R^n} \mathbf{c}^\top\xv \quad \text{s.t. } \Amat\xv \le \mathbf{b}, \; \xv \ge 0
$$
with $\Amat \in \R^{m\times n}$, $\mathbf{b} \in \R^m$
\item Constraints are to be understood \textbf{componentwise}
\item $\xv \ge 0$: ``non-negativity constraint''
\item $\mathbf{c}$: ``cost vector''
\end{framei}


\begin{framei}[sep=L]{LP: Converting to standard form}
\item $\min \longleftrightarrow \max$: multiply objective function by $-1$
\item $\leq \; \longleftrightarrow \; \geq$: multiply inequality by $-1$
\item $= \; \longleftrightarrow \; \leq,\geq$: replace $\mathbf{a}_i^\top\xv=b_i$ by $\mathbf{a}_i^\top\xv\ge b_i$ \textit{and} $\mathbf{a}_i^\top\xv\le b_i$
\item No non-negativity constraint: replace $x_i$ by $x_i^+ - x_i^-$ with $x_i^+, x_i^- \ge 0$ (positive and negative part)
\end{framei}


\begin{framei}{LP: Standard form example}
\item Example:
$$
\min - x_1 - x_2 \quad \text{s.t. } x_1 + 2x_2 \le 1, \; 2x_1 + x_2 \le 1, \; x_1, x_2 \ge 0
$$
\item Can also be formulated as:
$$
\max (1, 1) \begin{pmatrix} x_1 \\ x_2 \end{pmatrix} \quad \text{s.t. }  \begin{pmatrix}  1 &  2 \\  2 &  1 \end{pmatrix} \xv \le \begin{pmatrix}  1 \\  1 \end{pmatrix}, \; \xv \ge 0
$$
\end{framei}


\begin{framei}{Geometric interpretation: Feasible set}
\item $i$-th inequality constraint: $\mathbf{a}_i^\top \xv \le b_i$
\item Points $\{ \xv: \mathbf{a}_i^\top \xv = b_i\}$ form a hyperplane in $\R^n$ \\
($\mathbf{a}_i$ is perpendicular to the hyperplane and called \textbf{normal vector})
\item Points $\{\xv: \mathbf{a}_i^\top \xv \ge b_i\}$ lie on the side of the hyperplane into which the normal vector points (``half-space'')
\item Each inequality divides the space into two halves
\item \textbf{Claim:} Points satisfying \textbf{all} inequalities form a \textbf{convex polytope}
\imageC[0.4]{figure_man/cons-linear-pro-example.png}
\end{framei}


\begin{framei}{Geometric interpretation: Polytopes}
\item A \textbf{polytope} is a generalized polygon in arbitrary dimensions
\item A polytope consists of several sub-polytopes:
\begin{itemize}
    \item $0$-polytope: point
    \item $1$-polytope: line
    \item $2$-polytope: polygon, ...
\end{itemize}
\item \textbf{General:}
\begin{itemize}
    \item $d$-polytope is formed from several $(d-1)$-polytopes (``facets'')
    \item $(d-1)$-polytope is formed from several $(d-2)$-polytopes
\end{itemize}
\end{framei}


\begin{framei}{Geometric interpretation: Convexity}
\item Points $\{\xv: \mathbf{a}_i^\top \xv = b_i\}$ lie on the boundary of the polytope
\item Polytope $\{ \xv: \Amat \xv \le \mathbf{b}\}$ is convex: For $\xv_1,\xv_2 \in \mathcal{S}$ and $t \in [0, 1]$
\begin{align*}
    \Amat(\xv_1 + t(\xv_2 - \xv_1)) &= \Amat\xv_1 + t(\Amat\xv_2 - \Amat\xv_1) \\
    &= (1 -t)\underbrace{\Amat\xv_1}_{\le \mathbf{b}} + t\underbrace{\Amat\xv_2}_{\le \mathbf{b}} \le (1 - t) \mathbf{b} + t \mathbf{b} = \mathbf{b}
\end{align*}
\item Polytope $\{ \xv: \Amat \xv \le \mathbf{b}\}$ is an $n$-\textbf{simplex}, i.e., convex hull of $n + 1$ \textit{affinely independent} points
\end{framei}


\begin{framei}{Geometric interpretation: Objective function}
\item \textbf{Linear case:} Contour lines form a hyperplane
\item \textbf{Observe:} $\mathbf{c}$ is gradient and perpendicular to contour lines
\item Solution ``touches'' the polygon
\imageC[0.45]{figure_man/cons-opposite-direction.png}
\end{framei}


\begin{framei}{Solutions to LP}
\item There are 3 cases for linear programming:
\begin{enumerate}
\item Feasible set is \textbf{empty} $\Rightarrow$ LP is infeasible
\item Feasible set is \textbf{unbounded}
\item Feasible set is \textbf{bounded} $\Rightarrow$ LP has at least one solution
\end{enumerate}
\imageC[0.8]{figure_man/cons-solutions-lp.png}
\end{framei}


\begin{framei}{Solutions to LP}
\item If LP is solvable and constrained (neither case 1 nor case 2), there is always an optimal point that can \textbf{not} be convexly combined from other points in the polytope
\item The optimal solution is then a corner, edge or side of the polytope
\end{framei}

\endlecture
\end{document}
