\documentclass[11pt,compress,t,notes=noshow, xcolor=table]{beamer}

\input{../../style/preamble}
\input{../../latex-math/basic-math}
\input{../../latex-math/basic-ml}


\newcommand{\titlefigure}{figure_man/convex_programs.png}
\newcommand{\learninggoals}{
\item Definition
\item Max. Likelihood 
\item Normal regression
\item Risk Minimization
}


%\usepackage{animate} % only use if you want the animation for Taylor2D

\title{Optimization in Machine Learning}
%\author{Bernd Bischl}
\date{}

\begin{document}

\lecturechapter{LP Duality}
\lecture{Optimization in Machine Learning}
\sloppy
%%%%%%%%%%%%%%%%%%%%%%%%%%%%%%%%%%%%%%%%%%%%%%%%%%%%%%%%%%%%%%%%%%%%%%%%%%%%%%%%%%%
\begin{vbframe}{Duality: introductory example}

\textbf{Example:}

A bakery sells brownies for $50$ ct and mini cheesecakes for $80$ ct each. The two products contain the following ingredients

\begin{center}
\begin{tabular}{ r c c c}
    & \text{Chocolate} & \text{Sugar} & \text{Creamcheese} \\
    \hline
  \text{Brownie} & 3 & 2 & 2 \\
  \text{Cheesecake} & 0 & 4 & 5
\end{tabular}
\end{center}

A student wants to minimize his expenses, but at the same time eat at least $6$ units of chocolate, $10$ units of sugar and $8$ units of creamcheese.

\framebreak

He is therefore confronted with the following optimization problem:

\begin{eqnarray*}
\min_{\bm{x}\in \R^2} && 50x_1 + 80x_2 \\
\text{s.t. } && 3x_1 \ge 6 \\
&& 2x_1 + 4x_2 \ge 10 \\
&& 2x_1 + 5x_2 \ge 8 \\
&& \bm{x} \ge 0
\end{eqnarray*}

\framebreak

The solution of the Simplex algorithm:
\vspace{0.3cm}

\footnotesize
\begin{verbatim}
res = solveLP(cvec = c, bvec = b, Amat = A)
summary(res)
##
##
## Results of Linear Programming / Linear Optimization
##
## Objective function (Minimum): 220
##
## Solution
## opt
## 1 2.0
## 2 1.5
\end{verbatim}
% \col

%<<echo = F>>=
%A = - matrix(c(3, 2, 2, 0, 4, 5), ncol = 2)
%b = - c(6, 10, 8)
%c = c(50, 80)
%@

%<<>>=
%res = solveLP(cvec = c, bvec = b, Amat = A)
%summary(res)
%@

\framebreak
\normalsize
The baker informs the supplier that he needs at least $6$ units of chocolate, $10$ units of sugar and $8$ units of creamcheese to meet the student's requirements.

\lz

The supplier asks himself how he must set the prices for chocolate, sugar and creamcheese ($\alpha_1, \alpha_2, \alpha_3$) such that he can

\begin{itemize}
\item maximize his revenue
$$
\max_{\bm{\alpha} \in \R^3}6 \alpha_1 + 10 \alpha_2 + 8 \alpha_3
$$
\item and at the same time ensure that the baker buys from him (purchase cost $\le$ selling price)
\begin{eqnarray*}
3\alpha_1 + 2\alpha_2 + 2\alpha_3 &\le& 50 \qquad \text{Brownie} \\
4\alpha_2 + 5\alpha_3 &\le& 80 \qquad \text{Cheesecake}
\end{eqnarray*}

\end{itemize}

\framebreak

The presented example is known as a \textbf{dual problem}. The variables $\alpha_i$ are called \textbf{dual variables}.

\lz

In an economic context, dual variables can often be interpreted as \textbf{shadow prices} for certain goods.

\lz

If we solve the dual problem, we see that the dual problem has the same objective function value as the primal problem. This is later referred to as \textbf{strong duality}.

% \lz
%
% Der Lieferant sollte eine Einheit Schokolade für mindestens $\alpha_1$ verkaufen, der Bäcker aber höchstens zu $\alpha_1$ einkaufen.
\framebreak
\footnotesize
\begin{verbatim}
res = solveLP(cvec = c, bvec = b, Amat = A, maximum = T)
summary(res)
##
##
## Results of Linear Programming / Linear Optimization
##
## Objective function (Maximum): 220
##
## Solution
## opt
## 1 3.333333
## 2 20.000000
## 3 0.000000
\end{verbatim}
% \col

%<<echo = F>>=
%A = matrix(c(3, 0, 2, 4, 2, 5), ncol = 3)
%b = c(50, 80)
%c = c(6, 10, 8)
%@

%<<>>=
%res = solveLP(cvec = c, bvec = b, Amat = A, maximum = T)
%summary(res)
%@


% \textbf{Beispiel:} Eine Firma stellt Tische und Stühle her und verkauft diese für $30$ Euro (Tisch),  $20$ Euro (Stuhl). Hierfür gibt es $3$ Maschinen, die jeweils nur eine gewisse Zeit zur Verfügung stehen.
%
% \lz
%
% In der Tabelle ist aufgelistet, wie viel Stunden die jeweilige Maschine für Stuhl bzw. Tisch benötigt.
%
% \begin{center}
% \begin{tabular}{ r | c c | c}
%   \text{Maschine}  & \text{Stuhl} & \text{Tisch} & \text{Verfügbarkeit (in h)} \\
%     \hline
%   M_1 & 3 & 2 & 2 \\
%   M_2 & 0 & 4 & 5 \\
%   M_3 & 0 & 4 & 5
%
% \end{tabular}
% \end{center}

\end{vbframe}

\normalsize
\begin{vbframe}{Mathematical intuition}

The example explained duality from an economic point of view. But what is the mathematical intuition behind duality?

\lz

\textbf{Idea: } In minimization problems one is often interested in \textbf{lower bounds} of the objective function. How could we derive a lower bound for the problem above?

\lz

If we \enquote{skilfully} multiply the three inequalities by factors and add factors (similar to a linear system), we can find a lower bound.

\framebreak
\vspace*{-1.6cm}

\begin{eqnarray*}
\min_{\bm{x}\in \R^2} && 50x_1 + 80x_2  \\
\text{s.t. } && 3x_1 \ge 6 \quad \vert\textcolor{green}{\cdot 5}\\
&& 2x_1 + 4x_2 \ge 10 \quad \vert \textcolor{green}{\cdot 5} \\
&& 2x_1 + 5x_2 \ge 8 \quad \vert \textcolor{green}{\cdot 12} \\
&& \bm{x} \ge 0
\end{eqnarray*}

\vspace*{-0.2cm}

If we add up the constraints we obtain

\vspace*{-0.5cm}
\begin{eqnarray*}
& & \textcolor{green}{5} \cdot (3x_1) + \textcolor{green}{5} \cdot (2x_1 + 4x_2) + \textcolor{green}{12} \cdot (2x_1 + 5x_2)
\\ &=& 15x_1 + 10x_1 + 24x_1 + 20x_2 + 60x_2 = 49 x_1 + 80 x_2 \\
&\ge& 30 + 50 + 96 = 176
\end{eqnarray*}

Since $x_1 \ge 0$ we found a lower bound because

$$
50x_1 + 80 x_2 \ge 49 x_1 + 80 x_2 \ge 176.
$$

\framebreak

Maybe we could have transformed the equations more cleverly and then found a \textbf{better} lower bound. We replace the multipliers $5, 5, 12$ with $\alpha_1, \alpha_2, \alpha_3$ and demand (as a condition for a lower bound).

% \begin{eqnarray*}
% \min_{\bm{x}\in \R^2} && 50x_1 + 80x_2  \\
% \text{u. d. N. } && 3x_1 \ge 6 \quad \vert \textcolor{green}{\cdot \alpha_1}\\
% && 2x_1 + 4x_2 \ge 10 \quad \vert \textcolor{green}{\cdot \alpha_2} \\
% && 2x_1 + 5x_2 \ge 8 \quad \vert \textcolor{green}{\cdot \alpha_3} \\
% && \bm{x} \ge 0
% \end{eqnarray*}
%
% \framebreak

\vspace*{-0.5cm}

\begin{eqnarray*}
&& 50x_1 + 80x_2 \ge \textcolor{green}{\alpha_1} (3x_1) + \textcolor{green}{\alpha_2} (2x_1 + 4x_2) + \textcolor{green}{\alpha_3} (2x_1 + 5x_2) \\ &=& (3 \textcolor{green}{\alpha_1} + 2  \textcolor{green}{\alpha_2} + 2 \textcolor{green}{\alpha_3}) x_1 + (4  \textcolor{green}{\alpha_2} + 5 \textcolor{green}{\alpha_3}) x_2
\end{eqnarray*}

or equivalently

\vspace*{-0.5cm}
\begin{eqnarray*}
3 \textcolor{green}{\alpha_1} + 2  \textcolor{green}{\alpha_2} + 2 \textcolor{green}{\alpha_3} \le 50, \\
4  \textcolor{green}{\alpha_2} + 5 \textcolor{green}{\alpha_3} \le 80.
\end{eqnarray*}

The lower bound (right-hand side of the inequality) is given by

$$
\textcolor{green}{\alpha_1} \cdot 6 + \textcolor{green}{\alpha_2} \cdot 10 + \textcolor{green}{\alpha_3} \cdot 8.
$$

\framebreak

We are interested in a \textbf{largest possible} lower bound, because it gives us the most information. As an optimization problem

\begin{eqnarray*}
\max_{\bm{\alpha} \in \R^3} && 6 \alpha_1 + 10 \alpha_2 + 8 \alpha_3 \\
\text{s.t. } && 3\alpha_1 + 2\alpha_2 + 2\alpha_3 \le 50 \\
&& 4\alpha_2 + 5\alpha_3 \le 80\\
&& \bm{\alpha} \ge 0.
\end{eqnarray*}


\end{vbframe}



\begin{vbframe}{Duality}

We denote

\vspace*{-0.2cm}
\begin{eqnarray*}
\max_{\bm{\alpha} \in \R^m} && g(\bm{\alpha}) := \bm{\alpha}^T\bm{b}\\
\text{s.t. } && \bm{\alpha}^T\bm{A} \le \bm{c}^T \\
&& \bm{\alpha} \ge 0
\end{eqnarray*}

as the \textbf{dual problem} of

\begin{eqnarray*}
\min_{\bm{x} \in \R^n} && f(\bm{x}) := \bm{c}^T\bm{x}\\
\text{s.t. } && \bm{A}\bm{x} \ge \bm{b} \\
&& \bm{x} \ge 0
\end{eqnarray*}

(\textbf{primal problem}).

\framebreak

Duality overview:

\begin{footnotesize}
\begin{center}
    \begin{tabular}{c | c | c | c}
    & $\begin{array}{c} \text{Primal} \\ \text{(minimize)} \end{array}$ & $\begin{array}{c} \text{Dual} \\ \text{(maximize)} \end{array}$ \\
    \hline
    \multirow{3}{*}{$\begin{array}{c} \text{condition} \end{array}$} & $\le$ & $\le 0$ & \multirow{3}{*}{variable} \\
  & $\ge$ & $\ge 0$ & \\
    & $=$ & unconstrained & \\
    \hline
    \multirow{3}{*}{variable} & $\ge 0 $ & $\le$ & \multirow{3}{*}{$\begin{array}{c} \text{condition} \end{array}$}\\
    & $\le 0$ &  $\ge$ & \\
    & unconstrained & $=$ & \\
\hline
    \end{tabular}
\end{center}
\end{footnotesize}
\end{vbframe}

\begin{vbframe}{Duality theorem}

In general, the \textbf{weak duality theorem}$^{(*)}$ applies to all feasible $\bm{x}, \bm{\alpha}$

$$
g(\bm{\alpha}) = \bm{\alpha}^T\bm{b} \le \bm{c}^T\bm{x}  = f(\bm{x})
$$

The value of the dual function is therefore \textbf{always} a lower bound for the objective function value of the primal problem.

\lz

\textbf{Proof:}
$\bm{\alpha}^T\bm{b} \overset{\bm{Ax} \ge b}{\le}\bm{\alpha}^T\bm{Ax} \overset{\bm{\alpha}^T\bm{A} \le \bm{c}^T}{\le}\bm{c}^T\bm{x}$

\framebreak

The \textbf{strong duality theorem} states that if one of the two problems has a constrained solution, then the other also has a constrained solution. The objective function values are the same in this case.

$$
g(\bm{\alpha}^*) = (\bm{\alpha}^*)^T\bm{b} = \bm{c}^T\bm{x}^* = f(\bm{x}^*).
$$

In this case, the dual problem can be solved instead of the primal problem, which can lead to enormous runtime advantages, especially with many constraints and few variables.

\lz

The \textbf{dual simplex algorithm}, which has emerged as a standard procedure for Linear programming, is based on this idea.


\end{vbframe}

\endlecture
\end{document}


