\documentclass[11pt,compress,t,notes=noshow, xcolor=table]{beamer}

\input{../../style/preamble}
\input{../../latex-math/basic-math}

% sets
\newcommand{\CC}[1]{\mathcal{C}^{#1}}         % class C^k
\newcommand{\CCinf}{\mathcal{C}^{\infty}}     % class C^infty


% objectives, domains, points
\renewcommand{\S}{\mathcal{S}}              % domain of f
\newcommand{\xv}{\bm{x}}                  % vector x (bold)
\newcommand{\yv}{\bm{y}}                  % vector y (bold)
\newcommand{\vv}{\bm{v}}                  % vector v (bold)
\newcommand{\wv}{\bm{w}}                  % vector w (bold)
\newcommand{\xvs}{\bm{x}^\ast}            % x-star,  theoretical optimum
\newcommand{\xs}{x^\ast}                  % x-star,  theoretical optimum
\newcommand{\fxvs}{f(\xvs)}               % f at x-star, theoretical optimum
\renewcommand{\H}{\mathbf{H}}                     % Hessian matrix



% some other stuff
\providecommand{\eps}{\epsilon}               % epsilon
\newcommand{\A}{\mathbf{A}}                     % Hessian matrix
\newcommand{\bv}{\mathbf{b}}                     % Hessian matrix
\newcommand{\zv}{\mathbf{z}}                     % Hessian matrix
\newcommand{\V}{\mathbf{V}}                     % Hessian matrix


\newcommand{\vvmax}{\bm{v}_\text{max}}                  % vector v (bold)
\newcommand{\vvmin}{\bm{v}_\text{min}}                  % vector v (bold)
\newcommand{\lammax}{\lambda_\text{max}}                  % vector v (bold)
\newcommand{\lammin}{\lambda_\text{min}}                  % vector v (bold)

\newcommand{\sumid}{\sum\limits_{i=1}^d} % summation from i=1 to n


\title{Optimization in Machine Learning}

\begin{document}

\titlemeta{
First order methods
}{
Weaknesses of GD -- Curvature
}{
figure_man/taylor_2D_quadratic.png
}{
\item Effects of curvature
\item Step size effect in GD
}

\begin{framei}{Reminder: Local quadratic geometry}
\item Locally approximate smooth function by quadratic Taylor polynomial:
$$\fx \approx f(\xtil) + \nabla f(\xtil)^\top(\xv-\xtil) +
\frac 12(\xv-\xtil)^\top\nabla^2 f(\xtil)(\xv-\xtil)$$
\imageC[0.5][https://daniloroccatano.blog]{figure_man/taylor_2D_quadratic.png}
\end{framei}


\begin{framei}{Reminder: Local quadratic geometry}
\item Study Hessian $\mathbf{H} = \nabla^2 f(\xv^{[t]})$ in GD to discuss effect of curvature 
\item \textbf{Recall} for quadratic forms:
\begin{itemizeM}
\item Eigenvector $\textbf{v}_\text{max}$ ($\textbf{v}_\text{min}$) is direction of largest (smallest) curvature
\item $\mathbf{H}$ called ill-conditioned if $\kappa(\mathbf{H}) = |\lambda_\text{max}| / |\lambda_\text{min}|$ is large
\end{itemizeM}
\vfill
\splitV{\imageC[0.9]{figure/ill-cond_1.png}
}{\imageC[0.9]{figure/ill-cond_2.png}}
\end{framei}


\begin{framei}{Effects of curvature}
\item Intuitively, curvature determines reliability of a GD step
\imageC[0.8][https://www.deeplearningbook.org]{figure_man/curvature.png}
\centering\small
Quadratic objective $f$ (blue) with gradient approximation (dashed green). \textbf{Left:} $f$ decreases faster than $\nabla f$ predicts. \textbf{Center:} $\nabla f$ predicts decrease correctly. \textbf{Right:} $f$ decreases more slowly than $\nabla f$ predicts.
\end{framei}


\begin{framei}{Curvature and step size in GD}
\item\textbf{Worst case:} $\mathbf{H}$ is ill-conditioned.What does this mean for GD?
\vfill
\begin{itemizeM}
\item Quadratic Taylor polynomial of $f$ around $\tilde{\xv}$ (with gradient $\mathbf{g} = \nabla f$)
$$f(\xv) \approx f(\xtil) + (\xv - \xtil)^\top \mathbf{g} + \frac{1}{2} (\xv - \xtil)^\top \mathbf{H} (\xv - \xtil)$$
\item GD step with step size $\alpha > 0$ yields
$$f(\xtil-\alpha \mathbf{g}) \approx f(\xtil) - \alpha \mathbf{g}^\top\mathbf{g} + \frac{1}{2} \alpha^2 \mathbf{g}^\top \H\,\mathbf{g}$$ 
\item If $\mathbf{g}^\top \bm{H} \mathbf{g} > 0$, we can solve for optimal step size $\alpha^\ast$:
$$\alpha^\ast = \frac{\mathbf{g}^\top \mathbf{g}}{\mathbf{g}^\top \H\, \mathbf{g}}$$
\end{itemizeM}
\end{framei}


\begin{framei}{Curvature and step size in GD}
\item If $\mathbf{g}$ points along $\mathbf{v}_{\text{max}}$ (largest curvature), optimal step size is
$$\alpha^\ast = \frac{\mathbf{g}^\top \mathbf{g}}{\mathbf{g}^\top \H \mathbf{g}} = \frac{\mathbf{g}^\top \mathbf{g}}{\lammax \mathbf{g}^\top \mathbf{g}} = \frac{1}{\lammax}.$$ 
$\Rightarrow$ \textit{Large} step sizes can be problematic.
\item If $\mathbf{g}$ points along $\mathbf{v}_{\text{min}}$ (smallest curvature), then analogously
$$\alpha^* = \frac{1}{\lammin}.$$
$\Rightarrow$ \textit{Small} step sizes can be problematic.
\item \textbf{Ideally}: Perform large step along $\mathbf{v}_\text{min}$ but small step along $\mathbf{v}_\text{max}$.
\end{framei}


\begin{framei}{Curvature and step size in GD}
\item What if $\mathbf{g}$ is not aligned with eigenvectors?
\item Consider 2D case: Decompose $\mathbf{g}$ (black) into $\textcolor{red}{\mathbf{v}_{\text{max}}}$ and $\textcolor{blue}{\mathbf{v}_{\text{min}}}$
\imageC[0.7]{figure/curvature_1.png}
\item Ideally, perform \textbf{large} step along $\textcolor{blue}{\mathbf{v}_{\text{min}}}$ but \textbf{small} step along $\textcolor{red}{\mathbf{v}_{\text{max}}}$
\item However, gradient almost only points along $\textcolor{red}{\mathbf{v}_\text{max}}$
\end{framei}


\begin{framei}{Curvature and step size in GD}
\item GD is not aware of curvatures and can only walk along $\mathbf{g}$
\item Large step sizes result in ``zig-zag'' behaviour.
\item Small step sizes result in weak performance.
\imageC[0.8]{figure/curvature_2.png}
\centering\footnotesize Poorly conditioned quadratic form. GD with large (red) and small (blue) step size. For both, convergence to optimum is slow.
\end{framei}


\begin{framei}{Curvature and step size in GD}
\item Large step sizes for ill-conditioned Hessian can even increase
$$f(\xtil - \alpha \mathbf{g}) \approx f(\xtil) - \alpha \mathbf{g}^\top \mathbf{g} + \frac{1}{2} \alpha^2 \mathbf{g}^\top \H \mathbf{g}$$
if $$\frac{1}{2} \alpha^2 \mathbf{g}^\top \H \mathbf{g} > \alpha \mathbf{g}^\top\mathbf{g} \quad \Leftrightarrow \quad \alpha > 2\frac{\mathbf{g}^\top \mathbf{g}}{\mathbf{g}^\top \H \mathbf{g}}.$$
\item Ill-conditioning in practice: Monitor gradient norm and objective
\imageC[0.6][https://www.deeplearningbook.org]{figure_man/no_critical.png}
\end{framei}


\begin{framei}{Curvature and step size in GD}
\item If gradient norms $\|\mathbf{g}\|$ increase, GD is not converging since $\mathbf{g} \not= 0$.
\item Even if $\|\mathbf{g}\|$ increases, objective may stay approximately constant:
$$\underbrace{f(\xtil-\alpha \mathbf{g})}_{\text{$\approx$ constant}} \approx f(\xtil) - \alpha \underbrace{\mathbf{g}^\top \mathbf{g}}_{\text{increases}} + \frac{1}{2} \alpha^2 \underbrace{\mathbf{g}^\top \H \mathbf{g}}_{\text{increases}}$$
\end{framei}

\endlecture
\end{document}
