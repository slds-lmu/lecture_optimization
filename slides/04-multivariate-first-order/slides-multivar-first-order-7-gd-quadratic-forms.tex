\documentclass[11pt,compress,t,notes=noshow, xcolor=table]{beamer}

\input{../../style/preamble}
\input{../../latex-math/basic-math}
\input{../../latex-math/basic-ml}


\newcommand{\titlefigure}{figure_man/momentum_illustration_medium.png}
\newcommand{\learninggoals}{
\item Definition
\item Max. Likelihood 
\item Normal regression
\item Risk Minimization
}


%\usepackage{animate} % only use if you want the animation for Taylor2D

\title{Optimization in Machine Learning}
%\author{Bernd Bischl}
\date{}

\begin{document}

\lecturechapter{First order methods: GD on quadratic forms}
\lecture{Optimization in Machine Learning}
\sloppy
%%%%%%%%%%%%%%%%%%%%%%%%%%%%%%%%%%%%%%%%%%%%%%%%%%%%%%%%%%%%%%%%%%%%%%%%%%%%%%%%%%%

\begin{vbframe}{GD on quadratic forms}

\begin{itemize}
	\item We consider the quadratic function $q(\xv) = \xv^\top \bm{A} \xv - \bm{b}^\top \xv$. 
	\item We assume that $A$ is symmetric and invertible 
	\item The optimal solution is $\xv^\ast = A^{-1} b$ 
	\item As $\nabla q(\xv) = A \xv - b$, the iterations of gradient descent are
\end{itemize}
	$$
	\xv^{[ t+1 ] } = \xv^{[ t ] } - \alpha (A \xv - b)
	$$

\begin{figure}
	\includegraphics[height=0.3\textwidth, keepaspectratio]{figure_man/gd.png} \\
\end{figure}

\begin{footnotesize}
The following slides follow the blogpost by Goh, "Why Momentum Really Works", Distill, 2017. \url{http://doi.org/10.23915/distill.00006}
\end{footnotesize}

\framebreak

For $A$, there exists an eigenvalue decomposition: 

$$
	A = Q \, \text{diag}(\lambda_1, ..., \lambda_n) Q^\top
$$

where the columns of $Q$ contain the eigenvectors $\boldsymbol{e}_i$ and the eigenvalues $\lambda_i$ are sorted from smallest to biggest eigenvalue. 

With this, we perform a change of basis $w^{[ t ] } = Q^\top (x^{[ t ] } - x^\ast)$ to its eigenspace, where all dimensions act independently. 

\begin{figure}
	\includegraphics[height=0.35\textwidth, keepaspectratio]{figure_man/gd_eigenspace.png} \\
\end{figure}
\framebreak
If we now perform GD on $\boldsymbol{w}$, we get

\begin{eqnarray*}
	w_i^{[ t+1 ] } &=& w_i^{[ t ] } - \alpha \lambda_i w_i^{[ t ] } \\
	&=& (1- \alpha \lambda_i) w_i^{[ t ] } = (1- \alpha \lambda_i)^{t+1 } w_i^{[ 0 ] }
\end{eqnarray*}

\begin{figure}
	\includegraphics[height=0.35\textwidth, keepaspectratio]{figure_man/gd_eigenspace.png} \\
\end{figure}

\framebreak

Moving now back to the original space, we get

$$
	\xv^{[ t ] } - \xv^\ast = Q \boldsymbol{w}^{[ t ] } = \sum_{i}^{d} w_i^{[0]} ( 1 - \alpha \lambda_i)^t e_i 
$$

This allows a very intuitive interpretation: each element of $w^{[0]}$ is the component of the error in the initial guess in the eigenbasis and decays with a rate of $1 - \alpha \lambda_i$. 

For most step sizes, the eigenvectors with the largest eigenvalues converge the fastest. 

\framebreak

We now consider the contribution of each eigenvector to the total loss

$$
	q(\xv^{[ t ] }) - q(\xv^\ast) = \frac{1}{2} \sum_{i}^{d} (1 - \alpha \lambda_i)^{2t} \lambda_i (w_i^{[0]})^2 
$$

\begin{figure}
	\includegraphics[height=0.35\textwidth, keepaspectratio]{figure_man/gd_conv.png} \\
\end{figure}

%\framebreak

%The considerations above already gave some guidance on how to choose the optimal step size $\alpha$. In order to converge, $| 1 - \alpha \lambda_i| $ must be strictly less than $1$. 

%\begin{itemize}
%	\item use eigenvectors of $A$
%	\item GD in closed form 
%	\item decomposing the error, plot with convergence
%	\item choosing step size, show dependence on condition number 
%	\item maybe with example of polynomial regression
%\end{itemize}
\end{vbframe}

\begin{vbframe}{Dynamics of momentum}
	
We investigate now the behavior of gradient descent with momentum on quadratic forms. 

Recap: \textbf{Momentum update}

\begin{eqnarray*}
 \boldsymbol{\nu}^{[ t+1 ] } &=& \varphi \boldsymbol{\nu}^{[ t ] } + \alpha \nabla f(\xv^{[ t ] }) \\
 \xv^{[ t+1 ] } &=& \xv^{[ t ] } - \boldsymbol{\nu}^{[ t+1 ] }, 
\end{eqnarray*}

which simplifies to 

\begin{eqnarray*}
	\boldsymbol{\nu}^{[ t+1 ] } &=& \varphi \boldsymbol{\nu}^{[ t ] } + \alpha (A \xv^{[ t ] } - b)  \\
	\xv^{[ t+1 ] } &=& \xv^{[ t ] } - \boldsymbol{\nu}^{[ t+1 ] }, 
\end{eqnarray*}

for the quadratic form. 

\framebreak

Changing the basis as before with $\boldsymbol{w}^{[ t ] } = Q^\top (\xv^{[ t ] } - \xv^\ast)$ and $\boldsymbol{u}^{[ t ] } = Q \boldsymbol{\nu}^{[ t ] }$, we get the following set of equations, where each component acts independently, although $w_i^{[ t ] }$ and $u_i^{[ t ] }$ are coupled: 

\begin{eqnarray*}
	u_i^{[ t+1 ] } &=& \varphi u_i^{[ t ] } + \alpha \lambda_i w_i^{[ t ] }, \\
	w_i^{[ t+1 ] } &=& w_i^{[ t ] } - u_i^{[ t+1 ] }
\end{eqnarray*}

We rewrite this: 

\begin{equation*}
\begin{pmatrix}
	1 & 0 \\
	1 & 1 
\end{pmatrix}  \begin{pmatrix}
u_i^{[ t+1 ] } \\
w_i^{[ t+1 ] }
\end{pmatrix} = 
\begin{pmatrix}
	\varphi & \alpha \lambda_i \\
	0 & 1 
\end{pmatrix}  \begin{pmatrix}
	u_i^{[ t ] } \\
	w_i^{[ t ] }
\end{pmatrix}
\end{equation*}

and invert the matrix on the LHS: 

\begin{equation*}
\begin{pmatrix}
		u_i^{[ t+1 ] } \\
		w_i^{[ t+1 ] }
	\end{pmatrix} = 
	\begin{pmatrix}
		\varphi & \alpha \lambda_i \\
		-\varphi & 1 - \alpha \lambda_i
	\end{pmatrix}  \begin{pmatrix}
		u_i^{[ t ] } \\
		w_i^{[ t ] }
	\end{pmatrix} = R^{ t+1 }  \begin{pmatrix}
	u_i^{0} \\
	w_i^{0}
\end{pmatrix}
\end{equation*}

\framebreak

Taking a $2 \times 2$ matrix to the $t^{th}$ power reduces to a formula involving the eigenvalues of $R$, $\sigma_1$ and $\sigma_2$:

$$
R^t=\begin{cases}
\sigma_1^t R_1 - \sigma_2^t R_2, & \text{if} \, \sigma_1 \neq \sigma_2 \\
\sigma_1^t (t R/\sigma_1 - (t-1) I), & \text{if} \, \sigma_1 = \sigma_2 
\end{cases}
$$

where $R_j = \frac{R - \sigma_j I}{\sigma_1 - \sigma_2}$.


\vspace*{1.5cm}	
In contrast to gradient descent, where we got one geometric series, we have two coupled series with real or complex values.

\framebreak
\begin{figure}
	\includegraphics[width=0.5\textwidth, keepaspectratio]{figure_man/momentum_convergence.png} \\
	\begin{footnotesize} 
		The achieved convergence rate is therefore the slowest of the two, $\text{max} \{|\sigma_1|, |\sigma_2| \}$. Each region shows a different convergence behavior. 
	\end{footnotesize}
\end{figure}

\framebreak

%Our convergence criterion is $\text{max} \{|\sigma_1|, |\sigma_2| \} < 1$
%\begin{figure}
%	\includegraphics[height=0.35\textwidth, keepaspectratio]{figure_man/momentum_convergence.png} \\
%	\begin{footnotesize} 
%		Depending on $\text{max} \{|\sigma_1|, |\sigma_2| \}$, each region shows different convergence behavior. 
%	\end{footnotesize}
%\end{figure}

\framebreak
\vspace*{1.0cm}
\begin{figure}
	\includegraphics[height=0.35\textwidth, keepaspectratio]{figure_man/momentum_conv_ripples.png} ~~ \includegraphics[height=0.35\textwidth, keepaspectratio]{figure_man/momentum_ripples.png} \\
	\begin{footnotesize} 
		The eigenvalues of $R$ are complex and we see low frequency ripples. 
	\end{footnotesize}
\end{figure}

\framebreak
\vspace*{1.0cm}
\begin{figure}
	\includegraphics[height=0.35\textwidth, keepaspectratio]{figure_man/momentum_conv_mono.png} ~~ \includegraphics[height=0.35\textwidth, keepaspectratio]{figure_man/momentum_mono.png} \\
	\begin{footnotesize} 
		Here, both eigenvalues of $R$ are positive with their norm being less than $1$. This behavior resembles gradient descent. 
	\end{footnotesize}
\end{figure}

\framebreak
\vspace*{1.0cm}
\begin{figure}
	\includegraphics[height=0.35\textwidth, keepaspectratio]{figure_man/momentum_conv_1step.png} ~~ \includegraphics[height=0.35\textwidth, keepaspectratio]{figure_man/momentum_1step.png} \\
	\begin{footnotesize} 
		The step size is $\alpha = 1/\lambda_i$ and $\varphi = 0$ - we converge in one step.
	\end{footnotesize}
\end{figure}
\framebreak
\vspace*{1.0cm}
\begin{figure}
	\includegraphics[height=0.35\textwidth, keepaspectratio]{figure_man/momentum_conv_osc.png} ~~ \includegraphics[height=0.35\textwidth, keepaspectratio]{figure_man/momentum_osc.png} \\
	\begin{footnotesize} 
		When $\alpha > 1/\lambda_i$, the iterates flip sign every iteration. 
	\end{footnotesize}
\end{figure}
\framebreak
\vspace*{1.0cm}
\begin{figure}
	\includegraphics[height=0.35\textwidth, keepaspectratio]{figure_man/momentum_conv_div.png} ~~ \includegraphics[height=0.35\textwidth, keepaspectratio]{figure_man/momentum_div.png} \\
	\begin{footnotesize} 
		If  $\text{max} \{|\sigma_1|, |\sigma_2| \} > 1$, the iterates diverge. 
	\end{footnotesize}
\end{figure}

\framebreak
\begin{itemize}
	\item Finally, we investigate the role of $\varphi$. 
	\item We can think of gradient descent with momentum as a damped harmonic oscillator: a weight on a spring. We pull the weight down and study the path back to the equilibrium in phase space (looking at the position and the velocity). 
	\item Depending on the choice of $\varphi$, the rate of return to the equilibrium position is affected. 
\end{itemize}

\begin{figure}
	\includegraphics[height=0.35\textwidth, keepaspectratio]{figure_man/phasespace_underdamping.png} 
\end{figure}

\framebreak

\begin{figure}
	\includegraphics[height=0.35\textwidth, keepaspectratio]{figure_man/momentum_damping.png} 
	\begin{footnotesize}
		 
		Left: If $\varphi$ is too large, we are underdamping. The spring oscillates back and forth and misses the optimum. 
		
		Middle: The best value of $\varphi$ lies in the middle. 
		
		Right: If $\varphi$ is too small, we are overdamping, meaning that the spring experiences too much friction and stops before reaching the equilibrium. 
	\end{footnotesize}
\end{figure}

\end{vbframe}


\endlecture
\end{document}

