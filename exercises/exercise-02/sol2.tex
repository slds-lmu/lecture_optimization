\documentclass[a4paper]{article}
\usepackage[]{graphicx}\usepackage[]{xcolor}
% maxwidth is the original width if it is less than linewidth
% otherwise use linewidth (to make sure the graphics do not exceed the margin)
\makeatletter
\def\maxwidth{ %
  \ifdim\Gin@nat@width>\linewidth
    \linewidth
  \else
    \Gin@nat@width
  \fi
}
\makeatother

\definecolor{fgcolor}{rgb}{0.345, 0.345, 0.345}
\newcommand{\hlnum}[1]{\textcolor[rgb]{0.686,0.059,0.569}{#1}}%
\newcommand{\hlstr}[1]{\textcolor[rgb]{0.192,0.494,0.8}{#1}}%
\newcommand{\hlcom}[1]{\textcolor[rgb]{0.678,0.584,0.686}{\textit{#1}}}%
\newcommand{\hlopt}[1]{\textcolor[rgb]{0,0,0}{#1}}%
\newcommand{\hlstd}[1]{\textcolor[rgb]{0.345,0.345,0.345}{#1}}%
\newcommand{\hlkwa}[1]{\textcolor[rgb]{0.161,0.373,0.58}{\textbf{#1}}}%
\newcommand{\hlkwb}[1]{\textcolor[rgb]{0.69,0.353,0.396}{#1}}%
\newcommand{\hlkwc}[1]{\textcolor[rgb]{0.333,0.667,0.333}{#1}}%
\newcommand{\hlkwd}[1]{\textcolor[rgb]{0.737,0.353,0.396}{\textbf{#1}}}%
\let\hlipl\hlkwb

\usepackage{framed}
\makeatletter
\newenvironment{kframe}{%
 \def\at@end@of@kframe{}%
 \ifinner\ifhmode%
  \def\at@end@of@kframe{\end{minipage}}%
  \begin{minipage}{\columnwidth}%
 \fi\fi%
 \def\FrameCommand##1{\hskip\@totalleftmargin \hskip-\fboxsep
 \colorbox{shadecolor}{##1}\hskip-\fboxsep
     % There is no \\@totalrightmargin, so:
     \hskip-\linewidth \hskip-\@totalleftmargin \hskip\columnwidth}%
 \MakeFramed {\advance\hsize-\width
   \@totalleftmargin\z@ \linewidth\hsize
   \@setminipage}}%
 {\par\unskip\endMakeFramed%
 \at@end@of@kframe}
\makeatother

\definecolor{shadecolor}{rgb}{.97, .97, .97}
\definecolor{messagecolor}{rgb}{0, 0, 0}
\definecolor{warningcolor}{rgb}{1, 0, 1}
\definecolor{errorcolor}{rgb}{1, 0, 0}
\newenvironment{knitrout}{}{} % an empty environment to be redefined in TeX

\usepackage{alltt}
\newcommand{\SweaveOpts}[1]{}  % do not interfere with LaTeX
\newcommand{\SweaveInput}[1]{} % because they are not real TeX commands
\newcommand{\Sexpr}[1]{}       % will only be parsed by R




\usepackage[utf8]{inputenc}
%\usepackage[ngerman]{babel}
\usepackage{a4wide,paralist}
\usepackage{amsmath, amssymb, xfrac, amsthm}
\usepackage{dsfont}
%\usepackage[usenames,dvipsnames]{xcolor}
\usepackage{amsfonts}
\usepackage{graphicx}
\usepackage{caption}
\usepackage{subcaption}
\usepackage{framed}
\usepackage{multirow}
\usepackage{bytefield}
\usepackage{csquotes}
\usepackage[breakable, theorems, skins]{tcolorbox}
\usepackage{hyperref}
\usepackage{cancel}
\usepackage{bm}


\input{../../style/common}

\tcbset{enhanced}

\DeclareRobustCommand{\mybox}[2][gray!20]{%
	\iffalse
	\begin{tcolorbox}[   %% Adjust the following parameters at will.
		breakable,
		left=0pt,
		right=0pt,
		top=0pt,
		bottom=0pt,
		colback=#1,
		colframe=#1,
		width=\dimexpr\linewidth\relax,
		enlarge left by=0mm,
		boxsep=5pt,
		arc=0pt,outer arc=0pt,
		]
		#2
	\end{tcolorbox}
	\fi
}

\DeclareRobustCommand{\myboxshow}[2][gray!20]{%
%	\iffalse
	\begin{tcolorbox}[   %% Adjust the following parameters at will.
		breakable,
		left=0pt,
		right=0pt,
		top=0pt,
		bottom=0pt,
		colback=#1,
		colframe=#1,
		width=\dimexpr\linewidth\relax,
		enlarge left by=0mm,
		boxsep=5pt,
		arc=0pt,outer arc=0pt,
		]
		#2
	\end{tcolorbox}
%	\fi
}


%exercise numbering
\renewcommand{\theenumi}{(\alph{enumi})}
\renewcommand{\theenumii}{\roman{enumii}}
\renewcommand\labelenumi{\theenumi}


\font \sfbold=cmssbx10

\setlength{\oddsidemargin}{0cm} \setlength{\textwidth}{16cm}


\sloppy
\parindent0em
\parskip0.5em
\topmargin-2.3 cm
\textheight25cm
\textwidth17.5cm
\oddsidemargin-0.8cm
\pagestyle{empty}

\newcommand{\kopf}[1]{
\hrule
\vspace{.15cm}
\begin{minipage}{\textwidth}
%akwardly i had to put \" here to make it compile correctly
	{\sf\bf Optimization in Machine Learning \hfill Exercise sheet #1\\
	 \url{https://slds-lmu.github.io/website_optimization/} \hfill WS 2024/2025}
\end{minipage}
\vspace{.05cm}
\hrule
\vspace{1cm}}

\newcommand{\kopfic}[1]{
\hrule
\vspace{.15cm}
\begin{minipage}{\textwidth}
%akwardly i had to put \" here to make it compile correctly
	{\sf\bf Optimization in Machine Learning \hfill Live Session #1\\
	 \url{https://slds-lmu.github.io/website_optimization/} \hfill WS 2024/2025}
\end{minipage}
\vspace{.05cm}
\hrule
\vspace{1cm}}

\newcommand{\kopfsl}[1]{
\hrule
\vspace{.15cm}
\begin{minipage}{\textwidth}
%akwardly i had to put \" here to make it compile correctly
	{\sf\bf Optimization in Machine Learning \hfill Exercise sheet #1\\
	 \url{https://slds-lmu.github.io/website_optimization/} \hfill WS 2024/2025}
\end{minipage}
\vspace{.05cm}
\hrule
\vspace{1cm}}

\newenvironment{allgemein}
	{\noindent}{\vspace{1cm}}

\newcounter{aufg}
\newenvironment{aufgabe}[1]
	{\refstepcounter{aufg}\textbf{Exercise \arabic{aufg}: #1}\\ \noindent}
	{\vspace{0.5cm}}

\newcounter{loes}
\newenvironment{loesung}
	{\refstepcounter{loes}\textbf{Solution \arabic{loes}:}\\\noindent}
	{\bigskip}
	
\newenvironment{bonusaufgabe}
	{\refstepcounter{aufg}\textbf{Exercise \arabic{aufg}*\footnote{This
	is a bonus exercise.}:}\\ \noindent}
	{\vspace{0.5cm}}

\newenvironment{bonusloesung}
	{\refstepcounter{loes}\textbf{Solution \arabic{loes}*:}\\\noindent}
	{\bigskip}

\input{../../latex-math/basic-math.tex}
\input{../../latex-math/basic-ml.tex}

\begin{document}
% !Rnw weave = knitr

\kopfsl{2}{Mathematical Concepts 2}

\loesung{Convexity}{

\begin{enumerate}
\item Let $x,y \in \R$ and $t \in [0, 1]$ then it holds that
\begin{align*}
(f + g)(x + t(y-x)) &= f(x + t(y-x)) + g(x + t(y-x)) \\
&\leq f(x) + t(f(y)-f(x)) + g(x) + t(g(y)-g(x)) & \text{($f,g$ are convex)} \\
& = f(x) + g(x) + t(f(y) + g(y) -(f(x) + g(x))) \\
& = (f + g)(x) + t((f + g)(y) - (f + g)(x)).
\end{align*}
\item Let $x,y \in \R$ and $t \in [0, 1]$ then it holds that
\begin{align*}
(g \circ f)(x + t(y-x)) &= g(f(x + t(y-x))) \\
&\leq g(f(x) + t(f(y)-f(x))) &\text{($g$ is non-decreasing, $f$ is convex)}\\
&\leq g(f(x)) + t(g(f(y)) - g(f(x)))) & \text{($g$ is convex)} \\
&= (g\circ f)(x) + t((g \circ f)(y) - (g \circ f)(x)).
\end{align*}

\end{enumerate}
}

\loesung{Convexity}{

Consider the bivariate function $f: \R^2 \to \R, (x_1, x_2) \mapsto \exp(\pi \cdot x_1) - \sin(\pi \cdot x_2) + \pi \cdot x_1 \cdot x_2$

\begin{enumerate}
\item	$\nabla f(\mathbf{x}) = \pi \cdot (\exp(\pi x_1) +   x_2,  -\cos(\pi x_2) +  x_1)^\top$
	\item $\nabla^2 f(\mathbf{x}) = \pi \cdot \begin{pmatrix}\pi \exp(\pi x_1) &  1 \\
	 1 & \pi \sin(\pi x_2)  \end{pmatrix}$
	\item  $T_{1,\mathbf{a}}(\mathbf{x}) = f(\mathbf{a}) + \nabla f(\mathbf{a})^\top(\mathbf{x}-\mathbf{a}) = 1 + \pi \cdot (2, 1) \cdot (x_1, x_2 - 1)^\top = 1 - \pi + 2\pi x_1 + \pi x_2$
	\item \begin{align*}T_{2,\mathbf{a}}(\mathbf{x}) &= T_{1,\mathbf{a}}(\mathbf{x}) + \frac{1}{2} (\mathbf{x} - \mathbf{a})^\top\nabla^2 f(\mathbf{a})(\mathbf{x} - \mathbf{a}) \\
	&= T_{1,\mathbf{a}}(\mathbf{x}) + \frac{1}{2} \mathbf{x}^\top\nabla^2 f(\mathbf{a})\mathbf{x} - \mathbf{x}^\top\nabla^2 f(\mathbf{a})\mathbf{a} + \frac{1}{2}\mathbf{a}^\top\nabla^2 f(\mathbf{a})\mathbf{a}\end{align*}
	With $\nabla^2 f(\mathbf{a}) = \begin{pmatrix}\pi^2  &  \pi \\
	\pi & 0 \end{pmatrix}$ we get that 
	\begin{align*}
	T_{2,\mathbf{a}}(\mathbf{x}) = T_{1,\mathbf{a}}(\mathbf{x}) &+ 0.5\pi^2x_1^2 \\
	  &+ \pi x_1 x_2 - \pi x_1 \\
	  &+ 0.
	\end{align*}.
	\item $T_{2,\mathbf{a}}(\mathbf{x})$ is multivariate polynomial of degree 2 which means its Hessian is constant and we can directly see that $\mathbf{H}:= \nabla^2 T_{2,\mathbf{a}}(\mathbf{x}) = \nabla^2 f(\mathbf{a}).$
	For the eigenvalues of the Hessian it must hold that
	\begin{align*}
	&\det (\mathbf{H} - \lambda\mathbf{I}) &= 0 \\
	\iff& \det \begin{pmatrix}\pi^2 - \lambda & \pi \\ 
	\pi & -\lambda\end{pmatrix} &= 0 
 \\
		\iff& (\pi^2 - \lambda) \cdot (-\lambda) - \pi^2 &= 0 \\
		\iff& \lambda^2 -\pi^2\lambda - \pi^2 &= 0.
		\end{align*}
	From which it follows that
	$\lambda_{1,2} = \frac{\pi^2 \pm \sqrt{\pi^4 + 4\pi^2}}{2} \Rightarrow \lambda_1 \approx 10.785, \lambda_2 \approx -0.915.$ Since $\lambda_2 < 0\;T_{2,\mathbf{a}}$  is not convex.
\end{enumerate}
}


\loesung{Optimality in 1d}{

Let $f: [-1,2] \rightarrow \R, x \mapsto \exp(x^3 - 2x^2)$
\begin{enumerate}
	%
	\item $f'(x) = \exp(x^3 - 2x^2)\cdot(3x^2 - 4x)$
	\item 
\begin{knitrout}
\definecolor{shadecolor}{rgb}{0.969, 0.969, 0.969}\color{fgcolor}\begin{kframe}
\begin{alltt}
\hlkwd{library}\hlstd{(ggplot2)}

\hlstd{f} \hlkwb{<-} \hlkwa{function}\hlstd{(}\hlkwc{x}\hlstd{)} \hlkwd{exp}\hlstd{(x}\hlopt{^}\hlnum{3} \hlopt{-} \hlnum{2}\hlopt{*}\hlstd{x}\hlopt{^}\hlnum{2}\hlstd{)}
\hlstd{df} \hlkwb{<-} \hlkwa{function}\hlstd{(}\hlkwc{x}\hlstd{)} \hlkwd{f}\hlstd{(x)} \hlopt{*} \hlstd{(}\hlnum{3}\hlopt{*}\hlstd{x}\hlopt{^}\hlnum{2} \hlopt{-} \hlnum{4}\hlopt{*}\hlstd{x)}

\hlkwd{ggplot}\hlstd{(}\hlkwd{data.frame}\hlstd{(}\hlkwc{x} \hlstd{=} \hlkwd{seq}\hlstd{(}\hlopt{-}\hlnum{1}\hlstd{,} \hlnum{2}\hlstd{,} \hlkwc{by}\hlstd{=}\hlnum{0.005}\hlstd{)),} \hlkwd{aes}\hlstd{(x))} \hlopt{+}
  \hlkwd{geom_function}\hlstd{(}\hlkwc{fun} \hlstd{= f,} \hlkwd{aes}\hlstd{(}\hlkwc{linetype} \hlstd{=} \hlstr{"f"}\hlstd{))} \hlopt{+}
  \hlkwd{geom_function}\hlstd{(}\hlkwc{fun} \hlstd{= df,} \hlkwd{aes}\hlstd{(}\hlkwc{linetype} \hlstd{=} \hlstr{"f'"}\hlstd{))}    \hlopt{+}
  \hlkwd{scale_linetype_discrete}\hlstd{(}\hlkwc{name} \hlstd{=} \hlstr{"Function"}\hlstd{)}
\end{alltt}
\end{kframe}
\includegraphics[width=0.5\linewidth]{figure/1d-plot-1} 
\end{knitrout}
	\item $f$ is continuously differentiable $\Rightarrow$ candidates can only be stationary points and boundary points. \\
	Find stationary points, i.e., points where \\ 
	$f'(x) = 0 \iff  \underbrace{\exp(x^3 - 2x^2)}_{>0}\cdot(3x^2 - 4x) = 0 \iff 3x^2 - 4x = 0 \iff x(3x - 4) = 0 .$ \\
	$\Rightarrow x_1 = 0, x_2 = 4/3.$
	The other candidates are boundary points, i.e., $x_3 = -1, x_4 = 2.$
	\item $f''(x) = \exp(x^3 - 2x^2)\cdot(3x^2 - 4x)^2 + \exp(x^3 - 2x^2)\cdot(6x - 4)$ \\
	\item $f''(x_1) = \exp(0)\cdot(- 4) < 0$ \\
	  $\Rightarrow x_1$ is a local maximum \\[\baselineskip]
	  $f''(x_2) = \exp((4/3)^3 - 2(4/3)^2)\cdot(4) > 0$ \\
	  $\Rightarrow x_2$ is a local minimum. \\[\baselineskip]
	  The boundary points $x_3$ and $x_4$ are not considered as \textit{local} optima.
	\item $f(x_1) = \exp(0) = 1$ \\
	 $f(x_2) = \exp((4/3)^3 - 2(4/3)^2) \approx 0.3057$ \\
	  $f(x_3) = \exp(-3) \approx 0.05$ \\
	  $f(x_4) = \exp(0) = 1$ \\
	  $\Rightarrow x_1, x_4$ are global maxima.
	  $x_3$ is global minimum.
	
\end{enumerate}
}
\end{document}
