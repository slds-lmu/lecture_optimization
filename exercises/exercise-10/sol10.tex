\documentclass[a4paper]{article}
\usepackage[]{graphicx}\usepackage[]{xcolor}
% maxwidth is the original width if it is less than linewidth
% otherwise use linewidth (to make sure the graphics do not exceed the margin)
\makeatletter
\def\maxwidth{ %
  \ifdim\Gin@nat@width>\linewidth
    \linewidth
  \else
    \Gin@nat@width
  \fi
}
\makeatother

\definecolor{fgcolor}{rgb}{0.345, 0.345, 0.345}
\newcommand{\hlnum}[1]{\textcolor[rgb]{0.686,0.059,0.569}{#1}}%
\newcommand{\hlstr}[1]{\textcolor[rgb]{0.192,0.494,0.8}{#1}}%
\newcommand{\hlcom}[1]{\textcolor[rgb]{0.678,0.584,0.686}{\textit{#1}}}%
\newcommand{\hlopt}[1]{\textcolor[rgb]{0,0,0}{#1}}%
\newcommand{\hlstd}[1]{\textcolor[rgb]{0.345,0.345,0.345}{#1}}%
\newcommand{\hlkwa}[1]{\textcolor[rgb]{0.161,0.373,0.58}{\textbf{#1}}}%
\newcommand{\hlkwb}[1]{\textcolor[rgb]{0.69,0.353,0.396}{#1}}%
\newcommand{\hlkwc}[1]{\textcolor[rgb]{0.333,0.667,0.333}{#1}}%
\newcommand{\hlkwd}[1]{\textcolor[rgb]{0.737,0.353,0.396}{\textbf{#1}}}%
\let\hlipl\hlkwb

\usepackage{framed}
\makeatletter
\newenvironment{kframe}{%
 \def\at@end@of@kframe{}%
 \ifinner\ifhmode%
  \def\at@end@of@kframe{\end{minipage}}%
  \begin{minipage}{\columnwidth}%
 \fi\fi%
 \def\FrameCommand##1{\hskip\@totalleftmargin \hskip-\fboxsep
 \colorbox{shadecolor}{##1}\hskip-\fboxsep
     % There is no \\@totalrightmargin, so:
     \hskip-\linewidth \hskip-\@totalleftmargin \hskip\columnwidth}%
 \MakeFramed {\advance\hsize-\width
   \@totalleftmargin\z@ \linewidth\hsize
   \@setminipage}}%
 {\par\unskip\endMakeFramed%
 \at@end@of@kframe}
\makeatother

\definecolor{shadecolor}{rgb}{.97, .97, .97}
\definecolor{messagecolor}{rgb}{0, 0, 0}
\definecolor{warningcolor}{rgb}{1, 0, 1}
\definecolor{errorcolor}{rgb}{1, 0, 0}
\newenvironment{knitrout}{}{} % an empty environment to be redefined in TeX

\usepackage{alltt}
\newcommand{\SweaveOpts}[1]{}  % do not interfere with LaTeX
\newcommand{\SweaveInput}[1]{} % because they are not real TeX commands
\newcommand{\Sexpr}[1]{}       % will only be parsed by R




\usepackage[utf8]{inputenc}
%\usepackage[ngerman]{babel}
\usepackage{a4wide,paralist}
\usepackage{amsmath, amssymb, xfrac, amsthm}
\usepackage{dsfont}
%\usepackage[usenames,dvipsnames]{xcolor}
\usepackage{amsfonts}
\usepackage{graphicx}
\usepackage{caption}
\usepackage{subcaption}
\usepackage{framed}
\usepackage{multirow}
\usepackage{bytefield}
\usepackage{csquotes}
\usepackage[breakable, theorems, skins]{tcolorbox}
\usepackage{hyperref}
\usepackage{cancel}
\usepackage{bm}


\input{../../style/common}

\tcbset{enhanced}

\DeclareRobustCommand{\mybox}[2][gray!20]{%
	\iffalse
	\begin{tcolorbox}[   %% Adjust the following parameters at will.
		breakable,
		left=0pt,
		right=0pt,
		top=0pt,
		bottom=0pt,
		colback=#1,
		colframe=#1,
		width=\dimexpr\linewidth\relax,
		enlarge left by=0mm,
		boxsep=5pt,
		arc=0pt,outer arc=0pt,
		]
		#2
	\end{tcolorbox}
	\fi
}

\DeclareRobustCommand{\myboxshow}[2][gray!20]{%
%	\iffalse
	\begin{tcolorbox}[   %% Adjust the following parameters at will.
		breakable,
		left=0pt,
		right=0pt,
		top=0pt,
		bottom=0pt,
		colback=#1,
		colframe=#1,
		width=\dimexpr\linewidth\relax,
		enlarge left by=0mm,
		boxsep=5pt,
		arc=0pt,outer arc=0pt,
		]
		#2
	\end{tcolorbox}
%	\fi
}


%exercise numbering
\renewcommand{\theenumi}{(\alph{enumi})}
\renewcommand{\theenumii}{\roman{enumii}}
\renewcommand\labelenumi{\theenumi}


\font \sfbold=cmssbx10

\setlength{\oddsidemargin}{0cm} \setlength{\textwidth}{16cm}


\sloppy
\parindent0em
\parskip0.5em
\topmargin-2.3 cm
\textheight25cm
\textwidth17.5cm
\oddsidemargin-0.8cm
\pagestyle{empty}

\newcommand{\kopf}[1]{
\hrule
\vspace{.15cm}
\begin{minipage}{\textwidth}
%akwardly i had to put \" here to make it compile correctly
	{\sf\bf Optimization in machine learning \hfill Exercise sheet #1\\
	 \url{https://slds-lmu.github.io/website_optimization/} \hfill WS 2022/2023}
\end{minipage}
\vspace{.05cm}
\hrule
\vspace{1cm}}

\newcommand{\kopfic}[1]{
\hrule
\vspace{.15cm}
\begin{minipage}{\textwidth}
%akwardly i had to put \" here to make it compile correctly
	{\sf\bf Optimization in machine learning \hfill Live Session #1\\
	 \url{https://slds-lmu.github.io/website_optimization/} \hfill WS 2022/2023}
\end{minipage}
\vspace{.05cm}
\hrule
\vspace{1cm}}

\newcommand{\kopfsl}[1]{
\hrule
\vspace{.15cm}
\begin{minipage}{\textwidth}
%akwardly i had to put \" here to make it compile correctly
	{\sf\bf Optimization in machine learning \hfill Exercise sheet #1\\
	 \url{https://slds-lmu.github.io/website_optimization/} \hfill WS 2022/2023}
\end{minipage}
\vspace{.05cm}
\hrule
\vspace{1cm}}

\newenvironment{allgemein}
	{\noindent}{\vspace{1cm}}

\newcounter{aufg}
\newenvironment{aufgabe}[1]
	{\refstepcounter{aufg}\textbf{Exercise \arabic{aufg}: #1}\\ \noindent}
	{\vspace{0.5cm}}

\newcounter{loes}
\newenvironment{loesung}
	{\refstepcounter{loes}\textbf{Solution \arabic{loes}:}\\\noindent}
	{\bigskip}
	
\newenvironment{bonusaufgabe}
	{\refstepcounter{aufg}\textbf{Exercise \arabic{aufg}*\footnote{This
	is a bonus exercise.}:}\\ \noindent}
	{\vspace{0.5cm}}

\newenvironment{bonusloesung}
	{\refstepcounter{loes}\textbf{Solution \arabic{loes}*:}\\\noindent}
	{\bigskip}



\begin{document}
% !Rnw weave = knitr



\input{../../latex-math/basic-math.tex}
\input{../../latex-math/basic-ml.tex}

\kopfsl{10}{Linear Programming 1}

\aufgabe{Sparse Quantile Regression}{

\begin{enumerate}
\item Univariate sparse quantile regression: \\
$$\min_{(\beta_0, \beta_1) \in \R^2}\frac{1}{n}\sum^n_{i = 1}\rho_\tau(y^{(i)} - \beta_0 -\beta_1 x^{(i)}) \text{ s.t. } \vert \beta_1\vert \leq t$$
\begin{enumerate}
\item Decompose unconstrained parameters into positive and negative part: $\beta_i = \beta^+_i - \beta^-_i \quad i = 0,1$ 
\item Transform absolute value: $\vert \beta_1\vert \leq t \iff \beta^+_1 + \beta^-_1 \leq t$
\item We can write $\rho_\tau(\underbrace{y^{(i)} - \beta_0 -\beta_1 x^{(i)}}_{=:r^{(i)}}) = \tau \cdot r^{(i)} \mathds{1}_{\{r^{(i)} > 0\}} - (1-\tau) \cdot r^{(i)} \mathds{1}_{\{r^{(i)} \leq 0\}} = \tau  \cdot {r^{(i)}}^+ + (1-\tau)\cdot {r^{(i)}}^-$
\item Transform equality of the residuals into two inequalities: \\
${r^{(i)}}^+ - {r^{(i)}}^- = y^{(i)} - \beta_0 -\beta_1 x^{(i)} \iff {r^{(i)}}^+ - {r^{(i)}}^- \leq y^{(i)} - \beta_0 -\beta_1 x^{(i)}$ and \\ 
$-{r^{(i)}}^+ + {r^{(i)}}^- \leq -y^{(i)} + \beta_0 +\beta_1 x^{(i)}$
\end{enumerate}
With this we get the standard form:\\
$$\max_{\mathbf{z} \in \R^{4+2n}} \mathbf{c}^\top\mathbf{z}\quad $$
s.t. $$\mathbf{A}\mathbf{z} \leq \mathbf{b},$$ 
$$\mathbf{z} \geq \mathbf{0}$$ 
with $\mathbf{z} = \begin{pmatrix}\beta_0^+\\ \beta_0^-\\ \beta_1^+\\ \beta_1^- \\ {r^{(1)}}^+ \\ \vdots \\ {r^{(n)}}^+ \\ {r^{(1)}}^- \\ \vdots \\ {r^{(n)}}^-  \end{pmatrix}, \mathbf{c} = \begin{pmatrix}0\\ 0\\ 0\\ 0 \\ -\tau \\ \vdots \\ -\tau \\ -(1-\tau) \\ \vdots \\ -(1-\tau)  \end{pmatrix},
\mathbf{A} = \begin{pmatrix}0 & 0 & 1 & 1 & 0 & \cdots  & 0 \\ 
\bm{1}_n & -\bm{1}_n & \mathbf{x} & -\mathbf{x} & \mathbf{I}_n && -\mathbf{I}_n \\ 
-\bm{1}_n & \bm{1}_n & -\mathbf{x} & \mathbf{x} & -\mathbf{I}_n && \mathbf{I}_n \
 \end{pmatrix},  \mathbf{b} = \begin{pmatrix}t\\ \mathbf{y}\\ -\mathbf{y} \end{pmatrix}$
 \item 
\begin{knitrout}
\definecolor{shadecolor}{rgb}{0.969, 0.969, 0.969}\color{fgcolor}\begin{kframe}
\begin{alltt}
\hlkwd{library}\hlstd{(ggplot2)}

\hlkwd{set.seed}\hlstd{(}\hlnum{123}\hlstd{)}
\hlstd{n} \hlkwb{=} \hlnum{30}
\hlstd{x} \hlkwb{=} \hlkwd{runif}\hlstd{(n)}
\hlstd{y} \hlkwb{=} \hlnum{2} \hlopt{*} \hlstd{x} \hlopt{+} \hlkwd{rgamma}\hlstd{(n,} \hlkwc{shape} \hlstd{=} \hlnum{1}\hlstd{)}

\hlstd{remp} \hlkwb{=} \hlkwa{function}\hlstd{(}\hlkwc{beta}\hlstd{)\{}
 \hlstd{r} \hlkwb{=} \hlstd{y} \hlopt{-} \hlstd{beta[}\hlnum{1}\hlstd{]} \hlopt{-} \hlstd{beta[}\hlnum{2}\hlstd{]} \hlopt{*} \hlstd{x}
 \hlkwd{return}\hlstd{(}\hlkwd{sum}\hlstd{(}\hlkwd{ifelse}\hlstd{(r} \hlopt{>} \hlnum{0}\hlstd{, tau}\hlopt{*}\hlstd{r,} \hlopt{-}\hlstd{(}\hlnum{1}\hlopt{-}\hlstd{tau)}\hlopt{*}\hlstd{r)))}
\hlstd{\}}

\hlstd{tau} \hlkwb{=} \hlnum{0.4}
\hlstd{tval} \hlkwb{=} \hlnum{1.7}

\hlstd{b} \hlkwb{=} \hlkwd{seq}\hlstd{(}\hlopt{-}\hlnum{3}\hlstd{,} \hlnum{3}\hlstd{,} \hlkwc{by}\hlstd{=}\hlnum{0.05}\hlstd{)}
\hlstd{bb} \hlkwb{=} \hlkwd{expand.grid}\hlstd{(}\hlkwc{X1} \hlstd{= b,} \hlkwc{X2} \hlstd{= b)}
\hlstd{fbb} \hlkwb{=} \hlkwd{apply}\hlstd{(bb,} \hlnum{1}\hlstd{,} \hlkwa{function}\hlstd{(}\hlkwc{beta}\hlstd{)} \hlkwd{remp}\hlstd{(beta))}

\hlstd{df} \hlkwb{=} \hlkwd{data.frame}\hlstd{(}\hlkwc{bb} \hlstd{= bb,} \hlkwc{fbb} \hlstd{= fbb)}
\hlstd{remp_plot} \hlkwb{=} \hlkwd{ggplot}\hlstd{()} \hlopt{+}
 \hlkwd{geom_contour_filled}\hlstd{(}\hlkwc{data} \hlstd{= df,} \hlkwd{aes}\hlstd{(}\hlkwc{x} \hlstd{= bb.X1,} \hlkwc{y} \hlstd{= bb.X2,} \hlkwc{z} \hlstd{= fbb))} \hlopt{+}
 \hlkwd{xlab}\hlstd{(}\hlkwd{expression}\hlstd{(beta[}\hlnum{0}\hlstd{]))} \hlopt{+}
 \hlkwd{ylab}\hlstd{(}\hlkwd{expression}\hlstd{(beta[}\hlnum{1}\hlstd{]))} \hlopt{+}
 \hlkwd{geom_hline}\hlstd{(}\hlkwc{yintercept} \hlstd{= tval,} \hlkwc{color}\hlstd{=}\hlstr{"red"}\hlstd{)} \hlopt{+}
 \hlkwd{geom_hline}\hlstd{(}\hlkwc{yintercept} \hlstd{=} \hlopt{-}\hlstd{tval,} \hlkwc{color}\hlstd{=}\hlstr{"red"}\hlstd{)}

\hlstd{remp_plot}
\end{alltt}
\end{kframe}
\includegraphics[width=0.5\linewidth]{figure/remp-plot-1} 
\end{knitrout}
\item 
\begin{knitrout}
\definecolor{shadecolor}{rgb}{0.969, 0.969, 0.969}\color{fgcolor}\begin{kframe}
\begin{alltt}
\hlkwd{library}\hlstd{(linprog)}
\end{alltt}


{\ttfamily\noindent\itshape\color{messagecolor}{\#\# Loading required package: lpSolve}}\begin{alltt}
\hlstd{Amat} \hlkwb{=} \hlkwd{c}\hlstd{(}\hlnum{0}\hlstd{,} \hlnum{0}\hlstd{,} \hlnum{1}\hlstd{,} \hlnum{1}\hlstd{,} \hlkwd{rep}\hlstd{(}\hlnum{0}\hlstd{,} \hlnum{2}\hlopt{*}\hlstd{n))}
\hlstd{Amat} \hlkwb{=} \hlkwd{rbind}\hlstd{(Amat,} \hlkwd{cbind}\hlstd{(}\hlnum{1}\hlstd{,} \hlopt{-}\hlnum{1}\hlstd{, x,} \hlopt{-}\hlstd{x,} \hlkwd{diag}\hlstd{(n),} \hlopt{-}\hlkwd{diag}\hlstd{(n)))}
\hlstd{Amat} \hlkwb{=} \hlkwd{rbind}\hlstd{(Amat,} \hlkwd{cbind}\hlstd{(}\hlopt{-}\hlnum{1}\hlstd{,} \hlnum{1}\hlstd{,} \hlopt{-}\hlstd{x, x,} \hlopt{-}\hlkwd{diag}\hlstd{(n),} \hlkwd{diag}\hlstd{(n)))}

\hlstd{bvec} \hlkwb{=} \hlkwd{c}\hlstd{(tval, y,} \hlopt{-}\hlstd{y)}
\hlstd{cvec} \hlkwb{=} \hlkwd{c}\hlstd{(}\hlnum{0}\hlstd{,} \hlnum{0}\hlstd{,} \hlnum{0}\hlstd{,} \hlnum{0}\hlstd{,} \hlkwd{rep}\hlstd{(tau, n),} \hlkwd{rep}\hlstd{(}\hlnum{1}\hlopt{-}\hlstd{tau, n))}

\hlstd{res} \hlkwb{=} \hlkwd{solveLP}\hlstd{(cvec, bvec, Amat,} \hlkwc{maximum} \hlstd{=} \hlnum{FALSE}\hlstd{,} \hlkwc{lpSolve} \hlstd{=} \hlnum{TRUE}\hlstd{)}

\hlstd{beta} \hlkwb{=} \hlkwd{c}\hlstd{(res}\hlopt{$}\hlstd{solution[}\hlnum{1}\hlstd{]} \hlopt{-} \hlstd{res}\hlopt{$}\hlstd{solution[}\hlnum{2}\hlstd{], res}\hlopt{$}\hlstd{solution[}\hlnum{3}\hlstd{]} \hlopt{-} \hlstd{res}\hlopt{$}\hlstd{solution[}\hlnum{4}\hlstd{])}

\hlkwd{ggplot}\hlstd{(}\hlkwd{data.frame}\hlstd{(}\hlkwc{x} \hlstd{= x,} \hlkwc{y} \hlstd{= y),} \hlkwd{aes}\hlstd{(}\hlkwc{x}\hlstd{=x,} \hlkwc{y}\hlstd{=y))} \hlopt{+}
 \hlkwd{geom_point}\hlstd{()} \hlopt{+}
 \hlkwd{geom_abline}\hlstd{(}\hlkwc{intercept} \hlstd{= beta[}\hlnum{1}\hlstd{],} \hlkwc{slope} \hlstd{= beta[}\hlnum{2}\hlstd{])}
\end{alltt}
\end{kframe}
\includegraphics[width=0.5\linewidth]{figure/remp-lpsolve-1} 
\begin{kframe}\begin{alltt}
\hlstd{remp_plot} \hlopt{+}   \hlkwd{geom_point}\hlstd{(}\hlkwc{data}\hlstd{=}\hlkwd{data.frame}\hlstd{(}\hlkwc{x} \hlstd{= beta[}\hlnum{1}\hlstd{],} \hlkwc{y} \hlstd{= beta[}\hlnum{2}\hlstd{]),}
            \hlkwd{aes}\hlstd{(}\hlkwc{x}\hlstd{=x,} \hlkwc{y}\hlstd{=y),} \hlkwc{color}\hlstd{=}\hlstr{"lightblue"}\hlstd{)}
\end{alltt}
\end{kframe}
\includegraphics[width=0.5\linewidth]{figure/remp-lpsolve-2} 
\end{knitrout}
\item From the lecture, we know that the dual problem must be
$$\max_{\bm{\alpha} \in \R^{2n + 1}} -\bm{\alpha}^\top\mathbf{b}$$ 
s.t.
$$-\bm{\alpha}^\top \mathbf{A} \leq -\mathbf{c}^\top$$
$$\bm{\alpha} \geq \mathbf{0}$$
\end{enumerate}
}
\end{document}
