\documentclass[a4paper]{article}
\usepackage[]{graphicx}\usepackage[]{xcolor}
% maxwidth is the original width if it is less than linewidth
% otherwise use linewidth (to make sure the graphics do not exceed the margin)
\makeatletter
\def\maxwidth{ %
  \ifdim\Gin@nat@width>\linewidth
    \linewidth
  \else
    \Gin@nat@width
  \fi
}
\makeatother

\definecolor{fgcolor}{rgb}{0.345, 0.345, 0.345}
\newcommand{\hlnum}[1]{\textcolor[rgb]{0.686,0.059,0.569}{#1}}%
\newcommand{\hlstr}[1]{\textcolor[rgb]{0.192,0.494,0.8}{#1}}%
\newcommand{\hlcom}[1]{\textcolor[rgb]{0.678,0.584,0.686}{\textit{#1}}}%
\newcommand{\hlopt}[1]{\textcolor[rgb]{0,0,0}{#1}}%
\newcommand{\hlstd}[1]{\textcolor[rgb]{0.345,0.345,0.345}{#1}}%
\newcommand{\hlkwa}[1]{\textcolor[rgb]{0.161,0.373,0.58}{\textbf{#1}}}%
\newcommand{\hlkwb}[1]{\textcolor[rgb]{0.69,0.353,0.396}{#1}}%
\newcommand{\hlkwc}[1]{\textcolor[rgb]{0.333,0.667,0.333}{#1}}%
\newcommand{\hlkwd}[1]{\textcolor[rgb]{0.737,0.353,0.396}{\textbf{#1}}}%
\let\hlipl\hlkwb

\usepackage{framed}
\makeatletter
\newenvironment{kframe}{%
 \def\at@end@of@kframe{}%
 \ifinner\ifhmode%
  \def\at@end@of@kframe{\end{minipage}}%
  \begin{minipage}{\columnwidth}%
 \fi\fi%
 \def\FrameCommand##1{\hskip\@totalleftmargin \hskip-\fboxsep
 \colorbox{shadecolor}{##1}\hskip-\fboxsep
     % There is no \\@totalrightmargin, so:
     \hskip-\linewidth \hskip-\@totalleftmargin \hskip\columnwidth}%
 \MakeFramed {\advance\hsize-\width
   \@totalleftmargin\z@ \linewidth\hsize
   \@setminipage}}%
 {\par\unskip\endMakeFramed%
 \at@end@of@kframe}
\makeatother

\definecolor{shadecolor}{rgb}{.97, .97, .97}
\definecolor{messagecolor}{rgb}{0, 0, 0}
\definecolor{warningcolor}{rgb}{1, 0, 1}
\definecolor{errorcolor}{rgb}{1, 0, 0}
\newenvironment{knitrout}{}{} % an empty environment to be redefined in TeX

\usepackage{alltt}
\newcommand{\SweaveOpts}[1]{}  % do not interfere with LaTeX
\newcommand{\SweaveInput}[1]{} % because they are not real TeX commands
\newcommand{\Sexpr}[1]{}       % will only be parsed by R




\usepackage[utf8]{inputenc}
%\usepackage[ngerman]{babel}
\usepackage{a4wide,paralist}
\usepackage{amsmath, amssymb, xfrac, amsthm}
\usepackage{dsfont}
%\usepackage[usenames,dvipsnames]{xcolor}
\usepackage{amsfonts}
\usepackage{graphicx}
\usepackage{caption}
\usepackage{subcaption}
\usepackage{framed}
\usepackage{multirow}
\usepackage{bytefield}
\usepackage{csquotes}
\usepackage[breakable, theorems, skins]{tcolorbox}
\usepackage{hyperref}
\usepackage{cancel}
\usepackage{bm}


\input{../../style/common}

\tcbset{enhanced}

\DeclareRobustCommand{\mybox}[2][gray!20]{%
	\iffalse
	\begin{tcolorbox}[   %% Adjust the following parameters at will.
		breakable,
		left=0pt,
		right=0pt,
		top=0pt,
		bottom=0pt,
		colback=#1,
		colframe=#1,
		width=\dimexpr\linewidth\relax,
		enlarge left by=0mm,
		boxsep=5pt,
		arc=0pt,outer arc=0pt,
		]
		#2
	\end{tcolorbox}
	\fi
}

\DeclareRobustCommand{\myboxshow}[2][gray!20]{%
%	\iffalse
	\begin{tcolorbox}[   %% Adjust the following parameters at will.
		breakable,
		left=0pt,
		right=0pt,
		top=0pt,
		bottom=0pt,
		colback=#1,
		colframe=#1,
		width=\dimexpr\linewidth\relax,
		enlarge left by=0mm,
		boxsep=5pt,
		arc=0pt,outer arc=0pt,
		]
		#2
	\end{tcolorbox}
%	\fi
}


%exercise numbering
\renewcommand{\theenumi}{(\alph{enumi})}
\renewcommand{\theenumii}{\roman{enumii}}
\renewcommand\labelenumi{\theenumi}


\font \sfbold=cmssbx10

\setlength{\oddsidemargin}{0cm} \setlength{\textwidth}{16cm}


\sloppy
\parindent0em
\parskip0.5em
\topmargin-2.3 cm
\textheight25cm
\textwidth17.5cm
\oddsidemargin-0.8cm
\pagestyle{empty}

\newcommand{\kopf}[1]{
\hrule
\vspace{.15cm}
\begin{minipage}{\textwidth}
%akwardly i had to put \" here to make it compile correctly
	{\sf\bf Optimization in Machine Learning \hfill Exercise sheet #1\\
	 \url{https://slds-lmu.github.io/website_optimization/} \hfill WS 2024/2025}
\end{minipage}
\vspace{.05cm}
\hrule
\vspace{1cm}}

\newcommand{\kopfic}[1]{
\hrule
\vspace{.15cm}
\begin{minipage}{\textwidth}
%akwardly i had to put \" here to make it compile correctly
	{\sf\bf Optimization in Machine Learning \hfill Live Session #1\\
	 \url{https://slds-lmu.github.io/website_optimization/} \hfill WS 2024/2025}
\end{minipage}
\vspace{.05cm}
\hrule
\vspace{1cm}}

\newcommand{\kopfsl}[1]{
\hrule
\vspace{.15cm}
\begin{minipage}{\textwidth}
%akwardly i had to put \" here to make it compile correctly
	{\sf\bf Optimization in Machine Learning \hfill Exercise sheet #1\\
	 \url{https://slds-lmu.github.io/website_optimization/} \hfill WS 2024/2025}
\end{minipage}
\vspace{.05cm}
\hrule
\vspace{1cm}}

\newenvironment{allgemein}
	{\noindent}{\vspace{1cm}}

\newcounter{aufg}
\newenvironment{aufgabe}[1]
	{\refstepcounter{aufg}\textbf{Exercise \arabic{aufg}: #1}\\ \noindent}
	{\vspace{0.5cm}}

\newcounter{loes}
\newenvironment{loesung}[1]
	{\refstepcounter{loes}\textbf{Solution \arabic{loes}: #1}\\ \noindent}
	{\bigskip}
	
\newenvironment{bonusaufgabe}
	{\refstepcounter{aufg}\textbf{Exercise \arabic{aufg}*\footnote{This
	is a bonus exercise.}:}\\ \noindent}
	{\vspace{0.5cm}}

\newenvironment{bonusloesung}
	{\refstepcounter{loes}\textbf{Solution \arabic{loes}*:}\\\noindent}
	{\bigskip}

\input{../../latex-math/basic-math.tex}
\input{../../latex-math/basic-ml.tex}

\begin{document}
% !Rnw weave = knitr

\kopfsl{6}{Multivariate Optimization 1}

\loesung{Gradient Descent}{
You are given the following data situation:
\begin{knitrout}
\definecolor{shadecolor}{rgb}{0.969, 0.969, 0.969}\color{fgcolor}\begin{kframe}
\begin{alltt}
\hlkwd{library}\hldef{(ggplot2)}

\hlkwd{set.seed}\hldef{(}\hlnum{314}\hldef{)}
\hldef{n} \hlkwb{<-} \hlnum{100}
\hldef{X} \hlkwb{=} \hlkwd{cbind}\hldef{(}\hlkwd{rnorm}\hldef{(n,} \hlopt{-}\hlnum{5}\hldef{,} \hlnum{5}\hldef{),}
  \hlkwd{rnorm}\hldef{(n,} \hlopt{-}\hlnum{10}\hldef{,} \hlnum{10}\hldef{))}
\hldef{X_design} \hlkwb{=} \hlkwd{cbind}\hldef{(}\hlnum{1}\hldef{, X)}

\hldef{z} \hlkwb{<-} \hlnum{2}\hlopt{*}\hldef{X[,}\hlnum{1}\hldef{]} \hlopt{+} \hlnum{3}\hlopt{*}\hldef{X[,}\hlnum{2}\hldef{]}
\hldef{pr} \hlkwb{<-} \hlnum{1}\hlopt{/}\hldef{(}\hlnum{1}\hlopt{+}\hlkwd{exp}\hldef{(}\hlopt{-}\hldef{z))}
\hldef{y} \hlkwb{<-} \hlkwd{as.integer}\hldef{(pr} \hlopt{>} \hlnum{0.5}\hldef{)}
\hldef{df} \hlkwb{<-} \hlkwd{data.frame}\hldef{(}\hlkwc{X} \hldef{= X,} \hlkwc{y} \hldef{= y)}

\hlkwd{ggplot}\hldef{(df)} \hlopt{+}
  \hlkwd{geom_point}\hldef{(}\hlkwd{aes}\hldef{(}\hlkwc{x} \hldef{= X.1,} \hlkwc{y} \hldef{= X.2,} \hlkwc{color}\hldef{=}\hlkwd{as.factor}\hldef{(y)))} \hlopt{+}
  \hlkwd{xlab}\hldef{(}\hlkwd{expression}\hldef{(x[}\hlnum{1}\hldef{]))} \hlopt{+}
  \hlkwd{ylab}\hldef{(}\hlkwd{expression}\hldef{(x[}\hlnum{2}\hldef{]))} \hlopt{+}
  \hlkwd{labs}\hldef{(}\hlkwc{colour} \hldef{=} \hlsng{"y"}\hldef{)}
\end{alltt}
\end{kframe}
\includegraphics[width=0.5\linewidth]{figure/mv-plot-1} 
\end{knitrout}

\begin{enumerate}
\item
We start with
\begin{align*}
  \mathcal{R}_\text{emp}(\tilde{\bm{\theta}}) &= \sum^n_{i=1} \log(1 + \exp(\tilde{\bm{\theta}}^\top \mathbf{x}^{(i)})) - y^{(i)}\tilde{\bm{\theta}}^\top \mathbf{x}^{(i)} \\
  &= \sum^n_{i=1} \begin{cases}
    \log(1 + \exp(\tilde{\bm{\theta}}^\top \mathbf{x}^{(i)})) & \text{if $y^{(i)} = 0$}, \\
    \log(1 + \exp(\tilde{\bm{\theta}}^\top \mathbf{x}^{(i)})) - \tilde{\bm{\theta}}^\top \mathbf{x}^{(i)} & \text{if $y^{(i)} = 1$}.
  \end{cases}
\end{align*}

Since $\tilde{\bm{\theta}}$ perfectly classifies the data, we know that
\begin{equation*}
  \begin{cases}
    \tilde{\bm{\theta}}^\top \xv^{(i)} < 0 & \text{if $y^{(i)} = 0$}, \\
    \tilde{\bm{\theta}}^\top \xv^{(i)} \geq 0 & \text{if $y^{(i)} = 1$}.
  \end{cases}
\end{equation*}

Hence, we can focus on the functions $g(z) = \log(1 + \exp(-z))$ and $h(z) = \log(1 + \exp(z)) - z$ for $z > 0$ and study their monotonicity.

We compute
\begin{equation*}
  g'(z) = \frac{1}{1 + \exp(-z)} \cdot \exp(-z) \cdot (-1) = - \underbrace{\frac{\exp(-z)}{1 + \exp(-z)}}_{> 0} < 0
\end{equation*}
and
\begin{equation*}
  h'(z) = \underbrace{\frac{\exp(z)}{1 + \exp(z)}}_{< 1} - 1 < 0.
\end{equation*}
Therefore, both $g$ and $h$ are strictly monotonically decreasing.

It follows that $\mathcal{R}_\text{emp}(\tilde{\bm{\theta}}) > \mathcal{R}_\text{emp}(\alpha \tilde{\bm{\theta}})$ for $\alpha > 1$.

\item
\begin{knitrout}
\definecolor{shadecolor}{rgb}{0.969, 0.969, 0.969}\color{fgcolor}\begin{kframe}
\begin{alltt}
\hldef{lambda} \hlkwb{=} \hlnum{0}

\hldef{f} \hlkwb{<-} \hlkwa{function}\hldef{(}\hlkwc{theta}\hldef{,} \hlkwc{lambda}\hldef{) lambda} \hlopt{*} \hldef{theta} \hlopt \hldef{theta} \hlopt{+}
  \hlkwd{sum}\hldef{(}\hlopt{-}\hldef{y} \hlopt{*} \hldef{X} \hlopt \hldef{theta} \hlopt{+} \hlkwd{log}\hldef{(}\hlnum{1} \hlopt{+} \hlkwd{exp}\hldef{(X} \hlopt \hldef{theta)))}

\hldef{x} \hlkwb{=} \hlkwd{seq}\hldef{(}\hlopt{-}\hlnum{1}\hldef{,} \hlnum{5}\hldef{,} \hlkwc{by}\hldef{=}\hlnum{0.1}\hldef{)}
\hldef{xx} \hlkwb{=} \hlkwd{expand.grid}\hldef{(}\hlkwc{X1} \hldef{= x,} \hlkwc{X2} \hldef{= x)}

\hldef{fxx} \hlkwb{=} \hlkwd{log}\hldef{(}\hlkwd{apply}\hldef{(xx,} \hlnum{1}\hldef{,} \hlkwa{function}\hldef{(}\hlkwc{t}\hldef{)} \hlkwd{f}\hldef{(t, lambda)))}
\hldef{df} \hlkwb{=} \hlkwd{data.frame}\hldef{(}\hlkwc{xx} \hldef{= xx,} \hlkwc{fxx} \hldef{= fxx)}

\hlkwd{ggplot}\hldef{()} \hlopt{+}
    \hlkwd{geom_contour_filled}\hldef{(}\hlkwc{data} \hldef{= df,} \hlkwd{aes}\hldef{(}\hlkwc{x} \hldef{= xx.X1,} \hlkwc{y} \hldef{= xx.X2,} \hlkwc{z} \hldef{= fxx))} \hlopt{+}
    \hlkwd{xlab}\hldef{(}\hlkwd{expression}\hldef{(theta[}\hlnum{1}\hldef{]))} \hlopt{+}
    \hlkwd{ylab}\hldef{(}\hlkwd{expression}\hldef{(theta[}\hlnum{2}\hldef{]))}
\end{alltt}
\end{kframe}
\includegraphics[width=0.5\linewidth]{figure/mv-plot_r_emp-1} 
\end{knitrout}

\item $\frac{\partial}{\partial \bm{\theta}}\mathcal{R}_{\text{emp}} = \sum^n_{i=1} \frac{\exp(\bm{\theta}^\top \mathbf{x}^{(i)})}{1 + \exp(\bm{\theta}^\top \mathbf{x}^{(i)})}{\mathbf{x}^{(i)}}^\top - y^{(i)}{\mathbf{x}^{(i)}}^\top$ 
\item Note that we visualize form the first iteration on ($t = 1$) and not from the initial starting point $\bm{\theta}^{[0]} = (0,0)^\top$ but after having already made one GD step.

\begin{knitrout}
\definecolor{shadecolor}{rgb}{0.969, 0.969, 0.969}\color{fgcolor}\begin{kframe}
\begin{alltt}
\hlkwd{library}\hldef{(gridExtra)}

\hldef{plot_fun} \hlkwb{<-} \hlkwa{function}\hldef{(}\hlkwc{gd_fun}\hldef{,} \hlkwc{lambda}\hldef{)\{}
  \hldef{theta} \hlkwb{=} \hlkwd{c}\hldef{(}\hlnum{0}\hldef{,}\hlnum{0}\hldef{)}
  \hldef{thetas} \hlkwb{=} \hlkwa{NULL}
  \hldef{thetas_norm} \hlkwb{=} \hlkwa{NULL}
  \hldef{fs} \hlkwb{=} \hlkwa{NULL}
  \hlkwa{for}\hldef{(i} \hlkwa{in} \hlnum{1}\hlopt{:}\hlnum{500}\hldef{)\{}
    \hldef{theta} \hlkwb{=} \hlkwd{gd_fun}\hldef{(theta)}
    \hldef{thetas_norm} \hlkwb{=} \hlkwd{rbind}\hldef{(thetas_norm,} \hlkwd{sqrt}\hldef{(theta} \hlopt \hldef{theta))}
    \hldef{thetas} \hlkwb{=} \hlkwd{rbind}\hldef{(thetas,} \hlkwd{t}\hldef{(theta))}
    \hldef{fs} \hlkwb{=} \hlkwd{rbind}\hldef{(fs,} \hlkwd{f}\hldef{(theta, lambda))}
  \hldef{\}}

  \hldef{df_trace} \hlkwb{=} \hlkwd{as.data.frame}\hldef{(thetas)}
  \hldef{df_trace}\hlopt{$}\hldef{t} \hlkwb{=} \hlnum{1}\hlopt{:}\hlkwd{nrow}\hldef{(df_trace)}
  \hldef{trace_plot} \hlkwb{=} \hlkwd{ggplot}\hldef{()} \hlopt{+}
    \hlkwd{geom_point}\hldef{(}\hlkwc{data} \hldef{= df_trace,} \hlkwd{aes}\hldef{(}\hlkwc{x}\hldef{=V1,} \hlkwc{y}\hldef{=V2,} \hlkwc{colour}\hldef{=t))} \hlopt{+}
    \hlkwd{xlab}\hldef{(}\hlkwd{expression}\hldef{(theta[}\hlnum{1}\hldef{]))} \hlopt{+}
    \hlkwd{ylab}\hldef{(}\hlkwd{expression}\hldef{(theta[}\hlnum{2}\hldef{]))}
  \hldef{norm_plot} \hlkwb{=} \hlkwd{ggplot}\hldef{(}\hlkwd{data.frame}\hldef{(}\hlkwc{norms} \hldef{= thetas_norm,} \hlkwc{t} \hldef{=} \hlnum{1}\hlopt{:}\hlkwd{nrow}\hldef{(thetas_norm)))} \hlopt{+}
    \hlkwd{geom_line}\hldef{(}\hlkwd{aes}\hldef{(}\hlkwc{x} \hldef{= t,} \hlkwc{y} \hldef{= norms))} \hlopt{+}
    \hlkwd{scale_x_continuous}\hldef{(}\hlkwc{breaks} \hldef{=} \hlkwd{c}\hldef{(}\hlnum{1}\hldef{,} \hlnum{100}\hldef{,} \hlnum{200}\hldef{,} \hlnum{300}\hldef{,} \hlnum{400}\hldef{,} \hlnum{500}\hldef{),} \hlkwc{limits} \hldef{=} \hlkwd{c}\hldef{(}\hlnum{1}\hldef{,} \hlnum{500}\hldef{))} \hlopt{+}
    \hlkwd{ylab}\hldef{(}\hlkwd{expression}\hldef{(}\hlkwd{paste}\hldef{(}\hlsng{"||"}\hldef{, theta,} \hlsng{"||"}\hldef{[}\hlnum{2}\hldef{])))}
  \hldef{remp_plot} \hlkwb{=} \hlkwd{ggplot}\hldef{(}\hlkwd{data.frame}\hldef{(}\hlkwc{f} \hldef{= fs,} \hlkwc{t} \hldef{=} \hlnum{1}\hlopt{:}\hlkwd{nrow}\hldef{(thetas_norm)))} \hlopt{+}
    \hlkwd{geom_line}\hldef{(}\hlkwd{aes}\hldef{(}\hlkwc{x} \hldef{= t,} \hlkwc{y} \hldef{= f))} \hlopt{+}
    \hlkwd{scale_x_continuous}\hldef{(}\hlkwc{breaks} \hldef{=} \hlkwd{c}\hldef{(}\hlnum{1}\hldef{,} \hlnum{100}\hldef{,} \hlnum{200}\hldef{,} \hlnum{300}\hldef{,} \hlnum{400}\hldef{,} \hlnum{500}\hldef{),} \hlkwc{limits} \hldef{=} \hlkwd{c}\hldef{(}\hlnum{1}\hldef{,} \hlnum{500}\hldef{))} \hlopt{+}
    \hlkwd{ylab}\hldef{(}\hlkwd{expression}\hldef{(R[emp]))}
  \hlkwd{grid.arrange}\hldef{(trace_plot, norm_plot, remp_plot,} \hlkwc{ncol}\hldef{=}\hlnum{3}\hldef{)}
\hldef{\}}

\hldef{df_t} \hlkwb{<-} \hlkwa{function}\hldef{(}\hlkwc{theta}\hldef{,} \hlkwc{lambda}\hldef{) lambda} \hlopt{*} \hlkwd{t}\hldef{(theta)} \hlopt{-}\hldef{(}\hlkwd{t}\hldef{(y)} \hlopt \hldef{X)} \hlopt{+}
  \hlkwd{t}\hldef{(}\hlnum{1}\hlopt{/}\hldef{(}\hlnum{1} \hlopt{+} \hlkwd{exp}\hldef{(}\hlopt{-}\hldef{X} \hlopt \hldef{theta)))} \hlopt \hldef{X}

\hldef{gd_step} \hlkwb{<-} \hlkwa{function}\hldef{(}\hlkwc{theta}\hldef{,} \hlkwc{alpha}\hldef{,} \hlkwc{lambda}\hldef{)} \hlkwd{return}\hldef{(theta} \hlopt{-} \hldef{alpha} \hlopt{*} \hlkwd{df_t}\hldef{(theta, lambda)[}\hlnum{1}\hldef{,])}

\hlcom{## Alpha = 0.01}
\hldef{gd_fun} \hlkwb{<-} \hlkwa{function}\hldef{(}\hlkwc{theta}\hldef{)} \hlkwd{return}\hldef{(}\hlkwd{gd_step}\hldef{(theta,} \hlnum{0.01}\hldef{, lambda))}
\hlkwd{plot_fun}\hldef{(gd_fun,} \hlnum{0}\hldef{)}
\end{alltt}
\end{kframe}
\includegraphics[width=0.5\linewidth]{figure/mv-plot_gd_step-1} 
\begin{kframe}\begin{alltt}
\hlcom{## Alpha = 0.02}
\hldef{gd_fun} \hlkwb{<-} \hlkwa{function}\hldef{(}\hlkwc{theta}\hldef{)} \hlkwd{return}\hldef{(}\hlkwd{gd_step}\hldef{(theta,} \hlnum{0.02}\hldef{, lambda))}
\hlkwd{plot_fun}\hldef{(gd_fun,} \hlnum{0}\hldef{)}
\end{alltt}
\end{kframe}
\includegraphics[width=0.5\linewidth]{figure/mv-plot_gd_step-2} 
\end{knitrout}

Gradient descent will in theory not converge since $\mathcal{R}_\text{emp}$ has no minimum (a)
\item

\begin{knitrout}
\definecolor{shadecolor}{rgb}{0.969, 0.969, 0.969}\color{fgcolor}\begin{kframe}
\begin{alltt}
\hlcom{## Lambda = 1, alpha = 0.01}
\hldef{gd_fun} \hlkwb{<-} \hlkwa{function}\hldef{(}\hlkwc{theta}\hldef{)} \hlkwd{return}\hldef{(}\hlkwd{gd_step}\hldef{(theta,} \hlnum{0.01}\hldef{,} \hlnum{1}\hldef{))}
\hlkwd{plot_fun}\hldef{(gd_fun,} \hlnum{1}\hldef{)}
\end{alltt}
\end{kframe}
\includegraphics[width=0.5\linewidth]{figure/mv-plot_gd_step_reg-1} 
\begin{kframe}\begin{alltt}
\hlcom{## Lambda = 1, alpha = 0.02}
\hldef{gd_fun} \hlkwb{<-} \hlkwa{function}\hldef{(}\hlkwc{theta}\hldef{)} \hlkwd{return}\hldef{(}\hlkwd{gd_step}\hldef{(theta,} \hlnum{0.02}\hldef{,} \hlnum{1}\hldef{))}
\hlkwd{plot_fun}\hldef{(gd_fun,} \hlnum{1}\hldef{)}
\end{alltt}
\end{kframe}
\includegraphics[width=0.5\linewidth]{figure/mv-plot_gd_step_reg-2} 
\end{knitrout}

\item 
\begin{knitrout}
\definecolor{shadecolor}{rgb}{0.969, 0.969, 0.969}\color{fgcolor}\begin{kframe}
\begin{alltt}
\hldef{lambda} \hlkwb{=} \hlnum{1}

\hldef{fxx_reg} \hlkwb{=} \hlkwd{log}\hldef{(}\hlkwd{apply}\hldef{(xx,} \hlnum{1}\hldef{,} \hlkwa{function}\hldef{(}\hlkwc{t}\hldef{)} \hlkwd{f}\hldef{(t, lambda)))}
\hldef{df_reg} \hlkwb{=} \hlkwd{data.frame}\hldef{(}\hlkwc{xx} \hldef{= xx,} \hlkwc{fxx} \hldef{= fxx_reg)}

\hlkwd{ggplot}\hldef{()} \hlopt{+}
    \hlkwd{geom_contour_filled}\hldef{(}\hlkwc{data} \hldef{= df_reg,} \hlkwd{aes}\hldef{(}\hlkwc{x} \hldef{= xx.X1,} \hlkwc{y} \hldef{= xx.X2,} \hlkwc{z} \hldef{= fxx))} \hlopt{+}
    \hlkwd{xlab}\hldef{(}\hlkwd{expression}\hldef{(theta[}\hlnum{1}\hldef{]))} \hlopt{+}
    \hlkwd{ylab}\hldef{(}\hlkwd{expression}\hldef{(theta[}\hlnum{2}\hldef{]))}
\end{alltt}
\end{kframe}
\includegraphics[width=0.5\linewidth]{figure/mv-plot_r_emp_reg-1} 
\end{knitrout}
\item
\begin{knitrout}
\definecolor{shadecolor}{rgb}{0.969, 0.969, 0.969}\color{fgcolor}\begin{kframe}
\begin{alltt}
\hldef{gd_backtracking_step} \hlkwb{<-} \hlkwa{function}\hldef{(}\hlkwc{theta}\hldef{,} \hlkwc{alpha}\hldef{,} \hlkwc{gamma}\hldef{,} \hlkwc{tau}\hldef{,} \hlkwc{lambda}\hldef{)\{}
    \hldef{ftheta} \hlkwb{=} \hlkwd{f}\hldef{(theta, lambda)}
    \hldef{dftheta} \hlkwb{=} \hlkwd{df_t}\hldef{(theta, lambda)[}\hlnum{1}\hldef{,]}
    \hlkwa{for}\hldef{(i} \hlkwa{in} \hlnum{1}\hlopt{:}\hlnum{1000}\hldef{) \{}
      \hldef{theta_prop} \hlkwb{=} \hldef{theta} \hlopt{-} \hldef{alpha} \hlopt{*} \hldef{dftheta}
      \hlkwa{if}\hldef{(}\hlkwd{f}\hldef{(theta_prop, lambda)} \hlopt{<=} \hldef{ftheta} \hlopt{-} \hldef{gamma} \hlopt{*} \hldef{alpha} \hlopt{*} \hlkwd{t}\hldef{(dftheta)} \hlopt \hldef{dftheta)\{}
        \hlkwd{return}\hldef{(theta_prop)}
      \hldef{\}}\hlkwa{else}\hldef{\{}
        \hldef{alpha} \hlkwb{=} \hldef{tau} \hlopt{*} \hldef{alpha}
      \hldef{\}}
    \hldef{\}}
    \hlkwd{return}\hldef{(theta)}
\hldef{\}}

\hlcom{## Lambda = 1, alpha = 0.01}
\hldef{gd_fun} \hlkwb{<-} \hlkwa{function}\hldef{(}\hlkwc{theta}\hldef{)} \hlkwd{return}\hldef{(}\hlkwd{gd_backtracking_step}\hldef{(theta,} \hlnum{0.01}\hldef{,} \hlnum{0.9}\hldef{,} \hlnum{0.5}\hldef{,} \hlnum{1}\hldef{))}
\hlkwd{plot_fun}\hldef{(gd_fun,} \hlnum{1}\hldef{)}
\end{alltt}
\end{kframe}
\includegraphics[width=0.5\linewidth]{figure/mv-plot_gd_backtrack-1} 
\begin{kframe}\begin{alltt}
\hlcom{## Lambda = 1, alpha = 0.02}
\hldef{gd_fun} \hlkwb{<-} \hlkwa{function}\hldef{(}\hlkwc{theta}\hldef{)} \hlkwd{return}\hldef{(}\hlkwd{gd_backtracking_step}\hldef{(theta,} \hlnum{0.02}\hldef{,} \hlnum{0.9}\hldef{,} \hlnum{0.5}\hldef{,} \hlnum{1}\hldef{))}
\hlkwd{plot_fun}\hldef{(gd_fun,} \hlnum{1}\hldef{)}
\end{alltt}
\end{kframe}
\includegraphics[width=0.5\linewidth]{figure/mv-plot_gd_backtrack-2} 
\end{knitrout}
\end{enumerate}
}
\end{document}
