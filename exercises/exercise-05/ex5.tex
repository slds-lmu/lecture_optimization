\documentclass[a4paper]{article}
\usepackage[]{graphicx}\usepackage[]{xcolor}
% maxwidth is the original width if it is less than linewidth
% otherwise use linewidth (to make sure the graphics do not exceed the margin)
\makeatletter
\def\maxwidth{ %
  \ifdim\Gin@nat@width>\linewidth
    \linewidth
  \else
    \Gin@nat@width
  \fi
}
\makeatother

\definecolor{fgcolor}{rgb}{0.345, 0.345, 0.345}
\newcommand{\hlnum}[1]{\textcolor[rgb]{0.686,0.059,0.569}{#1}}%
\newcommand{\hlstr}[1]{\textcolor[rgb]{0.192,0.494,0.8}{#1}}%
\newcommand{\hlcom}[1]{\textcolor[rgb]{0.678,0.584,0.686}{\textit{#1}}}%
\newcommand{\hlopt}[1]{\textcolor[rgb]{0,0,0}{#1}}%
\newcommand{\hlstd}[1]{\textcolor[rgb]{0.345,0.345,0.345}{#1}}%
\newcommand{\hlkwa}[1]{\textcolor[rgb]{0.161,0.373,0.58}{\textbf{#1}}}%
\newcommand{\hlkwb}[1]{\textcolor[rgb]{0.69,0.353,0.396}{#1}}%
\newcommand{\hlkwc}[1]{\textcolor[rgb]{0.333,0.667,0.333}{#1}}%
\newcommand{\hlkwd}[1]{\textcolor[rgb]{0.737,0.353,0.396}{\textbf{#1}}}%
\let\hlipl\hlkwb

\usepackage{framed}
\makeatletter
\newenvironment{kframe}{%
 \def\at@end@of@kframe{}%
 \ifinner\ifhmode%
  \def\at@end@of@kframe{\end{minipage}}%
  \begin{minipage}{\columnwidth}%
 \fi\fi%
 \def\FrameCommand##1{\hskip\@totalleftmargin \hskip-\fboxsep
 \colorbox{shadecolor}{##1}\hskip-\fboxsep
     % There is no \\@totalrightmargin, so:
     \hskip-\linewidth \hskip-\@totalleftmargin \hskip\columnwidth}%
 \MakeFramed {\advance\hsize-\width
   \@totalleftmargin\z@ \linewidth\hsize
   \@setminipage}}%
 {\par\unskip\endMakeFramed%
 \at@end@of@kframe}
\makeatother

\definecolor{shadecolor}{rgb}{.97, .97, .97}
\definecolor{messagecolor}{rgb}{0, 0, 0}
\definecolor{warningcolor}{rgb}{1, 0, 1}
\definecolor{errorcolor}{rgb}{1, 0, 0}
\newenvironment{knitrout}{}{} % an empty environment to be redefined in TeX

\usepackage{alltt}
\newcommand{\SweaveOpts}[1]{}  % do not interfere with LaTeX
\newcommand{\SweaveInput}[1]{} % because they are not real TeX commands
\newcommand{\Sexpr}[1]{}       % will only be parsed by R




\usepackage[utf8]{inputenc}
%\usepackage[ngerman]{babel}
\usepackage{a4wide,paralist}
\usepackage{amsmath, amssymb, xfrac, amsthm}
\usepackage{dsfont}
%\usepackage[usenames,dvipsnames]{xcolor}
\usepackage{amsfonts}
\usepackage{graphicx}
\usepackage{caption}
\usepackage{subcaption}
\usepackage{framed}
\usepackage{multirow}
\usepackage{bytefield}
\usepackage{csquotes}
\usepackage[breakable, theorems, skins]{tcolorbox}
\usepackage{hyperref}
\usepackage{cancel}
\usepackage{bm}


\input{../../style/common}

\tcbset{enhanced}

\DeclareRobustCommand{\mybox}[2][gray!20]{%
	\iffalse
	\begin{tcolorbox}[   %% Adjust the following parameters at will.
		breakable,
		left=0pt,
		right=0pt,
		top=0pt,
		bottom=0pt,
		colback=#1,
		colframe=#1,
		width=\dimexpr\linewidth\relax,
		enlarge left by=0mm,
		boxsep=5pt,
		arc=0pt,outer arc=0pt,
		]
		#2
	\end{tcolorbox}
	\fi
}

\DeclareRobustCommand{\myboxshow}[2][gray!20]{%
%	\iffalse
	\begin{tcolorbox}[   %% Adjust the following parameters at will.
		breakable,
		left=0pt,
		right=0pt,
		top=0pt,
		bottom=0pt,
		colback=#1,
		colframe=#1,
		width=\dimexpr\linewidth\relax,
		enlarge left by=0mm,
		boxsep=5pt,
		arc=0pt,outer arc=0pt,
		]
		#2
	\end{tcolorbox}
%	\fi
}


%exercise numbering
\renewcommand{\theenumi}{(\alph{enumi})}
\renewcommand{\theenumii}{\roman{enumii}}
\renewcommand\labelenumi{\theenumi}


\font \sfbold=cmssbx10

\setlength{\oddsidemargin}{0cm} \setlength{\textwidth}{16cm}


\sloppy
\parindent0em
\parskip0.5em
\topmargin-2.3 cm
\textheight25cm
\textwidth17.5cm
\oddsidemargin-0.8cm
\pagestyle{empty}

\newcommand{\kopf}[1]{
\hrule
\vspace{.15cm}
\begin{minipage}{\textwidth}
%akwardly i had to put \" here to make it compile correctly
	{\sf\bf Optimization in machine learning \hfill Exercise sheet #1\\
	 \url{https://slds-lmu.github.io/website_optimization/} \hfill WS 2023/2024}
\end{minipage}
\vspace{.05cm}
\hrule
\vspace{1cm}}

\newcommand{\kopfic}[1]{
\hrule
\vspace{.15cm}
\begin{minipage}{\textwidth}
%akwardly i had to put \" here to make it compile correctly
	{\sf\bf Optimization in machine learning \hfill Live Session #1\\
	 \url{https://slds-lmu.github.io/website_optimization/} \hfill WS 2023/2024}
\end{minipage}
\vspace{.05cm}
\hrule
\vspace{1cm}}

\newcommand{\kopfsl}[1]{
\hrule
\vspace{.15cm}
\begin{minipage}{\textwidth}
%akwardly i had to put \" here to make it compile correctly
	{\sf\bf Optimization in machine learning \hfill Exercise sheet #1\\
	 \url{https://slds-lmu.github.io/website_optimization/} \hfill WS 2023/2024}
\end{minipage}
\vspace{.05cm}
\hrule
\vspace{1cm}}

\newenvironment{allgemein}
	{\noindent}{\vspace{1cm}}

\newcounter{aufg}
\newenvironment{aufgabe}[1]
	{\refstepcounter{aufg}\textbf{Exercise \arabic{aufg}: #1}\\ \noindent}
	{\vspace{0.5cm}}

\newcounter{loes}
\newenvironment{loesung}
	{\refstepcounter{loes}\textbf{Solution \arabic{loes}:}\\\noindent}
	{\bigskip}
	
\newenvironment{bonusaufgabe}
	{\refstepcounter{aufg}\textbf{Exercise \arabic{aufg}*\footnote{This
	is a bonus exercise.}:}\\ \noindent}
	{\vspace{0.5cm}}

\newenvironment{bonusloesung}
	{\refstepcounter{loes}\textbf{Solution \arabic{loes}*:}\\\noindent}
	{\bigskip}



\begin{document}
% !Rnw weave = knitr



\input{../../latex-math/basic-math.tex}
\input{../../latex-math/basic-ml.tex}

\kopfsl{5}{Univariate Optimization 1}

\aufgabe{Golden Ratio, Brent's Method}{

You are given the following data situation:
\begin{knitrout}
\definecolor{shadecolor}{rgb}{0.969, 0.969, 0.969}\color{fgcolor}\begin{kframe}
\begin{alltt}
\hlkwd{library}\hlstd{(ggplot2)}

\hlkwd{set.seed}\hlstd{(}\hlnum{123}\hlstd{)}

\hlstd{X} \hlkwb{=} \hlkwd{matrix}\hlstd{(}\hlkwd{runif}\hlstd{(}\hlnum{100}\hlstd{),} \hlkwc{ncol} \hlstd{=} \hlnum{2}\hlstd{)}
\hlstd{y} \hlkwb{=} \hlopt{-}\hlstd{((X} \hlopt \hlkwd{c}\hlstd{(}\hlopt{-}\hlnum{1}\hlstd{,} \hlnum{1}\hlstd{)} \hlopt{+} \hlkwd{rnorm}\hlstd{(}\hlnum{50}\hlstd{,} \hlnum{0}\hlstd{,} \hlnum{0.1}\hlstd{)} \hlopt{<} \hlnum{0}\hlstd{)} \hlopt{*} \hlnum{2} \hlopt{-} \hlnum{1}\hlstd{)}
\hlstd{df} \hlkwb{=} \hlkwd{as.data.frame}\hlstd{(X)}
\hlstd{df}\hlopt{$}\hlstd{type} \hlkwb{=} \hlkwd{as.character}\hlstd{(y)}

\hlkwd{ggplot}\hlstd{(df)} \hlopt{+}
  \hlkwd{geom_point}\hlstd{(}\hlkwd{aes}\hlstd{(}\hlkwc{x} \hlstd{= V1,} \hlkwc{y} \hlstd{= V2,} \hlkwc{color}\hlstd{=type))} \hlopt{+}
  \hlkwd{xlab}\hlstd{(}\hlkwd{expression}\hlstd{(theta[}\hlnum{1}\hlstd{]))} \hlopt{+}
  \hlkwd{ylab}\hlstd{(}\hlkwd{expression}\hlstd{(theta[}\hlnum{2}\hlstd{]))}
\end{alltt}
\end{kframe}
\includegraphics[width=0.5\linewidth]{figure/univ-plot-1} 
\end{knitrout}
\\In the following we want to estimate a linear SVM without intercept and with $\lambda = 1$. We assume we know that $\bm{\theta}_2 = 2.$
\begin{enumerate}
  \item Show that if $f:\R^2 \rightarrow \R$ is convex then $g_c: \R \rightarrow \R, x \mapsto f(x, c)\quad \forall c\in\R$ is convex.
  \item  Explain why the non-geometric primal linear SVM formulation should be used rather than the geometric one if we want to find $\bm{\theta}_1$ via the golden ratio algorithm\footnote{We choose this algorithm for educational purposes; in practice, we typically use more advanced algorithms.}.
	\item Find $\bm{\theta}_1$ via the golden ratio algorithm. Implement the algorithm in $\texttt{R}.$ For the termination criterion, use an absolute error of 0.01. Use $[-3, 3]$ as the starting interval.
	\item Given the three points $(x_1, y_1), (x_2, y_2), (x_3, y_3)$ show that the parameters $a, b, c \in \R$ of the interpolating parabola can be found via 
	$\begin{pmatrix}a \\ b \\ c \end{pmatrix} = \begin{pmatrix}  x_1^2 & x_1 & 1 \\ x_2^2 & x_2 & 1 \\ x_3^2 & x_3 & 1 \end{pmatrix}^{-1} \begin{pmatrix}y_1 \\ y_2 \\ y_3 \end{pmatrix}$ when the parabola equation is given by $p(x) = ax^2 + bx + c.$ 
	\item Find $\bm{\theta}_1$ via Brent's method\footnote{We choose this algorithm for educational purposes; in practice, we typically use more advanced algorithms.}. Implement a simplified version\footnote{Only check if the proposed point is in the current interval} of the algorithm in $\texttt{R}.$ For the termination criterion, use an absolute error of 0.01. Use $[-3, 3]$ as the starting interval. For the first step, use a golden ratio step. 
\item Now, assume we do not know $\bm{\theta}_2.$ Our initial guess is $\bm{\theta}_2 = 0.$ We now alternately minimize w.r.t. either $\bm{\theta}_1$ or $\bm{\theta}_2$ via the golden ratio method (the starting interval is always reset to $[-3, 3]$) while the other parameter is held constant. We switch to minimizing the other parameter when the absolute error is smaller than 0.01. Repeat this procedure 10 times.
\item How does the optimization trace of f) look in parameter space?
\item Can we find the optimal parameters of the linear SVM with our procedure in f)? \\
\textit{Hint}: a)

\end{enumerate}
}
\end{document}
